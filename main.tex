\documentclass[a4paper, titlepage, openany]{book}
\usepackage[utf8]{inputenc}
\usepackage[T5]{fontenc}
\usepackage[vietnamese]{babel}
\usepackage[a4paper]{geometry}
\usepackage{amsmath, amssymb, blindtext, hyperref, icomma, multicol, tikz, wrapfig, xcolor}
\usetikzlibrary{calc, shadings}

\geometry{bindingoffset=0.5cm, inner=2cm, outer=2cm, top=2cm, bottom=2cm}

\title{\Huge Ôn tập Vật Lí}
\author{Bùi Nhật Minh}
\date{\today}

% Tạo bài tập
\newcounter{exercise}
\newcommand{\exercise}{
   \refstepcounter{exercise}
   \noindent\textbf{Bài \arabic{exercise}:\label{ex\arabic{exercise}}}
}

% Tạo lời giải
\newcounter{solution}
\newcommand{\solution}{
   \refstepcounter{solution}
   \noindent\textbf{Lời giải \ref{ex\arabic{solution}}:\label{sol\arabic{solution}}}
}

\begin{document}

\maketitle

\setcounter{chapter}{-1}
\tableofcontents

\chapter{Kiến thức toán học nền tảng}

\ % Lùi đầu dòng

Phần này bao gồm các kiến thức toán học cần thiết để xây dựng lí thuyết của môn vật lí, giả sử rằng bạn đọc đã có kiến thức đại số và một chút hình học từ ghế nhà trường. Chương này sẽ bao hàm những phần không nằm trong chương trình trung học phổ thông và có thể cả chương trình đại học. Bạn đọc có thể tìm hiểu một cách sơ cấp hay gợi nhớ lại về các khái niệm toán học mà không cần tập trung vào việc chứng minh chặt chẽ các tính chất toán học. Tác giả mong muốn thông qua chương này, bạn đọc có thể có một cảm nhận và từ đó có kĩ năng để áp dụng các khái niệm toán giải quyết các yêu cầu thực tế. Hơn thế nữa, có một niềm cảm hứng để tìm hiểu chuyên sâu các phân môn của toán học.

\section{Giải tích}

\subsection{Những hàm cơ bản}

\ % Lùi đầu dòng

Trong phần này, các hàm số quen thuộc sẽ được nhắc lại. Đây là những hàm hay thấy nhất trong quá trình học phần lớn các môn khoa học tự nhiên.

Đầu tiên, chúng ta có hàm \textit{đa thức}, thông thường được biểu diễn dưới dạng $$f(x)=P_n(x)=\sum_{i = 0}^n a_i x^i = a_nx^n + a_{n-1}x^{n-1} + \cdots + a_1x + a_0$$ với $n$ là một số nguyên không âm, $a_i$ là các số thực, gọi là các \textit{hệ số}, với mọi $i$ nguyên nằm trong đoạn $[0, n]$ và $a_n \neq 0$. Khi này, $n$ được gọi là \textit{bậc} của đa thức. Khi đa thức có bậc bằng $0$, hay $f(x) = P_0(x) = a_0$, thì được gọi là \textit{đa thức hằng} hay \textit{hàm hằng}. Một trường hợp đặc biệt là khi $f(x) = 0$ \footnote{Sẽ có nhiều sách viết "$f(x) \equiv 0$"\/ thay vì "$f(x) = 0$"\/ để phân biệt giữa khẳng định hai hàm là như nhau so với một phương trình. Tôi không muốn bạn đọc bị vướng víu với nhiều kí hiệu lạ, cho nên tôi sẽ cố gắng dùng những kí hiệu cũ. Bạn đọc có thể tự suy luận ý nghĩa thông qua ngữ cảnh.}, khi này hàm không có bậc nhưng vẫn được gọi là hàm hằng \footnote{Đa số những nhà toán học không coi $f(x) = 0$ là đa thức bậc $0$ do nhiều tính chất của đa thức bị phá vỡ khi gặp trường hợp này. Do đó, $f(x) = 0$ chỉ được coi là "hàm hằng"\/ chứ không phải "\emph{đa thức} hằng". Trong cuốn sách này, trở về sau sẽ chỉ có thuật ngữ "hàm hằng"\/ được sử dụng.}.

Mỗi số hạng của đa thức có dạng $x^n$ với $n$ nguyên.

\subsection{Số ảo và số phức}

\ % Lùi đầu dòng

Trước khi đến với sô thực với số phức, chúng ta bắt đầu tiếp cận với định nghĩa đơn vị ảo. Cụ thể, đơn vị ảo được kí hiệu là $\mathbf{i}$ \footnote{Phần lớn các sách sẽ kí hiệu số ảo là chữ $i$ thông thường. Tôi kí hiệu thành chữ $\mathbf{i}$ đứng in đậm để, thứ nhất, bảo toàn chữ $i$ cho nhiệm vụ khác, và thứ hai, nhấn mạnh rằng nó mang tính chiều.} và thỏa mãn $$\mathbf{i}^2 = -1 \text{ hay } \mathbf{i} = \sqrt{-1}.$$

Để có số ảo, ta nhân một số thực $b \neq 0$ với đơn vị ảo để thành $\mathbf{i}b$. Một số phức bao gồm thành phần thực và thành phẩn ảo cộng vào. Viết dưới dạng kí hiệu, một số phức có dạng là $$z=a+\mathbf{i}b$$ với $a, b$ thực.

Thật sự, qua định nghĩa trên, rất khó cho nhiều người không thường xuyên thường thức về toán ngay lập tức tìm ra và cảm nhận được ý nghĩa thực tiễn của số ảo. Chúng ta không thể tưởng tượng được số ảo một cách trực quan như các số mà chúng ta thường thấy ở ngoài cuộc sống như $5$ cái bút hay $\frac{1}{3}$ giờ. Đi kèm với đó, kể cả trên lí thuyết toán của ghế nhà trường, cũng sẽ không xảy ra trường hợp nào để cho một số nhân với chính nó ra một số âm.

Theo quan điểm cá nhân, số âm trong đời xuống ít khi được sử dụng. Chẳng mấy ai ưa nói "lãi $-500000$ đồng"\/ so với "lỗ $500000$ đồng". Một cách tương tự, trong phần lớn quá trình phát triển của toán học, các nhà toán học xưa thường có mặc cảm với những số âm. Các phương trình sẽ luôn được viết lại thành nhiều trường hợp để tránh chúng. Ví dụ, nếu phương trình bậc hai được viết dưới dạng hiện đại là $x^2 + ax+b=0$ với $a,b$ là hai số thực (có thể âm), thì trong quá khữ, phương trình này được chia ra làm ba trường hợp
\begin{align*}
   &x^2+ax = b;\\
   &x^2+b =ax;\\
   &x^2 =ax+b
\end{align*}
với $a,b$ là hai số thực luôn dương. Và cũng từ sự mặc cảm với số âm, họ cho rằng nghiệm của phương trình cũng phải là một số dương. Tương tự với Các-đa-nô \footnote{Gerolamo Cardano (1501-1576).}, khi giải phương trình bậc ba, ông cũng đưa về các trường hợp như trên. Cụ thể, chúng ta xem xét một trường hợp của bài toán: $$x^3 = ax+b.$$ Giải phương trình, chúng ta có được nghiệm $$x=\sqrt[3]{\frac{b}{2} + \sqrt{\frac{b^2}{4}-\frac{a^3}{27}}}+\sqrt[3]{\frac{b}{2} - \sqrt{\frac{b^2}{4}-\frac{a^3}{27}}}.$$ Tuy nhiên, sau khi thay những giá trị cụ thể vào $a$ và $b$, Các-đa-nô đã phát hiện ra một vấn đề. Khi $a=15$ và $b=4$, nghiệm trả ra cho phương trình $x^3 = 15x+4$ theo công thức vừa trên là $$x=\sqrt[3]{2+\sqrt{-121}}+\sqrt[3]{2-\sqrt{-121}}$$ mặc dù phương trình có một nghiệm bình thường là $x=4$ (với kiến thức toán học hiện đại, chúng ta có thể giải ra hai nghiệm cũng thực khác là $-2 \pm \sqrt{3}$). Nhận ra điều đó, Các-đa-nô đã khẳng định rằng công thức này của ông không áp dụng được trong trường hợp xảy ra căn của một số âm. Tuy nhiên, một học trò của ông, Bom-be-li \footnote{Rafael Bombelli (1526-1572).}, lại phủ nhận điều trên. Bom-be-li nhận định rằng tồn tại một kiểu số khác số thực sẽ có giá trị bằng "căn âm". Ông phân tích rõ cách nhân số thực với số ảo, và sau đó là phương pháp cộng và trừ các số phức với nhau. Sau khi xây dựng một nền tảng đại số cho kiểu số mới này, ông đã tính được căn bậc ba của hai số phức lần lượt là $\sqrt[3]{2+\sqrt{-121}}=2+\sqrt{-1}$ và $\sqrt[3]{2-\sqrt{-121}}=2-\sqrt{-1}$. Cộng hai số vào, hiển nhiên sẽ có được nghiệm $4$ như mong muốn.

\subsection{Giới hạn}
\ % Lùi đầu dòng



\section{Bài tập}
\exercise Phác thảo đồ thị của những hàm sau:
\begin{multicols}{3}
\begin{enumerate}
   \item $f(x) = x + 2$; 
   \item $f(x) = x^2 + 2x + 1$;
   \item $f(x) = x^3 - 9x^2 + 24x - 16$;
   \item $f(x) = 2$
\end{enumerate}
\end{multicols}

\solution

\chapter{Cơ bản của xử lí số liệu trong vật lí}
\exercise Khoảng cách trung bình từ trái đất đến mặt trời là $1,5 \cdot 10^8$ km. Giả sử quỹ đạo của trái đất quanh mặt trời là tròn và mặt trời được đặt tại gốc của hệ quy chiếu.
\begin{enumerate}
   \item Tính tốc độ di chuyển trung bình của trái đất quanh mặt trời dưới dạng dặm trên giờ ($1 \;\text{dặm}=1,6093\;\text{km}$).
   \item Ước lượng góc $\theta$ giữa véc-tơ vị trí của trái đất bây giờ và vị trí sau đó $4$ tháng.
   \item Tính khoảng cách giữa hai vị trí đó.
\end{enumerate}
\solution
\begin{enumerate}
   \item Giả sử trái đất quay quanh mặt trời trong $365,25$ ngày. Quãng đường mà trái đất đi được trong thời gian này là chu vi của quỹ đạo tròn $2 \pi \cdot 1,5 \cdot 10^8$ km. Từ đó, ta có thể tính được tốc độ trung bình của trái đất quanh mặt trời là $\frac{2 \pi \cdot 1,5 \cdot 10^8\ \text{km}}{365,25\ \text{ngày}}$. Thực hiện quy đổi để được:
      \[
         \frac{2 \pi \cdot 1,5 \cdot 10^8\ \text{km}}{365,25\ \text{ngày}}
         \cdot \frac{1\ \text{dặm}}{1,6903\ \text{km}}
         \cdot \frac{1\ \text{ngày}}{24\ \text{h}}
         = \boxed{6,4\cdot 10^4\ \frac{\text{dặm}}{\text{h}}}.
      \]
   \item Trái đất quay quanh mặt trời trong $12$ tháng, tương đương với một góc quay $360^{\circ}$ so với gốc là mặt trời. Coi như các tháng có độ dài như nhau. Ta có $\theta$ chính là góc quay của trái đất trong $4$ tháng, tương đương với:
   \[
      \theta = \frac{360^{\circ}}{12\ \text{tháng}} \cdot 4\ \text{tháng}= \boxed{120^{\circ}}.
   \]
   \item
\end{enumerate}

\begin{figure}[h!]
   \centering
   \begin{tikzpicture}
      \draw[thick] (0,0) circle (2cm);
      \filldraw[black] (0,0) circle (2pt) node[anchor=north] {O};
      \draw[->, thick, >=latex, line width=0.5mm] (0,0) -- (0:2cm) node[midway, below] {$r$};
      \draw[->, thick, >=latex, line width=0.5mm] (0,0) -- (120:2cm);
      \draw[thick, line width=0.5mm] (0:2cm) -- (120:2cm) node[midway, above] {$d$};
      \filldraw[black] (0:2cm) circle (2pt) node[anchor=west] {A};
      \filldraw[black] (120:2cm) circle (2pt) node[anchor=south] {B};
      \node at (60:0.3cm) {$\theta$};
      \draw[thick] (0:0.5cm) arc[start angle=0, end angle=120, radius=0.5cm];
   \end{tikzpicture}
   \caption{Quỹ đạo trái đất}
   \label{fig:earth}
\end{figure}

Gọi $A$ là vị trí của trái đất bây giờ, $B$ là vị trí của trái đất sau $4$ tháng theo như hình \ref{fig:earth}. Coi một đơn vị trên tọa độ bằng độ dài bán kính của quỹ đạo tròn, tức là $r=1,5\cdot10^8$ km. Ta có tọa độ điểm $A$ là $(1;0)$. Tọa độ điểm $B$ là $\left(\cos(120^{\circ}); \sin(120^{\circ})\right)=\left(-\frac{1}{2}; \frac{\sqrt{3}}{2}\right)$. Từ đó, ta có khoảng cách giữa hai vị trí đó là: $$d = r\cdot \sqrt{\left(1-\left(-\frac{1}{2}\right)\right)^2 + \left(\frac{\sqrt{3}}{2}\right)^2}=\boxed{2,6\cdot10^8\ \text{km}}.$$

\exercise Khối lượng riêng (bằng khối lượng của vật chia cho thể tích của vật đó) của nước là $1,00 \frac{\text{g}}{\text{cm}^3}$.
\begin{enumerate}
   \item Tính giá trị này theo ki-lô-gam trên mét khối.
   \item $1,00$ lít nước nặng bao nhiêu ki-lô-gam, bao nhiêu pao (lb)? Biết $1\ \text{lb} = 0,45\ \text{kg}$ (chính xác).
\end{enumerate}

\solution
\begin{enumerate}
   \item Thực hiện quy đổi, ta có:
   \begin{align*}
      1,00 \frac{\text{g}}{\text{cm}^3} &= \left(1,00 \frac{\text{g}}{\text{cm}^3}\right)\cdot\frac{1\ \text{kg}}{1000\ \text{g}}\cdot\left(\frac{100\ \text{cm}}{1\ \text{m}}\right)^3 \\
      &= \boxed{1,00\cdot 10^3\ \frac{\text{kg}}{\text{m}^3}}.
   \end{align*}
   \item Khối lượng của $1,00$ lít nước là
   \begin{align*}
      1,00\ \text{L} \cdot \left(1,00\cdot 10^3\ \frac{\text{kg}}{\text{m}^3}\right)&= 1,00\ \text{L} \cdot \left(1,00\cdot 10^3\ \frac{\text{kg}}{\text{m}^3}\right) \cdot \frac{1\ \text{m}^3}{1000\ \text{L}} \\
      &= \boxed{1,00\cdot 10^0\ \text{kg}}.
   \end{align*}
   Theo đơn vị pao (lb), ta có:
   \[
      1,00\cdot 10^0\ \text{kg} = 1,00\cdot 10^0\ \text{kg} \cdot \frac{1\ \text{lb}}{0,45\ \text{kg}} = \boxed{2,22\cdot 10^0\ \text{lb}}.
   \]
\end{enumerate}

\exercise Trong hệ thời gian cổ Trung Hoa, từ triều đại Thanh trở về trước (trừ một số năm), một ngày được chia thành $100$ khắc. Sau triều đại này (trừ một số năm), một ngày được chia thành $96$ khắc. Coi một ngày có $24$ giờ và mọi số liệu là chính xác tuyệt đối.
\begin{enumerate}
   \item Tính số giây (hệ đo lường hiện đại) trong một khắc trong cả hai thời kì.
   \item Tính tỉ lệ về độ dài của hai khắc trong hai thời kì.
\end{enumerate}

\solution
\begin{enumerate}
   \item Số giây trong một ngày là $$24\ \text{h} \cdot \frac{60\ \text{phút}}{1\ \text{h}} \cdot \frac{60\ \text{giây}}{1\ \text{phút}} = 86400\ \text{giây}.$$
\end{enumerate}
Từ triều đại Thanh trở về trước, số giây trong một khắc là $$\frac{86400\ \text{giây}}{100\ \text{khắc}_{\text{trước}}} = \boxed{864 \frac{\text{giây}}{\text{khắc}_{\text{trước}}}}.$$
Sau triều đại Thanh, số giây trong một khắc là $$\frac{86400\ \text{giây}}{96\ \text{khắc}_{\text{sau}}} = \boxed{900 \frac{\text{giây}}{\text{khắc}_{\text{sau}}}}.$$
\begin{enumerate}
   \item[2] Tỉ lệ độ dài thời gian một khắc trước và sau là $$\frac{1\ \text{khắc}_{\text{trước}}}{1\ \text{khắc}_{\text{sau}}} = \frac{1\ \text{khắc}_{\text{trước}}}{1\ \text{khắc}_{\text{sau}}}\cdot \frac{864\ \text{giây}}{1\ \text{khắc}_{\text{trước}}}\cdot\frac{1\ \text{khắc}_{\text{sau}}}{900\ \text{giây}}=\boxed{0,96}.$$
\end{enumerate}

\exercise Một vòng đĩa tròn như trong hình \ref{fig:vong_dia} có đường kính $4,50$ cm rỗng ở giữa một lỗ đường kính $1,25$ cm. Đĩa dày $1,50$ mm. Biết rằng đĩa được làm từ chất liệu có khối lượng riêng là $8600 \frac{\text{kg}}{\text{m}^3}$. Tính khôi lượng vòng đĩa theo gram.

\begin{figure}
   \centering
   \begin{tikzpicture}
      % Draw the shape
      \draw[bottom color=gray!20] (-2.25, 0) arc[start angle=180, end angle=360, x radius = 2.25cm, y radius = 2.4cm];
      \draw[bottom color=gray!70, top color=gray!20] (0, 0) circle (2.25cm);
      \draw[fill=white] (0, 0) circle (0.625cm);
      \draw[bottom color = gray!20] (-0.625, 0) arc[start angle=180, end angle=0, radius = 0.625cm];
      \draw[fill=white] (-0.625, 0) arc[start angle=180, end angle=0, x radius = 0.625cm, y radius = 0.5cm];
      % Draw the dimension
      \draw[<->] (-2.25,-2.5) -- (2.25,-2.5);
      \node at (0, -2.7) {$4.50$ cm};
      \draw[<->] (-0.625,0) -- (0.625,0);
      \node at (0, -0.2) {$1.25$ cm};
      \draw[<->] (0,0.5) -- (0,0.625);
      \node[anchor=west] at (-0.05, 0.7) {$1.50$ mm};
   \end{tikzpicture}
   \caption{Vòng đĩa tròn}
   \label{fig:vong_dia}
\end{figure}

\solution

Đặt $D=4,50\ \text{cm}=4,50\times 10^{-2}\ \text{m}$, $d=1,25\ \text{cm}=1,25\times 10^{-2}\ \text{m}$, $h=1,50\ \text{mm}=1,50\times 10^{-3}\ \text{m}$ và $\mathcal{D}=8600 \frac{\text{kg}}{\text{m}^3}=8,6\times 10^3 \frac{\text{kg}}{\text{m}^3}\cdot \frac{10^3 \text{g}}{\text{kg}}=8,6\times 10^6 \frac{\text{g}}{\text{m}^3}$.

Nhận thấy rằng đĩa có dạng trụ, diện tích mặt đáy là $$S=\pi\cdot \left(\frac{D}{2}\right)^2-\pi\cdot \left(\frac{d}{2}\right)^2=\frac{\pi \left(D^2-d^2\right)}{4}.$$

Thể tích của đĩa là $V=S\cdot h=\frac{\pi \cdot h\cdot \left(D^2-d^2\right)}{4}.$ Nhân với khối lượng riêng, ta có khối lượng của đĩa là $$m=\mathcal{D}\cdot V=\frac{\pi \cdot h\cdot \mathcal{D}\cdot \left(D^2-d^2\right)}{4}.$$ Thay số trực tiếp với sự để ý đến số chữ số có nghĩa, ta có kết quả $m=\boxed{1,89\times 10^1\ \text{kg}}$.

\exercise Khối lượng của một chất lỏng được mô hình hóa bởi phương trình $m=A\cdot t^{0,8}-B\cdot t$. Nếu như $t$ được tính bằng giây và $m$ được tính bằng ki-lô-gram, thì đơn vị của $A$ và $B$ là gì?

\solution

Để có thể cộng trừ các phần tử, chúng cần phải có cùng đơn vị. Do vậy, đơn vị của $A\cdot t^{0,8}$ và $B\cdot t$ là kg. Từ quy tắc nhân chia các đơn vị, ta có:
\begin{equation*}
   \begin{cases}
     A\cdot \text{s}^{0,8} &=\text{kg} \\
     B\cdot\text{s} &=\text{kg}
   \end{cases}
   \iff
   \begin{cases}
      A &=\frac{\text{kg}}{\text{s}^{0,8}} \\
      B&=\frac{\text{kg}}{\text{s}}
   \end{cases}.
\end{equation*}

Vậy đơn vị của $A$ là $\boxed{\frac{\text{kg}}{\text{s}^{0,8}}}$ và đơn vị của $B$ là $\boxed{\frac{\text{kg}}{\text{s}}}$.

\chapter{Chuyển động}

\exercise Một ô tô đi $40$ km trên một đường thẳng với tốc độ không đổi $40\ \frac{\text{km}}{\text{h}}$. Sau đó, nó đi thêm theo chiều đó $60$ km với tốc độ không đổi $50 \frac{\text{km}}{\text{h}}$. Các giá trị đo được tính đến hai chữ số có nghĩa.
\begin{enumerate}
   \item Tính vận tốc trung bình trên cả quãng đường.
   \item Tính tốc độ trung bình trên cả quãng đường.
   \item Nếu xe quay đầu trước khi đi $50$ km lúc sau, giữ nguyên các số liệu khác, thì vận tốc trung bình và tốc độ trung bình có thay đổi không. Tại sao?
   \item Vẽ đồ thị vị trí $x$ theo thời gian $t$ và từ đó chỉ ra cách tính vận tốc trung bình.
\end{enumerate}
\solution

Coi chiều chuyển động ban đầu là chiều dương.

\begin{enumerate}
   \item Thời gian đi $40$ km đầu là $$40\ \text{km}\div 40\ \frac{\text{km}}{\text{h}}=1,0\ \text{h}.$$
\end{enumerate}

Thời gian đi $50$ km sau là $$60\ \text{km}\div 50\ \frac{\text{km}}{\text{h}}=1,2\ \text{h}.$$

Do hai quãng đường là cùng chiều nên ta có độ dịch chuyển của xe tổng cộng là $$\Delta x=40\ \text{km} + 60\ \text{km} = 100\ \text{km}$$ và tổng thời gian đi là $$\Delta t =1,0\ \text{h}+1,2\ \text{h}=2,2\ \text{h}.$$

Từ đó, ta có vận tốc trung bình là $$\bar{v} = \frac{\Delta x}{\Delta t} =\boxed{4,5\times10^1\ \frac{\text{km}}{\text{h}}}.$$

\begin{enumerate}
   \item[2.] Dễ thấy tổng quãng đường đi là $d=100\ \text{km}$. Tốc độ trung bình là $\bar{s} = \frac{d}{\Delta t}=\boxed{4,5\times10^1\ \frac{\text{km}}{\text{h}}}.$
   \item[3.] Thời gian không thay đổi. Có độ dịch chuyển thay đổi còn $\Delta x = 40\ \text{km} - 60\ \text{km} = -20\ \text{km}$ nhưng tổng quãng đường thì không. Do đó, $\boxed{\text{tốc độ trung bình giữ nguyên}}$ nhưng $\boxed{\text{vận tốc trung bình thay đổi}}$.
   \item[4.] Ta có đồ thị ở hình \ref{fig:do_thi_xe} bằng việc vẽ mối quan hệ $x(t)$ xong nối điểm đầu và điểm cuối. Vận tốc trung bình là độ dốc của đường thẳng nối hai điểm này.
\end{enumerate}

\begin{figure}[h]
   \centering
   \begin{tikzpicture}[scale=1.2]
      \draw[->] (0,0) -- (7,0) node[right] {$t$ (h)};
      \draw[->] (0,0) -- (0,5.5) node[above] {$x$ (km)};
      \node[below] at (3.5,-0.5) {Thời gian};
      \node[rotate=90, above] at (-0.6,2.75) {Vận tốc};
      \draw (0,0) -- (3,2);
      \draw (3,2) -- (6.6,5);
      \filldraw (0,0) circle (1.5pt);
      \filldraw (3,2) circle (1.5pt);
      \filldraw (6.6,5) circle (1.5pt);
      \draw (0,0) -- (-0.08,0) node[left] {$0$};
      \draw (0,2) -- (-0.08,2) node[left] {$40$};
      \draw (0,5) -- (-0.08,5) node[left] {$100$};
      \draw (0,0) -- (0,-0.08) node[below] {$0$};
      \draw (3,0) -- (3,-0.08) node[below] {$1{,}0$};
      \draw (6.6,0) -- (6.6,-0.08) node[below] {$2{,}2$};

      \draw[dashed] (6.6,5) -- (6.6,0);
      \node[left] at (6.6,2.5) {$\Delta x = 100\ \text{km}$};
      \draw[dashed] (0,0) -- (6.6,0);
      \node[above] at (3.3,0) {$\Delta t = 2,2\ \text{h}$};
      \draw[ultra thick] (0,0) -- (6.6,5);
   \end{tikzpicture}
   \caption{Đồ thị vị trí xe-thời gian chạy}
   \label{fig:do_thi_xe}
\end{figure}

\exercise Một máy bay phản lực đang bay ngang ở độ cao $h=42$ mét. Đột nhiên nó bay vào vùng đất dốc lên góc $\theta=4,2^\circ$ (xem hình \ref{fig:may_bay_doc}). Với tốc độ bay là $v=1300\ \frac{\text{km}}{\text{h}}$, thời gian tính từ lúc bay vào vùng đất dốc mà người phi công có để điều chỉnh máy bay là bao nhiêu? Tất cả các số liệu được đo đến hai chữ số có nghĩa.

\begin{figure}[h]
   \centering
   \begin{tikzpicture}[scale=1.2]
      \draw (0,{7*sin(4.2)}) -- (7, 0);
      \draw (7, 0) -- (11, 0);
      \draw[dashed] (0, 0) -- (7, 0);
      \draw (5,0) arc[start angle=180, end angle=175.8, radius=2];

      \draw[->] (5.2,0.5) -- ({7+2*cos(175.8)},{2*sin(175.8)});
      \node[anchor=west] at (5.2,0.5) {$\theta=4,2^\circ$};

      \draw (7,2.5) -- (7.5,2.5) -- (7.5,2.7) -- (7,2.5);
      \filldraw[fill=black] (7.5,2.5) -- (8.1,2.5) -- (7.5,2.7) -- cycle;
      \draw (8.1,2.5) -- (8.3, 2.5) -- (8.3, 2.7) -- cycle;

      \draw[<->, dashed] (7,0) -- (7,2.5);
      \node[anchor=west] at (7,1.25) {$h = 42$ m};
      \draw[->] (7,2.5) -- (4,2.5);
      \node[anchor=south] at (5,2.5) {$v=1300\ \frac{\text{km}}{\text{h}}$};
   \end{tikzpicture}
   \caption{Vị trí máy bay trong vùng dốc lên}
   \label{fig:may_bay_doc}
\end{figure}

\solution

Khoảng cách từ máy bay đến điểm va chạm với mặt đất là $$d=\frac{h}{\tan{(\theta)}}.$$ Từ đó, ta có được thời gian cho phép là $$t=\frac{d}{v}=\frac{h}{v\tan{(\theta)}}.$$

Thay số trực tiếp, với để ý đến sự quy đổi $v=1300\ \frac{\text{km}}{\text{h}}=1300\ \frac{\text{km}}{\text{h}}\frac{1000\ \text{m}}{1\ \text{km}}\frac{1\ \text{h}}{3600\ \text{s}}=361\ \frac{\text{m}}{\text{s}}$, ta có $$t=\boxed{1,6\times 10^0\ s}.$$

\exercise Cho biết vị trí của một vật chuyển động thẳng được xác định bằng $x(t) = a\cdot t^2+b\cdot t+c$. Xác định vị trí, vận tốc và gia tốc của vật tại thời điểm $t=t_0$.

\solution

Vị trí của vật tại $t=t_0$ là $$x\left(t_0\right)=\boxed{a\cdot t_0^2+b\cdot t_0+c}.$$

Vận tốc của vật tại $t=t_0$ là $$v\left(t_0\right)=\left.\frac{\mathrm{d}x(t)}{\mathrm{d}t}\right|_{t=t_0}=\boxed{2a\cdot t_0+b}.$$

Gia tốc của vật tại $t=t_0$ là $$a\left(t_0\right)=\left.\frac{\mathrm{d}v(t)}{\mathrm{d}t}\right|_{t=t_0}=\boxed{2a}.$$

\exercise Phác họa đồ thị vị trí - thời gian và gia tốc thời gian của một người chạy bộ nếu đồ thị vận tốc - thời gian của người đó được biểu diễn trên đồ thị
\begin{enumerate}
   \item hình \ref{fig:chay_phan_a};
   \item hình \ref{fig:chay_phan_b}.
\end{enumerate}
Các số liệu được coi như chính xác tuyệt đối. Bạn có thể giả sử người đó bắt đầu chạy từ vị trí $x = 0$.

\begin{figure}[h]
   \centering
   \begin{minipage}[t]{0.48\textwidth}
      \centering
      \begin{tikzpicture}
         \draw[->] (0,0) -- (5.5,0) node[right] {$t$ (s)};
         \draw[->] (0,0) -- (0,4.5) node[above] {$v\left(\frac{\text{m}}{\text{s}}\right)$};
         \node[below] at (2.75,-0.5) {Thời gian};
         \node[rotate=90, above] at (-0.5,2.25) {Vận tốc};
         
         \draw[thick] (0,0) -- (1, 4);
         \draw[thick] (1,4) -- (2, 4);
         \draw[thick] (2,4) -- (4,3);
         \draw[thick] (4,3) -- (5, 0);

         \foreach \x/\y in {1/4, 2/4, 4/3} {
            \draw[dashed] (\x,0) -- (\x,\y);
         }
         \draw[dashed] (0,4) -- (1,4);
         \draw[dashed] (4,3) -- (0,3);
         \foreach \x in {0,1,2,4,5} {
            \draw (\x,0) -- (\x,-0.08) node[below] {$\x$};
         }
         \foreach \y in {0, 3, 4} {
            \draw (0,\y) -- (-0.08,\y) node[left] {$\y$};
         }
      \end{tikzpicture}
      \caption{Phần 1}
      \label{fig:chay_phan_a}
   \end{minipage}
   \hfill
   \begin{minipage}[t]{0.48\textwidth}
      \centering
      \begin{tikzpicture}
         \draw[->] (0,0) -- (5.5,0) node[right] {$t$ (s)};
         \draw[->] (0,0) -- (0,4.5) node[above] {$v\left(\frac{\text{m}}{\text{s}}\right)$};

         \node[below] at (2.75,-0.5) {Thời gian};
         \node[rotate=90, above] at (-0.5,2.25) {Vận tốc};
         
         \draw[domain=0:4, smooth, variable=\x, thick] plot ({\x}, {-\x*(4*\x^3-31*\x^2+77*\x-68)/6});
         \draw[thick] (4,0) -- (5,0);
         \foreach \x in {0,1,2,3,4,5} {
            \draw (\x,0) -- (\x,-0.080) node[below] {$\x$};
         }
         \foreach \y in {0,2,3,4} {
            \draw (0,\y) -- (-0.08,\y) node[left] {$\y$};
         }

         \foreach \x/\y in {1/3, 2/2, 3/4} {
            \draw[dashed] (\x,0) -- (\x,\y);
            \draw[dashed] (0,\y) -- (\x,\y);
         }
         
      \end{tikzpicture}
      \caption{Phần 2}
      \label{fig:chay_phan_b}
   \end{minipage}
\end{figure}

\solution
\begin{enumerate}
   \item Ta chia quá trình chạy làm $4$ phần.
\end{enumerate}

\begin{itemize}
   \item Phần 1 $\left(0\ \text{s}\leq t \leq 1\ \text{s}\right)$: Vận tốc tăng đều từ $0$ đến $4\ \frac{\text{m}}{\text{s}}$. Chuyển động là nhanh dần với gia tốc không đổi là $\left.a(t)\right|_{t\in\left[0\ \text{s};1\ \text{s}\right]}=\frac{v(1\ \text{s})-v(0\ \text{s})}{1\ \text{s}-0\ \text{s}}=4\ \frac{\text{m}}{\text{s}^2}$.
   
Sau khoảng thời gian $t$, độ dịch chuyển là $\left.x(t)\right|_{t\in\left[0\ \text{s};1\ \text{s}\right]} - x(0\ \text{s}) = \frac{\left.a(t)\right|_{t\in\left[0\ \text{s};1\ \text{s}\right]}\cdot t^2}{2} + \left.v(t)\right|_{t\in\left[0\ \text{s};1\ \text{s}\right]}\cdot t$. Từ đó ta có $x(t) = 2\ \frac{\text{m}}{\text{s}^2}\cdot t^2$ với $0\ \text{s}\leq t \leq 1\ \text{s}$ và $x(1\ \text{s}) = 2\ \text{m}$.
   
   \item Phần 2 $\left(1\ \text{s}\leq t \leq 2\ \text{s}\right)$: Vận tốc không đổi ở $\left.v(t)\right|_{t\in\left[1\ \text{s};2\ \text{s}\right]} = 4\ \frac{\text{m}}{\text{s}}$ (chuyển động thẳng đều). 
   
Qua đó, ta có $\left.x(t)\right|_{t\in\left[1\ \text{s};2\ \text{s}\right]} = x(1\ \text{s}) + \left.v(t)\right|_{t\in\left[1\ \text{s};2\ \text{s}\right]}\cdot\left(t - 1\ \text{s}\right) = 4\ \frac{\text{m}}{\text{s}}\cdot t - 2\ \text{m}$ và $x(2\ \text{s}) = 6\ \text{m}$.
\end{itemize}

Phần 3 $\left(2\ \text{s}\leq t \leq 4\ \text{s}\right)$ và phần 4 $\left(4\ \text{s}\leq t \leq 5\ \text{s}\right)$ làm tương tự như phần 1. Ta được
\begin{equation*}
   \begin{cases}
     \left.a(t)\right|_{t\in\left[2\ \text{s};4\ \text{s}\right]} &= -\frac{1}{2}\ \frac{\text{m}}{\text{s}^2}\\
     \left.a(t)\right|_{t\in\left[4\ \text{s};5\ \text{s}\right]} &= -3\ \frac{\text{m}}{\text{s}^2}\\
   \end{cases}
\end{equation*}
và qua đó
\begin{equation*}
   \begin{cases}
     \left.x(t)\right|_{t\in\left[2\ \text{s};4\ \text{s}\right]} &= -\frac{1}{4}\ \frac{\text{m}}{\text{s}^2}\cdot\left(t - 2\ \text{s}\right)^2 + 4\ \frac{\text{m}}{\text{s}}\cdot \left(t - 2\ \text{s}\right) + 6\ \text{m}\\
     \left.x(t)\right|_{t\in\left[4\ \text{s};5\ \text{s}\right]} &= -\frac{3}{2}\ \frac{\text{m}}{\text{s}^2}\cdot\left(t - 4\ \text{s}\right)^2 + 3\ \frac{\text{m}}{\text{s}}\cdot \left(t - 4\ \text{s}\right) + 13\ \text{m}\\
   \end{cases}
\end{equation*}

\begin{equation*}
         \iff
   \begin{cases}
     \left.x(t)\right|_{t\in\left[2\ \text{s};4\ \text{s}\right]} &= -\frac{1}{4}\ \frac{\text{m}}{\text{s}^2}\cdot t^2 + 5\ \frac{\text{m}}{\text{s}}\cdot t - 3\ \text{m}\\
     \left.x(t)\right|_{t\in\left[4\ \text{s};5\ \text{s}\right]} &= -\frac{3}{2}\ \frac{\text{m}}{\text{s}^2}\cdot t^2 + 15\ \frac{\text{m}}{\text{s}}\cdot t - 23\ \text{m}\\
   \end{cases}.
\end{equation*}

Cuối cùng, chúng ta có thể biểu diễn vị trí của người chạy trên đồ thị như hình \ref{fig:vt_tg1}.

\begin{figure}[h]
   \centering
   \fbox{
      \begin{tikzpicture}
         \draw[->] (0,0) -- (5.5,0) node[right] {$t$ (s)};
         \draw[->] (0,0) -- (0,4.5) node[above] {$x\left(\text{m}\right)$};
         \node[below] at (2.75,-0.5) {Thời gian};
         \node[rotate=90, above] at (-0.5,2.25) {Vị trí};
         
         \draw[domain=0:1, smooth, variable=\t, thick] plot ({\t}, {\t^2 / 2});
         \draw[domain=1:2, smooth, variable=\t, thick] plot ({\t}, {\t-1/2});
         \draw[domain=2:4, smooth, variable=\t, thick] plot ({\t}, {-1/16*(\t-2)^2+(\t-2)+3/2});
         \draw[domain=4:5, smooth, variable=\t, thick] plot ({\t}, {-3/8*\t^2+15/4*\t-23/4});
         \foreach \x/\y in {1/2, 2/6, 4/13, 5/14.5} {
            \draw[dashed] (\x,0) -- (\x,\y/4);
            \draw[dashed] (0,\y/4) -- (\x,\y/4);
         }
         \foreach \x in {0,1,2,4,5} {
            \draw (\x,0) -- (\x,-0.08) node[below] {$\x$};
         }
         \foreach \y in {0, 2, 6, 13, 14.5} {
            \draw (0,\y/4) -- (-0.08,\y/4) node[left] {$\y$};
         }
      \end{tikzpicture}
   }
   \caption{Đồ thị vị trí - thời gian cho phần 1}
   \label{fig:vt_tg1}
\end{figure}

\begin{enumerate}
   \item[2.] 
\end{enumerate}
Chúng ta có thể phác họa đồ thị vị trí - thời gian bằng việc xấp xỉ đồ thị vận tốc - thời gian dưới dạng đường gấp khúc nối các điểm đã biết thể hiện ở \ref{fig:xx_p2}.

Từ đây, thực hiện tương tự như phần 1 để có phương trình vị trí - thời gian
\begin{equation*}
   x(t) = \begin{cases}
      \frac{3}{2}\ \frac{\text{m}}{\text{s}^2}\cdot t^2 &\quad \text{với } 0 \leq t < 1\ \text{s}\\
      -\frac{1}{2}\ \frac{\text{m}}{\text{s}^2}\cdot t^2 + 4\ \frac{\text{m}}{\text{s}}\cdot t - 2\ \text{m}&\quad \text{với } 1\ \text{s} \leq t < 2\ \text{s}\\
      1\ \frac{\text{m}}{\text{s}^2}\cdot t^2 - 2\ \frac{\text{m}}{\text{s}}\cdot t + 4\ \text{m}&\quad \text{với } 2\ \text{s} \leq t < 3\ \text{s}\\
      -2\ \frac{\text{m}}{\text{s}^2}\cdot t^2+16\ \frac{\text{m}}{\text{s}}\cdot t-23\ \text{m}&\quad \text{với } 3\ \text{s} \leq t < 4\ \text{s}\\
      9\ \text{m}&\quad \text{với } 4\ \text{s} \leq t \leq 5\ \text{s}
   \end{cases}
\end{equation*}
và ta vẽ được đồ thị ở hình \ref{fig:vttgxxp2}.

\begin{figure}[h]
   \centering
   \begin{minipage}[t]{0.48\textwidth}
      \centering
      \begin{tikzpicture}
         \draw[->] (0,0) -- (5.5,0) node[right] {$t$ (s)};
         \draw[->] (0,0) -- (0,5) node[above] {$v\left(\frac{\text{m}}{\text{s}}\right)$};

         \node[below] at (2.75,-0.5) {Thời gian};
         \node[rotate=90, above] at (-0.5,2.25) {Vận tốc};
         
         \draw[thick] (0,0) -- (1,3) -- (2,2) -- (3,4) -- (4,0) -- (5,0);
         \foreach \x in {0,1,2,3,4,5} {
            \draw (\x,0) -- (\x,-0.080) node[below] {$\x$};
         }
         \foreach \y in {0,2,3,4} {
            \draw (0,\y) -- (-0.08,\y) node[left] {$\y$};
         }

         \foreach \x/\y in {1/3, 2/2, 3/4} {
            \draw[dashed] (\x,0) -- (\x,\y);
            \draw[dashed] (0,\y) -- (\x,\y);
         }
      \end{tikzpicture}
      \caption{Vận tốc - thời gian xấp xỉ của phần 2}
      \label{fig:xx_p2}
   \end{minipage}
   \hfill
   \begin{minipage}[t]{0.48\textwidth}
      \centering
      \fbox{
         \begin{tikzpicture}
            \draw[->] (0,0) -- (5.5,0) node[right] {$t$ (s)};
            \draw[->] (0,0) -- (0,5) node[above] {$x\left(\text{m}\right)$};
            \node[below] at (2.75,-0.5) {Thời gian};
            \node[rotate=90, above] at (-0.5,2.25) {Vị trí};
            
            \draw[domain=0:1, smooth, variable=\t, thick] plot ({\t}, {(3*\t^2 / 2) / 2});
            \draw[domain=1:2, smooth, variable=\t, thick] plot ({\t}, {(-\t^2 / 2 + 4*\t - 2)/2});
            \draw[domain=2:3, smooth, variable=\t, thick] plot ({\t}, {(\t^2 -2*\t +4)/2});
            \draw[domain=3:4, smooth, variable=\t, thick] plot ({\t}, {(-2*\t^2+16*\t-23)/2});
            \draw[thick] (4,4.5) -- (5,4.5);

            \foreach \x/\y in {1/1.5, 2/4, 3/7, 4/9} {
               \draw[dashed] (\x,0) -- (\x,\y/2);
               \draw[dashed] (0,\y/2) -- (\x,\y/2);
            }
            \draw[dashed] (5,0) -- (5,4.5);
            \foreach \x in {0,1,2,3,4,5} {
               \draw (\x,0) -- (\x,-0.08) node[below] {$\x$};
            }
            \foreach \y in {0, 1.5, 4, 7, 9} {
               \draw (0,\y/2) -- (-0.08,\y/2) node[left] {$\y$};
            }
         \end{tikzpicture}
      }
      \caption{Vị trí - thời gian (xấp xỉ) cho phần 2}
      \label{fig:vttgxxp2}
   \end{minipage}
\end{figure}

\begin{figure}[h!]
   \centering
   \fbox{
      \begin{tikzpicture}
         \draw[->] (0,0) -- (5.5,0) node[right] {$t$ (s)};
         \draw[->] (0,0) -- (0,6) node[above] {$v\left(\frac{\text{m}}{\text{s}}\right)$};

         \node[below] at (2.75,-0.5) {Thời gian};
         \node[rotate=90, above] at (-0.5,2.25) {Vận tốc};
         
         \draw[domain=0:4, smooth, variable=\x, thick] plot ({\x}, {-\x^2*(48*\x^3-465*\x^2+1540*\x-2040)/720});
         \draw[thick] (4,{496/90}) -- (5,{496/90});
         \foreach \x in {0,1,2,3,4,5} {
            \draw (\x,0) -- (\x,-0.080) node[below] {$\x$};
         }
         \draw (0,{496/90}) -- (-0.08,{496/90}) node[left] {$\approx 11$};

         \foreach \x/\y in {4/{496/45}, 5/{496/45}} {
            \draw[dashed] (\x,0) -- (\x,{\y/2});
         }
         \draw[dashed] (0,{496/90}) -- (4,{496/90});
      \end{tikzpicture}
   }
   \caption{Đồ thị vị trí - thời gian cho phần 2}
   \label{fig:vttgp2}
\end{figure}

Trong thực tiễn, chúng ta hay xấp xỉ những quá trình không tuyến tính qua hữu hạn những điểm đo rồi nội suy tuyến tính (nối các điểm bằng các đoạn thẳng) như đã làm. Còn nhiều phương pháp nội suy nữa còn có thể được tìm thấy trong những tài liệu về phương pháp tính và giải tích số. Thông thường, với càng nhiều điểm thì độ chính xác càng lớn.

Trong trường hợp mà bạn nhận ra phương trình vận tốc - thời gian được cho là
\begin{equation*}
   v(t) =
   \begin{cases}
      \displaystyle \frac{\displaystyle -t\left(4\ \frac{\text{m}}{\text{s}^5}\cdot t^3-31\ \frac{\text{m}}{\text{s}^4}\cdot t^2+77\ \frac{\text{m}}{\text{s}^3}\cdot t-68\ \frac{\text{m}}{\text{s}^2}\right)}{6} &\quad \text{với } 0 \leq t < 4 \\
      0&\quad \text{với } 4 \leq t \leq 5
   \end{cases}
\end{equation*}
thì bạn có thể thực hiện nguyên hàm trên hàm này để tính được vị trí vật là
\begin{equation*}
   \displaystyle 
   x(t) =
   \begin{cases}
      \displaystyle \frac{\displaystyle -t^2\left(48\ \frac{\text{m}}{\text{s}^5}\cdot t^3-465\ \frac{\text{m}}{\text{s}^4}\cdot t^2+1540\ \frac{\text{m}}{\text{s}^3}\cdot t-2040\ \frac{\text{m}}{\text{s}^2}\right)}{360} &\quad \text{với } 0 \leq t < 4 \\
      \displaystyle \frac{496}{45}\ \text{m}&\quad \text{với } 4 \leq t \leq 5
   \end{cases}
\end{equation*}
và ta có đồ thị như hình \ref{fig:vttgp2}.

\exercise Hai xe hơi có tốc độ lần lượt là $v_1 = 50,0\ \frac{\text{km}}{\text{h}}$ và $v_2 = 60,0\ \frac{\text{km}}{\text{h}}$ đi ngược chiều với nhau trên một con đường hẹp. Hai xe phát hiện lẫn nhau khi khoảng cách giữa hai xe là $d = 400\ \text{m}$. Cả hai xe đồng thời giảm tốc với cùng một gia tốc hãm đều là $a$. Tính giá trị tối thiểu của $a$ nếu biết hai xe không xảy ra va chạm. Số liệu được đo tới $3$ chữ số có nghĩa.

\solution

Gọi quãng đường đi được trong khi hãm phanh của hai xe lần lượt là $d_1$ và $d_2$.

Trong quá trình hãm đến vận tốc bằng $0$, tổng quãng đường đi của cả hai xe phải không vượt quá khoảng cách $d$. Vì vậy, ta có bất đẳng thức $$d_1 + d_2 \leq d.$$

Trong khi đó, quãng đường xe thứ nhất đã di chuyển là $d_1 = \frac{0^2 - v_1^2}{2(-a)} = \frac{v_1^2}{2a}$. Tương tự, ta có quãng đường mà xe thứ hai di chuyển trong khoảng thời gian này là $d_2 = \frac{v_2^2}{2a}$. Từ đó, thay vào phương trình ở trên để được $$
   \frac{v_1^2}{2a} + \frac{v_2^2}{2a} \le d
   \iff a \geq \frac{v_1^2+v_2^2}{2d}.
$$

Thay số trực tiếp, ta có gia tốc hãm tối thiểu phải là $\boxed{7,63 \times 10^3 \frac{\text{km}}{\text{h}^2}}$.

\exercise Để dừng xe ban đầu bạn cần một thời gian phản ứng để bắt đầu phanh, rồi xe mới đi chậm dần nhờ có một gia tốc hãm không đổi. Giả sử quãng được đi được trong hai pha này là $186$ ft nếu vận tốc ban đầu là $50\ \frac{\text{dặm}}{\text{h}}$. Còn trong một trường hợp khác, quãng được đi được trong hai pha này là $80$ ft nếu vận tốc ban đầu là $30\ \frac{\text{dặm}}{\text{h}}$. Biết thời gian phản ứng là cố định và $1$ dặm $= 5280$ ft, tính thời gian phản ứng và độ lớn của gia tốc hãm.

\solution

Gọi thời gian phản ứng là $t_p$, vận tốc đầu là $v_0$, gia tốc hãm là $a$.

Trong khoảng thời gian phản ứng, xe đi được $v_0t_p$. Và trong khoảng thời gian hãm, xe đi được $\frac{0^2-v_0^2}{2(-a)}=\frac{v_0^2}{2a}$. Cho nên, tổng quãng đượt đi được trong hai pha là 
\begin{equation}
\Delta x = v_0 t + \frac{v_0^2}{2a}
\label{eq:stopping_distance}
\end{equation}

Trước khi thay số ta thực hiện quy đổi $$50\ \frac{\text{dặm}}{\text{h}}=50\ \frac{\text{dặm}}{\text{h}}\cdot\frac{5280\ \text{ft}}{1\ \text{dặm}}\cdot\frac{1\ \text{h}}{3600\ \text{s}}\approx 73\ \frac{\text{ft}}{\text{s}},$$ tương tự, $30\ \frac{\text{dặm}}{\text{h}}=44\ \frac{\text{ft}}{\text{s}}$. Từ đó, thay số vào phương trình \ref{eq:stopping_distance} để có hệ
\begin{equation*}
   \begin{cases}
      186\ \text{ft} = 73\ \frac{\text{ft}}{\text{s}}\cdot t_p + \frac{\left(73\ \frac{\text{ft}}{\text{s}}\right)^2}{2a} \\
      80\ \text{ft} = 44\ \frac{\text{ft}}{\text{s}}\cdot t_p + \frac{\left(44\ \frac{\text{ft}}{\text{s}}\right)^2}{2a} 
   \end{cases}.
\end{equation*}
Giải hệ phương trình, ta có thời gian phản ứng là $t_p=0,97\ \text{s}$ và gia tốc hãm là $a = 26\ \frac{\text{ft}}{\text{s}^2}$.

\begin{wrapfigure}{r}{0.6\textwidth} % 'r' for right, 'l' for left, width of the figure
    \centering
    \begin{tikzpicture}
      \draw[->] (0,0) -- (8,0) node[right] {$t$};
      \draw[->] (0,0) -- (0,6) node[above] {$h$};

      \node[below] at (4,-0.25) {Thời gian};
      \node[rotate=90, above] at (-0.25,3) {Độ cao};

      \draw[domain=0.8:7.2, smooth, variable=\x, thick] plot ({\x}, {-4 * (\x - 1) * (\x - 7)/9 + 1});
      \draw[<->] (1,1) -- (7,1);
      \node[anchor=south] at (4,1) {$\Delta T_t$};
      \draw[<->] (3,41/9) -- (5,41/9);
      \node[anchor=north] at (4,41/9) {$\Delta T_c$};
      \draw[<->] (0.5,1) -- (0.5,41/9);
      \node[anchor=west] at (0.5,25/9) {$H$};

      \draw[dashed] (0,1) -- (1,1);
      \draw[dashed] (7,1) -- (8,1);
      \draw[dashed] (0,41/9) -- (3,41/9);
      \draw[dashed] (5,41/9) -- (8,41/9);

   \end{tikzpicture}

   \caption{Đồ thị thời gian - độ cao của quả bóng thủy tinh}
   \label{fig:tgdcqbtt}
\end{wrapfigure}

\exercise Tại Phòng Thí nghiệm Vật lí Quốc gia ở Anh, người ta thực hiện xác định gia tốc trọng trường $g$ theo thí nghiệm sau: Ném một quả bóng thủy tinh lên theo chiều thẳng đứng trong ống chân không và cho nó rơi xuống. Gọi $\Delta T_t$ trên hình \ref{fig:tgdcqbtt} là thời gian khoảng giữa hai lần quả bóng đi qua một điểm thấp nào đó. $\Delta T_c$ là khoảng thời gian giữa hai lần quả bóng đi qua một điểm cao hơn và $H$ là khoảng cách giữa hai điểm. Chứng minh rằng $$g=\frac{8H}{\Delta T_t^2 - \Delta T_c^2}.$$

\solution

Gọi vận tốc khi bóng bắt đầu bay lên từ vị trị thấp là $v_0$. Sau một khoảng thời gian $\Delta T_t$, quả bóng quay lại vị trí cũ, do vậy, ta có phương trình $0 = -\frac{g \Delta T_t^2}{2} + v_0 \Delta T_t$. Thực hiện biến đổi tương đương để có $$v_0=\frac{g \Delta T_t}{2}.$$

Nhận thấy rằng đồ thị có tính đối xứng. Sử dụng điều đó, ta tính được khoảng thời gian quả bóng lên một độ cao $H$ là $t=\frac{\Delta T_t-\Delta T_c}{2}$. Qua đó, ta có phương trình thứ hai là $$H = -\frac{g t^2}{2} + v_0 t=-\frac{g \left(\frac{\Delta T_t-\Delta T_c}{2}\right)^2}{2} + v_0 \left(\frac{\Delta T_t-\Delta T_c}{2}\right).$$

Thế giá trị của $v_0$ vào phương trình và tiếp tục thực hiện biến đổi, ta có:
\begin{align*}
   H &= -\frac{g \left(\Delta T_t-\Delta T_c\right)^2}{8} + \frac{g \Delta T_t}{2} \left(\frac{\Delta T_t-\Delta T_c}{2}\right) \\
   &= -g\left(\frac{\Delta T_t^2}{8} - \frac{\Delta T_t\Delta T_c}{4} + \frac{\Delta T_c^2}{8}\right) + g\left(\frac{\Delta T_t^2}{4} - \frac{\Delta T_t\Delta T_c}{4}\right) \\
   &= g\cdot \frac{\Delta T_t^2 - \Delta T_c^2}{8} \\
   \iff g &= \frac{8H}{\Delta T_t^2 - \Delta T_c^2}.
\end{align*}

Ta có điều phải chứng minh.

\exercise Một nghệ sĩ tung hứng các quả bóng lên theo phương thẳng đứng. Quả bóng sẽ lên cao hơn bao nhiêu nếu thời gian bóng trong không khí tăng gấp $n$ lần ($n \in \mathbb{R}^+$)?

\solution

Có thời gian để quả bóng bay từ tay lên trên vị trí cao nhất bằng một nửa thời gian bóng trong không khí. Nếu thời gian bóng trong không khí tăng gấp $n$ lần so với thời gian trong không khí gốc, thì cùng chia cho $2$, ta cũng sẽ có thời gian bóng bay từ tay lên trên vị trí cao nhất cũng tăng gấp $n$ lần so với thời gian gốc để bay lên vị trí cao nhất.

Gọi $t_1$ là thời gian gốc để bóng bay từ tay lên vị trí cao nhất, $t_2 = n t_1$ là thời gian bay khi đã tăng $n$ lần. Gọi $h_1, h_2$ lần lượt là độ cao bóng đi được tương ứng với hai khoảng thời gian $t_1, t_2$. Để ý rằng khi lên vị trí cao nhất thì vận tốc bóng là $0$; ta có hệ phương trình

\begin{equation*}
   \begin{cases}
      h_1 &= \frac{gt_1^2}{2} \\
      h_2 &= \frac{gt_2^2}{2} = \frac{g\left(nt_1\right)^2}{2}
   \end{cases}
   \implies h_2 = n^2 h_1.
\end{equation*}

Từ đó, ta có quả bóng cao lên $\boxed{n^2 - 1 \text{ lần độ cao gốc}}$.

\begin{thebibliography}{1}
\bibitem{Agarwal2011}
Agarwal, R.P., Perera, K., Pinelas, S. (2011). \textit{History of Complex Numbers}. In: An Introduction to Complex Analysis. Springer, Boston, MA. \url{https://doi.org/10.1007/978-1-4614-0195-7_50}
\end{thebibliography}

\end{document}

