\documentclass[a4paper, titlepage, openany]{book}
\usepackage[utf8]{inputenc}
\usepackage[T5]{fontenc}
\usepackage[vietnamese]{babel}
\usepackage[a4paper]{geometry}
\usepackage{amsmath, amssymb, hyperref, icomma, tikz, xcolor}
\usetikzlibrary{calc, shadings}

\geometry{bindingoffset=0.5cm, inner=2cm, outer=2cm, top=2cm, bottom=2cm}

\title{\Huge Ôn tập Vật Lí}
\author{Bùi Nhật Minh}
\date{\today}

% Tạo bài tập
\newcounter{exercise}
\newcommand{\exercise}{
   \refstepcounter{exercise}
   \noindent\textbf{Bài \arabic{exercise}:\label{ex\arabic{exercise}}}
}

% Tạo lời giải
\newcounter{solution}
\newcommand{\solution}{
   \refstepcounter{solution}
   \noindent\textbf{Lời giải \ref{ex\arabic{solution}}:\label{sol\arabic{solution}}}
}

\begin{document}

\maketitle

\chapter{Cơ bản của xử lí số liệu trong vật lí}
\exercise Khoảng cách trung bình từ trái đất đến mặt trời là $1,5 \cdot 10^8$ km. Giả sử quỹ đạo của trái đất quanh mặt trời là tròn và mặt trời được đặt tại gốc của hệ quy chiếu.
\begin{enumerate}
   \item Tính tốc độ di chuyển trung bình của trái đất quanh mặt trời dưới dạng dặm trên giờ ($1 \;\text{dặm}=1,6093\;\text{km}$).
   \item Ước lượng góc $\theta$ giữa véc-tơ vị trí của trái đất bây giờ và vị trí sau đó $4$ tháng.
   \item Tính khoảng cách giữa hai vị trí đó.
\end{enumerate}
\solution
\begin{enumerate}
   \item Giả sử trái đất quay quanh mặt trời trong $365,25$ ngày. Quãng đường mà trái đất đi được trong thời gian này là chu vi của quỹ đạo tròn $2 \pi \cdot 1,5 \cdot 10^8$ km. Từ đó, ta có thể tính được tốc độ trung bình của trái đất quanh mặt trời là $\frac{2 \pi \cdot 1,5 \cdot 10^8\ \text{km}}{365,25\ \text{ngày}}$. Thực hiện quy đổi để được:
      \[
         \frac{2 \pi \cdot 1,5 \cdot 10^8\ \text{km}}{365,25\ \text{ngày}}
         \cdot \frac{1\ \text{dặm}}{1,6903\ \text{km}}
         \cdot \frac{1\ \text{ngày}}{24\ \text{h}}
         = \boxed{6,4\cdot 10^4\ \frac{\text{dặm}}{\text{h}}}.
      \]
   \item Trái đất quay quanh mặt trời trong $12$ tháng, tương đương với một góc quay $360^{\circ}$ so với gốc là mặt trời. Coi như các tháng có độ dài như nhau. Ta có $\theta$ chính là góc quay của trái đất trong $4$ tháng, tương đương với:
   \[
      \theta = \frac{360^{\circ}}{12\ \text{tháng}} \cdot 4\ \text{tháng}= \boxed{120^{\circ}}.
   \]
   \item
\end{enumerate}

\begin{figure}[h!]
   \centering
   \begin{tikzpicture}
      \draw[thick] (0,0) circle (2cm);
      \filldraw[black] (0,0) circle (2pt) node[anchor=north] {O};
      \draw[->, thick, >=latex, line width=0.5mm] (0,0) -- (0:2cm) node[midway, below] {$r$};
      \draw[->, thick, >=latex, line width=0.5mm] (0,0) -- (120:2cm);
      \draw[thick, line width=0.5mm] (0:2cm) -- (120:2cm) node[midway, above] {$d$};
      \filldraw[black] (0:2cm) circle (2pt) node[anchor=west] {A};
      \filldraw[black] (120:2cm) circle (2pt) node[anchor=south] {B};
      \node at (60:0.3cm) {$\theta$};
      \draw[thick] (0:0.5cm) arc[start angle=0, end angle=120, radius=0.5cm];
   \end{tikzpicture}
   \caption{Quỹ đạo trái đất}
   \label{fig:earth}
\end{figure}

Gọi $A$ là vị trí của trái đất bây giờ, $B$ là vị trí của trái đất sau $4$ tháng theo như hình \ref{fig:earth}. Coi một đơn vị trên tọa độ bằng độ dài bán kính của quỹ đạo tròn, tức là $r=1,5\cdot10^8$ km. Ta có tọa độ điểm $A$ là $(1;0)$. Tọa độ điểm $B$ là $\left(\cos(120^{\circ}); \sin(120^{\circ})\right)=\left(-\frac{1}{2}; \frac{\sqrt{3}}{2}\right)$. Từ đó, ta có khoảng cách giữa hai vị trí đó là: $$d = r\cdot \sqrt{\left(1-\left(-\frac{1}{2}\right)\right)^2 + \left(\frac{\sqrt{3}}{2}\right)^2}=\boxed{2,6\cdot10^8\ \text{km}}.$$

\exercise Khối lượng riêng (bằng khối lượng của vật chia cho thể tích của vật đó) của nước là $1,00 \frac{\text{g}}{\text{cm}^3}$.
\begin{enumerate}
   \item Tính giá trị này theo ki-lô-gam trên mét khối.
   \item $1,00$ lít nước nặng bao nhiêu ki-lô-gam, bao nhiêu pao (lb)? Biết $1\ \text{lb} = 0,45\ \text{kg}$ (chính xác).
\end{enumerate}

\solution
\begin{enumerate}
   \item Thực hiện quy đổi, ta có:
   \begin{align*}
      1,00 \frac{\text{g}}{\text{cm}^3} &= \left(1,00 \frac{\text{g}}{\text{cm}^3}\right)\cdot\frac{1\ \text{kg}}{1000\ \text{g}}\cdot\left(\frac{100\ \text{cm}}{1\ \text{m}}\right)^3 \\
      &= \boxed{1,00\cdot 10^3\ \frac{\text{kg}}{\text{m}^3}}.
   \end{align*}
   \item Khối lượng của $1,00$ lít nước là
   \begin{align*}
      1,00\ \text{L} \cdot \left(1,00\cdot 10^3\ \frac{\text{kg}}{\text{m}^3}\right)&= 1,00\ \text{L} \cdot \left(1,00\cdot 10^3\ \frac{\text{kg}}{\text{m}^3}\right) \cdot \frac{1\ \text{m}^3}{1000\ \text{L}} \\
      &= \boxed{1,00\cdot 10^0\ \text{kg}}.
   \end{align*}
   Theo đơn vị pao (lb), ta có:
   \[
      1,00\cdot 10^0\ \text{kg} = 1,00\cdot 10^0\ \text{kg} \cdot \frac{1\ \text{lb}}{0,45\ \text{kg}} = \boxed{2,22\cdot 10^0\ \text{lb}}.
   \]
\end{enumerate}

\exercise Trong hệ thời gian cổ Trung Hoa, từ triều đại Thanh trở về trước (trừ một số năm), một ngày được chia thành $100$ khắc. Sau triều đại này (trừ một số năm), một ngày được chia thành $96$ khắc. Coi một ngày có $24$ giờ và mọi số liệu là chính xác tuyệt đối.
\begin{enumerate}
   \item Tính số giây (hệ đo lường hiện đại) trong một khắc trong cả hai thời kì.
   \item Tính tỉ lệ về độ dài của hai khắc trong hai thời kì.
\end{enumerate}

\solution
\begin{enumerate}
   \item Số giây trong một ngày là $$24\ \text{h} \cdot \frac{60\ \text{phút}}{1\ \text{h}} \cdot \frac{60\ \text{giây}}{1\ \text{phút}} = 86400\ \text{giây}.$$
\end{enumerate}
Từ triều đại Thanh trở về trước, số giây trong một khắc là $$\frac{86400\ \text{giây}}{100\ \text{khắc}_{\text{trước}}} = \boxed{864 \frac{\text{giây}}{\text{khắc}_{\text{trước}}}}.$$
Sau triều đại Thanh, số giây trong một khắc là $$\frac{86400\ \text{giây}}{96\ \text{khắc}_{\text{sau}}} = \boxed{900 \frac{\text{giây}}{\text{khắc}_{\text{sau}}}}.$$
\begin{enumerate}
   \item[2] Tỉ lệ độ dài thời gian một khắc trước và sau là $$\frac{1\ \text{khắc}_{\text{trước}}}{1\ \text{khắc}_{\text{sau}}} = \frac{1\ \text{khắc}_{\text{trước}}}{1\ \text{khắc}_{\text{sau}}}\cdot \frac{864\ \text{giây}}{1\ \text{khắc}_{\text{trước}}}\cdot\frac{1\ \text{khắc}_{\text{sau}}}{900\ \text{giây}}=\boxed{0,96}.$$
\end{enumerate}

\exercise Một vòng đĩa tròn như trong hình \ref{fig:vong_dia} có đường kính $4,50$ cm rỗng ở giữa một lỗ đường kính $1,25$ cm. Đĩa dày $1,50$ mm. Biết rằng đĩa được làm từ chất liệu có khối lượng riêng là $8600 \frac{\text{kg}}{\text{m}^3}$. Tính khôi lượng vòng đĩa theo gram.

\begin{figure}
   \centering
   \begin{tikzpicture}
      % Draw the shape
      \draw[bottom color=gray!20] (-2.25, 0) arc[start angle=180, end angle=360, x radius = 2.25cm, y radius = 2.4cm];
      \draw[bottom color=gray!70, top color=gray!20] (0, 0) circle (2.25cm);
      \draw[fill=white] (0, 0) circle (0.625cm);
      \draw[bottom color = gray!20] (-0.625, 0) arc[start angle=180, end angle=0, radius = 0.625cm];
      \draw[fill=white] (-0.625, 0) arc[start angle=180, end angle=0, x radius = 0.625cm, y radius = 0.5cm];
      % Draw the dimension
      \draw[<->] (-2.25,-2.5) -- (2.25,-2.5);
      \node at (0, -2.7) {$4.50$ cm};
      \draw[<->] (-0.625,0) -- (0.625,0);
      \node at (0, -0.2) {$1.25$ cm};
      \draw[<->] (0,0.5) -- (0,0.625);
      \node[anchor=west] at (-0.05, 0.7) {$1.50$ mm};
   \end{tikzpicture}
   \caption{Vòng đĩa tròn}
   \label{fig:vong_dia}
\end{figure}

\solution

Đặt $D=4,50\ \text{cm}=4,50\times 10^{-2}\ \text{m}$, $d=1,25\ \text{cm}=1,25\times 10^{-2}\ \text{m}$, $h=1,50\ \text{mm}=1,50\times 10^{-3}\ \text{m}$ và $\mathcal{D}=8600 \frac{\text{kg}}{\text{m}^3}=8,6\times 10^3 \frac{\text{kg}}{\text{m}^3}\cdot \frac{10^3 \text{g}}{\text{kg}}=8,6\times 10^6 \frac{\text{g}}{\text{m}^3}$.

Nhận thấy rằng đĩa có dạng trụ, diện tích mặt đáy là $$S=\pi\cdot \left(\frac{D}{2}\right)^2-\pi\cdot \left(\frac{d}{2}\right)^2=\frac{\pi \left(D^2-d^2\right)}{4}.$$

Thể tích của đĩa là $V=S\cdot h=\frac{\pi \cdot h\cdot \left(D^2-d^2\right)}{4}.$ Nhân với khối lượng riêng, ta có khối lượng của đĩa là $$m=\mathcal{D}\cdot V=\frac{\pi \cdot h\cdot \mathcal{D}\cdot \left(D^2-d^2\right)}{4}.$$ Thay số trực tiếp với sự để ý đến số chữ số có nghĩa, ta có kết quả $m=\boxed{1,89\times 10^1\ \text{kg}}$.

\exercise Khối lượng của một chất lỏng được mô hình hóa bởi phương trình $m=A\cdot t^{0,8}-B\cdot t$. Nếu như $t$ được tính bằng giây và $m$ được tính bằng ki-lô-gram, thì đơn vị của $A$ và $B$ là gì?

\solution

Để có thể cộng trừ các phần tử, chúng cần phải có cùng đơn vị. Do vậy, đơn vị của $A\cdot t^{0,8}$ và $B\cdot t$ là kg. Từ quy tắc nhân chia các đơn vị, ta có:
\begin{equation*}
   \begin{cases}
     A\cdot \text{s}^{0,8} &=\text{kg} \\
     B\cdot\text{s} &=\text{kg}
   \end{cases}
   \iff
   \begin{cases}
      A &=\frac{\text{kg}}{\text{s}^{0,8}} \\
      B&=\frac{\text{kg}}{\text{s}}
   \end{cases}.
\end{equation*}

Vậy đơn vị của $A$ là $\boxed{\frac{\text{kg}}{\text{s}^{0,8}}}$ và đơn vị của $B$ là $\boxed{\frac{\text{kg}}{\text{s}}}$.

\chapter{Chuyển động trên một đường thẳng}
\exercise Một ô tô đi $40$ km trên một đường thẳng với tốc độ không đổi $40\ \frac{\text{km}}{\text{h}}$. Sau đó, nó đi thêm theo chiều đó $60$ km với tốc độ không đổi $50 \frac{\text{km}}{\text{h}}$. Các giá trị đo được tính đến hai chữ số có nghĩa.
\begin{enumerate}
   \item Tính vận tốc trung bình trên cả quãng đường.
   \item Tính tốc độ trung bình trên cả quãng đường.
   \item Nếu xe quay đầu trước khi đi $50$ km lúc sau, giữ nguyên các số liệu khác, thì vận tốc trung bình và tốc độ trung bình có thay đổi không. Tại sao?
   \item Vẽ đồ thị vị trí $x$ theo thời gian $t$ và từ đó chỉ ra cách tính vận tốc trung bình.
\end{enumerate}
\solution

Coi chiều chuyển động ban đầu là chiều dương.

\begin{enumerate}
   \item Thời gian đi $40$ km đầu là $$40\ \text{km}\div 40\ \frac{\text{km}}{\text{h}}=1,0\ \text{h}.$$
\end{enumerate}

Thời gian đi $50$ km sau là $$60\ \text{km}\div 50\ \frac{\text{km}}{\text{h}}=1,2\ \text{h}.$$

Do hai quãng đường là cùng chiều nên ta có độ dịch chuyển của xe tổng cộng là $$\Delta x=40\ \text{km} + 60\ \text{km} = 100\ \text{km}$$ và tổng thời gian đi là $$\Delta t =1,0\ \text{h}+1,2\ \text{h}=2,2\ \text{h}.$$

Từ đó, ta có vận tốc trung bình là $$\bar{v} = \frac{\Delta x}{\Delta t} =\boxed{4,5\times10^1\ \frac{\text{km}}{\text{h}}}.$$

\begin{enumerate}
   \item[2.] Dễ thấy tổng quãng đường đi là $d=100\ \text{km}$. Tốc độ trung bình là $\bar{s} = \frac{d}{\Delta t}=\boxed{4,5\times10^1\ \frac{\text{km}}{\text{h}}}.$
   \item[3.] Thời gian không thay đổi. Có độ dịch chuyển thay đổi còn $\Delta x = 40\ \text{km} - 60\ \text{km} = -20\ \text{km}$ nhưng tổng quãng đường thì không. Do đó, $\boxed{\text{tốc độ trung bình giữ nguyên}}$ nhưng $\boxed{\text{vận tốc trung bình thay đổi}}$.
   \item[4.] Ta có đồ thị ở hình \ref{fig:do_thi_xe} bằng việc vẽ mối quan hệ $x(t)$ xong nối điểm đầu và điểm cuối. Vận tốc trung bình là độ dốc của đường thẳng nối hai điểm này.
\end{enumerate}

\begin{figure}
   \centering
   \begin{tikzpicture}[scale=1.2]
      \draw[->] (0,0) -- (7,0) node[right] {$t$ (h)};
      \draw[->] (0,0) -- (0,5.5) node[above] {$x$ (km)};
      \draw (0,0) -- (3,2);
      \draw (3,2) -- (6.6,5);
      \filldraw (0,0) circle (1.5pt);
      \filldraw (3,2) circle (1.5pt);
      \filldraw (6.6,5) circle (1.5pt);
      \draw (0,0) -- (-0.08,0) node[left] {$0$};
      \draw (0,2) -- (-0.08,2) node[left] {$40$};
      \draw (0,5) -- (-0.08,5) node[left] {$100$};
      \draw (0,0) -- (0,-0.08) node[below] {$0$};
      \draw (3,0) -- (3,-0.08) node[below] {$1{,}0$};
      \draw (6.6,0) -- (6.6,-0.08) node[below] {$2{,}2$};

      \draw[dashed] (6.6,5) -- (6.6,0);
      \node[left] at (6.6,2.5) {$\Delta x = 100\ \text{km}$};
      \draw[dashed] (0,0) -- (6.6,0);
      \node[above] at (3.3,0) {$\Delta t = 2,2\ \text{h}$};
      \draw[ultra thick] (0,0) -- (6.6,5);
   \end{tikzpicture}
   \caption{Đồ thị vị trí xe-thời gian chạy}
   \label{fig:do_thi_xe}
\end{figure}

\exercise Một máy bay phản lực đang bay ngang ở độ cao $h=42$ mét. Đột nhiên nó bay vào vùng đất dốc lên góc $\theta=4,2^\circ$ (xem hình \ref{fig:may_bay_doc}). Với tốc độ bay là $v=1300\ \frac{\text{km}}{\text{h}}$, thời gian tính từ lúc bay vào vùng đất dốc mà người phi công có để điều chỉnh máy bay là bao nhiêu? Tất cả các số liệu được đo đến hai chữ số có nghĩa.

\begin{figure}
   \centering
   \begin{tikzpicture}[scale=1.2]
      \draw (0,{7*sin(4.2)}) -- (7, 0);
      \draw (7, 0) -- (11, 0);
      \draw[dashed] (0, 0) -- (7, 0);
      \draw (5,0) arc[start angle=180, end angle=175.8, radius=2];

      \draw[->] (5.2,0.5) -- ({7+2*cos(175.8)},{2*sin(175.8)});
      \node[anchor=west] at (5.2,0.5) {$\theta=4,2^\circ$};

      \draw (7,2.5) -- (7.5,2.5) -- (7.5,2.7) -- (7,2.5);
      \filldraw[fill=black] (7.5,2.5) -- (8.1,2.5) -- (7.5,2.7) -- cycle;
      \draw (8.1,2.5) -- (8.3, 2.5) -- (8.3, 2.7) -- cycle;

      \draw[<->, dashed] (7,0) -- (7,2.5);
      \node[anchor=west] at (7,1.25) {$h = 42$ m};
      \draw[->] (7,2.5) -- (4,2.5);
      \node[anchor=south] at (5,2.5) {$v=1300\ \frac{\text{km}}{\text{h}}$};
   \end{tikzpicture}
   \caption{Vị trí máy bay trong vùng dốc lên}
   \label{fig:may_bay_doc}
\end{figure}

\solution

Khoảng cách từ máy bay đến điểm va chạm với mặt đất là $$d=\frac{h}{\tan{(\theta)}}.$$ Từ đó, ta có được thời gian cho phép là $$t=\frac{d}{v}=\frac{h}{v\tan{(\theta)}}.$$

Thay số trực tiếp, với để ý đến sự quy đổi $v=1300\ \frac{\text{km}}{\text{h}}=1300\ \frac{\text{km}}{\text{h}}\frac{1000\ \text{m}}{1\ \text{km}}\frac{1\ \text{h}}{3600\ \text{s}}=361\ \frac{\text{m}}{\text{s}}$, ta có $$t=\boxed{1,6\times 10^0\ s}.$$

\exercise Cho biết vị trí của một vật chuyển động thẳng được xác định bằng $x(t) = a\cdot t^2+b\cdot t+c$. Xác định vị trí, vận tốc và gia tốc của vật tại thời điểm $t=t_0$.

\solution

Vị trí của vật tại $t=t_0$ là $x\left(t_0\right)=\boxed{a\cdot t_0^2+b\cdot t_0+c}$.



\end{document}