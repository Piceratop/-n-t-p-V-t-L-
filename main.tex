\documentclass[a4paper, titlepage, openany]{book}
\usepackage[utf8]{inputenc}
\usepackage[T5]{fontenc}
\usepackage[vietnamese]{babel}
\usepackage[a4paper]{geometry}
\usepackage{amsmath, amssymb, blindtext, calc, float, hyperref, icomma, multicol, steinmetz, tikz, tikz-3dplot, wrapfig, xcolor, xparse, xeCJK}
\usetikzlibrary{arrows.meta, calc, shadings}
\geometry{bindingoffset=0.5cm, inner=2cm, outer=2cm, top=2cm, bottom=2cm}
\setCJKmainfont[AutoFakeSlant=0.15,AutoFakeBold=2.0]{Nom Na Tong}

\DeclareMathOperator{\Arg}{Arg}

\title{\Huge Ôn tập Vật Lí}
\author{Bùi Nhật Minh}
\date{\today}

\newcounter{exercise}

\NewDocumentCommand{\exercise}{o}{
   \refstepcounter{exercise}
   \noindent\textbf{Bài \arabic{exercise}:}
   \IfNoValueTF{#1}
     {\label{ex\arabic{exercise}}}
     {\label{#1}}
}

% Counter for solutions
\newcounter{solution}

\NewDocumentCommand{\solution}{o}{
   \refstepcounter{solution}
   \IfNoValueTF{#1}
      {\noindent\textbf{Lời giải cho bài~\ref{ex\arabic{solution}}:}}
      {\noindent\textbf{Lời giải cho bài~\ref{#1}}:}
   \IfNoValueTF{#1}
      {\label{sol\arabic{solution}}}
      {\label{sol-#1}}
}

\newcommand\dblquote[1]{\textquotedblleft #1\textquotedblright}

\begin{document}

\maketitle

\setcounter{chapter}{-1}
\tableofcontents

\chapter{Kiến thức toán học nền tảng}

\ % Lùi đầu dòng

Phần này bao gồm các kiến thức toán học cần thiết để xây dựng lí thuyết của môn vật lí (hoặc ít nhất để đọc tài liệu này), giả sử rằng bạn đọc đã có kiến thức đại số và một chút hình học từ ghế nhà trường. Chương này sẽ bao hàm những phần không nằm trong chương trình trung học phổ thông và có thể cả chương trình đại học. Bạn đọc có thể tìm hiểu một cách sơ cấp hay gợi nhớ lại về các khái niệm toán học mà không cần tập trung vào việc chứng minh chặt chẽ các tính chất toán học. Tác giả mong muốn thông qua chương này, bạn đọc có thể có một cảm nhận và từ đó có kĩ năng để áp dụng các khái niệm toán giải quyết các yêu cầu thực tế. Hơn thế nữa, có một niềm cảm hứng để tìm hiểu chuyên sâu các phân môn của toán học. Ngoài ra, vì các bạn đọc đang đọc về tài liệu nghiêng về Vật Lí, tác giả sẽ không tập trung nhiều vào chặt chẽ toán học. Các định nghĩa và chứng minh vẫn sẽ được đưa ra, không quá chặt chẽ nhưng đủ ý, nhằm làm bước đệm cho bạn đọc nếu có mong muốn tìm hiểu sâu hơn về toán.

\section{Đồ thị}

\subsection{Trục số một chiều}

\ % Lùi đầu dòng

Đồ thị là cầu nối đầu tiên giữa đại số và hình học mà chắc là bạn đọc đã được họcn Thông thường, nhắc đến đồ thị, chúng thường được dùng để biểu thị mặt phẳng hai chiều hoặc không gian ba chiều. Nhưng, đồ thị cơ bản nhất chỉ có một chiều, hay tên gọi khác là \emph{trục số}. 

\def \pointSize {1.5pt}

\begin{wrapfigure}{r}{0.4\textwidth}
   \centering
   \begin{tikzpicture}
      \draw[->] (-3,0) -- (2,0) node[right] {Trục số};
      \filldraw (0, 0) circle (\pointSize) node[below] {$O(0)$};
      \filldraw (1.5, 0) circle (\pointSize) node[below] {$P(x_P)$};
      \filldraw (-1.5, 0) circle (\pointSize) node[below] {$P_-(-x_P)$};
   \end{tikzpicture}
   \caption{Trục số một chiều}
   \label{fig:truc so mot chieu}
\end{wrapfigure}

Đặt một điểm trên trục làm gốc tọa độ $0$, từ đó chúng ta có thể biểu diễn mọi số thực trên trục số này. Nói một cách không chính thống, với một số $x_P$ dương bất kì, đánh dấu cách $O$ một đoạn bằng $x_P$ đơn vị độ dài theo hướng trục, ta có điểm $P$ biểu diễn $x_P$. Viết tắt cách biểu diễn, được $P(x_P)$. Ngược lại, nếu chúng ta muốn đánh dấu số $x_{P_-}=-x_P$ mang giá trị âm, chúng ta dịch ngược lại chiều trục như trên hình \ref{fig:truc so mot chieu}.

\begin{wrapfigure}{r}{0.4\textwidth}
   \centering
   \begin{tikzpicture}
      \draw[->] (-3,0) -- (2,0) node[right] {$x$};
      \filldraw (0, 0) circle (\pointSize) node[below] {$O$};
      \pgfmathsetmacro{\xP}{1.2}
      \pgfmathsetmacro{\xQ}{-2}
      \pgfmathsetmacro{\h}{0.4}
      \filldraw (\xP, 0) circle (\pointSize) node[below] {$P(x_P)$};
      \filldraw (\xQ, 0) circle (\pointSize) node[below] {$Q(x_Q)$};
      \draw[thick] (\xQ,0) -- (\xP,0);
      \node[above] at ({(\xP+\xQ)/2}, {\h/2}) {$d(P;Q)$};
      \draw (\xP,0) -- (\xP,\h);
      \draw (\xQ,0) -- (\xQ,\h);
      \draw[<->, >=latex, shorten >=\pointSize, shorten <=\pointSize] (\xP,{\h/2}) -- (\xQ,{\h/2});
   \end{tikzpicture}
   \caption{Khoảng cách trên trục số}
   \label{fig:truc so mot chieu}
\end{wrapfigure}

Khi có nhiều điểm ở trên đồ thị, chúng ta sẽ mong muốn tính những thông số liên quan tới những điểm đó. Do chương này mang tính giới thiệu, chúng ta sẽ chỉ tập trung vào một đặc điểm nhất định, \emph{khoảng cách}. Trên một trục số như hình \ref{fig:toa do vuong goc}, cho hai điểm $P(x_P)$ và $Q(x_Q)$, khoảng cách giữa chúng là $$d(P;Q)=\sqrt{(x_P-x_Q)^2}=|x_P-x_Q|.$$

\exercise[ex:0.1] Biểu diễn nhóm các điểm sau trên trục số. Tính khoảng cách giữa hai điểm phân biệt bất kì trong nhóm đó.
\begin{enumerate}
   \item $A(2)$, $B(-3)$, và $C(4)$;
   \item $D(\pi)$, $E(-\pi)$, $F(0)$, và $G\left(\frac{\pi}{2}\right)$;
   \item $H(0{,}\bar{3})$ và $I(\sqrt{2})$;
   \item $J\left(\frac{355}{113}\right)$, $K\left(\frac{9801}{2206\sqrt{2}}\right)$ và $L\left(\sqrt[4]{\frac{2143}{22}}\right)$;
   \item $M(x)$ và $N(2x)$ với $x\in\mathbb{R}$.
\end{enumerate}

\solution[ex:0.1] 
\begin{figure}[h]
   \centering
   \begin{tikzpicture}[scale=1]

      \begin{scope}[yshift=2cm]
         \draw[->] (-5,0) -- (5,0) node[right] {$x$};
         \filldraw (0, 0) circle (\pointSize) node[above] {$O(0)$};
         \filldraw (2, 0) circle (\pointSize) node[below] {$A(2)$};
         \filldraw (-3, 0) circle (\pointSize) node[below] {$B(-3)$};
         \filldraw (4, 0) circle (\pointSize) node[below] {$C(4)$};
      \end{scope}

      \begin{scope}[yshift=0cm]
         \draw[->] (-5,0) -- (5,0) node[right] {$x$};
         \filldraw ({pi}, 0) circle (\pointSize) node[below] {$D(\pi)$};
         \filldraw ({-pi}, 0) circle (\pointSize) node[below] {$E(-\pi)$};
         \filldraw (0, 0) circle (\pointSize) node[below] {$F(0)$};
         \filldraw ({pi/2}, 0) circle (\pointSize) node[below] {$G\left(\frac{\pi}{2}\right)$};
      \end{scope}

      \begin{scope}[yshift=-2cm, xshift=-5cm]
         \draw[->] (0,0) -- (10,0) node[right] {$x$};
         \filldraw (0, 0) circle (\pointSize) node[above] {$O(0)$};
         \filldraw ({2/3}, 0) circle (\pointSize) node[below] {$H\left(0{,}\bar{3}\right)$};
         \filldraw ({2*sqrt(2)}, 0) circle (\pointSize) node[below] {$I(\sqrt{2})$};
         \filldraw (4, 0) circle (\pointSize) node[above] {$C(4)$};
      \end{scope}

      \begin{scope}[yshift=-4cm, xshift=-5cm]

         \draw[->] (0,0) -- (10,0) node[right] {$x$};
         \filldraw (0,0) circle (\pointSize) node[above] {$P(3{,}1415926)$};
         \filldraw (3,0) circle (\pointSize) node[above] {$Q(3{,}1415927)$};
         \filldraw (6,0) circle (\pointSize) node[above] {$R(3{,}1415928)$};
         \filldraw (9,0) circle (\pointSize) node[above] {$S(3{,}1415929)$};
         \filldraw (9.61062, 0) circle (\pointSize) node[below] {$J\left(\frac{355}{113}\right)$};
         \filldraw (3.9, 0) circle (\pointSize) node[below] {$K\left(\frac{9801}{2206\sqrt{2}}\right)$};
         \filldraw (1.577475, 0) circle (\pointSize) node[below] {$L\left(\sqrt[4]{\frac{2143}{22}}\right)$};
      \end{scope}

   \end{tikzpicture}
   \caption{Bốn trục số cho các phần từ $1$ đến $4$ của bài \ref{ex:0.1}}
   \label{fig:bon truc}
\end{figure}

Ta có đồ thị cho các phần từ $1$ đến $4$ như hình \ref{fig:bon truc}.

Cần lưu ý rằng, để biểu diễn thuận lợi nhất, các trục số khi biểu diễn sốcần được chọn những tỉ lệ khác nhau và tại những vị trí khác nhau.

Các khoảng cách giữa hai điểm phân biệt đôi một là
\begin{enumerate}
   \item \begin{alignat*}{2}
      &d(A;B) = d(B;A) = \left|2 - (-3)\right| = 5; \\
      &d(B;C) = d(C;B) = \left|4 - (-3)\right| = 7; \\
      &d(C;A) = d(A;C) = \left|4 - 2\right| = 2.
   \end{alignat*}
   \item \begin{alignat*}{2}
      &d(D;E) = d(E;D) = \left|\pi - (-\pi)\right| = 2\pi; \\
      &d(E;F) = d(F;E) = \left|(-\pi) - 0\right| = \pi; \\
      &d(F;G) = d(G;F) = \left|0 - \frac{\pi}{2}\right| = \frac{\pi}{2}; \\
      &d(G;D) = d(D;G) = \left|\frac{\pi}{2} - \pi\right| = \frac{\pi}{2}; \\
      &d(D;F) = d(F;D) = \left|\pi - 0\right| = \pi;\\
      &d(E;G) = d(G;E) = \left|(-\pi) - \frac{\pi}{2}\right| = \frac{3\pi}{2}.
   \end{alignat*}
   \item \begin{alignat*}{2}
      &d(H;I) = d(I;H) = \left|0{,}\bar{3} - \sqrt{2}\right| = \frac{1-3\sqrt{2}}{3}.
   \end{alignat*}
   \item \begin{alignat*}{2}
      &d(J;K) = d(K;J) = \left|\frac{355}{113} - \frac{9801}{2206\sqrt{2}}\right| = \frac{1566260-1107513\sqrt{2}}{498556} \approx 1{,}9034\times 10^{-7}; \\
      &d(K;L) = d(L;K) = \left|\frac{9801}{2206\sqrt{2}} - \sqrt[4]{\frac{2143}{22}}\right| = \frac{107811\sqrt{2}-2206\sqrt[4]{22818664}}{48532} \approx 7{,}7431\times 10^{-8}; \\
      &d(L;J) = d(J;L) = \left|\sqrt[4]{\frac{2143}{22}} - \frac{355}{113}\right| = \frac{7810-113\sqrt[4]{22818664}}{2486} \approx 2{,}6777\times 10^{-7}.
   \end{alignat*}
\end{enumerate}

\begin{figure}[h]
   \centering
   \begin{tikzpicture}[scale=1]
      \draw[->] (-5,0) -- (5,0) node[right] {$x$};
      \filldraw (0, 0) circle (\pointSize) node[above] {$O(0)$};
      \filldraw (pi, 0) circle (\pointSize) node[below] {$J\left(\frac{355}{113}\right)$};
      \node[below] at (pi, -0.6) {$K\left(\frac{9801}{2206\sqrt{2}}\right)$};
      \node[below] at (pi, -1.2) {$L\left(\sqrt[4]{\frac{2143}{22}}\right)$};
      \node[above] at (pi, 0) {$\approx \pi$};
   \end{tikzpicture}
   \caption{Xấp xỉ vị trí điểm trên trục số cho phần $4$ của bài \ref{ex:0.1}}
   \label{fig:truc bon xx}
\end{figure}

Trong vật lí, việc tính toán chính xác đến như ở phần $4$ là không cần thiết và nhiều khi còn không chính xác. Luôn luôn có sai số khi đo đạc, và trong phần lớn trường hợp, khi kết hợp sai số này vào trong tính toán thì các giá trị khoảng cách như trên gần như vô nghĩa. Cho nên, về mặt thực tiễn, chúng ta hoàn toàn có thể thay thế đồ thị của $4$ như hình \ref{fig:truc bon xx} và khoảng cách chúng ta có thể tính xấp xỉ là $$d(J,K) = d(K,J) \approx d(K,L) = d(L,K) \approx d(L,J) = d(J,L) \approx 0.$$

\begin{figure}[h]
   \centering
   \begin{tikzpicture}[scale=1]
      \draw[->] (-5,0) -- (5,0) node[right] {$x$};
      \filldraw (0, 0) circle (\pointSize) node[above] {$O(0)$} node[below] {$M_0(x)$};
      \node[below] at (0, -0.6) {$N_0(2x)$};
      \filldraw ({e / 2}, 0) circle (\pointSize) node[below] {$M_+\left(x\right)$};
      \filldraw (e, 0) circle (\pointSize) node[below] {$N_+\left(2x\right)$};
      \filldraw ({-sqrt(3)}, 0) circle (\pointSize) node[below] {$M_-\left(x\right)$};
      \filldraw ({-sqrt(3) * 2}, 0) circle (\pointSize) node[below] {$N_-\left(2x\right)$};
   \end{tikzpicture}
   \caption{Ba trường hợp cho vị trí tương đối của $M$, $N$, $O$ cho phần $5$ của bài \ref{ex:0.1}}
   \label{fig:truc phan 5}
\end{figure}

Để vẽ được đồ thị cho phần $5$, chúng ta sẽ xét vị trí tương đối giữa $M$, $N$ kèm theo gốc $O$ để quy chiếu như biểu diễn ở hình \ref{fig:truc phan 5}. Cụ thể, khi $x>0$, điểm $M$ và $N$ được biểu diễn thành hai điểm $M_+$ và $N_+$. Tương tự, khi $x<0$, $M$ và $N$ biểu diễn hai điểm $M_-$ và $N_-$. Một trường hợp đặc biệt là khi $x=0$, $M$ và $N$ đều có tọa độ là $0$, cho nên hai điểm đó và gốc cùng chia sẻ vị trí với nhau.

Khoảng cách giữa hai điểm $M$ và $N$ luôn là $$d(M;N)=d(N;M)=|x-2x|=|x|.$$

\subsection{Mặt phẳng hai chiều và hệ tọa độ vuông góc}

\begin{figure}[h]
   \centering
   \begin{minipage}[b]{0.48\textwidth}
      \centering
      \begin{tikzpicture}
         \draw[->] (-2,0) -- (2,0);
         \draw[->] (0,-2) -- (0,2);
         \node[right] at (2,0) {Trục hoành};
         \node[above] at (0,2) {Trục tung};
         \filldraw (0, 0) circle (1.5pt);
         \node[below left] at (0, 0) {$O(0; 0)$};

         \node at (1,1) {$\boxed{\text{I}}$};
         \node at (-1,1) {$\boxed{\text{II}}$};
         \node at (1,-1) {$\boxed{\text{IV}}$};
         \node at (-1,-1) {$\boxed{\text{III}}$};

         \draw (1,0) -- (1,-0.08) node[below] {$x_P$};
         \draw (0,1.5) -- (-0.08,1.5) node[left] {$y_P$};

         \filldraw (1, 1.5) circle (1.5pt);
         \node[above right] at (1, 1.5) {$P(x_P; y_P)$};

         \draw[dashed] (1, 1.5) -- (1, 0);
         \draw[dashed] (1, 1.5) -- (0, 1.5);
      \end{tikzpicture}
      \caption{Hệ tọa độ vuông góc}
      \label{fig:toa do vuong goc}
   \end{minipage}
   \hfill
   \begin{minipage}[b]{0.48\textwidth}
      \centering
      \begin{tikzpicture}
         \draw[->] (-2,0) -- (2,0);
         \draw[->] (0,-2) -- (0,2);
         \node[right] at (2,0) {$x$};
         \node[above] at (0,2) {$y$};
         \filldraw (0, 0) circle (1.5pt);
         \node[below left] at (0, 0) {$O$};

         \pgfmathsetmacro{\xP}{1}
         \pgfmathsetmacro{\yP}{1.5}
         \pgfmathsetmacro{\xQ}{-1.2}
         \pgfmathsetmacro{\yQ}{-1}

         \filldraw (\xP, \yP) circle (1.5pt);
         \node[above right] at (\xP, \yP) {$P(x_P; y_P)$};

         \filldraw (\xQ, \yQ) circle (1.5pt);
         \node[below right] at (\xQ, \yQ) {$Q(x_Q; y_Q)$};

         \draw[thick] (\xP, \yP) -- (\xQ, \yQ);
         \draw[dashed] (\xP, \yP) -- (\xP, \yQ);
         \draw[dashed] (\xP, \yQ) -- (\xQ, \yQ);

         \node[above left] at ({(\xP+\xQ)/2}, {(\yP+\yQ)/2}) {$d(P;Q)$};
      \end{tikzpicture}
      \caption{Khoảng cách giữa hai điểm}
      \label{fig:khoang cach 2d}
   \end{minipage}
\end{figure}


Mở rộng lên mặt phẳng hai chiều, nếu chúng ta đặt hai trục vuông góc với nhau và giao nhau tại gốc $O(0)$ của mỗi trục, khi đó, chúng ta có thể xác định vị trị của điểm trên mặt phẳng chứa hai trục theo biểu diễn đại số bằng cách dóng điểm đó lên trục mà sau này được gọi là \emph{tọa độ}. Đây được gọi là \emph{hệ tọa độ vuông góc} (hay \emph{hệ tọa độ Đề-các}\footnote{René Descartes (1596-1650)}). Như ở hình \ref{fig:toa do vuong goc}, trục nằm ngang được gọi là \emph{trục hoành}, trục dọc được gọi là \emph{trục tung}. Tùy trong từng trường hợp, vị trí và hướng chỉ của các trục có thể thay đổi. Với mỗi điểm, vị trí khi dóng điểm đó vào trục hoành gọi là \emph{hoành độ}, vào trục tung gọi là \emph{tung độ}. Tiếp tục lấy ví dụ từ hình \ref{fig:toa do vuong goc}, điểm $P$ có tọa độ là $(x_P;y_P)$ và được kí hiệu là $P(x_P;y_P)$. Thêm vào đó, hai trục chia mặt phẳng thành bốn góc phần tư, từ góc phần tư thứ I đến góc phần tư thứ IV bao gồm các điểm thỏa mãn tính chất sau:
\begin{itemize}
   \item Góc phần tư thứ I: $x>0$, $y>0$;
   \item Góc phần tư thứ II: $x<0$, $y>0$;
   \item Góc phần tư thứ III: $x<0$, $y<0$;
   \item Góc phần tư thứ IV: $x>0$, $y<0$.
\end{itemize}
Về mặt hình học, khi tọa độ được vẽ thông thường, góc phần tư thứ I nằm ở vị trí trên cùng bên phải, và các góc phần tư còn lại lần lượt được đánh số theo ngược chiều kim đồng hồ. Khi tọa độ bị thay đổi thì vị trí các góc phần tư cũng thay đổi theo, nhưng vẫn thỏa mãn điều kiện đại số ở trên. Các điểm trên trục không xác định thuộc bất cứ góc phần tư nào.

Giống như trên trục một chiều, khi có hai điểm trên mặt phẳng thì chúng ta có thể tính khoảng cách giữa chúng. Một cách chi tiết, cho hai điểm $P(x_P;y_P)$ và $Q(x_Q;y_Q)$, theo định lí Pi-ta-go, khoảng cách giữa hai điểm đó là $$d(P;Q)=\sqrt{(x_P-x_Q)^2+(y_P-y_Q)^2}.$$

\exercise[ex:0.2] Biểu diễn các điểm sau trên hệ tọa độ vuông góc: $A(2;3)$, $B(-1;2)$, $C(-3;0)$, $D(0;4)$, $P(12t;-3t)$, $Q(20t;12t)$ (với $t \in \mathbb{R}$). Xác định góc phần tư hoặc trục tọa độ của mỗi điểm. Sau đó, tính khoảng cách giữa những cặp điểm sau: $A$ và $B$, $C$ và $D$, $P$ và $Q$.

\solution[ex:0.2]

\begin{figure}[h]
   \centering
   \begin{tikzpicture}
      \draw[->] (-4,0) -- (4,0);
      \draw[->] (0,-1) -- (0,5);
      \node[right] at (4,0) {$x$};
      \node[above] at (0,5) {$y$};
      \filldraw (0, 0) circle (\pointSize) node[below right] {$O(0;0)$};
      \filldraw (2, 3) circle (\pointSize) node[above right] {$A(2;3)$};
      \filldraw (-1, 2) circle (\pointSize) node[below] {$B(-1;2)$};
      \filldraw (-3, 0) circle (\pointSize) node[below] {$C(-3;0)$};
      \filldraw (0, 4) circle (\pointSize) node[above right] {$D(0;4)$};
      \draw[thick] (2, 3) -- (-1, 2);
      \draw[thick] (-3, 0) -- (0, 4);

      \node at (1,1) {$\boxed{\text{I}}$};
      \node at (-1,1) {$\boxed{\text{II}}$};
   \end{tikzpicture}
   \caption{Biểu diễn các điểm $A$, $B$, $C$, $D$ trong bài \ref{ex:0.2}}
   \label{fig:toa do vuong goc bai tap}
\end{figure}

Các góc phần tư hay trục số mà các điểm thuộc về có thể được xác định như hình \ref{fig:toa do vuong goc bai tap}. Theo một cách khác, về mặt đại số, có:
\begin{itemize}
   \item $A(2;3)$: $x>0$, $y>0 \implies A$ thuộc góc phần tư thứ I;
   \item $B(-1;2)$: $x<0$, $y>0 \implies B$ thuộc góc phần tư thứ II;
   \item $C(-3;0)$: $x<0$, $y=0 \implies C$ thuộc trục hoành;
   \item $D(0;4)$: $x=0$, $y>0 \implies D$ thuộc trục tung.
\end{itemize}

\begin{figure}[h]
   \centering
   \begin{tikzpicture}
      \draw[->] (-4.5,0) -- (4,0);
      \draw[->] (0,-3.5) -- (0,2);
      \node[right] at (4,0) {$x$};
      \node[above] at (0,2) {$y$};
      \filldraw (0, 0) circle (\pointSize) node[above left] {$O(0;0)$};
      \filldraw (0, 0) circle (\pointSize) node[below right] {$P_{t=0}$};
      \filldraw (0, 0) circle (\pointSize) node[above right] {$Q_{t=0}$};

      \node at (1,1) {$\boxed{\text{I}}$};
      \node at (-1,-1) {$\boxed{\text{III}}$};
      \node at (-1,1) {$\boxed{\text{II}}$};
      \node at (1,-1) {$\boxed{\text{IV}}$};
      \pgfmathsetmacro{\t}{0.1}
      \filldraw ({12*\t}, {-3*\t}) circle (\pointSize) node[below right] {$P_{t_+}$};
      \filldraw ({20*\t}, {12*\t}) circle (\pointSize) node[above right] {$Q_{t_+}$};
      \draw[thick] ({12*\t}, {-3*\t}) -- ({20*\t}, {12*\t});
      \filldraw ({12*(-\t-0.1)}, {-3*(-\t-0.1)}) circle (\pointSize) node[below right] {$P_{t_-}$};
      \filldraw ({20*(-\t-0.1)}, {12*(-\t-0.1)}) circle (\pointSize) node[below left] {$Q_{t_-}$};
      \draw[thick] ({12*(-\t-0.1)}, {-3*(-\t-0.1)}) -- ({20*(-\t-0.1)}, {12*(-\t-0.1)});
   \end{tikzpicture}
   \caption{Biểu diễn các điểm $P$, $Q$ trong bài \ref{ex:0.2}} theo các trường hợp
   \label{fig:toa do vuong goc PQ}
\end{figure}

Để xác định được vị trí của hai điểm $P$ và $Q$, cần phải xét giá trị của $t$. Nếu $t$ dương, thì $P$ và $Q$ sẽ có tọa độ là $P_{t_+}(12t;-3t)$ và $Q_{t_+}(20t;12t)$ với $x_{P_{t_+}}>0$, $y_{P_{t_+}}<0$ và $x_{Q_{t_+}}>0$, $y_{Q_{t_+}}>0$. Khi này, chúng ta có thể kết luận rằng $P$ thuộc góc phần tư thứ IV và $Q$ thuộc góc phần tư thứ I. Ngược lại, nếu $t$ âm, thì $P$ và $Q$ sẽ có tọa độ là $P_{t_-}(12t;-3t)$ và $Q_{t_-}(20t;12t)$ với $x_{P_{t_-}}<0$, $y_{P_{t_-}}>0$ và $x_{Q_{t_-}}<0$, $y_{Q_{t_-}}<0$. Khi này, $P$ thuộc góc phần tư thứ II và $Q$ thuộc góc phần tư thứ III. Cuối cùng, nếu $t=0$, thì cả hai điểm đều có tọa độ là $(0;0)$, tức là chúng trùng với gốc tọa độ.

Khoảng cách giữa những cặp điểm được yêu cầu là:
\begin{itemize}
   \item $d(A;B) = \sqrt{\left(2-(-1)\right)^2+\left(3-2\right)^2} = \sqrt{10} \approx 3{,}1623$;
   \item $d(C;D) = \sqrt{\left(-3-0\right)^2+\left(0-4\right)^2} = 5$;
   \item $d(P;Q) = \sqrt{\left(12t-20t\right)^2+\left(-3t-12t\right)^2} = 13|t|$.
\end{itemize}

\subsection{Không gian ba chiều và hướng tam diện}

\begin{figure}[h]
   \centering
   \begin{minipage}[b]{0.48\textwidth}
      \centering
      \tdplotsetmaincoords{70}{130}
      \begin{tikzpicture}[tdplot_main_coords]
         \coordinate (P) at (1.5,2,1);
         
         \draw[->] (-2.5,0,0) -- (2.5,0,0) node[anchor=north east]{Trục hoành};
         \draw[->] (0,-2.5,0) -- (0,2.5,0) node[anchor=north west]{Trục tung};
         \draw[->] (0,0,-1) -- (0,0,2) node[anchor=south]{Trục cao/Trục đứng/Trục sâu};
         
         \filldraw (0, 0, 0) circle (1.5pt) node[above left] {$O(0;0;0)$};
         \filldraw (P) circle (1.5pt) node[above] {$P$};
         
         \draw[dashed] (P) -- (1.5, 0, 0) node[above left] {$x_P$};
         \draw[dashed] (P) -- (0, 2, 0) node[above] {$y_P$};
         \draw[dashed] (P) -- (0, 0, 1) node[left] {$z_P$};
         \draw[dashed] (P) -- (1.5, 2, 0) -- (0, 0, 0);
         \draw[dashed] (1.5, 2, 0) -- (0, 2, 0);
         \draw[dashed] (1.5, 2, 0) -- (1.5, 0, 0);

         
      \end{tikzpicture}
      \caption{Hệ tọa độ vuông góc ba chiều}
      \label{fig:toa do vuong goc ba chieu}
   \end{minipage}
   \hfill
   \begin{minipage}[b]{0.48\textwidth}
      \centering
      \tdplotsetmaincoords{60}{60}
      \begin{tikzpicture}[tdplot_main_coords]
         \coordinate (P) at (1.5,2,1);
         \coordinate (Q) at (-2,-1,-0.5);
         
         \draw[->] (-2.5,0,0) -- (2.5,0,0) node[anchor=north east]{$x$};
         \draw[->] (0,-2.5,0) -- (0,2.5,0) node[anchor=north west]{$y$};
         \draw[->] (0,0,-1.5) -- (0,0,1.5) node[anchor=south]{$z$};
         
         \filldraw (P) circle (1.5pt) node[above] {$P$};
         \filldraw (Q) circle (1.5pt) node[below] {$Q$};
         \draw[thick] (P) -- (Q);
         \node[above
         ] at (-0.3, 0.5, 0.3) {$d(P;Q)$};

      \end{tikzpicture}
      \caption{Khoảng cách giữa hai điểm trong không gian ba chiều}
      \label{fig:khoang cach ba chieu}
   \end{minipage}
\end{figure}

Đương nhiên sẽ có một vài trường hợp mà biểu diễn hai chiều không thể đủ. Khi này, mở rộng hơn nữa, chúng ta cũng có thể làm những điều trên không gian ba chiều tương tự với khi ở trục số một chiều hay mặt phẳng hai chiều. Khi đó, chúng ta sẽ có một hệ tọa độ ba chiều với ba trục vuông góc với nhau, được gọi là \emph{hệ tọa độ vuông góc ba chiều}. Mỗi điểm trong không gian sẽ có tọa độ là $(x;y;z)$ với $x$, $y$, $z$ là các hoành độ, tung độ và cao độ tương ứng. Khoảng cách giữa hai điểm trong không gian ba chiều được tính theo công thức $$d(P;Q)=\sqrt{(x_P-x_Q)^2+(y_P-y_Q)^2+(z_P-z_Q)^2}.$$

\begin{wrapfigure}{r}{0.5\textwidth}
   \centering
   \tdplotsetmaincoords{20}{10}
   \begin{tikzpicture}[tdplot_main_coords]         
      \node at (1.5, 1.5, -1.5) {$\boxed{\text{V}}$};
      \node at (-1.5, 1.5, -1.5) {$\boxed{\text{VI}}$};
      \node at (-1.5, -1.5, -1.5) {$\boxed{\text{VII}}$};
      \node at (1.5, -1.5, -1.5) {$\boxed{\text{VIII}}$};
      \draw[fill=gray!30, opacity=0.4] (-2.5,-2.5,0) -- (2.5,-2.5,0) -- (2.5,2.5,0) -- (-2.5,2.5,0) -- cycle;

      \draw[->] (-2.5,0,0) -- (2.5,0,0) node[anchor=north east]{$x$};
      \draw[->] (0,-2.5,0) -- (0,2.5,0) node[anchor=north west]{$y$};
      \draw[->] (0,0,-2.5) -- (0,0,2.5) node[anchor=south]{$z$};
      \filldraw (0, 0, 0) circle (\pointSize) node[above right] {$O(0;0;0)$};
      \node[fill=white, inner sep=2pt] at (1.5, 1.5, 1.5) {$\boxed{\text{I}}$};
      \node[fill=white, inner sep=2pt] at (-1.5, 1.5, 1.5) {$\boxed{\text{II}}$};
      \node[fill=white, inner sep=2pt] at (-1.5, -1.5, 1.5) {$\boxed{\text{III}}$};
      \node[fill=white, inner sep=2pt] at (1.5, -1.5, 1.5) {$\boxed{\text{IV}}$};
      

   \end{tikzpicture}
   \caption{Góc phần tám không gian}
   \label{fig:goc phan tam khong gian}
\end{wrapfigure}

Và cũng tương tự như với mặt phẳng hai chiều, ba trục sẽ chia không gian thành tám phần, gọi là \emph{góc phần tám không gian}. Các phần này được đánh số từ I đến VIII như sau: Nhìn từ phía dương của trục cao, các góc phần tám được đánh dấu ngược chiều kim đồng hồ. Các góc phần tám I, II, III, IV nằm trên mặt phẳng $Oxy$ và được xác định tương tự như các góc phần tư trong mặt phẳng hai chiều. Các góc phần tám V, VI, VII, VIII nằm dưới mặt phẳng $Oxy$ và được xác định tương tự như trên. Các góc phần tám này được biểu diễn trong hình \ref{fig:goc phan tam khong gian}. Về mặt đại số, 

\begin{itemize}
   \item Góc phần tám I: $x>0$, $y>0$, $z>0$;
   \item Góc phần tám II: $x<0$, $y>0$, $z>0$;
   \item Góc phần tám III: $x<0$, $y<0$, $z>0$;
   \item Góc phần tám IV: $x>0$, $y<0$, $z>0$;
   \item Góc phần tám V: $x>0$, $y>0$, $z<0$;
   \item Góc phần tám VI: $x<0$, $y>0$, $z<0$;
   \item Góc phần tám VII: $x<0$, $y<0$, $z<0$;
   \item Góc phần tám VIII: $x>0$, $y<0$, $z<0$.
\end{itemize}

\begin{figure}[h]
   \centering
   \begin{minipage}[b]{0.48\textwidth}
      \centering
      \tdplotsetmaincoords{20}{10}
      \begin{tikzpicture}[tdplot_main_coords]
         \draw[->] (-2.5,0,0) -- (2.5,0,0) node[anchor=north east]{$x$};
         \draw[->] (0,-2.5,0) -- (0,2.5,0) node[anchor=north west]{$y$};
         \draw[->] (0,0,-2.5) -- (0,0,2.5) node[anchor=south]{$z$};
         
         \draw[thick,->] (1.5,0,0) arc (0:90:1.5);
      \end{tikzpicture}
      \caption{Tam diện thuận}
      \label{fig:tam dien thuan}
   \end{minipage}
   \hfill
   \begin{minipage}[b]{0.48\textwidth}
      \centering
      \tdplotsetmaincoords{20}{10}
      \begin{tikzpicture}[tdplot_main_coords]
         \draw[->] (-2.5,0,0) -- (2.5,0,0) node[anchor=north east]{$y$};
         \draw[->] (0,-2.5,0) -- (0,2.5,0) node[anchor=north west]{$x$};
         \draw[->] (0,0,-2.5) -- (0,0,2.5) node[anchor=south]{$z$};
         
         \draw[thick,->] (0,1.5,0) arc (90:0:1.5);
      \end{tikzpicture}
      \caption{Tam diện nghịch}
      \label{fig:tam dien nghich}
   \end{minipage}
\end{figure}

Trên hệ tọa độ không gian, chúng ta cần phải quan tâm thêm xem là ba trục tạo thành \emph{hướng tam diện} nào. Ta nhìn từ phía dương của trục cao, khi này, nếu trục hoành xoay sang trục tung theo hướng ngược chiều kim đồng hồ, thì hướng tam diện được gọi là \emph{hướng tam diện thuận}. Ngược lại, nếu trục hoành xoay sang trục tung theo hướng cùng chiều kim đồng hồ, thì hướng tam diện được gọi là \emph{hướng tam diện nghịch}. Một cách khác là dùng quy tắc bàn tay phải: nắm tay phải vào trục cao, khi này, ngón tay cái chỉ hướng của trục cao. Nếu hướng nắm ngón tay theo hương quay từ trục hoành sang trục tung, thì hướng tam diện là thuận. Ngược lại, nếu hướng nắm ngón tay theo hướng quay từ trục tung sang trục hoành, thì hướng tam diện là nghịch.

Chúng ta đã có phân bổ vị trí của các góc phần tám trong hệ tọa độ tam diện thuận. Lặp lại lập luận với cùng biểu thức đại số, chúng ta có thể phân bổ vị trí của các góc phần tám trong hệ tọa độ tam diện nghịch. Thông thường, hệ tọa độ tam diện thuận được ưa dùng hơn.

\exercise Trung điểm của một đoạn thẳng $AB$ là điểm $M$ trong không gian khi và chỉ khi $M$ thỏa mãn $d(A;M) = d(B;M) = \frac{d(A;B)}{2}$. Chứng minh rằng với tọa độ của $M$ là $$M\left(\frac{x_A+x_B}{2}; \frac{y_A+y_B}{2}; \frac{z_A+z_B}{2}\right)$$ thì $M$ là trung điểm của đoạn thẳng nối hai điểm $A(x_A; y_A; z_A)$ và $B(x_B; y_B; z_B)$. Vẽ ví dụ với $A(1;2;3)$ và $B(-1;0;4)$.

\solution


\section{Hàm số}

\subsection{Định nghĩa hàm số}

\ % Lùi đầu dòng

Chúng ta gọi $f$ là một \emph{hàm số} (hay \emph{hàm}) đi từ tập $X$ đến tập $Y$ khi và chỉ khi với mọi $x\in X$, gọi là \emph{tập xác định}, thông qua mối liên hệ $f$ có một và chỉ một $y\in Y$ tương ứng với $x$. Khi này, chúng ta có thể viết mối liên hệ hàm số này dưới dạng biểu thức giải tích $y=f(x)$. Tuy nhiên, cần phải để ý rằng, thông qua định nghĩa này, mặc dù mọi $x$ trong $X$ phải có đầu ra trong $Y$, không phải mọi $y$ trong $Y$ đều phải có đầu vào trong $X$. Nói cách khác, tập tất cả các giá trị đầu ra có thể của $y=f(x)$, gọi là \emph{tập giá trị}, là tập con của tập $Y$.

Khi chúng ta có định nghĩa hàm số thì chúng ta cũng sẽ có những khái niệm liên quan. Khi $f$ là một hàm số, thì bất cứ giá trị $a$ thuộc tập xác định để $f(a) = 0$ đều được gọi là \emph{nghiệm} của $f$. Mở rộng ra, với $f$ và $g$ là hai hàm số, bất cứ giá trị $a$ thỏa mãn $f(a) = g(a)$ thì $a$ được gọi là nghiệm của \emph{phương trình} $f(x) = g(x)$. Hơn thế nữa, nếu thay dấu $=$ trong câu vừa trước bởi các dấu $<$, $>$, $\leq$, $\geq$, $\neq$ thì ta có định nghĩa cho nghiệm của \emph{bất phương trình}. Lấy ví dụ, với $f$ và $g$ là hai hàm số, giá trị $a$ để $f(a) \neq g(a)$ thì $a$ được gọi là nghiệm của bất phương trình $f(x) \neq g(x)$\footnote{Phần lớn các tại liệu khác quên mất sự tồn tại của dấu $\neq$ rồi. Trong tài liệu này, cũng không có nhiều cơ hội để dùng dấu $\neq$ cho nên tác giả sẽ cho nó nhiều \dblquote{đất diễn} nhất có thể.}.

\exercise


\solution

\subsection{Những hàm số sơ cấp}

\ % Lùi đầu dòng

Trong phần này, các hàm số quen thuộc sẽ được nhắc lại. Đây là những hàm hay thấy nhất trong quá trình học phần lớn các môn khoa học tự nhiên.

Đầu tiên, chúng ta có hàm \emph{đa thức}, thông thường được biểu diễn dưới dạng $$f(x)=P_n(x)=\sum_{i = 0}^n a_i x^i = a_nx^n + a_{n-1}x^{n-1} + \cdots + a_1x + a_0$$ với $n$ là một số nguyên không âm, $a_i$ là các số thực, gọi là các \emph{hệ số}, với mọi $i$ nguyên nằm trong đoạn $[0, n]$ và $a_n \neq 0$. Khi này, $n$ được gọi là \emph{bậc} của đa thức. Ví dụ:
\begin{itemize}
   \item $f(x) = 2x^2 + 3x + 1$ là một đa thức bậc $2$ với các hệ số $a_2 = 2$, $a_1 = 3$, $a_0 = 1$;
   \item $g(y) = y^3 - 4y$ là một đa thức bậc $3$ với các hệ số $b_3 = 1$, $b_2 = 0$, $b_1 = -4$, $b_0 = 0$;
   \item $h(z) = 5$ là một đa thức bậc $0$ với hệ số $c_0 = 5$;
\end{itemize}
Tính toán một số giá trị mẫu:
\begin{itemize}
   \item $p(1) = 7 \cdot 1^4 - 2 \cdot 1^2 + 9 = 14$ với $q(t)= 7t^4 - 2t^2 + 9$ là một đa thức bậc $4$ với các hệ số $d_4 = 7$, $d_3 = 0$, $d_2 = -2$, $d_1 = 0$, $d_0 = 9$;
   \item $q(2) = -3 \cdot 2 + 8 = 2$ với $q(r) = -3r + 8$ là một đa thức bậc $1$ với các hệ số $e_1 = -3$, $e_0 = 8$.
\end{itemize}
Khi đa thức có bậc bằng $0$, hay $f(x) = P_0(x) = a_0$, thì được gọi là \emph{đa thức hằng} hay \emph{hàm hằng}. Một trường hợp đặc biệt là khi $f(x) = 0$ \footnote{Sẽ có nhiều tài liệu viết \dblquote{$f(x) \equiv 0$} thay vì \dblquote{$f(x) = 0$} để phân biệt giữa khẳng định hai hàm là như nhau so với một phương trình. Tác giả không muốn bạn đọc bị vướng víu với nhiều kí hiệu lạ, cho nên Tác giả sẽ cố gắng dùng những kí hiệu cũ. Bạn đọc có thể tự suy luận ý nghĩa thông qua ngữ cảnh.}, khi này hàm không có bậc nhưng vẫn được gọi là hàm hằng \footnote{Đa số những nhà toán học không coi $f(x) = 0$ là đa thức bậc $0$ do nhiều tính chất của đa thức bị phá vỡ khi gặp trường hợp này. Do đó, $f(x) = 0$ chỉ được coi là \dblquote{hàm hằng} chứ không phải \dblquote{\textit{đa thức} hằng}. Trong tài liệu này, trở về sau sẽ chỉ có thuật ngữ \dblquote{hàm hằng} được sử dụng.}.

\exercise Phác thảo đồ thị của những hàm sau:
\begin{multicols}{2}
\begin{enumerate}
   \item $f(x) = x + 2$; 
   \item $f(x) = x^2 + 2x + 3$;
   \item $f(x) = x^3 - 9x^2 + 24x - 16$;
   \item $f(x) = 2$.
\end{enumerate}
\end{multicols}

\solution

Mỗi số hạng của đa thức có dạng $x^n$ với $n$ nguyên. Tuy nhiên, nếu chúng ta lấy $x$ lũy thừa với một số thực $a$ bất kì, khi đó chúng ta sẽ có \emph{hàm lũy thừa}. Dạng tổng quát của hàm này là $$f(x) = x^a$$ với $a$ thực. Ngoài ra, khi làm việc trên tập số thực, có một vài điều kiện xác định ngặt nghèo cho đầu vào $x$ đi kèm. Cụ thể:
\begin{itemize}
   \item Nếu $a$ là số nguyên dương $\left(a \in \mathbb{Z}^+\right)$ thì tập xác định là toàn bộ $\mathbb{R}$;
   \item Nếu $a$ là số nguyên không dương (âm hoặc bằng $0$, kí hiệu $a \in \mathbb{Z} \setminus\mathbb{Z}^+$) thì tập xác định là tập thực bỏ số $0$ $\left(\mathbb{R} \setminus \{0\}\right)$;
   \item Nếu $a$ không nguyên ($a \notin \mathbb{Z}$) thì tập xác định là toàn bộ số dương $\mathbb{R}^+$\footnote{Tại sao điều kiện lại phải nhiều trường hợp vậy? Do chúng ta đang làm việc trên tập số thực. Khi sang miền phức thì đầu vào hàm này sẽ có nhiều sự tự do hơn.}.
\end{itemize}
Ví dụ:
\begin{itemize}
   \item $f(x) = x^{\frac{1}{3}}$ là hàm lũy thừa với $a = \frac{1}{3}$, tập xác định là $\mathbb{R}^+$;
   \item $g(y) = y^{4}$ là hàm lũy thừa với $a = 4$, tập xác định là $\mathbb{R}$;
   \item $h(z) = z^{-3}$ là hàm lũy thừa với $a = -3$, tập xác định là $\mathbb{R} \setminus \{0\}$;
   \item $p(t) = t^{\pi}$ là hàm lũy thừa với $a = \pi$, tập xác định là $\mathbb{R}^+$;
   \item $q(u) = u^0 = 1$ là hàm lũy thừa với $a = 0$, tập xác định là $\mathbb{R} \setminus \{0\}$;
\end{itemize}
Tính toán một số giá trị mẫu:
\begin{itemize}
   \item $\textit{日}(-5) = (-5)^2 = 25$ với $\textit{日}\left(\textit{𠶎}\right) = \textit{𠶎}^2$ là hàm lũy thừa với $a = 2$;
   \item $\textit{月}(16) = 16^{2,5} = 1024$ với $\textit{月}\left(\textit{啛}\right) = \textit{啛}^{2,5}$ là hàm lũy thừa với $a = 2,5$;
   \item $\textit{丫}(-7) = (-7)^{-1} = -\frac{1}{7}$ với $\textit{丫}\left(\textit{低}\right) = \textit{低}^{-1} = \frac{1}{\textit{低}}$ là hàm lũy thừa với $a = -1$.
\end{itemize}

Hàm lũy thừa có trong nó những hàm quen thuộc mà có thể bạn đọc đã nhận ra, kể như hàm phân thức $x^{-b} = \frac{1}{x^b}$ hay hàm khai căn $x^{\frac{1}{c}} = \sqrt[n]{x}$. Những hàm này có cùng tập xác định với hàm lũy thừa kể trên\footnote{Ê, nếu hàm khai căn cũng chia sẻ cùng tập xác định với hàm lũy thừa thì $\sqrt[3]{-27} = (-27)^{\frac{1}{3}} = -3$ cũng không xác định à? Nhiều tài liệu khác vẫn cho phép khai căn mũ lẻ cho số âm, và nếu bạn đọc muốn thực hiện khai căn như vậy thì tác giả cũng không cấm. Tuy nhiên, khai căn là một phép tính đặc biệt. Khi xét đến trường số phức, \textit{hàm khai căn} không còn là một hàm chỉ trả ra một số mà là một tập số. Để muốn nó vẫn là một hàm theo nghĩa thường thì phải có quy ước, và theo quy ước đó, $$\sqrt[3]{-27} = \frac{3}{2} + \mathbf{i}\frac{3\sqrt{3}}{2}.$$}.

Khi tính toán đại số, có một số tính chất quen thuộc mà bạn đọc nên ghi nhớ. Với mọi $a, b$ thực và $x$ thực sao cho mọi tính toán có nghĩa, khi này:
\begin{itemize}
   \item $x^a\cdot x^b = x^{a+b}$;
   \item $\frac{x^a}{x^b} = x^{a - b}$;
   \item $(x^a)^b = x^{a\cdot b}$.
\end{itemize}

Một kiểu hàm có tên tương tự mà hay gây nhầm lẫn là \emph{hàm mũ}. Hàm này có dạng $$f(x) = a^x$$ với $a$ là một số thực dương. Ví dụ:
\begin{itemize}
   \item $f(x) = 2^x$ là hàm mũ với cơ số $a = 2$;
   \item $g(y) = 10^y$ là hàm mũ với cơ số $a = 10$;
   \item $h(z) = e^z$ là hàm mũ với cơ số $a = e$ (số Ơ-le).
\end{itemize}
Tính toán một số giá trị mẫu:
\begin{itemize}
   \item $f(3) = 2^3 = 8$ với $f(x) = 2^x$;
   \item $g(-1) = 10^{-1} = 0,1$ với $g(y) = 10^y$;
   \item $h(0) = e^0 = 1$ với $h(z) = e^z$.
\end{itemize}
Tương tự với hàm lũy thừa, hàm mũ cũng có những đẳng thức quen thuộc. Với $a, b, x$ là ba số thực sao cho mọi tính toán có nghĩa, chúng ta có:
\begin{itemize}
   \item $(a\cdot b)^x=a^x\cdot b^x$;
   \item $\left(\frac{a}{b}\right)^x = \frac{a^x}{b^x}$.
\end{itemize}

Để phân biệt giữa hàm lũy thừa và hàm mũ, mời bạn đọc tham khảo bảng \ref{tab:bảng so sánh lũy thừa số mũ}.
\begin{table}[h]
\centering
\begin{tabular}{|l|c|c|}
\hline
\textbf{Đặc điểm} & \textbf{Hàm lũy thừa} & \textbf{hàm mũ} \\
\hline
Dạng tổng quát & $f(x) = x^a$ & $f(x) = a^x$ \\
\hline
Biến số & $x$ ở cơ số & $x$ ở số mũ \\
\hline
Tham số & $a$ là số thực bất kỳ & $a$ là số thực dương khác $1$\\
\hline
Tập xác định & Phụ thuộc vào $a$ & $x \in \mathbb{R}$ \\
\hline
Ví dụ & $f(x) = x^2$, $f(x) = x^{-1}$ & $f(x) = 2^x$, $f(x) = e^x$ \\
\hline
\end{tabular}
\caption{So sánh hàm lũy thừa và hàm mũ}
\label{tab:bảng so sánh lũy thừa số mũ}
\end{table}

Người ta thường nói ngược lại của hàm lũy thừa là hàm khai căn, tỉ như nếu $f(x) = x^a$ thì (có thể) $x = \sqrt[a]{f(x)}$. Thế đối với hàm mũ $f(x) = a^x$ thì $x$ là gì của $f(x)$? Chúng ta bây giờ cần đến \emph{hàm lô-ga-rít (logarithm)} và bắt đầu phải dùng nhiều chữ khi gọi hàm: $$f(x) = \log_a {\left(x\right)}.$$

Chúng ta có mối liên hệ $f(x) = a^x \implies x = \log_a {\left(f(x)\right)}$. Hàm lô-ga-rít cơ số $a$ ($\log_a {\left(x\right)}$) chỉ xác định khi $a > 0$, $a \neq 1$ và $x > 0$. Đặc biệt, khi $a = 10$ thì hàm không cần phải viết cơ số và trở thành $f(x) = \log(x)$. Khi $a = e$, là số Ơ-le vừa được nhắc đến, thì hàm có thể được viết là $f(x) = \ln(x)$. Ví dụ:
\begin{itemize}
   \item $f(x) = \log_2 (x)$ là hàm lô-ga-rít với cơ số $a = 2$;
   \item $g(y) = \log_{10} (y) = \log(y)$ là hàm lô-ga-rít với cơ số $a = 10$;
   \item $h(z) = \log_e (z)=\ln(z)$ là hàm lô-ga-rít với cơ số $a = e$.
\end{itemize}
Tính toán một số giá trị mẫu:
\begin{itemize}
   \item $f(2) = \log_2 (8) = 3$ vì $2^3 = 8$;
   \item $g(100) = \log (100) = 2$ vì $10^2 = 100$;
   \item $h(e) = \ln (e) = 1$ vì $e^1 = e$.
\end{itemize} 

\section{Số ảo và số phức}

\ % Lùi đầu dòng

Trước khi đến với sô thực với số phức, chúng ta bắt đầu tiếp cận với định nghĩa đơn vị ảo. Cụ thể, \emph{đơn vị ảo} được kí hiệu là $\mathbf{i}$ \footnote{Phần lớn các tài liệu sẽ kí hiệu số ảo là chữ $i$ thông thường. Tác giả kí hiệu thành chữ $\mathbf{i}$ đứng in đậm để bảo toàn chữ $i$ cho nhiệm vụ khác.} và thỏa mãn $$\mathbf{i}^2 = -1 \text{ hay } \mathbf{i} = \sqrt{-1}.$$

Để có số ảo, ta nhân một số thực $b \neq 0$ với đơn vị ảo để thành $\mathbf{i}b$. Một số phức bao gồm thành phần thực và $\mathbf{i}$ lần phẩn ảo cộng vào. Viết dưới \emph{dạng chính tắc}, một số phức có dạng là $$z=a+\mathbf{i}b$$ với $a, b$ thực. Từ một số phức, chúng ta cũng có thể lấy ngược lại giá trị phần thực và phần ảo của nó lần lượt qua hai hàm $\Re{(z)}$ và $\Im{(z)}$ (hoặc $\operatorname*{Re}{(z)}$ và $\operatorname*{Im}{(z)}$). Cụ thể, với $z=a+\mathbf{i}b$ thì $\Re{(z)} = a$ và $\Im{(z)}=b$\footnote{Tại sao không gọi cả $\mathbf{i}b$ là phần ảo? Khi nói đến phần ảo, chúng ta đã ngầm định nó sẽ thuộc về số hạng mà có thừa số $\mathbf{i}$. Viết lại đơn vị ảo trở nên thừa thãi. Hơn nữa, sẽ dễ làm việc hơn khi mà cả $\Re{(z)}$ và $\Im{(z)}$ đều thực và ta không phải chia $\Im{(z)}$ cho $\mathbf{i}$ liên tục.}.

Chúng ta sẽ coi như có thể thực hiện các định luật đại số thông thường trên tập số phức. Coi $\mathbf{i}$ là một biến với $\mathbf{i}^2 = -1$. Để cộng (hay trừ) hai số phức $v = a + \mathbf{i}b$ và $w = c + \mathbf{i}d$, ta thực hiện cộng (hay trừ) các thành phần tương đương (phần thực với phần thực, phần ảo với phần ảo). Viết dưới dạng toán học:
\begin{equation*}
   \begin{cases}
      v + w = (a + c) + \mathbf{i}(b + d) \\ 
      v - w = (a - c) + \mathbf{i}(b - d) 
   \end{cases}.
\end{equation*}

Nhân hai số phức sẽ yêu cầu sử dụng tính chất phân phối giữa phép nhân với phép cộng, được thực hiện như sau
\begin{align*}
   v\cdot w&=\left(a + \mathbf{i}b\right)\cdot\left(c + \mathbf{i}d\right) \\
      &= ac + \mathbf{i}ad + \mathbf{i}bc + \mathbf{i}^2 bd \\
      &= (ac - bd) + \mathbf{i}(ad + bc).
\end{align*}

Trước khi chia hai số phức, chúng ta cần phải biết đến khái niệm số phức liên hợp và tính chất đặc biệt của nó. Một số phức $z = a+\mathbf{i}b$ sẽ có số phức liên hợp là $$\bar{z} = z^* = a - \mathbf{i}b.$$ Khi này, thực hiện phép nhân số phức $z$ với liên hợp của nó để có $$z\bar{z} = (a+\mathbf{i}b)(a-\mathbf{i}b) = a^2 + b^2.$$ Để ý rằng $a^2 + b^2$ là một số thực do $a, b$ đã là số thực từ định nghĩa, và cũng cần phải nhớ lại rằng khi nhân cả số bị chia và số chia với một số thì thương không đổi. Cho nên, để chia hai số phức, chúng ta nhân cả tử và mẫu với liên hợp của số chia $$\frac{v}{w} = \frac{a + \mathbf{i}b}{c + \mathbf{i}d} = \frac{\left(a + \mathbf{i}b\right)\left(c - \mathbf{i}d\right)}{\left(c + \mathbf{i}d\right)\left(c - \mathbf{i}d\right)} = \frac{\left(ac+bd\right) + \mathbf{i}\left(bc - ad\right)}{c^2+d^2}.$$ Từ đó, chúng ta đưa phép chia hai số phức thành phép chia số phức với số thực và có kết quả là $$\frac{v}{w} = \frac{a + \mathbf{i}b}{c + \mathbf{i}d} = \frac{ac+bd}{c^2+d^2}+\mathbf{i}\cdot\frac{bc - ad}{c^2+d^2}.$$

\begin{wrapfigure}{r}{0.5\textwidth}
    \centering
    \begin{tikzpicture}
      \draw[->] (-2,0) -- (4,0) node[right] {$\operatorname*{Re}(z)$};
      \draw[->] (0,-1) -- (0,3) node[above] {$\operatorname*{Im}(z)$};
      \filldraw (3,2) circle (1pt) node[anchor=west] {$z = 3 + 2\mathbf{i}$};
      \filldraw (0, 0) circle (1pt);
      \node[above] at (3,2) {$Z(3, 2)$};
      \draw[dashed, thick] (0, 0) -- (3, 2);
      \draw (0,0) -- (0.1, -0.15);
      \draw (3, 2) -- (3.1, 1.85);
      \draw[<->] (0.05, -0.075) -- (3.05, 1.925);
      \node[right] at ($(0.1, -0.1)!0.5!(3.1, 1.9)$) {$|z| = \sqrt{3^2 + 2^2} = \sqrt{13}$};
   \end{tikzpicture}

   \caption{Biểu diễn $z = 3 + 2\mathbf{i}$ trên mặt phẳng tọa độ}
   \label{fig:bieu_dien_so_phuc}
\end{wrapfigure}

Ngoài cách biểu diễn đại số, còn có cách biểu diễn hình học trên mặt phẳng tọa độ của số phức qua việc coi trục hoành và trục tung lần lượt biểu diễn phần thực và phần ảo của số phức. Cụ thể, số phức $z = a + \mathbf{i}b$ được biểu diễn bởi một điểm $Z(a,b)$ trên hệ tọa độ vuông góc. Khi này, $Z$ là \emph{ảnh} (hay đơn giản là \emph{điểm biểu diễn}) của $z$ và $(a,b)$ được gọi là \emph{tọa vị} (hay \emph{tọa độ phức}) của $z$.

Hình \ref{fig:bieu_dien_so_phuc} đã biểu diễn số phức $z = 3 + 2\mathbf{i}$ trên mặt phẳng tọa độ. Từ đây, chúng ta có thể phát hiện ra những đặc tính khác của $z$ khác tọa vị. Đầu tiên, chúng ta có thể đo khoảng cách từ ảnh $Z$ đến gốc $(0;0)$, và từ đó, chúng ta sẽ nhận được \emph{mô-đun (module)} của $z$, kí hiệu: $|z|$. Bạn đọc có thể để ý rằng kí hiệu giống như kí hiệu giá trị tuyệt đối của số thực. Cũng có thể hiểu được tại sao lại vậy nếu như bạn đọc nhớ cách biểu diễn khoảng cách hình học của giá trị tuyệt đối trên trục số thực. Khi chúng ta có một điểm biểu diễn một số thực $x$ trên một trục thì giá trị tuyệt đối của $x$ chính là khoảng cách từ $x$ đến điểm $0$.

\begin{figure}[h]
   \centering
   \begin{tikzpicture}
      \draw[<->] (-5, 0) -- (5, 0);

      \foreach \pt/\lbl in {0/0, 4/x, -2.5/y} {
         \filldraw (\pt, 0) circle (1pt);
         \draw (\pt, 0) -- (\pt, -0.2);
         \node[above] at (\pt, 0) {$\lbl$};
      }
      
      \draw[<->] (-2.5, -0.1) -- (0, -0.1);
      \node[below] at (-1.25, -0.1) {$|y|$};
      \draw[<->] (4, -0.1) -- (0, -0.1);
      \node[below] at (2, -0.1) {$|x|$};
   \end{tikzpicture}
   \caption{Giá trị tuyệt đối trên trục thực}
   \label{fig:gia_tri_tuyet_doi_thuc}
\end{figure}

Một cách tương tự, $|z|$ là khoảng cách từ $Z$ đến gốc tọa độ. Công thức Pi-ta-go được sử dụng để tính khoảng cách này: $$|z| = \sqrt{a^2+b^2}.$$ Cũng là vì lí do đó nên trong một số tài liệu, $|z|$ vẫn được gọi là giá trị tuyệt đối để đảm bảo tính nhất quán.

Trên trục số thực, mốt số cụ thể thì giá trị tuyệt đối của nó chỉ có một giá trị, nhưng nếu đầu ra là một giá trị tuyệt đối thì đầu vào có thể là $2$ số khác nhau. Để biết chính xác là số nào thì cần biết thêm dấu của số đó, hay nói một cách khác, hướng của số đó nếu nhìn từ vị trí gốc $0$. Một cách tương tự, một số phức $z$ chỉ ra được một giá trị mô-đun $|z|$ của nó, nhưng với một $|z|$ thì có thể có nhiều $z$ thỏa mãn. Để biết chính xác được giá trị của $z$ thì chúng ta cần phải biết thêm hướng của $z$. Tuy nhiên, việc xác định hướng này không chỉ đơn giản là nằm trái hay phải trên trục một chiều nữa, mà cần phải xác định vị trí trong mặt phẳng hai chiều. Một cách để thực hiện điều này là xác định \emph{góc} (hay \emph{a-gu-men}) của $z$.

Để xác định góc, chúng ta cần phải có $2$ tia. Một tia có thể được nối từ gốc đến điểm biểu diễn. Một tia còn lại có thể bám theo một trục cố định. Về quy ước, phía dương trục hoành, hay trục thực, được sử dụng làm bờ còn lại. Bạn đọc có thể nghĩ rằng là khi này ta đã có đủ điều kiện để xác định góc. Cũng đũng, đã đủ để từ số phức $z$ ra được góc của $z$. Nhưng từ góc của $z$ vẫn chưa đủ để ra được $z$. Hãy nhìn vào hình \ref{fig:hai_truong_hop_goc}:
\begin{figure}[h]
   \centering
   \begin{tikzpicture}
      \draw[->] (-1, 0) -- (5, 0);
      \draw[->] (0, -3) -- (0, 3);

      \filldraw (4,2) circle (1pt) node[anchor=west] {$z = 4 + 2\mathbf{i}$};
      \filldraw (4,-2) circle (1pt) node[anchor=west] {$z = 4 - 2\mathbf{i}$};

      \draw[dashed, ->, thick] (0, 0) -- (4, 2);
      \draw[dashed, ->, thick] (0, 0) -- (4, -2);

      \draw (0.5,0) arc[start angle=0, end angle={atan(2/4)}, radius=0.5];
      \node at (0.75,0.18) {$\theta$};

      \draw (0.75,0) arc[start angle=0, end angle={-atan(2/4)}, radius=0.75];
      \node at (0.9,-0.2) {$\theta$};
   \end{tikzpicture}
   \caption{$4+2\mathbf{i}$ và $4-2\mathbf{i}$ có độ lớn góc bằng nhau.}
   \label{fig:hai_truong_hop_goc}
\end{figure}


Để phân biệt hai góc này, người ta sử dụng khái niệm \emph{góc định hướng}. Một cách đơn giản, quay trục hoành ngược chiều kim đồng hồ cho đến khi chạm vào cạnh còn lại. Góc đã quay là độ lớn của góc định hướng. Khi quay thuận chiều kim đồng hồ thì góc đó quy ước là quay góc âm. Từ đó, chúng ta có thể phân biệt góc nhìn như hình \ref{fig:hai_truong_hop_goc_dinh_huong}:

\begin{figure}[h]
   \centering
   \begin{tikzpicture}
      \draw[->] (-2, 0) -- (5, 0);
      \draw[->] (0, -3) -- (0, 3);

      \filldraw (4,2) circle (1pt) node[anchor=west] {$z = 4 + 2\mathbf{i}$};
      \filldraw (4,-2) circle (1pt) node[anchor=west] {$z = 4 - 2\mathbf{i}$};

      \draw[dashed, ->, thick] (0, 0) -- (4, 2);
      \draw[dashed, ->, thick] (0, 0) -- (4, -2);

      \draw[->] (0.5,0) arc[start angle=0, end angle={atan(2/4)}, radius=0.5];
      \node at (0.75,0.18) {$\theta$};

      \draw[->] (0.75,0) arc[start angle=0, end angle={-atan(2/4)}, radius=0.75];
      \node at (1,-0.22) {$-\theta$};

      \draw[->] (1,0) arc[start angle=0, end angle={360-atan(2/4)}, radius=1];
      \node at (-1.5,0.7) {$2\pi - \theta$};
   \end{tikzpicture}
   \caption{$4+2\mathbf{i}$ và $4-2\mathbf{i}$ có độ lớn góc bằng nhau.}
   \label{fig:hai_truong_hop_goc_dinh_huong}
\end{figure}

Người ta kí hiệu góc của số phức là $\arg{(z)}$ hoặc $\Arg{(z)}$. Cũng giống như góc không định hướng, khi cộng thêm hay bớt đi $2\pi$ ra-đi-an (hay $360$ độ) thì \dblquote{hướng nhìn} cũng không thay đổi. Để cho $\arg{(z)}$ chỉ trả ra một giá trị duy nhất, quy ước là lấy góc trong nửa đoạn $\left(-\pi;\pi\right]$ (hay $\left(-180^\circ;180^\circ\right]$). Như ví dụ trong hình \ref{fig:hai_truong_hop_goc_dinh_huong}, $\arg{(4+2\mathbf{i})} = \theta = \arctan\left(\frac{2}{4}\right) \approx 0,464~\text{rad}$ (hay $26,565^\circ$) và $\arg{(4-2\mathbf{i})} = -\theta = -\arctan\left(\frac{2}{4}\right) \approx -0,464~\text{rad}$ (cũng có thể được viết lại là $-26,565^\circ$).

Như đã viết, có khoảng cách và hướng nhìn thì chúng ta sẽ có được vị trí số phức. Cách biểu diễn này được gọi là \emph{dạng lượng giác} của số phức. Đặt $r = |z|$ và $\varphi = \arg{(z)}$, dạng lượng giác của $z$ được kí hiệu là $z = r \angle \varphi = r \phase{\varphi}$. Quy đổi giữa dạng lượng giác và dạng chính tắc, khi $z = a + \mathbf{i}b = r \phase{\varphi}$ thì
\[
\left\{
\begin{aligned}
   a &= \Re{(z)} = r \cos{(\varphi)} \\ 
   b &= \Im{(z)} = r \sin{(\varphi)}
\end{aligned}
\right.
\]
và từ đó $z = r\left(\cos{(\varphi)} + \mathbf{i}\sin{(\varphi)}\right)$. Liên hợp của $z$ dưới dạng lượng giác là $\bar{z} = r\left(\cos{(\varphi)} - \mathbf{i}\sin{(\varphi)}\right) = r \phase{-\varphi}$.



Thật sự, rất khó cho nhiều người không thường xuyên thường thức về toán ngay lập tức tìm ra và cảm nhận được ý nghĩa thực tiễn của số ảo. Chúng ta không thể tưởng tượng được số ảo một cách trực quan như các số mà chúng ta thường thấy ở ngoài cuộc sống như $5$ cái bút hay $\frac{1}{3}$ giờ. Đi kèm với đó, kể cả trên lí thuyết toán của ghế nhà trường, cũng sẽ không xảy ra trường hợp nào để cho một số nhân với chính nó ra một số âm. 

Nhắc về số âm, theo quan điểm cá nhân, số âm trong đời xuống vốn đã ít khi được sử dụng. Chẳng mấy ai ưa nói \dblquote{lãi $-500000$ đồng} so với \dblquote{lỗ $500000$ đồng}. Một cách tương tự, nhìn về phương diện lịch sử, trong phần lớn quá trình phát triển của toán học, các nhà toán học xưa thường có mặc cảm với những số âm. Các phương trình sẽ luôn được viết lại thành nhiều trường hợp để tránh chúng. Ví dụ, nếu phương trình bậc hai được viết dưới dạng hiện đại là $x^2 + ax+b=0$ với $a,b$ là hai số thực (có thể âm), thì trong quá khữ, phương trình này được chia ra làm ba trường hợp
\begin{align*}
   &x^2+ax = b;\\
   &x^2+b =ax;\\
   &x^2 =ax+b
\end{align*}
với $a,b$ là hai số thực luôn dương. Và cũng từ sự mặc cảm với số âm, họ cho rằng nghiệm của phương trình cũng phải là một số dương. Tương tự với Các-đa-nô \footnote{Gerolamo Cardano (1501-1576).}, khi giải phương trình bậc ba, ông cũng đưa về các trường hợp như trên. Cụ thể, chúng ta xem xét một trường hợp của bài toán: $$x^3 = ax+b.$$ Giải phương trình, chúng ta có được nghiệm $$x=\sqrt[3]{\frac{b}{2} + \sqrt{\frac{b^2}{4}-\frac{a^3}{27}}}+\sqrt[3]{\frac{b}{2} - \sqrt{\frac{b^2}{4}-\frac{a^3}{27}}}.$$ Tuy nhiên, sau khi thay những giá trị cụ thể vào $a$ và $b$, Các-đa-nô đã phát hiện ra một vấn đề. Khi $a=15$ và $b=4$, nghiệm trả ra cho phương trình $x^3 = 15x+4$ theo công thức vừa trên là $$x=\sqrt[3]{2+\sqrt{-121}}+\sqrt[3]{2-\sqrt{-121}}$$ mặc dù phương trình có một nghiệm bình thường là $x=4$ (với kiến thức toán học hiện đại, chúng ta có thể giải ra hai nghiệm cũng thực khác là $-2 \pm \sqrt{3}$). Nhận ra điều đó, Các-đa-nô đã khẳng định rằng công thức này của ông không áp dụng được trong trường hợp xảy ra căn của một số âm. Tuy nhiên, một học trò của ông, Bom-be-li \footnote{Rafael Bombelli (1526-1572).}, lại phủ nhận điều trên. Bom-be-li nhận định rằng tồn tại một kiểu số khác số thực sẽ có giá trị bằng \dblquote{căn âm}. Ông chỉ rõ sự khác biệt giữa kiểu số mới này và kiểu số thực thông thường, và đi kèm theo là phương pháp thực hiện đại số trên kiểu số mới. Áp dụng những nền tảng đó, ông đã tính được căn bậc ba của hai số phức lần lượt là $\sqrt[3]{2+\sqrt{-121}}=2+\sqrt{-1}$ và $\sqrt[3]{2-\sqrt{-121}}=2-\sqrt{-1}$. Cộng hai số vào, hiển nhiên sẽ có được nghiệm $4$ như mong muốn.

Với sự xây dựng ban đầu của Bom-be-li làm gốc, trong những thế kỉ sau, tên gọi và lí thuyết về cách biểu diễn số phức được hình thành.

\section{Bài tập tổng hợp}


\chapter{Cơ bản của xử lí số liệu trong vật lí}
\exercise Khoảng cách trung bình từ trái đất đến mặt trời là $1{,}5 \cdot 10^8$ km. Giả sử quỹ đạo của trái đất quanh mặt trời là tròn và mặt trời được đặt tại gốc của hệ quy chiếu.
\begin{enumerate}
   \item Tính tốc độ di chuyển trung bình của trái đất quanh mặt trời dưới dạng dặm trên giờ ($1 \;\text{dặm}=1{,}6093\;\text{km}$).
   \item Ước lượng góc $\theta$ giữa véc-tơ vị trí của trái đất bây giờ và vị trí sau đó $4$ tháng.
   \item Tính khoảng cách giữa hai vị trí đó.
\end{enumerate}
\solution
\begin{enumerate}
   \item Giả sử trái đất quay quanh mặt trời trong $365{,}25$ ngày. Quãng đường mà trái đất đi được trong thời gian này là chu vi của quỹ đạo tròn $2 \pi \cdot 1{,}5 \cdot 10^8$ km. Từ đó, ta có thể tính được tốc độ trung bình của trái đất quanh mặt trời là $\frac{2 \pi \cdot 1{,}5 \cdot 10^8\ \text{km}}{365{,}25\ \text{ngày}}$. Thực hiện quy đổi để được:
      \[
         \frac{2 \pi \cdot 1{,}5 \cdot 10^8\ \text{km}}{365{,}25\ \text{ngày}}
         \cdot \frac{1\ \text{dặm}}{1{,}6903\ \text{km}}
         \cdot \frac{1\ \text{ngày}}{24\ \text{h}}
         = \boxed{6{,}4\cdot 10^4\ \frac{\text{dặm}}{\text{h}}}.
      \]
   \item Trái đất quay quanh mặt trời trong $12$ tháng, tương đương với một góc quay $360^{\circ}$ so với gốc là mặt trời. Coi như các tháng có độ dài như nhau. Ta có $\theta$ chính là góc quay của trái đất trong $4$ tháng, tương đương với:
   \[
      \theta = \frac{360^{\circ}}{12\ \text{tháng}} \cdot 4\ \text{tháng}= \boxed{120^{\circ}}.
   \]
   \item
\end{enumerate}

\begin{wrapfigure}{r}{0.3\textwidth}
   \centering
   \begin{tikzpicture}
      \draw[thick] (0,0) circle (2cm);
      \filldraw[black] (0,0) circle (2pt) node[anchor=north] {O};
      \draw[->, thick, >=latex, line width=0.5mm] (0,0) -- (0:2cm) node[midway, below] {$r$};
      \draw[->, thick, >=latex, line width=0.5mm] (0,0) -- (120:2cm);
      \draw[thick, line width=0.5mm] (0:2cm) -- (120:2cm) node[midway, above] {$d$};
      \filldraw[black] (0:2cm) circle (2pt) node[anchor=west] {A};
      \filldraw[black] (120:2cm) circle (2pt) node[anchor=south] {B};
      \node at (60:0.3cm) {$\theta$};
      \draw[thick] (0:0.5cm) arc[start angle=0, end angle=120, radius=0.5cm];
   \end{tikzpicture}
   \caption{Quỹ đạo trái đất}
   \label{fig:earth}
\end{wrapfigure}

Gọi $A$ là vị trí của trái đất bây giờ, $B$ là vị trí của trái đất sau $4$ tháng theo như hình \ref{fig:earth}. Coi một đơn vị trên tọa độ bằng độ dài bán kính của quỹ đạo tròn, tức là $r=1{,}5\cdot10^8$ km. Ta có tọa độ điểm $A$ là $(1;0)$. Tọa độ điểm $B$ là $\left(\cos(120^{\circ}); \sin(120^{\circ})\right)=\left(-\frac{1}{2}; \frac{\sqrt{3}}{2}\right)$. Từ đó, ta có khoảng cách giữa hai vị trí đó là: $$d = r\cdot \sqrt{\left(1-\left(-\frac{1}{2}\right)\right)^2 + \left(\frac{\sqrt{3}}{2}\right)^2}=\boxed{2{,}6\cdot10^8\ \text{km}}.$$

\exercise Khối lượng riêng (bằng khối lượng của vật chia cho thể tích của vật đó) của nước là $1{,}00 \;\frac{\text{g}}{\text{cm}^3}$.
\begin{enumerate}
   \item Tính giá trị này theo ki-lô-gam trên mét khối.
   \item $1{,}00$ lít nước nặng bao nhiêu ki-lô-gam, bao nhiêu pao (lb)? Biết $1\ \text{lb} = 0{,}45\ \text{kg}$ (chính xác).
\end{enumerate}

\solution
\begin{enumerate}
   \item Thực hiện quy đổi, ta có:
   \begin{align*}
      1{,}00\;\frac{\text{g}}{\text{cm}^3} &= \left(1{,}00\;\frac{\text{g}}{\text{cm}^3}\right)\cdot\frac{1\ \text{kg}}{1000\ \text{g}}\cdot\left(\frac{100\ \text{cm}}{1\ \text{m}}\right)^3 \\
      &= \boxed{1{,}00\cdot 10^3\ \frac{\text{kg}}{\text{m}^3}}.
   \end{align*}
   \item Khối lượng của $1{,}00$ lít nước là
   \begin{align*}
      1{,}00\ \text{L} \cdot \left(1{,}00\cdot 10^3\ \frac{\text{kg}}{\text{m}^3}\right)&= 1{,}00\ \text{L} \cdot \left(1{,}00\cdot 10^3\ \frac{\text{kg}}{\text{m}^3}\right) \cdot \frac{1\ \text{m}^3}{1000\ \text{L}} \\
      &= \boxed{1{,}00\cdot 10^0\ \text{kg}}.
   \end{align*}
   Theo đơn vị pao (lb), ta có:
   \[
      1{,}00\cdot 10^0\ \text{kg} = 1{,}00\cdot 10^0\ \text{kg} \cdot \frac{1\ \text{lb}}{0{,}45\ \text{kg}} = \boxed{2{,}22\cdot 10^0\ \text{lb}}.
   \]
\end{enumerate}

\exercise Trong hệ thời gian cổ Trung Hoa, từ triều đại Thanh trở về trước (trừ một số năm), một ngày được chia thành $100$ khắc. Sau triều đại này (trừ một số năm), một ngày được chia thành $96$ khắc. Coi một ngày có $24$ giờ và mọi số liệu là chính xác tuyệt đối.
\begin{enumerate}
   \item Tính số giây (hệ đo lường hiện đại) trong một khắc trong cả hai thời kì.
   \item Tính tỉ lệ về độ dài của hai khắc trong hai thời kì.
\end{enumerate}

\solution
\begin{enumerate}
   \item Số giây trong một ngày là $$24\ \text{h} \cdot \frac{60\ \text{phút}}{1\ \text{h}} \cdot \frac{60\ \text{giây}}{1\ \text{phút}} = 86400\ \text{giây}.$$
\end{enumerate}
Từ triều đại Thanh trở về trước, số giây trong một khắc là $$\frac{86400\ \text{giây}}{100\ \text{khắc}_{\text{trước}}} = \boxed{864 \frac{\text{giây}}{\text{khắc}_{\text{trước}}}}.$$
Sau triều đại Thanh, số giây trong một khắc là $$\frac{86400\ \text{giây}}{96\ \text{khắc}_{\text{sau}}} = \boxed{900 \frac{\text{giây}}{\text{khắc}_{\text{sau}}}}.$$
\begin{enumerate}
   \item[2] Tỉ lệ độ dài thời gian một khắc trước và sau là $$\frac{1\ \text{khắc}_{\text{trước}}}{1\ \text{khắc}_{\text{sau}}} = \frac{1\ \text{khắc}_{\text{trước}}}{1\ \text{khắc}_{\text{sau}}}\cdot \frac{864\ \text{giây}}{1\ \text{khắc}_{\text{trước}}}\cdot\frac{1\ \text{khắc}_{\text{sau}}}{900\ \text{giây}}=\boxed{0{,}96}.$$
\end{enumerate}

\exercise Một vòng đĩa tròn như trong hình \ref{fig:vong_dia} có đường kính $4{,}50$ cm rỗng ở giữa một lỗ đường kính $1{,}25$ cm. Đĩa dày $1{,}50$ mm. Biết rằng đĩa được làm từ chất liệu có khối lượng riêng là $8600\;\frac{\text{kg}}{\text{m}^3}$. Tính khôi lượng vòng đĩa theo gram.

\begin{figure}
   \centering
   \begin{tikzpicture}
      % Draw the shape
      \draw[bottom color=gray!20] (-2.25, 0) arc[start angle=180, end angle=360, x radius = 2.25cm, y radius = 2.4cm];
      \draw[bottom color=gray!70, top color=gray!20] (0, 0) circle (2.25cm);
      \draw[fill=white] (0, 0) circle (0.625cm);
      \draw[bottom color = gray!20] (-0.625, 0) arc[start angle=180, end angle=0, radius = 0.625cm];
      \draw[fill=white] (-0.625, 0) arc[start angle=180, end angle=0, x radius = 0.625cm, y radius = 0.5cm];
      % Draw the dimension
      \draw[<->] (-2.25,-2.5) -- (2.25,-2.5);
      \node at (0, -2.7) {$4{,}50$ cm};
      \draw[<->] (-0.625,0) -- (0.625,0);
      \node at (0, -0.2) {$1{,}25$ cm};
      \draw[<->] (0,0.5) -- (0,0.625);
      \node[anchor=west] at (-0.05, 0.7) {$1{,}50$ mm};
   \end{tikzpicture}
   \caption{Vòng đĩa tròn}
   \label{fig:vong_dia}
\end{figure}

\solution

Đặt $D=4{,}50\;\text{cm}=4{,}50\times 10^{-2}\ \text{m}$, $d=1{,}25\ \text{cm}=1{,}25\times 10^{-2}\;\text{m}$, $h=1{,}50\ \text{mm}=1{,}50\times 10^{-3}\ \text{m}$ và $\mathcal{D}=8600\;\frac{\text{kg}}{\text{m}^3}=8{,}6\times 10^3\;\frac{\text{kg}}{\text{m}^3}\cdot \frac{10^3\;\text{g}}{\text{kg}}=8{,}6\times 10^6\;\frac{\text{g}}{\text{m}^3}$.

Nhận thấy rằng đĩa có dạng trụ, diện tích mặt đáy là $$S=\pi\cdot \left(\frac{D}{2}\right)^2-\pi\cdot \left(\frac{d}{2}\right)^2=\frac{\pi \left(D^2-d^2\right)}{4}.$$

Thể tích của đĩa là $V=S\cdot h=\frac{\pi \cdot h\cdot \left(D^2-d^2\right)}{4}.$ Nhân với khối lượng riêng, ta có khối lượng của đĩa là $$m=\mathcal{D}\cdot V=\frac{\pi \cdot h\cdot \mathcal{D}\cdot \left(D^2-d^2\right)}{4}.$$ Thay số trực tiếp với sự để ý đến số chữ số có nghĩa, ta có kết quả $m=\boxed{1{,}89\times 10^1\ \text{kg}}$.

\exercise Khối lượng của một chất lỏng được mô hình hóa bởi phương trình $m=A\cdot t^{0{,}8}-B\cdot t$. Nếu như $t$ được tính bằng giây và $m$ được tính bằng ki-lô-gram, thì đơn vị của $A$ và $B$ là gì?

\solution

Để có thể cộng trừ các phần tử, chúng cần phải có cùng đơn vị. Do vậy, đơn vị của $A\cdot t^{0{,}8}$ và $B\cdot t$ là kg. Từ quy tắc nhân chia các đơn vị, ta có:
\begin{equation*}
   \begin{cases}
     A\cdot \text{s}^{0{,}8} &=\text{kg} \\
     B\cdot\text{s} &=\text{kg}
   \end{cases}
   \iff
   \begin{cases}
      A &=\frac{\text{kg}}{\text{s}^{0{,}8}} \\
      B&=\frac{\text{kg}}{\text{s}}
   \end{cases}.
\end{equation*}

Vậy đơn vị của $A$ là $\boxed{\frac{\text{kg}}{\text{s}^{0{,}8}}}$ và đơn vị của $B$ là $\boxed{\frac{\text{kg}}{\text{s}}}$.

\chapter{Chuyển động}

\exercise Một ô tô đi $40$ km trên một đường thẳng với tốc độ không đổi $40\;\frac{\text{km}}{\text{h}}$. Sau đó, nó đi thêm theo chiều đó $60$ km với tốc độ không đổi $50\;\frac{\text{km}}{\text{h}}$. Các giá trị đo được tính đến hai chữ số có nghĩa.
\begin{enumerate}
   \item Tính vận tốc trung bình trên cả quãng đường.
   \item Tính tốc độ trung bình trên cả quãng đường.
   \item Nếu xe quay đầu trước khi đi $50$ km lúc sau, giữ nguyên các số liệu khác, thì vận tốc trung bình và tốc độ trung bình có thay đổi không. Tại sao?
   \item Vẽ đồ thị vị trí $x$ theo thời gian $t$ và từ đó chỉ ra cách tính vận tốc trung bình.
\end{enumerate}
\solution

Coi chiều chuyển động ban đầu là chiều dương.

\begin{enumerate}
   \item Thời gian đi $40$ km đầu là $$40\ \text{km}\div 40\ \frac{\text{km}}{\text{h}}=1{,}0\ \text{h}.$$
\end{enumerate}

Thời gian đi $50$ km sau là $$60\ \text{km}\div 50\ \frac{\text{km}}{\text{h}}=1{,}2\ \text{h}.$$

Do hai quãng đường là cùng chiều nên ta có độ dịch chuyển của xe tổng cộng là $$\Delta x=40\ \text{km} + 60\ \text{km} = 100\ \text{km}$$ và tổng thời gian đi là $$\Delta t =1{,}0\ \text{h}+1{,}2\ \text{h}=2{,}2\ \text{h}.$$

Từ đó, ta có vận tốc trung bình là $$\bar{v} = \frac{\Delta x}{\Delta t} =\boxed{4{,}5\times10^1\ \frac{\text{km}}{\text{h}}}.$$

\begin{enumerate}
   \item[2.] Dễ thấy tổng quãng đường đi là $d=100\ \text{km}$. Tốc độ trung bình là $\bar{s} = \frac{d}{\Delta t}=\boxed{4{,}5\times10^1\ \frac{\text{km}}{\text{h}}}.$
   \item[3.] Thời gian không thay đổi. Có độ dịch chuyển thay đổi còn $\Delta x = 40\ \text{km} - 60\ \text{km} = -20\ \text{km}$ nhưng tổng quãng đường thì không. Do đó, $\boxed{\text{tốc độ trung bình giữ nguyên}}$ nhưng $\boxed{\text{vận tốc trung bình thay đổi}}$.
   \item[4.] Ta có đồ thị ở hình \ref{fig:do_thi_xe} bằng việc vẽ mối quan hệ $x(t)$ xong nối điểm đầu và điểm cuối. Vận tốc trung bình là độ dốc của đường thẳng nối hai điểm này.
\end{enumerate}

\begin{figure}[h]
   \centering
   \begin{tikzpicture}[scale=1.2]
      \draw[->] (0,0) -- (7,0) node[right] {$t$ (h)};
      \draw[->] (0,0) -- (0,5.5) node[above] {$x$ (km)};
      \node[below] at (3.5,-0.5) {Thời gian};
      \node[rotate=90, above] at (-0.6,2.75) {Vận tốc};
      \draw (0,0) -- (3,2);
      \draw (3,2) -- (6.6,5);
      \filldraw (0,0) circle (1.5pt);
      \filldraw (3,2) circle (1.5pt);
      \filldraw (6.6,5) circle (1.5pt);
      \draw (0,0) -- (-0.08,0) node[left] {$0$};
      \draw (0,2) -- (-0.08,2) node[left] {$40$};
      \draw (0,5) -- (-0.08,5) node[left] {$100$};
      \draw (0,0) -- (0,-0.08) node[below] {$0$};
      \draw (3,0) -- (3,-0.08) node[below] {$1{,}0$};
      \draw (6.6,0) -- (6.6,-0.08) node[below] {$2{,}2$};

      \draw[dashed] (6.6,5) -- (6.6,0);
      \node[left] at (6.6,2.5) {$\Delta x = 100\ \text{km}$};
      \draw[dashed] (0,0) -- (6.6,0);
      \node[above] at (3.3,0) {$\Delta t = 2{,}2\ \text{h}$};
      \draw[ultra thick] (0,0) -- (6.6,5);
   \end{tikzpicture}
   \caption{Đồ thị vị trí xe-thời gian chạy}
   \label{fig:do_thi_xe}
\end{figure}

\exercise Một máy bay phản lực đang bay ngang ở độ cao $h=42$ mét. Đột nhiên nó bay vào vùng đất dốc lên góc $\theta=4{,}2^\circ$ (xem hình \ref{fig:may_bay_doc}). Với tốc độ bay là $v=1300\ \frac{\text{km}}{\text{h}}$, thời gian tính từ lúc bay vào vùng đất dốc mà người phi công có để điều chỉnh máy bay là bao nhiêu? Tất cả các số liệu được đo đến hai chữ số có nghĩa.

\begin{figure}[h]
   \centering
   \begin{tikzpicture}[scale=1.2]
      \draw (0,{7*sin(4.2)}) -- (7, 0);
      \draw (7, 0) -- (11, 0);
      \draw[dashed] (0, 0) -- (7, 0);
      \draw (5,0) arc[start angle=180, end angle=175.8, radius=2];

      \draw[->] (5.2,0.5) -- ({7+2*cos(175.8)},{2*sin(175.8)});
      \node[anchor=west] at (5.2,0.5) {$\theta=4{,}2^\circ$};

      \draw (7,2.5) -- (7.5,2.5) -- (7.5,2.7) -- (7,2.5);
      \filldraw[fill=black] (7.5,2.5) -- (8.1,2.5) -- (7.5,2.7) -- cycle;
      \draw (8.1,2.5) -- (8.3, 2.5) -- (8.3, 2.7) -- cycle;

      \draw[<->, dashed] (7,0) -- (7,2.5);
      \node[anchor=west] at (7,1.25) {$h = 42$ m};
      \draw[->] (7,2.5) -- (4,2.5);
      \node[anchor=south] at (5,2.5) {$v=1300\ \frac{\text{km}}{\text{h}}$};
   \end{tikzpicture}
   \caption{Vị trí máy bay trong vùng dốc lên}
   \label{fig:may_bay_doc}
\end{figure}

\solution

Khoảng cách từ máy bay đến điểm va chạm với mặt đất là $$d=\frac{h}{\tan{(\theta)}}.$$ Từ đó, ta có được thời gian cho phép là $$t=\frac{d}{v}=\frac{h}{v\tan{(\theta)}}.$$

Thay số trực tiếp, với để ý đến sự quy đổi $v=1300\ \frac{\text{km}}{\text{h}}=1300\ \frac{\text{km}}{\text{h}}\frac{1000\ \text{m}}{1\ \text{km}}\frac{1\ \text{h}}{3600\ \text{s}}=361\ \frac{\text{m}}{\text{s}}$, ta có $$t=\boxed{1{,}6\times 10^0\ s}.$$

\exercise Cho biết vị trí của một vật chuyển động thẳng được xác định bằng $x(t) = a\cdot t^2+b\cdot t+c$. Xác định vị trí, vận tốc và gia tốc của vật tại thời điểm $t=t_0$.

\solution

Vị trí của vật tại $t=t_0$ là $$x\left(t_0\right)=\boxed{a\cdot t_0^2+b\cdot t_0+c}.$$

Vận tốc của vật tại $t=t_0$ là $$v\left(t_0\right)=\left.\frac{\mathrm{d}x(t)}{\mathrm{d}t}\right|_{t=t_0}=\boxed{2a\cdot t_0+b}.$$

Gia tốc của vật tại $t=t_0$ là $$a\left(t_0\right)=\left.\frac{\mathrm{d}v(t)}{\mathrm{d}t}\right|_{t=t_0}=\boxed{2a}.$$

\exercise Phác họa đồ thị vị trí - thời gian và gia tốc thời gian của một người chạy bộ nếu đồ thị vận tốc - thời gian của người đó được biểu diễn trên đồ thị
\begin{enumerate}
   \item hình \ref{fig:chay_phan_a};
   \item hình \ref{fig:chay_phan_b}.
\end{enumerate}
Các số liệu được coi như chính xác tuyệt đối. Bạn có thể giả sử người đó bắt đầu chạy từ vị trí $x = 0$.

\begin{figure}[h]
   \centering
   \begin{minipage}[t]{0.48\textwidth}
      \centering
      \begin{tikzpicture}
         \draw[->] (0,0) -- (5.5,0) node[right] {$t$ (s)};
         \draw[->] (0,0) -- (0,4.5) node[above] {$v\left(\frac{\text{m}}{\text{s}}\right)$};
         \node[below] at (2.75,-0.5) {Thời gian};
         \node[rotate=90, above] at (-0.5,2.25) {Vận tốc};
         
         \draw[thick] (0,0) -- (1, 4);
         \draw[thick] (1,4) -- (2, 4);
         \draw[thick] (2,4) -- (4,3);
         \draw[thick] (4,3) -- (5, 0);

         \foreach \x/\y in {1/4, 2/4, 4/3} {
            \draw[dashed] (\x,0) -- (\x,\y);
         }
         \draw[dashed] (0,4) -- (1,4);
         \draw[dashed] (4,3) -- (0,3);
         \foreach \x in {0,1,2,4,5} {
            \draw (\x,0) -- (\x,-0.08) node[below] {$\x$};
         }
         \foreach \y in {0, 3, 4} {
            \draw (0,\y) -- (-0.08,\y) node[left] {$\y$};
         }
      \end{tikzpicture}
      \caption{Phần 1}
      \label{fig:chay_phan_a}
   \end{minipage}
   \hfill
   \begin{minipage}[t]{0.48\textwidth}
      \centering
      \begin{tikzpicture}
         \draw[->] (0,0) -- (5.5,0) node[right] {$t$ (s)};
         \draw[->] (0,0) -- (0,4.5) node[above] {$v\left(\frac{\text{m}}{\text{s}}\right)$};

         \node[below] at (2.75,-0.5) {Thời gian};
         \node[rotate=90, above] at (-0.5,2.25) {Vận tốc};
         
         \draw[domain=0:4, smooth, variable=\x, thick] plot ({\x}, {-\x*(4*\x^3-31*\x^2+77*\x-68)/6});
         \draw[thick] (4,0) -- (5,0);
         \foreach \x in {0,1,2,3,4,5} {
            \draw (\x,0) -- (\x,-0.080) node[below] {$\x$};
         }
         \foreach \y in {0,2,3,4} {
            \draw (0,\y) -- (-0.08,\y) node[left] {$\y$};
         }

         \foreach \x/\y in {1/3, 2/2, 3/4} {
            \draw[dashed] (\x,0) -- (\x,\y);
            \draw[dashed] (0,\y) -- (\x,\y);
         }
         
      \end{tikzpicture}
      \caption{Phần 2}
      \label{fig:chay_phan_b}
   \end{minipage}
\end{figure}

\solution

1. Ta chia quá trình chạy làm $4$ phần.


\begin{itemize}
   \item Phần 1 $\left(0\ \text{s}\leq t \leq 1\ \text{s}\right)$: Vận tốc tăng đều từ $0$ đến $4\ \frac{\text{m}}{\text{s}}$. Chuyển động là nhanh dần với gia tốc không đổi là $\left.a(t)\right|_{t\in\left[0\ \text{s};1\ \text{s}\right]}=\frac{v(1\ \text{s})-v(0\ \text{s})}{1\ \text{s}-0\ \text{s}}=4\ \frac{\text{m}}{\text{s}^2}$.
   
Sau khoảng thời gian $t$, độ dịch chuyển là $\left.x(t)\right|_{t\in\left[0\ \text{s};1\ \text{s}\right]} - x(0\ \text{s}) = \frac{\left.a(t)\right|_{t\in\left[0\ \text{s};1\ \text{s}\right]}\cdot t^2}{2} + \left.v(t)\right|_{t\in\left[0\ \text{s};1\ \text{s}\right]}\cdot t$. Từ đó ta có $x(t) = 2\ \frac{\text{m}}{\text{s}^2}\cdot t^2$ với $0\ \text{s}\leq t \leq 1\ \text{s}$ và $x(1\ \text{s}) = 2\ \text{m}$.
   
   \item Phần 2 $\left(1\ \text{s}\leq t \leq 2\ \text{s}\right)$: Vận tốc không đổi ở $\left.v(t)\right|_{t\in\left[1\ \text{s};2\ \text{s}\right]} = 4\ \frac{\text{m}}{\text{s}}$ (chuyển động thẳng đều). 
   
Qua đó, ta có $\left.x(t)\right|_{t\in\left[1\ \text{s};2\ \text{s}\right]} = x(1\ \text{s}) + \left.v(t)\right|_{t\in\left[1\ \text{s};2\ \text{s}\right]}\cdot\left(t - 1\ \text{s}\right) = 4\ \frac{\text{m}}{\text{s}}\cdot t - 2\ \text{m}$ và $x(2\ \text{s}) = 6\ \text{m}$.
\end{itemize}

Phần 3 $\left(2\ \text{s}\leq t \leq 4\ \text{s}\right)$ và phần 4 $\left(4\ \text{s}\leq t \leq 5\ \text{s}\right)$ làm tương tự như phần 1. Ta được
\begin{equation*}
   \begin{cases}
     \left.a(t)\right|_{t\in\left[2\ \text{s};4\ \text{s}\right]} &= -\frac{1}{2}\ \frac{\text{m}}{\text{s}^2}\\
     \left.a(t)\right|_{t\in\left[4\ \text{s};5\ \text{s}\right]} &= -3\ \frac{\text{m}}{\text{s}^2}\\
   \end{cases}
\end{equation*}
và qua đó
\begin{equation*}
   \begin{cases}
     \left.x(t)\right|_{t\in\left[2\ \text{s};4\ \text{s}\right]} &= -\frac{1}{4}\ \frac{\text{m}}{\text{s}^2}\cdot\left(t - 2\ \text{s}\right)^2 + 4\ \frac{\text{m}}{\text{s}}\cdot \left(t - 2\ \text{s}\right) + 6\ \text{m}\\
     \left.x(t)\right|_{t\in\left[4\ \text{s};5\ \text{s}\right]} &= -\frac{3}{2}\ \frac{\text{m}}{\text{s}^2}\cdot\left(t - 4\ \text{s}\right)^2 + 3\ \frac{\text{m}}{\text{s}}\cdot \left(t - 4\ \text{s}\right) + 13\ \text{m}\\
   \end{cases}
\end{equation*}

\begin{equation*}
         \iff
   \begin{cases}
     \left.x(t)\right|_{t\in\left[2\ \text{s};4\ \text{s}\right]} &= -\frac{1}{4}\ \frac{\text{m}}{\text{s}^2}\cdot t^2 + 5\ \frac{\text{m}}{\text{s}}\cdot t - 3\ \text{m}\\
     \left.x(t)\right|_{t\in\left[4\ \text{s};5\ \text{s}\right]} &= -\frac{3}{2}\ \frac{\text{m}}{\text{s}^2}\cdot t^2 + 15\ \frac{\text{m}}{\text{s}}\cdot t - 23\ \text{m}\\
   \end{cases}.
\end{equation*}

Cuối cùng, chúng ta có thể biểu diễn vị trí của người chạy trên đồ thị như hình \ref{fig:vt_tg1}.

\begin{figure}[h]
   \centering
   \fbox{
      \begin{tikzpicture}
         \draw[->] (0,0) -- (5.5,0) node[right] {$t$ (s)};
         \draw[->] (0,0) -- (0,4.5) node[above] {$x\left(\text{m}\right)$};
         \node[below] at (2.75,-0.5) {Thời gian};
         \node[rotate=90, above] at (-0.5,2.25) {Vị trí};
         
         \draw[domain=0:1, smooth, variable=\t, thick] plot ({\t}, {\t^2 / 2});
         \draw[domain=1:2, smooth, variable=\t, thick] plot ({\t}, {\t-1/2});
         \draw[domain=2:4, smooth, variable=\t, thick] plot ({\t}, {-1/16*(\t-2)^2+(\t-2)+3/2});
         \draw[domain=4:5, smooth, variable=\t, thick] plot ({\t}, {-3/8*\t^2+15/4*\t-23/4});
         \foreach \x/\y in {1/2, 2/6, 4/13, 5/14.5} {
            \draw[dashed] (\x,0) -- (\x,\y/4);
            \draw[dashed] (0,\y/4) -- (\x,\y/4);
         }
         \foreach \x in {0,1,2,4,5} {
            \draw (\x,0) -- (\x,-0.08) node[below] {$\x$};
         }
         \foreach \y in {0, 2, 6, 13, 14.5} {
            \draw (0,\y/4) -- (-0.08,\y/4) node[left] {$\y$};
         }
      \end{tikzpicture}
   }
   \caption{Đồ thị vị trí - thời gian cho phần 1}
   \label{fig:vt_tg1}
\end{figure}

2. Chúng ta có thể phác họa đồ thị vị trí - thời gian bằng việc xấp xỉ đồ thị vận tốc - thời gian dưới dạng đường gấp khúc nối các điểm đã biết thể hiện ở \ref{fig:xx_p2}.

Từ đây, thực hiện tương tự như phần 1 để có phương trình vị trí - thời gian
\begin{equation*}
   x(t) = \begin{cases}
      \frac{3}{2}\ \frac{\text{m}}{\text{s}^2}\cdot t^2 &\quad \text{với } 0 \leq t < 1\ \text{s}\\
      -\frac{1}{2}\ \frac{\text{m}}{\text{s}^2}\cdot t^2 + 4\ \frac{\text{m}}{\text{s}}\cdot t - 2\ \text{m}&\quad \text{với } 1\ \text{s} \leq t < 2\ \text{s}\\
      1\ \frac{\text{m}}{\text{s}^2}\cdot t^2 - 2\ \frac{\text{m}}{\text{s}}\cdot t + 4\ \text{m}&\quad \text{với } 2\ \text{s} \leq t < 3\ \text{s}\\
      -2\ \frac{\text{m}}{\text{s}^2}\cdot t^2+16\ \frac{\text{m}}{\text{s}}\cdot t-23\ \text{m}&\quad \text{với } 3\ \text{s} \leq t < 4\ \text{s}\\
      9\ \text{m}&\quad \text{với } 4\ \text{s} \leq t \leq 5\ \text{s}
   \end{cases}
\end{equation*}
và ta vẽ được đồ thị ở hình \ref{fig:vttgxxp2}.

\begin{figure}[h]
   \centering
   \begin{minipage}[t]{0.48\textwidth}
      \centering
      \begin{tikzpicture}
         \draw[->] (0,0) -- (5.5,0) node[right] {$t$ (s)};
         \draw[->] (0,0) -- (0,5) node[above] {$v\left(\frac{\text{m}}{\text{s}}\right)$};

         \node[below] at (2.75,-0.5) {Thời gian};
         \node[rotate=90, above] at (-0.5,2.25) {Vận tốc};
         
         \draw[thick] (0,0) -- (1,3) -- (2,2) -- (3,4) -- (4,0) -- (5,0);
         \foreach \x in {0,1,2,3,4,5} {
            \draw (\x,0) -- (\x,-0.080) node[below] {$\x$};
         }
         \foreach \y in {0,2,3,4} {
            \draw (0,\y) -- (-0.08,\y) node[left] {$\y$};
         }

         \foreach \x/\y in {1/3, 2/2, 3/4} {
            \draw[dashed] (\x,0) -- (\x,\y);
            \draw[dashed] (0,\y) -- (\x,\y);
         }
      \end{tikzpicture}
      \caption{Vận tốc - thời gian xấp xỉ của phần 2}
      \label{fig:xx_p2}
   \end{minipage}
   \hfill
   \begin{minipage}[t]{0.48\textwidth}
      \centering
      \fbox{
         \begin{tikzpicture}
            \draw[->] (0,0) -- (5.5,0) node[right] {$t$ (s)};
            \draw[->] (0,0) -- (0,5) node[above] {$x\left(\text{m}\right)$};
            \node[below] at (2.75,-0.5) {Thời gian};
            \node[rotate=90, above] at (-0.5,2.25) {Vị trí};
            
            \draw[domain=0:1, smooth, variable=\t, thick] plot ({\t}, {(3*\t^2 / 2) / 2});
            \draw[domain=1:2, smooth, variable=\t, thick] plot ({\t}, {(-\t^2 / 2 + 4*\t - 2)/2});
            \draw[domain=2:3, smooth, variable=\t, thick] plot ({\t}, {(\t^2 -2*\t +4)/2});
            \draw[domain=3:4, smooth, variable=\t, thick] plot ({\t}, {(-2*\t^2+16*\t-23)/2});
            \draw[thick] (4,4.5) -- (5,4.5);

            \foreach \x/\y in {1/1.5, 2/4, 3/7, 4/9} {
               \draw[dashed] (\x,0) -- (\x,\y/2);
               \draw[dashed] (0,\y/2) -- (\x,\y/2);
            }
            \draw[dashed] (5,0) -- (5,4.5);
            \foreach \x in {0,1,2,3,4,5} {
               \draw (\x,0) -- (\x,-0.08) node[below] {$\x$};
            }
            \foreach \y in {0, 1.5, 4, 7, 9} {
               \draw (0,\y/2) -- (-0.08,\y/2) node[left] {$\y$};
            }
         \end{tikzpicture}
      }
      \caption{Vị trí - thời gian (xấp xỉ) cho phần 2}
      \label{fig:vttgxxp2}
   \end{minipage}
\end{figure}

\begin{figure}[h!]
   \centering
   \fbox{
      \begin{tikzpicture}
         \draw[->] (0,0) -- (5.5,0) node[right] {$t$ (s)};
         \draw[->] (0,0) -- (0,6) node[above] {$v\left(\frac{\text{m}}{\text{s}}\right)$};

         \node[below] at (2.75,-0.5) {Thời gian};
         \node[rotate=90, above] at (-0.5,2.25) {Vận tốc};
         
         \draw[domain=0:4, smooth, variable=\x, thick] plot ({\x}, {-\x^2*(48*\x^3-465*\x^2+1540*\x-2040)/720});
         \draw[thick] (4,{496/90}) -- (5,{496/90});
         \foreach \x in {0,1,2,3,4,5} {
            \draw (\x,0) -- (\x,-0.080) node[below] {$\x$};
         }
         \draw (0,{496/90}) -- (-0.08,{496/90}) node[left] {$\approx 11$};

         \foreach \x/\y in {4/{496/45}, 5/{496/45}} {
            \draw[dashed] (\x,0) -- (\x,{\y/2});
         }
         \draw[dashed] (0,{496/90}) -- (4,{496/90});
      \end{tikzpicture}
   }
   \caption{Đồ thị vị trí - thời gian cho phần 2}
   \label{fig:vttgp2}
\end{figure}

Trong thực tiễn, chúng ta hay xấp xỉ những quá trình không tuyến tính qua hữu hạn những điểm đo rồi nội suy tuyến tính (nối các điểm bằng các đoạn thẳng) như đã làm. Còn nhiều phương pháp nội suy nữa còn có thể được tìm thấy trong những tài liệu về phương pháp tính và giải tích số. Thông thường, với càng nhiều điểm thì độ chính xác càng lớn.

Trong trường hợp mà bạn nhận ra phương trình vận tốc - thời gian được cho là
\begin{equation*}
   v(t) =
   \begin{cases}
      \displaystyle \frac{\displaystyle -t\left(4\ \frac{\text{m}}{\text{s}^5}\cdot t^3-31\ \frac{\text{m}}{\text{s}^4}\cdot t^2+77\ \frac{\text{m}}{\text{s}^3}\cdot t-68\ \frac{\text{m}}{\text{s}^2}\right)}{6} &\quad \text{với } 0 \leq t < 4 \\
      0&\quad \text{với } 4 \leq t \leq 5
   \end{cases}
\end{equation*}
thì bạn có thể thực hiện nguyên hàm trên hàm này để tính được vị trí vật là
\begin{equation*}
   \displaystyle 
   x(t) =
   \begin{cases}
      \displaystyle \frac{\displaystyle -t^2\left(48\ \frac{\text{m}}{\text{s}^5}\cdot t^3-465\ \frac{\text{m}}{\text{s}^4}\cdot t^2+1540\ \frac{\text{m}}{\text{s}^3}\cdot t-2040\ \frac{\text{m}}{\text{s}^2}\right)}{360} &\quad \text{với } 0 \leq t < 4 \\
      \displaystyle \frac{496}{45}\ \text{m}&\quad \text{với } 4 \leq t \leq 5
   \end{cases}
\end{equation*}
và ta có đồ thị như hình \ref{fig:vttgp2}.

\exercise Hai xe hơi có tốc độ lần lượt là $v_1 = 50{,}0\ \frac{\text{km}}{\text{h}}$ và $v_2 = 60{,}0\ \frac{\text{km}}{\text{h}}$ đi ngược chiều với nhau trên một con đường hẹp. Hai xe phát hiện lẫn nhau khi khoảng cách giữa hai xe là $d = 400\ \text{m}$. Cả hai xe đồng thời giảm tốc với cùng một gia tốc hãm đều là $a$. Tính giá trị tối thiểu của $a$ nếu biết hai xe không xảy ra va chạm. Số liệu được đo tới $3$ chữ số có nghĩa.

\solution

Gọi quãng đường đi được trong khi hãm phanh của hai xe lần lượt là $d_1$ và $d_2$.

Trong quá trình hãm đến vận tốc bằng $0$, tổng quãng đường đi của cả hai xe phải không vượt quá khoảng cách $d$. Vì vậy, ta có bất đẳng thức $$d_1 + d_2 \leq d.$$

Trong khi đó, quãng đường xe thứ nhất đã di chuyển là $d_1 = \frac{0^2 - v_1^2}{2(-a)} = \frac{v_1^2}{2a}$. Tương tự, ta có quãng đường mà xe thứ hai di chuyển trong khoảng thời gian này là $d_2 = \frac{v_2^2}{2a}$. Từ đó, thay vào phương trình ở trên để được $$
   \frac{v_1^2}{2a} + \frac{v_2^2}{2a} \le d
   \iff a \geq \frac{v_1^2+v_2^2}{2d}.
$$

Thay số trực tiếp, ta có gia tốc hãm tối thiểu phải là $\boxed{7{,}63 \times 10^3 \frac{\text{km}}{\text{h}^2}}$.

\exercise Để dừng xe ban đầu bạn cần một thời gian phản ứng để bắt đầu phanh, rồi xe mới đi chậm dần nhờ có một gia tốc hãm không đổi. Giả sử quãng được đi được trong hai pha này là $186$ ft nếu vận tốc ban đầu là $50\ \frac{\text{dặm}}{\text{h}}$. Còn trong một trường hợp khác, quãng được đi được trong hai pha này là $80$ ft nếu vận tốc ban đầu là $30\ \frac{\text{dặm}}{\text{h}}$. Biết thời gian phản ứng là cố định và $1$ dặm $= 5280$ ft, tính thời gian phản ứng và độ lớn của gia tốc hãm.

\solution

Gọi thời gian phản ứng là $t_p$, vận tốc đầu là $v_0$, gia tốc hãm là $a$.

Trong khoảng thời gian phản ứng, xe đi được $v_0t_p$. Và trong khoảng thời gian hãm, xe đi được $\frac{0^2-v_0^2}{2(-a)}=\frac{v_0^2}{2a}$. Cho nên, tổng quãng đượt đi được trong hai pha là 
\begin{equation}
\Delta x = v_0 t + \frac{v_0^2}{2a}
\label{eq:stopping_distance}
\end{equation}

Trước khi thay số ta thực hiện quy đổi $$50\ \frac{\text{dặm}}{\text{h}}=50\ \frac{\text{dặm}}{\text{h}}\cdot\frac{5280\ \text{ft}}{1\ \text{dặm}}\cdot\frac{1\ \text{h}}{3600\ \text{s}}\approx 73\ \frac{\text{ft}}{\text{s}},$$ tương tự, $30\ \frac{\text{dặm}}{\text{h}}=44\ \frac{\text{ft}}{\text{s}}$. Từ đó, thay số vào phương trình \ref{eq:stopping_distance} để có hệ
\begin{equation*}
   \begin{cases}
      186\ \text{ft} = 73\ \frac{\text{ft}}{\text{s}}\cdot t_p + \frac{\left(73\ \frac{\text{ft}}{\text{s}}\right)^2}{2a} \\
      80\ \text{ft} = 44\ \frac{\text{ft}}{\text{s}}\cdot t_p + \frac{\left(44\ \frac{\text{ft}}{\text{s}}\right)^2}{2a} 
   \end{cases}.
\end{equation*}
Giải hệ phương trình, ta có thời gian phản ứng là $t_p=\boxed{0{,}97\ \text{s}}$ và gia tốc hãm là $a = \boxed{26\ \frac{\text{ft}}{\text{s}^2}}$.

\begin{wrapfigure}{r}{0.6\textwidth} % 'r' for right, 'l' for left, width of the figure
    \centering
    \begin{tikzpicture}
      \draw[->] (0,0) -- (8,0) node[right] {$t$};
      \draw[->] (0,0) -- (0,6) node[above] {$h$};

      \node[below] at (4,-0.25) {Thời gian};
      \node[rotate=90, above] at (-0.25,3) {Độ cao};

      \draw[domain=0.8:7.2, smooth, variable=\x, thick] plot ({\x}, {-4 * (\x - 1) * (\x - 7)/9 + 1});
      \draw[<->] (1,1) -- (7,1);
      \node[anchor=south] at (4,1) {$\Delta T_t$};
      \draw[<->] (3,41/9) -- (5,41/9);
      \node[anchor=north] at (4,41/9) {$\Delta T_c$};
      \draw[<->] (0.5,1) -- (0.5,41/9);
      \node[anchor=west] at (0.5,25/9) {$H$};

      \draw[dashed] (0,1) -- (1,1);
      \draw[dashed] (7,1) -- (8,1);
      \draw[dashed] (0,41/9) -- (3,41/9);
      \draw[dashed] (5,41/9) -- (8,41/9);

   \end{tikzpicture}

   \caption{Đồ thị thời gian - độ cao của quả bóng thủy tinh}
   \label{fig:tgdcqbtt}
\end{wrapfigure}

\exercise Tại Phòng Thí nghiệm Vật lí Quốc gia ở Anh, người ta thực hiện xác định gia tốc trọng trường $g$ theo thí nghiệm sau: Ném một quả bóng thủy tinh lên theo chiều thẳng đứng trong ống chân không và cho nó rơi xuống. Gọi $\Delta T_t$ trên hình \ref{fig:tgdcqbtt} là thời gian khoảng giữa hai lần quả bóng đi qua một điểm thấp nào đó. $\Delta T_c$ là khoảng thời gian giữa hai lần quả bóng đi qua một điểm cao hơn và $H$ là khoảng cách giữa hai điểm. Chứng minh rằng $$g=\frac{8H}{\Delta T_t^2 - \Delta T_c^2}.$$

\solution

Gọi vận tốc khi bóng bắt đầu bay lên từ vị trị thấp là $v_0$. Sau một khoảng thời gian $\Delta T_t$, quả bóng quay lại vị trí cũ, do vậy, ta có phương trình $0 = -\frac{g \Delta T_t^2}{2} + v_0 \Delta T_t$. Thực hiện biến đổi tương đương để có $$v_0=\frac{g \Delta T_t}{2}.$$

Nhận thấy rằng đồ thị có tính đối xứng. Sử dụng điều đó, ta tính được khoảng thời gian quả bóng lên một độ cao $H$ là $t=\frac{\Delta T_t-\Delta T_c}{2}$. Qua đó, ta có phương trình thứ hai là $$H = -\frac{g t^2}{2} + v_0 t=-\frac{g \left(\frac{\Delta T_t-\Delta T_c}{2}\right)^2}{2} + v_0 \left(\frac{\Delta T_t-\Delta T_c}{2}\right).$$

Thế giá trị của $v_0$ vào phương trình và tiếp tục thực hiện biến đổi, ta có:
\begin{align*}
   H &= -\frac{g \left(\Delta T_t-\Delta T_c\right)^2}{8} + \frac{g \Delta T_t}{2} \left(\frac{\Delta T_t-\Delta T_c}{2}\right) \\
   &= -g\left(\frac{\Delta T_t^2}{8} - \frac{\Delta T_t\Delta T_c}{4} + \frac{\Delta T_c^2}{8}\right) + g\left(\frac{\Delta T_t^2}{4} - \frac{\Delta T_t\Delta T_c}{4}\right) \\
   &= g\cdot \frac{\Delta T_t^2 - \Delta T_c^2}{8} \\
   \iff g &= \frac{8H}{\Delta T_t^2 - \Delta T_c^2}.
\end{align*}

Ta có điều phải chứng minh.

\exercise Một nghệ sĩ tung hứng các quả bóng lên theo phương thẳng đứng. Quả bóng sẽ lên cao hơn bao nhiêu nếu thời gian bóng trong không khí tăng gấp $n$ lần ($n \in \mathbb{R}^+$)?

\solution

Có thời gian để quả bóng bay từ tay lên trên vị trí cao nhất bằng một nửa thời gian bóng trong không khí. Nếu thời gian bóng trong không khí tăng gấp $n$ lần so với thời gian trong không khí gốc, thì cùng chia cho $2$, ta cũng sẽ có thời gian bóng bay từ tay lên trên vị trí cao nhất cũng tăng gấp $n$ lần so với thời gian gốc để bay lên vị trí cao nhất.

Gọi $t_1$ là thời gian gốc để bóng bay từ tay lên vị trí cao nhất, $t_2 = n t_1$ là thời gian bay khi đã tăng $n$ lần. Gọi $h_1, h_2$ lần lượt là độ cao bóng đi được tương ứng với hai khoảng thời gian $t_1, t_2$. Để ý rằng khi lên vị trí cao nhất thì vận tốc bóng là $0$; ta có hệ phương trình

\begin{equation*}
   \begin{cases}
      h_1 &= \frac{gt_1^2}{2} \\
      h_2 &= \frac{gt_2^2}{2} = \frac{g\left(nt_1\right)^2}{2}
   \end{cases}
   \implies h_2 = n^2 h_1.
\end{equation*}

Từ đó, ta có quả bóng cao lên hơn được $\boxed{n^2 - 1 \text{ lần độ cao gốc}}$.

\begin{wrapfigure}{l}{0.4\textwidth} % 'r' for right, 'l' for left, width of the figure
   \centering
   \begin{tikzpicture}
      \draw[->] (-0.5, 0) -- (4,0) node[right] {$x$};
      \draw[->] (0, -0.5) -- (0, 3) node[above] {$y$};
      \filldraw (0, 0) circle (1.5pt);
      \draw[-{Latex[width=1.5mm]}] (0, 0) -- (1.3, 0.8) node[above left] {$\vec{v_1}$};
      \draw[-{Latex[width=1.5mm]}] (3, 2) -- (2, 2);
      \node[above] at (2.5,2) {$\vec{u}$};
      \draw (0.5,0) arc[start angle=0, end angle={atan(8/13)}, radius=0.5];
      \node at (0.72,0.20) {$\alpha$};
      
      \node[below left] at (0, 0) {$O$};
      
   \end{tikzpicture}

   \caption{Hình minh họa cho bài \ref{ex:16}}
   \label{fig:mhb16}
\end{wrapfigure}

\exercise[ex:16] Như trong hình \ref{fig:mhb16}, một vật nhỏ có khối lượng $m$ chỉ di chuyển từ gốc $O$ trong mặt phẳng $Oxy$ được cung cấp một vận tốc ban đầu $\overrightarrow{v_1}$ trong vùng không gian có gió thổi với vận tốc $\vec{u} = -u \overrightarrow{e_x}$.



\begin{thebibliography}{1}
\bibitem{Agarwal2011}
Agarwal, R.P., Perera, K., Pinelas, S. (2011). \textit{History of Complex Numbers}. In: An Introduction to Complex Analysis. Springer, Boston, MA. \url{https://doi.org/10.1007/978-1-4614-0195-7_50}
\end{thebibliography}

\end{document}

