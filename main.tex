\documentclass[a4paper, titlepage, openany]{book}
\usepackage[utf8]{inputenc}
\usepackage[T5]{fontenc}
\usepackage[vietnamese]{babel}
\usepackage[a4paper]{geometry}
\usepackage{amsmath, amssymb, blindtext, calc, float, hyperref, icomma, makecell, multicol, steinmetz, tikz, tikz-3dplot, wrapfig, xcolor, xparse, xeCJK}
\usetikzlibrary{arrows.meta, calc, shadings}
\geometry{bindingoffset=0.5cm, inner=2cm, outer=2cm, top=2cm, bottom=2cm}
\setCJKmainfont[AutoFakeSlant=0.15,AutoFakeBold=2.0]{Nom Na Tong}
\setcellgapes{2pt}
\makegapedcells

\DeclareMathOperator{\Arg}{Arg}

\title{\Huge Ôn tập Vật Lí}
\author{Bùi Nhật Minh}
\date{\today}

\newcounter{exercise}

\NewDocumentCommand{\exercise}{o}{
   \refstepcounter{exercise}
   \noindent\textbf{Bài \arabic{exercise}:}
   \IfNoValueTF{#1}
      {\label{ex\arabic{exercise}}}
      {\label{#1}}
}

% Counter for solutions
\newcounter{solution}

\NewDocumentCommand{\solution}{o}{
   \refstepcounter{solution}
   \IfNoValueTF{#1}
      {\noindent\textbf{Lời giải cho bài~\ref{ex\arabic{solution}}:}}
      {\noindent\textbf{Lời giải cho bài~\ref{#1}}:}
   \IfNoValueTF{#1}
      {\label{sol\arabic{solution}}}
      {\label{sol-#1}}
}

\newcommand\dblquote[1]{\textquotedblleft #1\textquotedblright}
\newcommand{\chapdir}{chapter/}
\renewcommand{\thefigure}{\thechapter.\arabic{figure}}
\renewcommand{\thetable}{\thechapter.\arabic{table}}

\numberwithin{equation}{chapter}

\begin{document}

\maketitle

\setcounter{chapter}{-1}

\tableofcontents

\chapter*{Lời giới thiệu}
\addcontentsline{toc}{chapter}{Lời giới thiệu}

\chapter{Kiến thức toán học nền tảng}

\ % Lùi đầu dòng

Phần này bao gồm các kiến thức toán học cần thiết để xây dựng lí thuyết của môn vật lí (hoặc ít nhất để đọc tài liệu này), giả sử rằng bạn đọc đã có kiến thức đại số và một chút hình học từ ghế nhà trường. Chương này sẽ bao hàm những phần không nằm trong chương trình trung học phổ thông và có thể cả chương trình đại học. Bạn đọc có thể tìm hiểu một cách sơ cấp hay gợi nhớ lại về các khái niệm toán học mà không cần tập trung vào việc chứng minh chặt chẽ các tính chất toán học. Tác giả mong muốn thông qua chương này, bạn đọc có thể có một cảm nhận và từ đó có kĩ năng để áp dụng các khái niệm toán giải quyết các yêu cầu thực tế. Suy cho cùng, mấu chốt của nhiều ngành khoa học, bao gồm cả vật lí, là xây dựng những mô hình toán học biểu diễn môi trường để từ đó đưa ra những dự đoán hay xây dựng nhũng công trình cho tương lai.

Và vì bạn đọc đang đọc về tài liệu nghiêng về vật lí, Tác giả sẽ không tập trung nhiều vào chặt chẽ toán học. Các định nghĩa và chứng minh cần thiết vẫn sẽ được đưa ra, tuy nhiên không quá chặt chẽ nhưng đủ để ứng dụng, nhằm làm bước đệm cho bạn đọc nếu có mong muốn tìm hiểu sâu hơn về toán.

\section{Đồ thị}

\subsection{Trục số một chiều}

\ % Lùi đầu dòng

Đồ thị là cầu nối đầu tiên giữa đại số và hình học mà chắc là bạn đọc đã được học. Thông thường, nhắc đến đồ thị, chúng thường được dùng để biểu thị mặt phẳng hai chiều hoặc không gian ba chiều. Nhưng, đồ thị cơ bản nhất chỉ có một chiều, hay tên gọi khác là \defText{trục số}. 

{
   \begin{minipageindent}{0.55\textwidth}
      Đặt một điểm $O$ trên trục làm gốc tọa độ biểu diễn cho số $0$, từ đó chúng ta có thể biểu diễn mọi số thực trên trục số này. Nói một cách không chính thống, với một số $x_P$ dương bất kì, đánh dấu cách $O$ một đoạn bằng $x_P$ đơn vị độ dài theo hướng trục, chúng ta có điểm $P$ biểu diễn $x_P$. Viết tắt cách biểu diễn, được $\bm{P(x_P)}$. Ngược lại, nếu chúng ta muốn đánh dấu số $x_{P_-}=-x_P$ mang giá trị âm, chúng ta dịch ngược lại chiều trục như trên hình \ref{fig:do_thi:truc_so:truc_so_mot_chieu}.
      
      Khi có nhiều điểm ở trên đồ thị, chúng ta sẽ mong muốn tính những thông số liên quan tới những điểm đó. Do kiến thức toán hiện tại đang bị giới hạn, chúng ta sẽ chỉ tập trung vào một đặc điểm nhất định, \defText{khoảng cách}. Trên một trục số như hình \ref{fig:do_thi:truc_so:khoang_cach_truc_so}, cho hai điểm $P(x_P)$ và $Q(x_Q)$, khoảng cách giữa chúng là $$\defMath{d(P;Q)=\sqrt{(x_P-x_Q)^2}=|x_P-x_Q|}.$$
   \end{minipageindent}
   \hfill
   \begin{minipageindent}{0.4\textwidth}
      \begin{figure}[H]
         \centering
         \begin{tikzpicture}
            \draw[->] (-3,0) -- (2,0) node[right] {Trục số};
            \filldraw(0, 0) circle (\pointSize) node[below] {$O(0)$};
            \filldraw[color=colorEmphasisCyan] (1.5, 0) circle (\pointSize) node[below,color=colorEmphasisCyan] {$P(x_P)$};
            \filldraw[color=colorEmphasisCyan] (-1.5, 0) circle (\pointSize) node[below,color=colorEmphasisCyan] {$P_-(-x_P)$};
         \end{tikzpicture}
         \caption{Trục số một chiều}
         \label{fig:do_thi:truc_so:truc_so_mot_chieu}
      \end{figure}
      
      \begin{figure}[H]
         \centering
         \begin{tikzpicture}
            \draw[->] (-3,0) -- (2,0) node[right] {$x$};
            \pgfmathsetmacro{\xP}{1.2}
            \pgfmathsetmacro{\xQ}{-2}
            \pgfmathsetmacro{\h}{0.4}
            \draw[very thin] (\xP,0) -- (\xP,\h);
            \draw[very thin] (\xQ,0) -- (\xQ,\h);
            \filldraw[color=colorEmphasisCyan] (\xP, 0) circle (\pointSize) node[below, color=colorEmphasisCyan] {$P(x_P)$};
            \filldraw[color=colorEmphasisCyan] (\xQ, 0) circle (\pointSize) node[below, color=colorEmphasisCyan] {$Q(x_Q)$};
            \draw[very thick,color=colorEmphasisCyan] (\xQ,0) -- (\xP,0);
            \node[above,color=colorEmphasisCyan] at ({(\xP+\xQ)/2}, {\h/2}) {$d(P;Q)$};
            \draw[<->, >=latex, shorten >=\pointSize, shorten <=\pointSize, color=colorEmphasisCyan] (\xP,{\h/2}) -- (\xQ,{\h/2});
         \end{tikzpicture}
         \caption{Khoảng cách trên trục số}
         \label{fig:do_thi:truc_so:khoang_cach_truc_so}
      \end{figure}
   \end{minipageindent}
}

\exercise[ex:0.1] Biểu diễn nhóm các điểm sau trên trục số. Tính khoảng cách giữa hai điểm phân biệt bất kì trong nhóm đó.
\begin{enumerate}
   \item $A(2)$, $B(-3)$, và $C(4)$;
   \item $D(\pi)$, $E(-\pi)$, $F(0)$, và $G\left(\frac{\pi}{2}\right)$;
   \item $H(0{,}\overline{3})$ và $I(\sqrt{2})$;
   \item $J\left(\frac{355}{113}\right)$, $K\left(\frac{9801}{2206\sqrt{2}}\right)$ và $L\left(\sqrt[4]{\frac{2143}{22}}\right)$;
   \item $M(x)$ và $N(2x)$ với $x\in\mathbb{R}$.
\end{enumerate}

\solution[ex:0.1] 
\begin{figure}[h]
   \centering
   \begin{tikzpicture}
      \draw[->] (-5,0) -- (5,0) node[right] {$x$};
      \filldraw (0, 0) circle (\pointSize) node[above] {$O(0)$};
      \filldraw[color=colorEmphasisCyan] (2, 0) circle (\pointSize) node[below] {$A(2)$};
      \filldraw[color=colorEmphasisCyan] (-3, 0) circle (\pointSize) node[below] {$B(-3)$};
      \filldraw[color=colorEmphasisCyan] (4, 0) circle (\pointSize) node[below] {$C(4)$};
   \end{tikzpicture}
   \caption{Trục số cho phần 1 của bài \ref{ex:0.1}}
   \label{fig:do_thi:truc_so:truc_so_nguyen}
\end{figure}

\begin{figure}[h]
   \centering
   \begin{tikzpicture}
      \draw[->] (-5,0) -- (5,0) node[right] {$x$};
      \filldraw[color=colorEmphasisCyan] ({pi}, 0) circle (\pointSize) node[below] {$D(\pi)$};
      \filldraw[color=colorEmphasisCyan] ({-pi}, 0) circle (\pointSize) node[below] {$E(-\pi)$};
      \filldraw[color=colorEmphasisCyan] (0, 0) circle (\pointSize) node[below] {$F(0)$};
      \filldraw[color=colorEmphasisCyan] ({pi/2}, 0) circle (\pointSize) node[below] {$G\left(\frac{\pi}{2}\right)$};
   \end{tikzpicture}
   \caption{Trục số cho phần 2 của bài \ref{ex:0.1}}
   \label{fig:do_thi:truc_so:truc_so_pi}
\end{figure}

\begin{figure}[h]
   \centering
   \begin{tikzpicture}
      \draw[->] (0,0) -- (10,0) node[right] {$x$};
      \filldraw (0, 0) circle (\pointSize) node[above] {$O(0)$};
      \filldraw[color=colorEmphasisCyan] ({2/3}, 0) circle (\pointSize) node[below] {$H\left(0{,}\overline{3}\right)$};
      \filldraw[color=colorEmphasisCyan] ({2*sqrt(2)}, 0) circle (\pointSize) node[below] {$I(\sqrt{2})$};
      \filldraw (4, 0) circle (\pointSize) node[above] {$C(2)$};
   \end{tikzpicture}
   \caption{Trục số cho phần 3 của bài \ref{ex:0.1}}
   \label{fig:do_thi:truc_so:truc_so_thap_phan}
\end{figure}

\begin{figure}[H]
   \centering
   \begin{tikzpicture}
      \draw[->] (0,0) -- (10,0) node[right] {$x$};
      \filldraw (0,0) circle (\pointSize) node[above] {$P(3{,}1415926)$};
      \filldraw (3,0) circle (\pointSize) node[above] {$Q(3{,}1415927)$};
      \filldraw (6,0) circle (\pointSize) node[above] {$R(3{,}1415928)$};
      \filldraw (9,0) circle (\pointSize) node[above] {$S(3{,}1415929)$};
      \filldraw (9.61062, 0)[color=colorEmphasisCyan] circle (\pointSize) node[below] {$J\left(\frac{355}{113}\right)$};
      \filldraw (3.9, 0)[color=colorEmphasisCyan] circle (\pointSize) node[below] {$K\left(\frac{9801}{2206\sqrt{2}}\right)$};
      \filldraw (1.577475, 0)[color=colorEmphasisCyan] circle (\pointSize) node[below] {$L\left(\sqrt[4]{\frac{2143}{22}}\right)$};
   \end{tikzpicture}
   \caption{Trục số cho phần 4 của bài \ref{ex:0.1}}
   \label{fig:do_thi:truc_so:truc_so_xap_xi_pi}
\end{figure}



Ta có đồ thị cho các phần từ $1$ đến $4$ như các hình \ref{fig:do_thi:truc_so:truc_so_nguyen}, \ref{fig:do_thi:truc_so:truc_so_pi}, \ref{fig:do_thi:truc_so:truc_so_thap_phan}, và \ref{fig:do_thi:truc_so:truc_so_xap_xi_pi}.

Cần lưu ý rằng, để biểu diễn thuận lợi nhất, các trục số khi biểu diễn số cần được chọn những tỉ lệ khác nhau và tại những vị trí khác nhau.

Các khoảng cách giữa hai điểm phân biệt đôi một là
\begin{enumerate}
   \item \begin{alignat*}{2}
      &d(A;B) = d(B;A) = \left|2 - (-3)\right| = 5; \\
      &d(B;C) = d(C;B) = \left|4 - (-3)\right| = 7; \\
      &d(C;A) = d(A;C) = \left|4 - 2\right| = 2.
   \end{alignat*}
   \item \begin{alignat*}{2}
      &d(D;E) = d(E;D) = \left|\pi - (-\pi)\right| = 2\pi; \\
      &d(E;F) = d(F;E) = \left|(-\pi) - 0\right| = \pi; \\
      &d(F;G) = d(G;F) = \left|0 - \frac{\pi}{2}\right| = \frac{\pi}{2}; \\
      &d(G;D) = d(D;G) = \left|\frac{\pi}{2} - \pi\right| = \frac{\pi}{2}; \\
      &d(D;F) = d(F;D) = \left|\pi - 0\right| = \pi;\\
      &d(E;G) = d(G;E) = \left|(-\pi) - \frac{\pi}{2}\right| = \frac{3\pi}{2}.
   \end{alignat*}
   \item \begin{alignat*}{2}
      &d(H;I) = d(I;H) = \left|0{,}\overline{3} - \sqrt{2}\right| = \frac{1-3\sqrt{2}}{3};
   \end{alignat*}
   \item \begin{alignat*}{2}
      &d(J;K) = d(K;J) = \left|\frac{355}{113} - \frac{9801}{2206\sqrt{2}}\right| = \frac{1566260-1107513\sqrt{2}}{498556} \textcolor{colorEmphasisCyan}{\approx 1{,}9034\times 10^{-7}}; \\
      &d(K;L) = d(L;K) = \left|\frac{9801}{2206\sqrt{2}} - \sqrt[4]{\frac{2143}{22}}\right| = \frac{107811\sqrt{2}-2206\sqrt[4]{22818664}}{48532} \textcolor{colorEmphasisCyan}{\approx 7{,}7431\times 10^{-8}}; \\
      &d(L;J) = d(J;L) = \left|\sqrt[4]{\frac{2143}{22}} - \frac{355}{113}\right| = \frac{7810-113\sqrt[4]{22818664}}{2486} \textcolor{colorEmphasisCyan}{\approx 2{,}6777\times 10^{-7}};
   \end{alignat*}
\end{enumerate}

Trong vật lí, việc tính toán chính xác đến như ở phần $4$ là không cần thiết và nhiều khi còn không chính xác. Luôn luôn có sai số khi đo đạc, và trong phần lớn trường hợp, khi kết hợp sai số này vào trong tính toán thì các giá trị khoảng cách như trên gần như vô nghĩa. Cho nên, về mặt thực tiễn, chúng ta hoàn toàn có thể thay thế đồ thị của $4$ như hình \ref{fig:do_thi:truc_so:truc_so_bon_xx} và khi tính khoảng cách, chúng ta có thể tính xấp xỉ là $$d(J,K) = d(K,J) \approx d(K,L) = d(L,K) \approx d(L,J) = d(J,L) \approx 0.$$

\begin{figure}[H]
   \centering
   \begin{tikzpicture}
      \draw[->] (-5,0) -- (5,0) node[right] {$x$};
      \filldraw (0, 0) circle (\pointSize) node[above] {$O(0)$};
      \filldraw (pi, 0)[color=colorEmphasisCyan] circle (\pointSize) node[below] {$J\left(\frac{355}{113}\right)$};
      \node[below][color=colorEmphasisCyan] at (pi, -0.6) {$K\left(\frac{9801}{2206\sqrt{2}}\right)$};
      \node[below][color=colorEmphasisCyan] at (pi, -1.2) {$L\left(\sqrt[4]{\frac{2143}{22}}\right)$};
      \node[above][color=colorEmphasis] at (pi, 0) {$\approx \pi$};
   \end{tikzpicture}
   \caption{Xấp xỉ vị trí điểm trên trục số cho phần $4$ của bài \ref{ex:0.1}}
   \label{fig:do_thi:truc_so:truc_so_bon_xx}
\end{figure}

Để vẽ được đồ thị cho phần $5$, chúng ta sẽ xét vị trí tương đối giữa $M$, $N$ kèm theo gốc $O$ để quy chiếu như biểu diễn ở hình \ref{fig:truc phan 5}. Cụ thể, khi $x>0$, điểm $M$ và $N$ được biểu diễn thành hai điểm $M_+$ và $N_+$. Tương tự, khi $x<0$, $M$ và $N$ biểu diễn hai điểm $M_-$ và $N_-$. Một trường hợp đặc biệt là khi $x=0$, $M$ và $N$ đều có tọa độ là $0$, cho nên hai điểm đó và gốc cùng chia sẻ vị trí với nhau.

\begin{figure}[H]
   \centering
   \begin{tikzpicture}
      \draw[->] (-5,0) -- (5,0) node[right] {$x$};
      \node[above] at (0, 0) {$O(0)$};
      \filldraw[color=colorEmphasis] (0, 0) circle (\pointSize) node[below] {$M_0(x)$};
      \node[below, color=colorEmphasis] at (0, -0.6) {$N_0(2x)$};
      \filldraw[color=colorEmphasisCyan] ({e / 2}, 0) circle (\pointSize) node[below] {$M_+\left(x\right)$};
      \filldraw[color=colorEmphasisCyan] (e, 0) circle (\pointSize) node[below] {$N_+\left(2x\right)$};
      \filldraw[color=colorEmphasisGreen] ({-sqrt(3)}, 0) circle (\pointSize) node[below] {$M_-\left(x\right)$};
      \filldraw[color=colorEmphasisGreen] ({-sqrt(3) * 2}, 0) circle (\pointSize) node[below] {$N_-\left(2x\right)$};
   \end{tikzpicture}
   \caption{Ba trường hợp cho vị trí tương đối của $M$, $N$, $O$ cho phần $5$ của bài \ref{ex:0.1}}
   \label{fig:truc phan 5}
\end{figure}

Khoảng cách giữa hai điểm $M$ và $N$ luôn là $$d(M;N)=d(N;M)=|x-2x|=\boxed{|x|}.$$

\subsection{Mặt phẳng hai chiều và hệ tọa độ vuông góc}

\begin{figure}[h]
   \centering
   \begin{minipage}[b]{0.48\textwidth}
      \centering
      \begin{tikzpicture}
         \draw[->] (-2,0) -- (2,0);
         \draw[->] (0,-2) -- (0,2);
         \node[right] at (2,0) {Trục hoành};
         \node[above] at (0,2) {Trục tung};
         \filldraw (0, 0) circle (1.5pt);
         \node[below left] at (0, 0) {$O(0; 0)$};

         \node at (1,1) {$\boxed{\text{I}}$};
         \node at (-1,1) {$\boxed{\text{II}}$};
         \node at (1,-1) {$\boxed{\text{IV}}$};
         \node at (-1,-1) {$\boxed{\text{III}}$};

         \draw (1,0) -- (1,-0.08) node[below] {$x_P$};
         \draw (0,1.5) -- (-0.08,1.5) node[left] {$y_P$};

         \filldraw (1, 1.5) circle (1.5pt);
         \node[above right] at (1, 1.5) {$P(x_P; y_P)$};

         \draw[dashed] (1, 1.5) -- (1, 0);
         \draw[dashed] (1, 1.5) -- (0, 1.5);
      \end{tikzpicture}
      \caption{Hệ tọa độ vuông góc}
      \label{fig:toa do vuong goc}
   \end{minipage}
   \hfill
   \begin{minipage}[b]{0.48\textwidth}
      \centering
      \begin{tikzpicture}
         \draw[->] (-2,0) -- (2,0);
         \draw[->] (0,-2) -- (0,2);
         \node[right] at (2,0) {$x$};
         \node[above] at (0,2) {$y$};
         \filldraw (0, 0) circle (1.5pt);
         \node[below left] at (0, 0) {$O$};

         \pgfmathsetmacro{\xP}{1}
         \pgfmathsetmacro{\yP}{1.5}
         \pgfmathsetmacro{\xQ}{-1.2}
         \pgfmathsetmacro{\yQ}{-1}

         \filldraw (\xP, \yP) circle (1.5pt);
         \node[above right] at (\xP, \yP) {$P(x_P; y_P)$};

         \filldraw (\xQ, \yQ) circle (1.5pt);
         \node[below right] at (\xQ, \yQ) {$Q(x_Q; y_Q)$};

         \draw[thick] (\xP, \yP) -- (\xQ, \yQ);
         \draw[dashed] (\xP, \yP) -- (\xP, \yQ);
         \draw[dashed] (\xP, \yQ) -- (\xQ, \yQ);

         \node[above left] at ({(\xP+\xQ)/2}, {(\yP+\yQ)/2}) {$d(P;Q)$};
      \end{tikzpicture}
      \caption{Khoảng cách giữa hai điểm}
      \label{fig:khoang cach 2d}
   \end{minipage}
\end{figure}


Mở rộng lên mặt phẳng hai chiều, nếu chúng ta đặt hai trục vuông góc với nhau và giao nhau tại gốc $O(0)$ của mỗi trục, khi đó, chúng ta có thể xác định vị trị của điểm trên mặt phẳng chứa hai trục theo biểu diễn đại số bằng cách dóng điểm đó lên trục mà sau này được gọi là \emph{tọa độ}. Đây được gọi là \emph{hệ tọa độ vuông góc} (hay \emph{hệ tọa độ Đề-các}\footnote{René Descartes (1596-1650)}). Như ở hình \ref{fig:toa do vuong goc}, trục nằm ngang được gọi là \emph{trục hoành}, trục dọc được gọi là \emph{trục tung}. Tùy trong từng trường hợp, vị trí và hướng chỉ của các trục có thể thay đổi. Với mỗi điểm, vị trí khi dóng điểm đó vào trục hoành gọi là \emph{hoành độ}, vào trục tung gọi là \emph{tung độ}. Tiếp tục lấy ví dụ từ hình \ref{fig:toa do vuong goc}, điểm $P$ có tọa độ là $(x_P;y_P)$ và được kí hiệu là $P(x_P;y_P)$. Thêm vào đó, hai trục chia mặt phẳng thành bốn góc phần tư, từ góc phần tư thứ I đến góc phần tư thứ IV bao gồm các điểm thỏa mãn tính chất sau:
\begin{itemize}
   \item Góc phần tư thứ I: $x>0$, $y>0$;
   \item Góc phần tư thứ II: $x<0$, $y>0$;
   \item Góc phần tư thứ III: $x<0$, $y<0$;
   \item Góc phần tư thứ IV: $x>0$, $y<0$.
\end{itemize}
Về mặt hình học, khi tọa độ được vẽ thông thường, góc phần tư thứ I nằm ở vị trí trên cùng bên phải, và các góc phần tư còn lại lần lượt được đánh số theo ngược chiều kim đồng hồ. Khi tọa độ bị thay đổi thì vị trí các góc phần tư cũng thay đổi theo, nhưng vẫn thỏa mãn điều kiện đại số ở trên. Các điểm trên trục không xác định thuộc bất cứ góc phần tư nào.

Giống như trên trục một chiều, khi có hai điểm trên mặt phẳng thì chúng ta có thể tính khoảng cách giữa chúng. Một cách chi tiết, cho hai điểm $P(x_P;y_P)$ và $Q(x_Q;y_Q)$, theo định lí Pi-ta-go, khoảng cách giữa hai điểm đó là $$d(P;Q)=\sqrt{(x_P-x_Q)^2+(y_P-y_Q)^2}.$$

\exercise[ex:0.2] Biểu diễn các điểm sau trên hệ tọa độ vuông góc: $A(2;3)$, $B(-1;2)$, $C(-3;0)$, $D(0;4)$, $P(12t;-3t)$, $Q(20t;12t)$ (với $t \in \mathbb{R}$). Xác định góc phần tư hoặc trục tọa độ của mỗi điểm. Sau đó, tính khoảng cách giữa những cặp điểm sau: $A$ và $B$, $C$ và $D$, $P$ và $Q$.

\solution[ex:0.2]

\begin{figure}[h]
   \centering
   \begin{tikzpicture}
      \draw[->] (-4,0) -- (4,0);
      \draw[->] (0,-1) -- (0,5);
      \node[right] at (4,0) {$x$};
      \node[above] at (0,5) {$y$};
      \filldraw (0, 0) circle (\pointSize) node[below right] {$O(0;0)$};
      \filldraw (2, 3) circle (\pointSize) node[above right] {$A(2;3)$};
      \filldraw (-1, 2) circle (\pointSize) node[below] {$B(-1;2)$};
      \filldraw (-3, 0) circle (\pointSize) node[below] {$C(-3;0)$};
      \filldraw (0, 4) circle (\pointSize) node[above right] {$D(0;4)$};
      \draw[thick] (2, 3) -- (-1, 2);
      \draw[thick] (-3, 0) -- (0, 4);

      \node at (1,1) {$\boxed{\text{I}}$};
      \node at (-1,1) {$\boxed{\text{II}}$};
   \end{tikzpicture}
   \caption{Biểu diễn các điểm $A$, $B$, $C$, $D$ trong bài \ref{ex:0.2}}
   \label{fig:toa do vuong goc bai tap}
\end{figure}

Các góc phần tư hay trục số mà các điểm thuộc về có thể được xác định như hình \ref{fig:toa do vuong goc bai tap}. Theo một cách khác, về mặt đại số, có:
\begin{itemize}
   \item $A(2;3)$: $x>0$, $y>0 \implies A$ thuộc góc phần tư thứ I;
   \item $B(-1;2)$: $x<0$, $y>0 \implies B$ thuộc góc phần tư thứ II;
   \item $C(-3;0)$: $x<0$, $y=0 \implies C$ thuộc trục hoành;
   \item $D(0;4)$: $x=0$, $y>0 \implies D$ thuộc trục tung.
\end{itemize}

\begin{figure}[h]
   \centering
   \begin{tikzpicture}
      \draw[->] (-4.5,0) -- (4,0);
      \draw[->] (0,-3.5) -- (0,2);
      \node[right] at (4,0) {$x$};
      \node[above] at (0,2) {$y$};
      \filldraw (0, 0) circle (\pointSize) node[above left] {$O(0;0)$};
      \filldraw (0, 0) circle (\pointSize) node[below right] {$P_{t=0}$};
      \filldraw (0, 0) circle (\pointSize) node[above right] {$Q_{t=0}$};

      \node at (1,1) {$\boxed{\text{I}}$};
      \node at (-1,-1) {$\boxed{\text{III}}$};
      \node at (-1,1) {$\boxed{\text{II}}$};
      \node at (1,-1) {$\boxed{\text{IV}}$};
      \pgfmathsetmacro{\t}{0.1}
      \filldraw ({12*\t}, {-3*\t}) circle (\pointSize) node[below right] {$P_{t_+}$};
      \filldraw ({20*\t}, {12*\t}) circle (\pointSize) node[above right] {$Q_{t_+}$};
      \draw[thick] ({12*\t}, {-3*\t}) -- ({20*\t}, {12*\t});
      \filldraw ({12*(-\t-0.1)}, {-3*(-\t-0.1)}) circle (\pointSize) node[below right] {$P_{t_-}$};
      \filldraw ({20*(-\t-0.1)}, {12*(-\t-0.1)}) circle (\pointSize) node[below left] {$Q_{t_-}$};
      \draw[thick] ({12*(-\t-0.1)}, {-3*(-\t-0.1)}) -- ({20*(-\t-0.1)}, {12*(-\t-0.1)});
   \end{tikzpicture}
   \caption{Biểu diễn các điểm $P$, $Q$ trong bài \ref{ex:0.2}} theo các trường hợp
   \label{fig:toa do vuong goc PQ}
\end{figure}

Để xác định được vị trí của hai điểm $P$ và $Q$, cần phải xét giá trị của $t$. Nếu $t$ dương, thì $P$ và $Q$ sẽ có tọa độ là $P_{t_+}(12t;-3t)$ và $Q_{t_+}(20t;12t)$ với $x_{P_{t_+}}>0$, $y_{P_{t_+}}<0$ và $x_{Q_{t_+}}>0$, $y_{Q_{t_+}}>0$. Khi này, chúng ta có thể kết luận rằng $P$ thuộc góc phần tư thứ IV và $Q$ thuộc góc phần tư thứ I. Ngược lại, nếu $t$ âm, thì $P$ và $Q$ sẽ có tọa độ là $P_{t_-}(12t;-3t)$ và $Q_{t_-}(20t;12t)$ với $x_{P_{t_-}}<0$, $y_{P_{t_-}}>0$ và $x_{Q_{t_-}}<0$, $y_{Q_{t_-}}<0$. Khi này, $P$ thuộc góc phần tư thứ II và $Q$ thuộc góc phần tư thứ III. Cuối cùng, nếu $t=0$, thì cả hai điểm đều có tọa độ là $(0;0)$, tức là chúng trùng với gốc tọa độ.

Khoảng cách giữa những cặp điểm được yêu cầu là:
\begin{itemize}
   \item $d(A;B) = \sqrt{\left(2-(-1)\right)^2+\left(3-2\right)^2} = \sqrt{10} \approx 3{,}1623$;
   \item $d(C;D) = \sqrt{\left(-3-0\right)^2+\left(0-4\right)^2} = 5$;
   \item $d(P;Q) = \sqrt{\left(12t-20t\right)^2+\left(-3t-12t\right)^2} = 13|t|$.
\end{itemize}


\subsection{Không gian ba chiều và hướng tam diện}

\begin{figure}[H]
   \centering
   \begin{minipage}[b]{0.48\textwidth}
      \centering
      \tdplotsetmaincoords{70}{130}
      \begin{tikzpicture}[tdplot_main_coords]
         \coordinate (P) at (1.5,2,1);
         
         \draw[->] (-2.5,0,0) -- (2.5,0,0) node[anchor=north east]{Trục hoành};
         \draw[->] (0,-2.5,0) -- (0,2.5,0) node[anchor=north west]{Trục tung};
         \draw[->] (0,0,-1) -- (0,0,2) node[anchor=south]{Trục cao/Trục đứng/Trục sâu};
         
         \filldraw (0, 0, 0) circle (1.5pt) node[above left] {$O(0;0;0)$};
         \filldraw (P) circle (1.5pt) node[above] {$P$};
         
         \draw[dashed] (P) -- (1.5, 0, 0) node[above left] {$x_P$};
         \draw[dashed] (P) -- (0, 2, 0) node[above] {$y_P$};
         \draw[dashed] (P) -- (0, 0, 1) node[left] {$z_P$};
         \draw[dashed] (P) -- (1.5, 2, 0) -- (0, 0, 0);
         \draw[dashed] (1.5, 2, 0) -- (0, 2, 0);
         \draw[dashed] (1.5, 2, 0) -- (1.5, 0, 0);

         
      \end{tikzpicture}
      \caption{Hệ tọa độ vuông góc ba chiều}
      \label{fig:toa do vuong goc ba chieu}
   \end{minipage}
   \hfill
   \begin{minipage}[b]{0.48\textwidth}
      \centering
      \tdplotsetmaincoords{60}{60}
      \begin{tikzpicture}[tdplot_main_coords]
         \coordinate (P) at (1.5,2,1);
         \coordinate (Q) at (-2,-1,-0.5);
         
         \draw[->] (-2.5,0,0) -- (2.5,0,0) node[anchor=north east]{$x$};
         \draw[->] (0,-2.5,0) -- (0,2.5,0) node[anchor=north west]{$y$};
         \draw[->] (0,0,-1.5) -- (0,0,1.5) node[anchor=south]{$z$};
         
         \filldraw (P) circle (1.5pt) node[above] {$P$};
         \filldraw (Q) circle (1.5pt) node[below] {$Q$};
         \draw[thick] (P) -- (Q);
         \node[above
         ] at (-0.3, 0.5, 0.3) {$d(P;Q)$};

      \end{tikzpicture}
      \caption{Khoảng cách giữa hai điểm trong không gian ba chiều}
      \label{fig:khoang cach ba chieu}
   \end{minipage}
\end{figure}

\ % Lùi đầu dòng

Đương nhiên sẽ có một vài trường hợp mà biểu diễn hai chiều không thể đủ. Khi này, mở rộng hơn nữa, chúng ta cũng có thể làm những điều trên không gian ba chiều tương tự với khi ở trục số một chiều hay mặt phẳng hai chiều. Khi đó, chúng ta sẽ có một hệ tọa độ ba chiều với ba trục vuông góc với nhau, được gọi là \emph{hệ tọa độ vuông góc ba chiều}. Mỗi điểm trong không gian sẽ có tọa độ là $(x;y;z)$ với $x$, $y$, $z$ là các hoành độ, tung độ và cao độ tương ứng. Khoảng cách giữa hai điểm trong không gian ba chiều được tính theo công thức $$d(P;Q)=\sqrt{(x_P-x_Q)^2+(y_P-y_Q)^2+(z_P-z_Q)^2}.$$

\begin{wrapfigure}{R}{0.5\textwidth}
   \centering
   \tdplotsetmaincoords{20}{10}
   \begin{tikzpicture}[tdplot_main_coords]         
      \node at (1.5, 1.5, -1.5) {$\boxed{\text{V}}$};
      \node at (-1.5, 1.5, -1.5) {$\boxed{\text{VI}}$};
      \node at (-1.5, -1.5, -1.5) {$\boxed{\text{VII}}$};
      \node at (1.5, -1.5, -1.5) {$\boxed{\text{VIII}}$};
      \draw[fill=gray!30, opacity=0.4] (-2.5,-2.5,0) -- (2.5,-2.5,0) -- (2.5,2.5,0) -- (-2.5,2.5,0) -- cycle;

      \draw[->] (-2.5,0,0) -- (2.5,0,0) node[anchor=north east]{$x$};
      \draw[->] (0,-2.5,0) -- (0,2.5,0) node[anchor=north west]{$y$};
      \draw[->] (0,0,-2.5) -- (0,0,2.5) node[anchor=south]{$z$};
      \filldraw (0, 0, 0) circle (\pointSize) node[above right] {$O(0;0;0)$};
      \node[fill=white, inner sep=2pt] at (1.5, 1.5, 1.5) {$\boxed{\text{I}}$};
      \node[fill=white, inner sep=2pt] at (-1.5, 1.5, 1.5) {$\boxed{\text{II}}$};
      \node[fill=white, inner sep=2pt] at (-1.5, -1.5, 1.5) {$\boxed{\text{III}}$};
      \node[fill=white, inner sep=2pt] at (1.5, -1.5, 1.5) {$\boxed{\text{IV}}$};
      

   \end{tikzpicture}
   \caption{Góc phần tám không gian}
   \label{fig:goc phan tam khong gian}
\end{wrapfigure}

Và cũng tương tự như với mặt phẳng hai chiều, ba trục sẽ chia không gian thành tám phần, gọi là \emph{góc phần tám không gian}. Các phần này được đánh số từ I đến VIII như sau: Nhìn từ phía dương của trục cao, các góc phần tám được đánh dấu ngược chiều kim đồng hồ. Các góc phần tám I, II, III, IV nằm trên mặt phẳng $Oxy$ và được xác định tương tự như các góc phần tư trong mặt phẳng hai chiều. Các góc phần tám V, VI, VII, VIII nằm dưới mặt phẳng $Oxy$ và được xác định tương tự như trên. Các góc phần tám này được biểu diễn trong hình \ref{fig:goc phan tam khong gian}. Về mặt đại số, 

\begin{itemize}
   \item Góc phần tám I: $x>0$, $y>0$, $z>0$;
   \item Góc phần tám II: $x<0$, $y>0$, $z>0$;
   \item Góc phần tám III: $x<0$, $y<0$, $z>0$;
   \item Góc phần tám IV: $x>0$, $y<0$, $z>0$;
   \item Góc phần tám V: $x>0$, $y>0$, $z<0$;
   \item Góc phần tám VI: $x<0$, $y>0$, $z<0$;
   \item Góc phần tám VII: $x<0$, $y<0$, $z<0$;
   \item Góc phần tám VIII: $x>0$, $y<0$, $z<0$.
\end{itemize}

\begin{figure}[h]
   \centering
   \tdplotsetmaincoords{20}{10}
   \begin{minipage}[b]{0.48\textwidth}
      \centering
      \begin{tikzpicture}[tdplot_main_coords]
         \draw[->] (-2.5,0,0) -- (2.5,0,0) node[anchor=north east]{$x$};
         \draw[->] (0,-2.5,0) -- (0,2.5,0) node[anchor=north west]{$y$};
         \draw[->] (0,0,-2.5) -- (0,0,2.5) node[anchor=south]{$z$};
         
         \draw[thick,->] (1.5,0,0) arc (0:90:1.5);
      \end{tikzpicture}
      \caption{Tam diện thuận}
      \label{fig:tam dien thuan}
   \end{minipage}
   \hfill
   \begin{minipage}[b]{0.48\textwidth}
      \centering
      \begin{tikzpicture}[tdplot_main_coords]
         \draw[->] (-2.5,0,0) -- (2.5,0,0) node[anchor=north east]{$y$};
         \draw[->] (0,-2.5,0) -- (0,2.5,0) node[anchor=north west]{$x$};
         \draw[->] (0,0,-2.5) -- (0,0,2.5) node[anchor=south]{$z$};
         
         \draw[thick,->] (0,1.5,0) arc (90:0:1.5);
      \end{tikzpicture}
      \caption{Tam diện nghịch}
      \label{fig:tam dien nghich}
   \end{minipage}
\end{figure}

Trên hệ tọa độ không gian, chúng ta cần phải quan tâm thêm xem là ba trục tạo thành \emph{hướng tam diện} nào. Ta nhìn từ phía dương của trục cao, khi này, nếu trục hoành xoay sang trục tung theo hướng ngược chiều kim đồng hồ, thì hướng tam diện được gọi là \emph{hướng tam diện thuận}. Ngược lại, nếu trục hoành xoay sang trục tung theo hướng cùng chiều kim đồng hồ, thì hướng tam diện được gọi là \emph{hướng tam diện nghịch}. Một cách khác là dùng quy tắc bàn tay phải: nắm tay phải vào trục cao, khi này, ngón tay cái chỉ hướng của trục cao. Nếu hướng nắm ngón tay theo hương quay từ trục hoành sang trục tung, thì hướng tam diện là thuận. Ngược lại, nếu hướng nắm ngón tay theo hướng quay từ trục tung sang trục hoành, thì hướng tam diện là nghịch.

Chúng ta đã có phân bổ vị trí của các góc phần tám trong hệ tọa độ tam diện thuận. Lặp lại lập luận với cùng biểu thức đại số, chúng ta có thể phân bổ vị trí của các góc phần tám trong hệ tọa độ tam diện nghịch. Thông thường, hệ tọa độ tam diện thuận được ưa dùng hơn.

\exercise Trung điểm của một đoạn thẳng $AB$ là điểm $M$ trong không gian khi và chỉ khi $M$ thỏa mãn $d(A;M) = d(B;M) = \frac{d(A;B)}{2}$. Chứng minh rằng với tọa độ của $M$ là $$M\left(\frac{x_A+x_B}{2}; \frac{y_A+y_B}{2}; \frac{z_A+z_B}{2}\right)$$ thì $M$ là trung điểm của đoạn thẳng nối hai điểm $A(x_A; y_A; z_A)$ và $B(x_B; y_B; z_B)$. Vẽ ví dụ với $A(1;2;3)$ và $B(-1;0;4)$.

\solution

Áp dụng công thức khoảng cách để tính khoảng cách giữa hai điểm $A$ và $M$, có:

\begin{align*}
   d(A;M) &= \sqrt{\left(x_A - \frac{x_A+x_B}{2}\right)^2 + \left(y_A - \frac{y_A+y_B}{2}\right)^2 + \left(z_A - \frac{z_A+z_B}{2}\right)^2} \\
   &= \sqrt{\left(\frac{x_A-x_B}{2}\right)^2 + \left(\frac{y_A-y_B}{2}\right)^2 + \left(\frac{z_A-z_B}{2}\right)^2} \\
   &= \frac{1}{2} \sqrt{(x_A-x_B)^2 + (y_A-y_B)^2 + (z_A-z_B)^2} = \frac{d(A;B)}{2}.
\end{align*}

Một cách tương tự, chúng ta cúng có $d(B;M) = \frac{d(A;B)}{2}$. Như vậy, $M$ là trung điểm của đoạn thẳng nối hai điểm $A$ và $B$.

Vẽ đồ thị ví dụ với $A(1;2;3)$ và $B(-1;0;4)$, chúng ta được đồ thị ở hình \ref{fig:trung diem}.

Công thức về vị trí tọa độ trung điểm được cho trong bài là công thức đơn giản và hữu dụng. Bạn đọc nên học thuộc công thức này.

\begin{figure}[H]
   \centering
   \tdplotsetmaincoords{80}{80}
   \begin{tikzpicture}[tdplot_main_coords]
      \draw[->] (-2,0,0) -- (2,0,0) node[anchor=north west]{$y$};
      \draw[->] (0,-1,0) -- (0,3,0) node[anchor=north east]{$x$};
      \draw[->] (0,0,-1) -- (0,0,5) node[anchor=south]{$z$};
      
      \filldraw (1,2,3) circle (\pointSize) node[anchor=west] {$A(1;2;3)$};
      \filldraw (-1,0,4) circle (\pointSize) node[anchor=east] {$B(-1;0;4)$};
      \filldraw (0,1,3.5) circle (\pointSize) node[anchor=south west] {$M(0;1;\frac{7}{2})$};
      \draw[thick] (1,2,3) -- (0,1,3.5) -- (-1,0,4);
   \end{tikzpicture}
   \caption{Ví dụ trung điểm với $A(1;2;3)$ và $B(-1;0;4)$}
   \label{fig:trung diem}
\end{figure}



\section{Hàm số}

\subsection{Định nghĩa hàm số, phương trình, bất phương trình và hệ}

\ % Lùi đầu dòng

Chúng ta gọi $f$ là một \defText{hàm số} (hay \defText{hàm}) đi từ tập $X$ đến tập $Y$ khi và chỉ khi với mọi $x\in X$, gọi là \defText{tập xác định}, thông qua mối liên hệ $f$ có một và chỉ một $y\in Y$ tương ứng với $x$. Câu vừa rồi có thể được tóm gọn trong một vài kí hiệu: $$\begin{aligned}\defMath{f: X} &\defMath{\to Y} \\ \defMath{x} &\defMath{\mapsto y}\end{aligned}.$$ Khi này, chúng ta có thể viết hàm số này dưới dạng biểu thức giải tích $\defMath{y=f(x)}$, gọi $y$ là hàm của $x$. Ngoài ra, cần phải để ý rằng, thông qua định nghĩa này, mặc dù mọi $x$ trong $X$ phải có đầu ra trong $Y$, không phải mọi $y$ trong $Y$ đều phải có đầu vào trong $X$. Nói cách khác, tập tất cả các giá trị đầu ra có thể của $y=f(x)$, gọi là \defText{tập giá trị}, là tập con của tập $Y$. Nếu $x$ nằm ngoài tập giá trị $x$ thì $f(x)$ là \defText{không xác định} và không nhận bất cứ giá trị nào.

Khi chúng ta có định nghĩa hàm số thì chúng ta cũng sẽ có những khái niệm liên quan. Khi $f$ là một hàm số, thì bất cứ giá trị $a$ thuộc tập xác định để $f(a) = 0$ đều được gọi là \defText{nghiệm} của $f$. Mở rộng ra, với $f$ và $g$ là hai hàm số, bất cứ giá trị $a$ thỏa mãn $f(a) = g(a)$ thì $a$ được gọi là nghiệm của \defText{phương trình} $f(x) = g(x)$. Hơn thế nữa, nếu thay dấu $=$ trong câu vừa trước bởi các dấu $<$, $>$, $\leq$\footnote{Còn những kí hiệu khác cho dấu nhỏ hơn hoặc bằng là $\leqq$, $\leqslant$.}, $\geq$\footnote{Còn những kí hiệu khác cho dấu lơn hơn hoặc bằng là $\geqq$, $\geqslant$.}, $\neq$ thì chúng ta có định nghĩa cho nghiệm của \defText{bất phương trình}\footnote{Ngoài những dấu biểu diễn bất phương trình được kể, còn những dấu như $\nless$ (không nhỏ hơn), $\ngtr$ (không lớn hơn), $\nleq$, $\not \leqq$ hay $\nleqslant$ (không nhỏ hơn hoặc bằng), $\ngeq$, $\not \geqq$ hay $\ngeqslant$ (không lớn hơn hoặc bằng), và những dấu bị nguyền rủa $\lessgtr$ (nhỏ hơn hoặc lớn hơn), $\lesseqgtr$ hay $\lesseqqgtr$ (nhỏ hơn, lớn hơn hoặc bằng). Bạn đọc có thể sẽ muốn thêm các dấu $\not \lessgtr$ (không nhỏ hơn hay lớn hơn) và cặp dấu $\not \lesseqgtr$, $\not \lesseqqgtr$ (không nhỏ hơn, lớn hơn hay bằng) làm dấu cho bất phương trình. Tuy nhiên, trên tập số thực, $\not \lesseqgtr$ tương đương với dấu $=$, và bất phương trình với $\not \lesseqgtr$, $\not \lesseqqgtr$ thì không bao giờ thỏa mãn. Về mặt ứng dụng, ngoài những môn nặng về nền tảng của toán như đại số cao cấp, những dấu kể trên gần như không bao giờ được sử dụng.}. Lấy ví dụ, với $f$ và $g$ là hai hàm số, giá trị $a$ để $f(a) \neq g(a)$ thì $a$ được gọi là nghiệm của bất phương trình $f(x) \neq g(x)$. Kết hợp nhiều phương trình hay bất phương trình, chúng ta có một \defText{hệ}. Ví dụ:
$$
\begin{cases}
   f(x) = g(x) \\
   \alpha(y) \neq \beta(z)
\end{cases}.
$$
Để thỏa mãn hệ thì mỗi thành phần trong hệ đều phải thỏa mãn. Một khái niệm liên quan mật thiết là \defText{giải phương trình, bất phương trình, hay hệ} (để ngắn gọn, chúng ta sẽ gọi phương trình, bất phương trình và hệ thành một cụm từ chung là \dblquote{phương bất hệ}). Để làm được việc này, yêu cầu cần tìm tất cả các bộ số đẻ phương bất hệ được cho thỏa mãn. Trong trường hợp phương trình luôn đúng với mọi giá trị trong tập xác định, thì phương trình này được gọi là \defText{đẳng thức}. Một cách tương đương, nếu như bất phương trình đúng với tất cả các giá trị có thể của đầu vào thì được gọi là \defText{bất đẳng thức}.

Nếu chỉ có số với chữ không thì hàm số sẽ trở nên rất nhàm chán, cho nên người ta đã nghĩ ra phương pháp biểu diễn hàm số qua đồ thị. Để biểu diễn một hàm số $y=f(x)$ với $x$ và $y$ là hai số thực, cần vẽ tất cả các cặp tọa độ $(x; y)$ thỏa mãn hàm $f$ trên đồ thị. Trong trường hợp hàm có vô số điểm, chúng ta lấy một số giá trị để định hướng hình dạng của đồ thị và rồi sau đó nối các điểm lại\footnote{Mặc dù vậy, vẫn có trường hợp mà cách vẽ này hoàn toàn bất lực. Ví dụ như hàm Đi-rích-lê: $$f(x) =
\begin{cases}
   1 \text{ nếu } x\in \mathbb{Q} \\
   0 \text{ nếu } x\notin \mathbb{Q}
\end{cases}$$ với $\mathbb{Q}$ là tập số hữu tỉ. Hàm này liên tục nhảy bật từ $0$ đến $1$ và ngược lại, khiến cho việc vẽ đồ thị trở nên bất khả thi.}. Do hàm số biểu thị mối liên hệ giữa hai đại lượng, chúng ta dùng đồ thị hai chiều để biểu diễn mối liên hệ giữa chúng. Chúng ta sẽ lấy ví dụ cho hàm sau được cho trong bảng \ref{tab:ham_so_mot_bien:dinh_nghia:vddths} với tập xác định chỉ có $5$ số.

\begin{table}[H]
   \centering
   \begin{tabular}{|c|c|c|c|c|c|}
      \hline
      $x$ & $1$ & $2$ & $3$ & $4$ & $5$ \\
      \hline
      $y=f(x)$ & $1$ & $2$ & $5$ & $2$ & $3$ \\
      \hline
   \end{tabular}
   \caption{Ví dụ của $y = f(x)$}
   \label{tab:ham_so_mot_bien:dinh_nghia:vddths}
\end{table}

\noindent Chúng ta nhìn thấy rằng có $5$ bộ số $(x;y)$ là $(1;2)$, $(2;3)$, $(3;4)$, $(4;5)$, $(5;6)$ thỏa mãn hàm $f$ (theo đúng định nghĩa của hàm). Do đó, chúng ta có đồ thị như hình \ref{fig:ham_so_mot_bien:dinh_nghia:vddths}. 

\begin{figure}[H]
   \centering
   \begin{tikzpicture}
      \draw[->] (-1, 0) -- (6, 0) node[right] {$x$};
      \draw[->] (0, -1) -- (0, 6) node[above] {$y$};
      \foreach \x/\y in {1/1, 2/2, 3/5, 4/2, 5/3} {
         \filldraw (\x, \y) circle (\pointSize) node[below] {$(\x; \y)$};
      }
   \end{tikzpicture}
   \caption{Đồ thị cho ví dụ của $y = f(x)$ được cho ở bảng \ref{tab:ham_so_mot_bien:dinh_nghia:vddths}}
   \label{fig:ham_so_mot_bien:dinh_nghia:vddths}
\end{figure}

Một cách tương tự, chúng ta cũng có thể biểu diễn phương bất hệ thông qua việc vẽ đồ thị chứa các nghiệm của phương bất hệ đó. Có bao nhiêu ẩn số trong phương bất hệ, đồ thị sẽ có bấy nhiêu chiều. Giả sử như bạn đọc cần biểu diễn phương trình $x^2 - 1 = 0$ với $x$ xác định trên tập số thực. Để biểu diễn được phương trình này, trước hết cần phải thực hiện giải nó. Tác giả kì vọng bạn đọc có thể thực hiện được những biến đổi sau:
\begin{align*}
   x^2 - 1 &= 0 \\
   \iff x^2 &= 1 \\
   \iff x &\in \{-1; 1\}.
\end{align*}
Do phương trình chỉ có một ẩn nên chúng ta sẽ chọn trục số một chiều biểu diễn $x$ để thể hiện nghiệm của phương trình này, như hình \ref{fig:ham_so_mot_bien:dinh_nghia:vdgpt}.

\begin{figure}[h]
   \centering
   \begin{tikzpicture}
      \draw[->] (-3, 0) -- (3, 0) node[right] {$x$};
      \foreach \x in {-1, 1} {
         \filldraw[color=colorEmphasisCyan] (\x, 0) circle (\pointSize) node[below] {$(\x)$};
      }
   \end{tikzpicture}
   \caption{Biểu diễn nghiệm của $x^2 - 1 = 0$}
   \label{fig:ham_so_mot_bien:dinh_nghia:vdgpt}
\end{figure}

Về mặt lợi ích của việc sử dụng đồ thị, biểu diễn hình học các đại lượng đại số là một trong những cách hữu hiệu để mở rộng cảm nhận về đối tượng đang nghiên cứu.
      
\exercise Mỗi phần trong bài tập sau bao gồm mối liên hệ giữa $x$ và $y$. Trong mỗi phần, $y$ có phải là hàm của $x$ hay không? Trong trường hợp $y$ là hàm số của $x$, xác định tập xác định và tập giá trị của hàm số đó. Còn trong trường hợp ngược lại, giải thích tại sao $y$ lại không phải là hàm số của $x$.

\setcounter{subexercise}{1}
\arabic{subexercise}.
\begin{tabular}{|c|c|c|c|c|c|}
   \hline
   $x$ & $1$ & $2$ & $3$ & $4$ & $5$ \\
   \hline
   $y$ & $2$ & $3$ & $4$ & $5$ & $6$ \\
   \hline
\end{tabular};

2.
\begin{tabular}{|c|c|c|c|c|c|}
   \hline
   $x$ & $0$ & $-1$ & $1$ & $2$ & $-3$ \\
   \hline
   $y$ & $0$ & $0$ & $0$ & $0$ & $0$ \\
   \hline
\end{tabular};

3.
\begin{tabular}{|c|c|c|c|c|c|}
   \hline
   $x$ & $15$ & $15$ & $16$ & $16$ & $17$ \\
   \hline
   $y$ & $123$ & $134$ & $578$ & $426$ & $348$ \\
   \hline
\end{tabular};

4.
\begin{tabular}{|c|c|c|c|c|c|}
   \hline
   $x$ & $0$ & $-17$ & $3$ & $55$ & $-17$ \\
   \hline
   $y$ & $4586$ & $1024$ & $4586$ & $4586$ & $1024$ \\
   \hline
\end{tabular};

5. $x$ là số chỉ tháng và $y$ là số ngày trong tháng $x$.

\solution

 $y$ là hàm của $x$ với tập xác định $X = \{1; 2; 3; 4; 5\}$ và tập giá trị $Y = \{2; 3; 4; 5; 6\}$.

2. $y$ là hàm của $x$ với tập xác định $X = \{0; -1; 1; 2; -3\}$ và tập giá trị $Y = \{0\}$.

3. $y$ không phải là hàm của $x$ do khi $x$ có giá trị $15$ thì $y$ có hai giá trị $123$ và $134$.

4. $y$ là hàm của $x$ với tập xác định $X = \{0; -17; 3; 55\}$ và tập giá trị $Y = \{4586; 1024\}$. Lưu ý rằng bảng có cột bị lặp.

5. $y$ không là hàm của $x$ do khi $x = 2$ thì $y$ có hai giá trị $28$ và $29$. Mặc dù cách viết có thể ám chỉ $y=f(x)$ với $f$ là hàm số đại diện cho số ngày trong tháng, nhưng $f$ không phải là hàm số do điều ngoại lệ.

\exercise[ex:ham_so_mot_bien:dinh_nghia:intropt] Vẽ đồ thị của phương trình $\mathcal{P}$, với các định nghĩa được cho. Hàm có tập xác định là bộ số đầu vào cho ở trong bảng. Để ý số ẩn của phương trình để chọn số chiều của đồ thị cho phù hợp.

\begin{enumerate}
   \item
   \begin{tabular}{|c|c|c|c|c|c|c|}
      \hline
      $x$ & $-1$ & $1$ & $-2$ & $2$ & $-3$ & $3$\\
      \hline
      $f(x)$ & $0$ & $0$ & $4$ & $3$ & $7$ & $0$\\
      \hline
   \end{tabular} và $\mathcal{P}: f(x) = 0$;

   \item
   \begin{tabular}{|c|c|c|c|c|c|c|}
      \hline
      $x$ & $-1$ & $1$ & $-2$ & $2$ & $-3$ & $3$\\
      \hline
      $f(x)$ & $0$ & $0$ & $4$ & $3$ & $7$ & $0$\\
      \hline
   \end{tabular} và $\mathcal{P}: f(x) = x^2 - 1$;

   \item
   \begin{tabular}{|c|c|c|c|c|c|c|}
      \hline
      $x$ & $1$ & $2$ & $3$ & $4$ & $5$ & $6$\\
      \hline
      $f(x)$ & $2$ & $3$ & $5$ & $7$ & $11$ & $13$\\
      \hline
      $g(x)$ & $1$ & $3$ & $5$ & $7$ & $9$ & $11$\\
      \hline
   \end{tabular} và $\mathcal{P}: f(x) = g(x)$;

   \item
   \begin{tabular}{|c|c|c|c|c|c|c|}
      \hline
      $x$ & $1$ & $2$ & $3$ & $4$ & $5$ & $6$\\
      \hline
      $f(x)$ & $2$ & $3$ & $5$ & $7$ & $11$ & $13$\\
      \hline
      $g(x)$ & $1$ & $3$ & $5$ & $7$ & $9$ & $11$\\
      \hline
   \end{tabular} và $\mathcal{P}: f(x) = g(y)$;

   \item
   \begin{tabular}{|c|c|c|c|c|c|c|}
      \hline
      $x$ & $1$ & $2$ & $3$ & $4$ & $5$ & $6$\\
      \hline
      $f(x)$ & $2$ & $3$ & $5$ & $7$ & $11$ & $13$\\
      \hline
   \end{tabular} và $\mathcal{P}: f(x) = 2b-1$ với $b \in \mathbb{R}$;

   \item
   \begin{tabular}{|c|c|c|c|c|c|c|}
      \hline
      $x$ & $-1$ & $1$ & $-2$ & $2$ & $-3$ & $3$\\
      \hline
      $f(x)$ & $0$ & $0$ & $4$ & $3$ & $7$ & $0$\\
      \hline
   \end{tabular} và $\mathcal{P}: f(x) = f(2b - 1)$;

   \item 
   \begin{tabular}{|c|c|c|c|c|c|c|}
      \hline
      $x$ & $-1$ & $1$ & $-2$ & $2$ & $-3$ & $3$\\
      \hline
      $f(x)$ & $0$ & $0$ & $4$ & $3$ & $7$ & $0$\\
      \hline
   \end{tabular} và $\mathcal{P}: f(a) + f(b) = f(c)$;
\end{enumerate}

\solution[ex:ham_so_mot_bien:dinh_nghia:intropt]

{
   \begin{minipageindent}{0.48\textwidth}
      \indent  $\mathcal{P}$ là phương trình chỉ có một ẩn $x$, do đó đồ thị của $\mathcal{P}$ chỉ là đồ thị một chiều trên một trục số biểu diễn cho $x$.

      Có ba giá trị để $f(x)$ bằng $0$: $x\in \{-1; 1; -3\}$. Chúng ta có đồ thị của $\mathcal{P}$ ở hình \ref{fig:ham_so_mot_bien:dinh_nghia:dtp1}.
   \end{minipageindent}
   \hfill
   \begin{minipageindent}{0.48\textwidth}
      \begin{figure}[H]
         \centering
         \begin{tikzpicture}
            \draw[->] (-2, 0) -- (4, 0) node[right] {$x$};
            \foreach \x in {1, -1, 3} {
               \filldraw[color=colorEmphasisCyan] (\x, 0) circle (\pointSize) node[below] {$(\x)$};
            }
         \end{tikzpicture}
         \caption{Đồ thị phần 1 bài \ref{ex:ham_so_mot_bien:dinh_nghia:intropt}}
         \label{fig:ham_so_mot_bien:dinh_nghia:dtp1}
      \end{figure}
   \end{minipageindent}
}


{
   \begin{minipageindent}{0.48\textwidth}
      2. Tập xác định của $f(x)$ là $\{-1; 1; -2; 2; -3; 3\}$, do đó, để $\mathcal{P}$ thỏa mãn thì $x$ chỉ có thể nhận các giá trị trong vùng tập xác định.

      Kẻ bảng so sánh:
      \begin{table}[H]
         \centering
         \begin{tabular}{|c|c|c|c|c|c|c|}
            \hline
            $x$ & $-1$ & $1$ & $-2$ & $2$ & $-3$ & $3$\\
            \hline
            $f(x)$ & $0$ & $0$ & $4$ & $3$ & $7$ & $0$\\
            \hline
            $x^2-1$ & $0$ & $0$ & $3$ & $3$ & $7$ & $7$\\
            \hline
         \end{tabular}
         \caption{Giá trị của $f(x)$ và $x^2-1$ ứng với $x$}
         \label{tab:ham_so_mot_bien:dinh_nghia:values3}
      \end{table}

      Nhận thấy rằng $\mathcal{P}$ chỉ đúng khi $x\in \{-1; 1; -3; 2\}$ và chúng ta có đồ thị là hình \ref{fig:ham_so_mot_bien:dinh_nghia:dtp2}.
   \end{minipageindent}
   \hfill
   \begin{minipageindent}{0.48\textwidth}
      \begin{figure}[H]
         \centering
         \begin{tikzpicture}
            \draw[->] (-3.5, 0) -- (2.5, 0) node[right] {$x$};
            \foreach \x in {1, -1, -3, 2} {
               \filldraw[color=colorEmphasisCyan] (\x, 0) circle (\pointSize) node[below] {$(\x)$};
            }
         \end{tikzpicture}
         \caption{Đồ thị phần 2 bài \ref{ex:ham_so_mot_bien:dinh_nghia:intropt}}
         \label{fig:ham_so_mot_bien:dinh_nghia:dtp2}
      \end{figure}
   \end{minipageindent}
}

{
   \begin{minipageindent}{0.48\textwidth}
      3. Nhìn vào bảng được cho, có $f(x) = g(x)$ khi và chỉ khi $x\in \{2; 3; 4\}$. Do đó, đồ thị của $\mathcal{P}$ có được như hình \ref{fig:ham_so_mot_bien:dinh_nghia:dtp3}.
   \end{minipageindent}
   \hfill
   \begin{minipageindent}{0.48\textwidth}
      \begin{figure}[H]
         \centering
         \begin{tikzpicture}
            \draw[->] (0, 0) -- (5, 0) node[right] {$x$};
            \foreach \x in {2, 3, 4} {
               \filldraw[color=colorEmphasisCyan] (\x, 0) circle (\pointSize) node[below] {$(\x)$};
            }
         \end{tikzpicture}
         \caption{Đồ thị phần 3 bài \ref{ex:ham_so_mot_bien:dinh_nghia:intropt}}
         \label{fig:ham_so_mot_bien:dinh_nghia:dtp3}
      \end{figure}
   \end{minipageindent}
}

{
   \begin{minipageindent}{0.48\textwidth}
      4. $\mathcal{P}$ là phương trình có hai ẩn $x$ và $y$, do đó đồ thị của $\mathcal{P}$ là một mặt phẳng hai chiều. Coi như trục hoành biểu diễn cho $x$ và trục tung biểu diễn cho $y$. 

      Để có thể vẽ được đồ thị của $\mathcal{P}$, hiển nhiên nhìn ra được rằng cần phải có những điểm $(x;y)$ để hai giá trị $f(x)$ và $g(y)$ bằng nhau. Và để làm được điều đó, trước hết, chúng ta sẽ tìm xem giá trị bằng nhau của $f(x)$ với $g(y)$ này bằng bao nhiêu. Gọi chung giá trị bằng nhau này là $B_n$. Kẻ lại bảng so sánh thành bảng \ref{tab:ham_so_mot_bien:dinh_nghia:bn_values}, với $B_n$ là giá trị đầu ra và $x$, $y$ là giá trị lần lượt đưa vào hai hàm $f$ và $g$ để có giá trị đầu ra đó. Và từ đó, chúng ta có đồ thị của $\mathcal{P}$ là hình \ref{fig:ham_so_mot_bien:dinh_nghia:dtp4}.
   \end{minipageindent}
   \hfill
   \begin{minipageindent}{0.48\textwidth}
      \begin{table}[H]
         \centering
         \begin{tabular}{|c|c|c|c|c|}
            \hline
            $B_n$ & $3$ & $5$ & $7$ & $11$ \\
            \hline
            $x$ & $2$ & $3$ & $4$ & $5$ \\
            \hline
            $y$ & $2$ & $3$ & $4$ & $6$ \\
            \hline 
         \end{tabular}
         \caption{Giá trị của $x$ và $y$ ứng với $B_n$}
         \label{tab:ham_so_mot_bien:dinh_nghia:bn_values}
      \end{table}
   \end{minipageindent}
}

\begin{figure}[H]
   \centering
   \begin{tikzpicture}
      \draw[->] (0, 0) -- (3.5, 0) node[right] {$x$};
      \draw[->] (0, 0) -- (0, 3.5) node[above] {$y$};
      \filldraw[color=colorEmphasisCyan] (1, 1) circle (\pointSize) node[right] {$(2; 2)$};
      \filldraw[color=colorEmphasisCyan] (1.5, 1.5) circle (\pointSize) node[right] {$(3; 3)$};
      \filldraw[color=colorEmphasisCyan] (2.5, 3) circle (\pointSize) node[right] {$(5; 6)$};
      \filldraw[color=colorEmphasisCyan] (2, 2) circle (\pointSize) node[right] {$(4; 4)$};
   \end{tikzpicture}
   \caption{Đồ thị phần 4 bài \ref{ex:ham_so_mot_bien:dinh_nghia:intropt}}
   \label{fig:ham_so_mot_bien:dinh_nghia:dtp4}
\end{figure}

{
   \begin{minipageindent}{0.48\textwidth}
      5. $\mathcal{P}$ là phương trình có hai ẩn $x$ và $b$, do đó đồ thị của $\mathcal{P}$ là một mặt phẳng hai chiều. Coi như trục hoành biểu diễn cho $x$ và trục tung biểu diễn cho $b$.

      Tính giá trị của $b$ từ $f(x)$:
      \begin{align*}
         f(x) &= 2b - 1 \\
         \iff b &= \frac{f(x) + 1}{2}.
      \end{align*}
   \end{minipageindent}
   \hfill
   \begin{minipageindent}{0.48\textwidth}
      \begin{table}[H]
         \centering
         \begin{tabular}{|c|c|c|c|c|c|c|}
            \hline
            $x$ & $1$ & $2$ & $3$ & $4$ & $5$ & $6$\\
            \hline
            $f(x)$ & $2$ & $3$ & $5$ & $7$ & $11$ & $13$\\
            \hline
            $b$ & $\frac{3}{2}$ & $\frac{5}{2}$ & $3$ & $4$ & $6$ & $7$\\
            \hline
         \end{tabular}
         \caption{Giá trị của $b$ ứng với $x$}
         \label{tab:ham_so_mot_bien:dinh_nghia:b_values6}
      \end{table}
   \end{minipageindent}
}
Từ đây, chúng ta có thể thêm giá trị của $b$ vào bảng được cho thành bảng \ref{tab:ham_so_mot_bien:dinh_nghia:b_values6}. 

Qua bảng đó, vẽ được đồ thị của $\mathcal{P}$ như hình \ref{fig:ham_so_mot_bien:dinh_nghia:dtp5}.


\begin{figure}[H]
   \centering
   \begin{tikzpicture}
      \draw[->] (0, 0) -- (4, 0) node[right] {$x$};
      \draw[->] (0, 0) -- (0, 4) node[above] {$b$};
      \filldraw[color=colorEmphasisCyan] (0.5, 0.75) circle (\pointSize) node[below] {$\left(1; \frac{3}{2}\right)$};
      \filldraw[color=colorEmphasisCyan] (1, 1.25) circle (\pointSize) node[below] {$\left(2; \frac{5}{2}\right)$};
      \filldraw[color=colorEmphasisCyan] (1.5, 1.5) circle (\pointSize) node[right] {$(3; 3)$};
      \filldraw[color=colorEmphasisCyan] (2, 2) circle (\pointSize) node[right] {$(4; 4)$};
      \filldraw[color=colorEmphasisCyan] (2.5, 3) circle (\pointSize) node[right] {$(5; 6)$};
      \filldraw[color=colorEmphasisCyan] (3, 3.5) circle (\pointSize) node[right] {$(6; 7)$};
   \end{tikzpicture}
   \caption{Đồ thị phần 5 bài \ref{ex:ham_so_mot_bien:dinh_nghia:intropt}}
   \label{fig:ham_so_mot_bien:dinh_nghia:dtp5}
\end{figure}

{
   \begin{minipageindent}{0.48\textwidth}
      6. Nhìn vào bảng định nghĩa được cho, $f(x)$ có thể nhận các giá trị là $\{0; 3; 4; 7\}$.

      \textcolor{colorEmphasis}{Trường hợp một}: Khi $f(x) \neq 0$, chỉ có một giá trị đầu vào cho $f$ sao cho $f(x)$ đạt được giá trị đầu ra. Ví dụ, chỉ có đầu vào $x = 2$ mới có $f(x) = 3$. Do đó, khi $f(x) \neq 0$, $x = 2b-1$. Biến đổi đại số cơ bản để có $b = \frac{x + 1}{2}$. Lập bảng \ref{tab:ham_so_mot_bien:dinh_nghia:b_values7} để thấy được mối quan hệ giữa $x$ và $b$.

   \end{minipageindent}
   \hfill
   \begin{minipageindent}{0.48\textwidth}
      \begin{table}[H]
         \centering
         \begin{tabular}{|c|c|c|c|}
            \hline
            $x$ & $-2$ & $2$ & $-3$\\
            \hline
            $b = \frac{x+1}{2}$ & $-\frac{3}{2}$ & $2$ & $-1$\\
            \hline
         \end{tabular}
         \caption{Giá trị của cặp $(x; b)$ với $f(x) \neq 0$}
         \label{tab:ham_so_mot_bien:dinh_nghia:b_values7}
      \end{table}
   \end{minipageindent}
}

\textcolor{colorEmphasisCyan}{Trường hợp hai}: Khi $f(x) = 0$, $x$ và $2b-1$ có thể nhận bất cứ giá trị nào trong tập $\{-1; 1; -3\}$. Từ đó, có thể chọn $x \in \{-1; 1; 3\}$ và giải đại số để chọn $b \in \left\{\frac{-1+1}{2}; \frac{1+1}{2}; \frac{3+1}{2}\right\} = \{0; 1; 2\}$. Các cặp $(x; b)$ thỏa mãn là $(x; b)$ $\in$ $\{\left(-1; 0\right); \left(-1; 1\right); \left(-1; 2\right); \left(1; 0\right); \left(1; 1\right); \left(1; 2\right); \left(-3; 0\right); \left(-3; 1\right); \left(-3; 2\right)\}$.

Cuối cùng, kết hợp hai trường hợp, chúng ta có đồ thị cho $\mathcal{P}$:
\begin{figure}[H]
   \centering
   \begin{tikzpicture}
      \draw[->] (-4, 0) -- (3, 0) node[right] {$x$};
      \draw[->] (0, -2.5) -- (0, 2.5) node[above] {$b$};
      
      % Vẽ phần f(x) khác 0
      \filldraw[color=colorEmphasis] (-2, -1.5) circle (\pointSize) node[below] {$\left(-2; -\frac{3}{2}\right)$};
      \filldraw[color=colorEmphasis] (2, 2) circle (\pointSize) node[below] {$\left(2; 2\right)$};
      \filldraw[color=colorEmphasis] (-3, -1) circle (\pointSize) node[below] {$\left(-3; -1\right)$};
      
      \filldraw[color=colorEmphasisCyan] (-1, 0) circle (\pointSize) node[below] {$\left(-1; 0\right)$};
      \filldraw[color=colorEmphasisCyan] (1, 0) circle (\pointSize) node[below] {$\left(1; 0\right)$};
      \filldraw[color=colorEmphasisCyan] (-3, 0) circle (\pointSize) node[below] {$\left(-3; 0\right)$};
      
      \filldraw[color=colorEmphasisCyan] (-1, 1) circle (\pointSize) node[below] {$\left(-1; 1\right)$};
      \filldraw[color=colorEmphasisCyan] (1, 1) circle (\pointSize) node[below] {$\left(1; 1\right)$};
      \filldraw[color=colorEmphasisCyan] (-3, 1) circle (\pointSize) node[below] {$\left(-3; 1\right)$};
      
      \filldraw[color=colorEmphasisCyan] (-1, 2) circle (\pointSize) node[below] {$\left(-1; 2\right)$};
      \filldraw[color=colorEmphasisCyan] (1, 2) circle (\pointSize) node[below] {$\left(1; 2\right)$};
      \filldraw[color=colorEmphasisCyan] (-3, 2) circle (\pointSize) node[below] {$\left(-3; 2\right)$};
   \end{tikzpicture}
   \caption{Đồ thị phần 6 bài \ref{ex:ham_so_mot_bien:dinh_nghia:intropt}}
   \label{fig:ham_so_mot_bien:dinh_nghia:dtp7}
\end{figure}

7. $\mathcal{P}$ là phương trình có ba ẩn $a$, $b$ và $c$, do đó đồ thị của $\mathcal{P}$ là một không gian ba chiều với các trục hoành, trục tung và trục cao tương ứng là $a$, $b$ và $c$.

Theo $\mathcal{P}$, chúng ta cần phải chọn ba số trong tập giá trị của $f$ để hai trong ba số có tổng bằng số còn lại. Từ bảng, nhận thấy rằng, chỉ có thể có hai tổng $4 + 3 = 7$ và $0 + 0 = 0$.

Chúng ta cần tìm tất cả các bộ ba $(a, b, c)$ thỏa mãn $f(a) + f(b) = f(c)$. Xét hai trường hợp sau:

\textcolor{colorEmphasis}{Trường hợp một}: Tổng hai số khác 0. Để $f(a) + f(b) = f(c)$, chỉ có thể xảy ra khi $3 + 4 = 7$. Do đó, $(f(a); f(b); f(c))$ phải là $(3; 4; 7)$ hoặc $(4; 3; 7)$. Tra ngược lại bảng giá trị, chúng ta có hai nghiệm:
   \begin{itemize}
      \item $f(a)=3, f(b)=4 \implies a=2, b=-2$. Và
      \item $f(a)=4, f(b)=3 \implies a=-2, b=2$.
   \end{itemize}

Chỉ có $f(-3) = 7$ nên $c = -3$.

\textcolor{colorEmphasisCyan}{Trường hợp hai}: Tất cả bằng 0. Khi $f(a) = f(b) = f(c) = 0$, từ bảng định nghĩa, $f(x)=0$ khi $x \in \{-3; -1; 1\}$. Do đó, $a, b, c$ có thể nhận bất kỳ giá trị nào trong tập $\{-3; -1; 1\}$. Có tổng cộng $3^3 = 27$ bộ ba thỏa mãn trong trường hợp này.

Kết hợp hai trường hợp, đồ thị của $\mathcal{P}$ sẽ gồm 29 điểm trong không gian 3 chiều (2 điểm từ trường hợp 1 và 27 điểm từ trường hợp 2), được biểu diễn trong hình \ref{fig:ham_so_mot_bien:dinh_nghia:dtp8}.

\begin{figure}[H]
   \centering
   \tdplotsetmaincoords{80}{30}
   \begin{tikzpicture}[tdplot_main_coords]
      \draw[->] (-5, 0, 0) -- (2, 0, 0) node[right] {$a$};
      \draw[->] (0, -5, 0) -- (0, 4, 0) node[above] {$b$};
      \draw[->] (0, 0, -4) -- (0, 0, 2) node[above] {$c$};
      \filldraw[color=colorEmphasis] (2, -2, -3) circle (\pointSize) node[font=\scriptsize, anchor=north] {$\left(2; -2; -3\right)$};  
      \filldraw[color=colorEmphasis] (-2, 2, -3) circle (\pointSize) node[font=\scriptsize, anchor=south] {$\left(-2; 2; -3\right)$};  
      \foreach \x/\y/\z in {
         -3/-3/-3, -3/-3/-1, -3/-3/1, 
         -3/-1/-3, -3/-1/-1, -3/-1/1, -3/1/-3, -3/1/-1, -3/1/1,
         -1/-3/-3, -1/-3/-1, -1/-3/1, -1/-1/-3, -1/-1/-1, -1/-1/1,
         -1/1/-3, -1/1/-1, -1/1/1, 1/-3/-3, 1/-3/-1, 1/-3/1,
         1/-1/-3, 1/-1/-1, 1/-1/1, 1/1/-3, 1/1/-1, 1/1/1
      } {
         \filldraw[color=colorEmphasisCyan] (\x, \y, \z) circle (\pointSize);
         \node[font=\scriptsize, anchor=east, color=colorEmphasisCyan] at (\x, \y, \z) {$\left(\x; \y; \z\right)$};
      }
   \end{tikzpicture}
   \caption{Đồ thị phần 7 bài \ref{ex:ham_so_mot_bien:dinh_nghia:intropt}}
   \label{fig:ham_so_mot_bien:dinh_nghia:dtp8}
\end{figure}

\exercise[ex:ham_so_mot_bien:dinh_nghia:bpt1] Vẽ đồ thị của bất phương trình $\mathcal{P}$, với các định nghĩa đã cho. Hàm có tập xác định là bộ số đầu vào cho ở trong bảng.
\begin{enumerate}
   \item 
   \begin{tabular}{|c|c|c|c|c|c|c|}
      \hline
      $x$ & $0$ & $1$ & $2$ & $3$ & $4$ & $5$ \\
      \hline
      $f(x)$ & $-1$ & $-3$ & $-4$ & $-2$ & $-1$ & $-3$\\
      \hline
   \end{tabular} và $\mathcal{P}:f(x) \neq -3$;

   \item 
   \begin{tabular}{|c|c|c|c|c|c|c|}
      \hline
      $x$ & $-10$ & $-8$ & $-2$ & $2$ & $8$ & $10$ \\
      \hline
      $\alpha(x)$ & $4$ & $8$ & $0$ & $1$ & $6$ & $8$\\
      \hline
   \end{tabular} và $\mathcal{P}:\alpha(x) < 0$;

   \item 
   \begin{tabular}{|c|c|c|c|c|c|c|}
      \hline
      $x$ & $0$ & $6$ & $2$ & $-7$ & $-6$ & $3$ \\
      \hline
      $\beta(x)$ & $4$ & $7$ & $10$ & $3$ & $10$ & $9$\\
      \hline
   \end{tabular} và $\mathcal{P}:\beta(x) > x$;

   \item 
   \begin{tabular}{|c|c|c|c|c|c|c|}
      \hline
      $x$ & $-10$ & $-8$ & $-2$ & $2$ & $8$ & $10$ \\
      \hline
      $\alpha(x)$ & $4$ & $8$ & $0$ & $1$ & $6$ & $8$\\
      \hline
   \end{tabular},
   \begin{tabular}{|c|c|c|c|c|c|c|}
      \hline
      $x$ & $0$ & $6$ & $2$ & $-7$ & $-6$ & $3$ \\
      \hline
      $\beta(x)$ & $4$ & $7$ & $10$ & $3$ & $10$ & $9$\\
      \hline
   \end{tabular}, và $\mathcal{P}:\alpha(x) \geq 2\beta(y)$.
\end{enumerate}

\solution[ex:ham_so_mot_bien:dinh_nghia:bpt1]

{
   \begin{minipageindent}{0.48\textwidth}
      \setcounter{subexercise}{1}
      \arabic{subexercise}. Phần này tương đối đơn giản. Kiểm tra trên bảng, chúng ta thấy $f(x) = -3$ khi $x \in \{2; 5\}$. Thêm vào đó, tập xác định của $f$ là $\{0; 1; 2; 3; 4; 5\}$. Do đó, $f(x) \neq -3$ khi $x \in \{0; 1; 3; 4\}$. 

      Đồ thị của $\mathcal{P}$ là hình \ref{fig:ham_so_mot_bien:dinh_nghia:bpt1} ở bên.
   \end{minipageindent}
   \begin{minipageindent}{0.48\textwidth}
      \begin{figure}[H]
         \centering
         \begin{tikzpicture}
            \draw[->] (-1, 0) -- (5, 0) node[right] {$x$};
            \foreach \x in {0, 1, 3, 4} {
               \filldraw[color=colorEmphasisCyan] (\x, 0) circle (\pointSize) node[below] {$\left(\x\right)$};
            }
         \end{tikzpicture}
         \caption{Đồ thị phần 1 bài \ref{ex:ham_so_mot_bien:dinh_nghia:bpt1}}
         \label{fig:ham_so_mot_bien:dinh_nghia:bpt1}
      \end{figure}
   \end{minipageindent}
}

{
   \begin{minipageindent}{0.48\textwidth}
      2. Tra bảng trực tiếp, các giá trị $\alpha(x)$ không bao giờ nhỏ hơn $0$. Chúng ta không xét giá trị $x$ ngoài bảng do không thuộc tập xác định của hàm $\alpha$. Do đó, $\alpha(x) < 0$ là bất phương trình vô nghiệm.

      Và qua đó, vẽ được đồ thị của $\mathcal{P}$ là trục không đánh dấu như hình \ref{fig:ham_so_mot_bien:dinh_nghia:bpt2}.
   \end{minipageindent}
   \begin{minipageindent}{0.48\textwidth}
      \begin{figure}[H]
         \centering
         \begin{tikzpicture}
            \draw[->] (-1, 0) -- (5, 0) node[right] {$x$};
         \end{tikzpicture}
         \caption{Đồ thị phần 2 bài \ref{ex:ham_so_mot_bien:dinh_nghia:bpt1}}
         \label{fig:ham_so_mot_bien:dinh_nghia:bpt2}
      \end{figure}
   \end{minipageindent}
}

3. Xét trên tập xác định của $\beta$, chúng ta có $\beta(x) > x$ với mọi $x$ nằm trên bảng được cho. Một cách đơn giản, chúng ta có đồ thị là hình \ref{fig:ham_so_mot_bien:dinh_nghia:bpt3}.

\begin{figure}[H]
   \centering
   \begin{tikzpicture}
      \draw[->] (-8, 0) -- (8, 0) node[right] {$x$};
      \foreach \x in {0, 6, 2, -7, -6, 3} {
         \filldraw[color=colorEmphasisCyan] (\x, 0) circle (\pointSize) node[below] {$\left(\x\right)$};
      }
   \end{tikzpicture}
   \caption{Đồ thị phần 3 bài \ref{ex:ham_so_mot_bien:dinh_nghia:bpt1}}
   \label{fig:ham_so_mot_bien:dinh_nghia:bpt3}
\end{figure}

4. Chúng ta có thể kiểm tra trực tiếp $36$ cặp $(x; y)$ và sau đó vẽ đồ thị. Sau đây, tác giả sẽ chỉ những góc nhìn để có thể giảm số trường hợp cần kiểm tra.

Để ý rằng, giá trị lớn nhất có thể của $\alpha(x)$ là $8$. Mặt khác, để $\mathcal{P}$ thỏa mãn thì
\begin{align*}
   \alpha(x) &\geq 2\beta(y) \\
   \iff \beta(y) &\leq \frac{\alpha(x)}{2}. \\
\end{align*}
Qua đó, giá trị lớn nhất có thể của $\beta(y)$ là $4$. Theo bảng định nghĩa, $\beta(y)$ chỉ có thể nhận hai giá trị là $3$ hoặc $4$. 

\textcolor{colorEmphasisCyan}{Trường hợp một}: $\beta(y) = 3 \iff y = -7$. Khi này, để $\alpha(x) \geq 2\beta(x)$ hay $\alpha(x) \geq 6$ thì $\alpha(x)$ có thể nhận giá trị $8$ hoặc $6$. Do đó,
\begin{align*}
   &\alpha(x) = 6 \implies x = 8;\\
   &\alpha(x) = 8 \implies x \in \{-8; 10\}.
\end{align*}

\textcolor{colorEmphasis}{Trường hợp hai}: $\beta(y) = 4 \iff y = 0$. Khi này, để $\alpha(x) \geq 8$ thì $\alpha(x)$ chỉ có thể nhận bằng $8$. Do đó, $x \in \{-8; 10\}$.

Từ đây, chúng ta có đồ thị \ref{fig:ham_so_mot_bien:dinh_nghia:bpt4}.

\begin{figure}[H]
   \centering
   \begin{tikzpicture}
      \draw[->] (-4, 0) -- (4, 0) node[right] {$x$};
      \draw[->] (0, -3) -- (0, 0) node[above] {$y$};
      \foreach \x/\y/\pos in {8/-7/above, -8/-7/below, 10/-7/below} {
         \filldraw[color=colorEmphasisCyan] (\x/3, \y/3) circle (\pointSize) node [\pos] {$\left(\x; \y\right)$};
      }
      \foreach \x/\y/\pos in {-8/0/below, 10/0/below} {
         \filldraw[color=colorEmphasis] (\x/3, \y/3) circle (\pointSize) node [\pos] {$\left(\x; \y\right)$};
      }
   \end{tikzpicture}
   \caption{Đồ thị phần 4 bài \ref{ex:ham_so_mot_bien:dinh_nghia:bpt1}}
   \label{fig:ham_so_mot_bien:dinh_nghia:bpt4}
\end{figure}

\exercise[ex:ham_so_mot_bien:dinh_nghia:hpt1] Vẽ đồ thị của hệ phương trình $\mathcal{P}$, với các định nghĩa đã cho. Hàm có tập xác định là bộ số đầu vào cho ở trong bảng.
\begin{enumerate}
   \item 
   \begin{tabular}{|c|c|c|c|c|c|c|}
      \hline
      $x$ & $0$ & $1$ & $2$ & $3$ & $4$ & $5$ \\
      \hline
      $f(x)$ & $-1$ & $-1$ & $-1$ & $-2$ & $-3$ & $-3$\\
      \hline
   \end{tabular},
   \begin{tabular}{|c|c|c|c|c|c|c|}
      \hline
      $y$ & $0$ & $-2$ & $4$ & $-6$ & $8$ & $-10$\\
      \hline
      $g(y)$ & $-1$ & $-2$ & $-3$ & $-7$ & $-8$ & $-9$\\
      \hline
   \end{tabular},

   \noindent\begin{tabular}{|c|c|c|c|c|c|c|}
      \hline
      $z$ & $-1$ & $1$ & $-2$ & $0$ & $-4$ & $4$\\
      \hline
      $h(z)$ & $2$ & $1$ & $0$ & $-1$ & $-2$ & $-3$\\
      \hline
   \end{tabular} và $\mathcal{P}:f(x) = g(x) = h(x)$;

   \item
   \begin{tabular}{|c|c|c|c|c|c|c|}
      \hline
      $x$ & $0$ & $1$ & $2$ & $3$ & $4$ & $5$ \\
      \hline
      $f(x)$ & $-1$ & $-1$ & $-1$ & $-2$ & $-3$ & $-3$\\
      \hline
   \end{tabular},
   \begin{tabular}{|c|c|c|c|c|c|c|}
      \hline
      $y$ & $0$ & $-2$ & $4$ & $-6$ & $8$ & $-10$\\
      \hline
      $g(y)$ & $-1$ & $-2$ & $-3$ & $-7$ & $-8$ & $-9$\\
      \hline
   \end{tabular},

   \noindent\begin{tabular}{|c|c|c|c|c|c|c|}
      \hline
      $z$ & $-1$ & $1$ & $-2$ & $0$ & $-4$ & $4$\\
      \hline
      $h(z)$ & $2$ & $1$ & $0$ & $-1$ & $-2$ & $-3$\\
      \hline
   \end{tabular} và $\mathcal{P}:f(a) = g(b) = h(c)$;

   \item
   \begin{tabular}{|c|c|c|c|c|c|c|}
      \hline
      $x$ & $0$ & $1$ & $2$ & $3$ & $4$ & $5$ \\
      \hline
      $f(x)$ & $-1$ & $-1$ & $-1$ & $-2$ & $-3$ & $-3$\\
      \hline
   \end{tabular},
   \begin{tabular}{|c|c|c|c|c|c|c|}
      \hline
      $y$ & $0$ & $-2$ & $4$ & $-6$ & $8$ & $-10$\\
      \hline
      $g(y)$ & $-1$ & $-2$ & $-3$ & $-7$ & $-8$ & $-9$\\
      \hline
   \end{tabular},

   \noindent\begin{tabular}{|c|c|c|c|c|c|c|}
      \hline
      $z$ & $-1$ & $1$ & $-2$ & $0$ & $-4$ & $4$\\
      \hline
      $h(z)$ & $2$ & $1$ & $0$ & $-1$ & $-2$ & $-3$\\
      \hline
   \end{tabular} và $\mathcal{P}:\begin{cases}f(o) = g(p)\\f(p + 1) = h(q)\end{cases}$.

   \item
   \begin{tabular}{|c|c|c|c|c|c|c|}
      \hline
      $x$ & $0$ & $1$ & $2$ & $3$ & $4$ & $5$ \\
      \hline
      $f(x)$ & $-1$ & $-1$ & $-1$ & $-2$ & $-3$ & $-3$\\
      \hline
   \end{tabular},
   \begin{tabular}{|c|c|c|c|c|c|c|}
      \hline
      $y$ & $0$ & $-2$ & $4$ & $-6$ & $8$ & $-10$\\
      \hline
      $g(y)$ & $-1$ & $-2$ & $-3$ & $-7$ & $-8$ & $-9$\\
      \hline
   \end{tabular},

   \noindent\begin{tabular}{|c|c|c|c|c|c|c|}
      \hline
      $z$ & $-1$ & $1$ & $-2$ & $0$ & $-4$ & $4$\\
      \hline
      $h(z)$ & $2$ & $1$ & $0$ & $-1$ & $-2$ & $-3$\\
      \hline
   \end{tabular} và $\mathcal{P}:\begin{cases}f(m) = n\\g(n) = h(w)\end{cases}$.
\end{enumerate}

\solution[ex:ham_so_mot_bien:dinh_nghia:hpt1]

\setcounter{subexercise}{1}
\arabic{subexercise}. Giá trị đầu vào để $f, g, h$ đều có cùng một đầu ra là $x\in\{0;4\}$. Vậy, chúng ta có đồ thị như hình \ref{fig:hpt11}.

\begin{figure}[H]
   \centering
   \begin{tikzpicture}
      \draw[->] (-1, 0) -- (5, 0) node[right] {$x$};
      \filldraw[color=colorEmphasisCyan] (0, 0) circle (\pointSize) node[below] {$(0)$};
      \filldraw[color=colorEmphasisCyan] (4, 0) circle (\pointSize) node[below] {$(4)$};
   \end{tikzpicture}
   \caption{Đồ thị phần 1 bài \ref{ex:ham_so_mot_bien:dinh_nghia:hpt1}}
   \label{fig:hpt11}
\end{figure}

2. Trước hết, cần tìm những giá trị chung trong tập giá trị của $f, g, h$. Nhận thấy rằng, có $-1, -2$ và $-3$ là những giá trị chung trong đó. 
\begin{itemize}
   \item Với đầu ra là $-1$, chúng ta có $f(a) = g(b) = h(c) = -1$. Từ đó, chúng ta có $a \in \{0; 1; 2\}$ và $b = c = 0$.
   \item Trong trường hợp kết quả của hàm là $-2$, $f(a) = g(b) = h(c) = -2$. Từ đó, bộ ba $\left(a; b; c\right)$ có giá trị là $(3; -2; -4)$.
   \item Trong trường hợp kết quả của hàm là $-3$, $f(a) = g(b) = h(c) = -3$. Từ đó, $\left(a; b; c\right)$ $\in \left\{\left(4; 4; 4\right); \left(5; 4; 4\right)\right\}$.
\end{itemize}
Kết hợp ba trường hợp, xây dựng không gian tọa độ, chúng ta có hình \ref{fig:hpt12}.

\begin{figure}[H]
   \centering
   \tdplotsetmaincoords{80}{-10}
   \begin{tikzpicture}[tdplot_main_coords]
      \draw[->] (-1.5, 0, 0) -- (3, 0, 0) node[right] {$a$};
      \draw[->] (0, -1, 0) -- (0, 2.5, 0) node[above] {$b$};
      \draw[->] (0, 0, -2.5) -- (0, 0, 2.5) node[above] {$c$};
      \foreach \x/\y/\z/\pos in {
         0/0/0/below left,
         1/0/0/above,
         2/0/0/below} {
            \filldraw[color=colorEmphasisCyan] ({\x/2}, {\y/2}, {\z/2}) circle (\pointSize) node[\pos] {$\left(\x; \y; \z\right)$};
      }
      \filldraw[color=colorEmphasis] (1.5, -1, -2) circle (\pointSize) node[above] {$\left(3; -2; -4\right)$};
      \foreach \x/\y/\z/\pos in {
         4/4/4/above,
         5/4/4/below} {
            \filldraw[color=colorEmphasisGreen] ({\x/2}, {\y/2}, {\z/2}) circle (\pointSize) node[\pos] {$\left(\x; \y; \z\right)$};
      };
   \end{tikzpicture}
   \caption{Đồ thị phần 2 bài \ref{ex:ham_so_mot_bien:dinh_nghia:hpt1}}
   \label{fig:hpt12}
\end{figure}

3. Để $f$ và $g$ nhận cùng một giá trị thì giá trị đầu ra đó, theo bảng định nghĩa được cho, kết quả mà hàm trả ra phải là $-1$, $-2$ hoặc $-3$.
\begin{itemize}
   \item Tại $f(o) = g(p) = -1$, $o \in \{0; 1; 2\}$ và $p = 0$. Từ đó, $f(p + 1) = f(1) = -1$. Khi này, $h(q) = -1 \iff q = 0$.
   \item Tại $f(o) = g(p) = -2$, sau khi tra bảng, chúng ta thấy được rằng $\begin{cases}o = 3\\p = -2\end{cases}$; suy ra $f(p + 1) = f(-1)$, Tuy nhiên, $-1$ không thuộc tập xác định của $f$. Vậy, chúng ta sẽ loại trường hợp này.
   \item Tại $f(o) = g(p) = -3$, $o \in \{4; 5\}$ và $p = 4$. Từ đó, $f(p + 1) = f(5) = -3$. Khi này, $h(q) = -3 \iff q = 4$.
\end{itemize}
Cuối cùng, vẽ đồ thị để được hình \ref{fig:hpt13}.

\begin{figure}[H]
   \tdplotsetmaincoords{80}{-10}
   \centering
   \fbox{
      \begin{tikzpicture}[tdplot_main_coords]
         \draw[->] (-1, 0, 0) -- (3, 0, 0) node[right] {$o$};
         \draw[->] (0, -1, 0) -- (0, 2.5, 0) node[above] {$p$};
         \draw[->] (0, 0, -1) -- (0, 0, 2.5) node[above] {$q$};
         \foreach \x/\y/\z/\pos in {
            0/0/0/below left,
            1/0/0/above,
            2/0/0/below} {
               \filldraw[color=colorEmphasisCyan] ({\x/2}, {\y/2}, {\z/2}) circle (\pointSize) node[\pos] {$\left(\x; \y; \z\right)$};
         }
         \foreach \x/\y/\z/\pos in {
            4/4/4/above,
            5/4/4/below} {
               \filldraw[color=colorEmphasisGreen] ({\x/2}, {\y/2}, {\z/2}) circle (\pointSize) node[\pos] {$\left(\x; \y; \z\right)$};
         };
      \end{tikzpicture}
   }
   \caption{Đồ thị phần 3 bài \ref{ex:ham_so_mot_bien:dinh_nghia:hpt1}}
   \label{fig:hpt13}
\end{figure}

4. Theo đề, chúng ta cần tìm những bộ $(m;n;w)$ thỏa mãn $\mathcal{P}$, trong đó có $g(n) = h(w)$. Cho nên, $n$ phải thuộc tập xác định của $g$. Nhìn vào bảng, tập xác định đó là $\{0; -2; 4; -6; 8; -10\}$. Tuy nhiên, cũng có $f(m) = n$, cho nên $n$ vừa phải thuộc tập giá trị của $f$, hay $n \in \{-1; -2; -3\}$. Lấy giao của hai tập đó, chúng ta có $n = -2$. Từ đó, giải $f(m) = -2$ để có $m = 3$. Thêm vào đó, $h(w) = g(-2) = -2 \iff w = -4$.

Bộ số duy nhất thỏa mãn hệ phương trình $\mathcal{P}$ là $\left(m; n; w\right) = \left(3; -2; -4\right)$. Đồ thị của $\mathcal{P}$ là hình \ref{fig:hpt14}.

\begin{figure}[H]
   \tdplotsetmaincoords{80}{20}
   \centering
   \fbox{
      \begin{tikzpicture}[tdplot_main_coords]
         \draw[->] (-0.5, 0, 0) -- (2, 0, 0) node[right] {$m$};
         \draw[->] (0, -1.5, 0) -- (0, 0.5, 0) node[above] {$n$};
         \draw[->] (0, 0, -2.5) -- (0, 0, 0.5) node[above] {$w$};
         \filldraw[color=colorEmphasisCyan] (1.5, -1, -2) circle (\pointSize ) node[above] {$\left(3; -2; -4\right)$};
      \end{tikzpicture}
   }
   \caption{Đồ thị phần 4 bài \ref{ex:ham_so_mot_bien:dinh_nghia:hpt1}}
   \label{fig:hpt14}
\end{figure}

\subsection{Đa thức}

\ % Lùi đầu dòng

Một dạng hàm quen thuộc, được giới thiệu trong chương trình học trung học phổ thông, là đa thức, thông thường được biểu diễn dưới dạng $$f(x)=P_n(x)=\sum_{i = 0}^n a_i x^i = a_nx^n + a_{n-1}x^{n-1} + \cdots + a_1x + a_0$$ với $n$ là một số nguyên không âm, $a_i$ là các số thực, gọi là các \emph{hệ số}, với mọi $i$ nguyên nằm trong đoạn $[0, n]$ và $a_n \neq 0$. Khi này, $n$ được gọi là \emph{bậc} của đa thức. Ví dụ:
\begin{itemize}
   \item $f(x) = 2x^2 + 3x + 1$ là một đa thức bậc $2$ với các hệ số $a_2 = 2$, $a_1 = 3$, $a_0 = 1$;
   \item $g(y) = y^3 - 4y$ là một đa thức bậc $3$ với các hệ số $b_3 = 1$, $b_2 = 0$, $b_1 = -4$, $b_0 = 0$;
   \item $h(z) = 5$ là một đa thức bậc $0$ với hệ số $c_0 = 5$;
\end{itemize}
Tính toán một số giá trị mẫu:
\begin{itemize}
   \item $p(1) = 7 \cdot 1^4 - 2 \cdot 1^2 + 9 = 14$ với $q(t)= 7t^4 - 2t^2 + 9$ là một đa thức bậc $4$ với các hệ số $d_4 = 7$, $d_3 = 0$, $d_2 = -2$, $d_1 = 0$, $d_0 = 9$;
   \item $q(2) = -3 \cdot 2 + 8 = 2$ với $q(r) = -3r + 8$ là một đa thức bậc $1$ với các hệ số $e_1 = -3$, $e_0 = 8$.
\end{itemize}
Khi đa thức có bậc bằng $0$, hay $f(x) = P_0(x) = a_0$, thì được gọi là \emph{đa thức hằng} hay \emph{hàm hằng}. Một trường hợp đặc biệt là khi $f(x) = 0$ \footnote{Sẽ có nhiều tài liệu viết \dblquote{$f(x) \equiv 0$} thay vì \dblquote{$f(x) = 0$} để phân biệt giữa khẳng định hai hàm là như nhau so với một phương trình. Tác giả không muốn bạn đọc phải bận tâm với nhiều kí hiệu lạ, cho nên tác giả sẽ cố gắng dùng những kí hiệu cũ. Bạn đọc có thể tự suy luận ý nghĩa thông qua ngữ cảnh.}. Nếu hàm này là đa thức, theo định nghĩa, hàm này có bậc là $0$ và hệ số $a_0 = 0$. Tuy nhiên, cũng theo định nghĩa thì hệ số đầu phải khác $0$. Vì vậy, hàm không có bậc và không được gọi là đa thức. Nhưng, do hàm nhận giá trị cố định với mọi $x$ nên vẫn được gọi là hàm hằng \footnote{Đa số những nhà toán học không coi $f(x) = 0$ là đa thức bậc $0$ do nhiều tính chất của đa thức bị phá vỡ khi gặp trường hợp này. Do đó, $f(x) = 0$ chỉ được coi là \dblquote{hàm hằng} chứ không phải \dblquote{\textit{đa thức} hằng}. Trong tài liệu này, trở về sau sẽ chỉ có thuật ngữ \dblquote{hàm hằng} được sử dụng.}.

\exercise Phác thảo đồ thị của những hàm sau:
\begin{multicols}{2}
\begin{enumerate}
   \item $f(x) = x + 2$; 
   \item $f(x) = x^2 + 2x + 3$;
   \item $f(x) = x^3 - 9x^2 + 24x - 16$;
   \item $f(x) = 2$.
\end{enumerate}
\end{multicols}

\solution

Khả năng rất cao là bạn đọc có kết nối với mạng; vì vậy, bạn đọc có thể dùng những phần mềm vẽ đồ thị để nhanh chóng có hình vẽ. Tuy nhiên, nếu không có thiết bị điện tử thì bạn đọc vẫn có thể vẽ đồ thị bằng giấy và bút bằng cách lấy nhiều điểm ví dụ cho $x$ và tính toán giá trị $f(x)$ và sau đó nối chúng lại với nhau.

Bạn đọc có thể để ý rằng là không phải lúc nào cũng đặt gốc tọa độ ở vị trí chính giữa. Trong nhiều trường hợp việc đặt chính giữa sẽ làm mất đi đồ thị và làm cho đồ thị lệch ra khỏi khu vực vẽ. Hơn nữa, hai trục sẽ có tỉ lệ khác nhau. Điều quan trọng nhất của những bài vẽ đồ thị trong vật lí không phải là căn ke chính xác vị trí từng điểm, mà là nhận ra được dáng điệu của đồ thị và vị trí tương đối giữa các điểm trên đồ thị đó. Qua đó, chúng ta rút ra được những tính chất toán học cần thiết để phục vụ những yêu cầu cụ thể trong bài toán.

Dưới đây là đồ thị của các hàm đa thức trong bài:

\begin{figure}[H]
   \centering
   \begin{minipage}[t]{0.48\textwidth}
      \centering
      \fbox{
         \begin{tikzpicture}
            \draw[->] (-5, 0) -- (1, 0) node[right] {$x$};
            \draw[->] (0, -3) -- (0, 3) node[above] {$y$};
            \draw[thick] plot[domain=-5:1] (\x, {\x + 2});
            \filldraw (0, 2) circle (\pointSize) node[below right] {$\left(0; 2\right)$};
            \filldraw (-2, 0) circle (\pointSize) node[below right] {$\left(-2; 0\right)$};
         \end{tikzpicture}
      }
      \caption{Đồ thị của hàm $f(x) = x + 2$}
      \label{fig:ham_so:ham_da_thuc:x_2}
   \end{minipage}
   \hfill
   \begin{minipage}[t]{0.48\textwidth}
      \centering
      \fbox{
         \begin{tikzpicture}
            \draw[->] (-3, 0) -- (2, 0) node[right] {$x$};
            \draw[->] (0, 0) -- (0, 6) node[above] {$y$};
            \draw[thick] plot[domain=-3:1] (\x, {(\x + 1)^2 + 2});
            \filldraw (-1, 2) circle (\pointSize) node[below] {$\left(-1; 2\right)$};
            \filldraw (0, 3) circle (\pointSize) node[below right] {$\left(0; 3\right)$};
            \filldraw (-2, 3) circle (\pointSize) node[left] {$\left(-2; 3\right)$};
         \end{tikzpicture}
      }
      \caption{Đồ thị của hàm $f(x) = x^2 + 2x + 3$}
      \label{fig:ham_so:ham_da_thuc:x2_2x_3}
   \end{minipage}
\end{figure}
\begin{figure}[H]
   \centering
   \begin{minipage}[t]{0.48\textwidth}
      \centering
      \fbox{
         \begin{tikzpicture}
            \draw[->] (0, 0) -- (6, 0) node[right] {$x$};
            \draw[->] (0, -3) -- (0, 5) node[above] {$y$};
            \draw[thick] plot[domain=0.508:5.492] (\x, {((\x)^3 - 9*(\x)^2 + 24*(\x) - 16) / 2});
            \filldraw (2, 2) circle (\pointSize) node[above] {$\left(2; 4\right)$};
            \filldraw (4, 0) circle (\pointSize) node[below] {$\left(4; 0\right)$};
            \filldraw (1, 0) circle (\pointSize) node[below right] {$\left(1; 0\right)$};
            \filldraw (5, 2) circle (\pointSize) node[right] {$\left(5; 4\right)$};
         \end{tikzpicture}
      }
      \caption{Đồ thị của hàm $f(x) = x^3 - 9x^2 + 24x - 16$}
      \label{fig:ham_so:ham_da_thuc:x3_t9x2_24x_t16}
   \end{minipage}
   \hfill
   \begin{minipage}[t]{0.48\textwidth}
      \centering
      \fbox{
         \begin{tikzpicture}
            \draw[->] (-3, 0) -- (3, 0) node[right] {$x$};
            \draw[->] (0, -3) -- (0, 3) node[above] {$y$};
            \draw[thick] plot[domain=-3:3] (\x, {2});
            \filldraw (0, 2) circle (\pointSize) node[above left] {$\left(0; 2\right)$};
         \end{tikzpicture}
      }
      \caption{Đồ thị của hàm $f(x) = 2$}
      \label{fig:ham_so:ham_da_thuc:2}
   \end{minipage}
\end{figure}

\exercise Giải những phương trình sau. Các phương trình đều có ẩn là $x \in \mathbb{R}$.
\begin{multicols}{2}
   \begin{enumerate}
      \item $3x - 7 = 0$;
      \item $x - 9 = 5x + 3$;
      \item $\frac{1}{v}\cdot x - \frac{1}{v} \cdot x_0 = t$, với $v$, $x_0$, $t$ là những tham số thực;
      \item $6x^2 - 5x - 21 = 0$;
      \item $5x^2 - 50x + 125 = 0$;
      \item $x^2 + 2x + 4 = 0$;
      \item $x^2 + 2x + 4 = 8$;
      \item $5x^2 - 20x + 20 = x^2 - 4$;
      \item $\frac{1}{2}kx^2 + \frac{1}{2}mv^2 = \frac{1}{2}kx_0^2$, với $k$, $m$, $v$, $x_0$ là những tham số thực;
      \item $x^3 - \frac{11}{6}\cdot x^2 + x - \frac{1}{6} = 0$;
      \item $2x^3 - 2x^2 + 2x - 2 = 6 + 6x^2$.
   \end{enumerate}
\end{multicols}

\solution

1. Biến đổi tương đương phương trình để có:
\begin{align*}
   3x - 7 &= 0 \\
   \iff 3x &= 7\\
   \iff x &= \frac{7}{3}.
\end{align*}
Vậy tập nghiệm của phương trình là $\boxed{\displaystyle\left\{\frac{7}{3}\right\}}$.

2. Chuyển số hạng có thừa số $x$ về một phía, và số hạng tự do về phía còn lại để được:
\begin{align*}
   x - 9 &= 5x + 3 \\
   \iff (x - 9) + (9 - 5x) &= (5x + 3) + (9 - 5x) \\ 
   \iff -4x &= 12 \\
   \iff x &= -3.
\end{align*}
Vậy tập nghiệm của phương trình là $\boxed{\displaystyle\left\{-3\right\}}$.

3. Để giải phương trình có chứa tham số, chúng ta cần viết lại ẩn $x$ dưới dạng một biểu thức chỉ chứa tham số và hằng số. Cụ thể,
\begin{align*}
   \frac{1}{v}\cdot x - \frac{1}{v} \cdot x_0 &= t \\
   \iff \frac{x}{v} &= t + \frac{x_0}{v} \\
   \iff x &= vt + x_0.
\end{align*}
Vậy nghiệm của phương trình là $\boxed{\displaystyle\left\{vt + x_0\right\}}$.

4. Nếu như bạn đọc chưa biết, nếu như một đa thức $f(x)$ nhận $x = a$ là nghiệm thì $f(x)$ có thể được viết thành tích của $(x - a)$ nhân một đa thức $g(x)$ với bậc nhỏ hơn $1$ so với $f(x)$. Và nếu $g(x)$ lại có nghiệm $x = b$ thì chúng ta có thể viết $g(x) = (x-b)h(x)$ và qua đó có thể viết lại $f(x) = (x-a)(x-b)h(x)$. Một cách tổng quát nhất, nếu như $f(x)$ là phương trình bậc $n$ có $n$ nghiệm $a_1, a_2, \cdots, a_n$ thì có thể viết lại $$f(x) = A \prod_{i=1}^{n} (x - a_i) = A(x - a_1)(x - a_2)\cdots (x - a_n)$$ với $A$ là hệ số của số hạng có bậc lớn nhất trong đa thức $f(x)$.

Nhẩm nghiệm (bằng cách bấm máy tính) phương trình thì có $x = -\frac{3}{2}$ và $x = \frac{7}{3}$. Chúng ta kì vọng có thể viết lại phương trình dưới dạng $6\left(x - \left(-\frac{3}{2}\right)\right)\left(x - \frac{7}{3}\right) = 0$. Thực vậy, thực hiện phân tích nhân tử để có:
\begin{align*}
   &6x^2 - 5x - 21 = 0 \\
   \iff &6x^2 - 14x + 9x - 21 = 0 \\
   \iff &2x(3x - 7) + 3(3x - 7) = 0 \\
   \iff &(2x + 3)(3x - 7) = 0 \\
   \iff &\left[
      \begin{aligned}
         2x + 3 &= 0 \\
         3x - 7 &= 0
      \end{aligned}
   \right.
   \iff \left[
      \begin{aligned}
         x &= -\frac{3}{2} \\
         x &= \frac{7}{3}
      \end{aligned}
   \right..
\end{align*}
Vậy phương trình có nghiệm là $\boxed{\displaystyle\left\{-\frac{3}{2}; \frac{7}{3}\right\}}$.

5.
\begin{align*}
   5x^2 - 50x + 125 &= 0 \\
   \iff 5\left(x^2 - 10x + 25\right) &= 0 \\
   \iff 5(x - 5)^2 &= 0 \\
   \iff x - 5 &= 0 \\
   \iff x &= 5.
\end{align*}

Vậy tập nghiệm của phương trình có một phần tử duy nhất $\boxed{\displaystyle\left\{5\right\}}$.

6. Với những phương trình liên quan tới đa thức bậc hai không thể nhẩm ngay được nghiệm, chúng ta sẽ sử dụng phương pháp tách bình phương. Với phương trình được cho:
\begin{align}
   x^2 + 2x + 4 &= 0 \nonumber\\ 
   \iff x^2 + 2x + 1 &= -3 \nonumber\\
   \iff (x + 1)^2 &= -3. \label{eq:ham_so_mot_bien:ham_da_thuc:gptdt6}
\end{align}
Một số thực nhân với chính nó sẽ ra một số không âm. Cho nên phương trình \ref{eq:ham_so_mot_bien:ham_da_thuc:gptdt6} không thể đúng. Vậy phương trình \fbox{vô nghiệm} trên tập số thực.

7. 
\begin{align*}
   &x^2 + 2x + 4 = 8 \\ 
   \iff &x^2 + 2x + 1 = 5 \\
   \iff &(x + 1)^2 = 5 \\
   \iff &\left[
      \begin{aligned}
         x + 1 &= \sqrt{5} \\
         x + 1 &= -\sqrt{5}
      \end{aligned}
   \right. \\
   \iff &\left[
      \begin{aligned}
         x &= \sqrt{5} - 1 \\
         x &= -\sqrt{5} - 1
      \end{aligned}
   \right..
\end{align*}
Vậy tập nghiệm của phương trình là $\boxed{\displaystyle\left\{\sqrt{5} - 1; -\sqrt{5} - 1\right\}}$.

8. Phần này tác giả làm khác so với phần 2. Chuyển đổi toàn bộ phương trình về một vế để đưa về dạng phương trình $f(x) = 0$:
\begin{align*}
   &5x^2 - 20x + 20 = x^2 - 4 \\
   \iff &4x^2 - 20x + 24 = 0 \\
   \iff &4\left(x^2 - 5x + 6\right) = 0 \\
   \iff &4\left(x^2 - 2x - 3x + 6\right) = 0 \\
   \iff &4\left(x(x - 2) - 3(x - 2)\right) = 0 \\
   \iff &4(x - 3)(x - 2) = 0 \\
   \iff &\left[
      \begin{aligned}
         x - 3 &= 0 \\
         x - 2 &= 0
      \end{aligned}
   \right. \iff x \in \left\{3; 2\right\}. 
\end{align*}
Vậy phương trình có tập nghiệm $\boxed{\left\{3; 2\right\}}$.

9. Nhân cả hai vế với $2$ để khử phân số trong phương trình:
\begin{align}
   &\frac{1}{2}kx^2 + \frac{1}{2}mv^2 = \frac{1}{2}kx_0^2 \nonumber \\
   \iff &kx^2 + mv^2 = kx_0^2. \label{eq:ham_so_mot_bien:ham_da_thuc:gptdt9}
\end{align}
Xong, thực hiện chuyển vế để giữ thừa số chứa $x^2$ ở một bên, phương trình \ref{eq:ham_so_mot_bien:ham_da_thuc:gptdt9} tương đương với
\begin{align*}
   (\ref{eq:ham_so_mot_bien:ham_da_thuc:gptdt9}) \iff &kx^2 = kx_0^2 - mv^2 \\
   \iff & x^2 = x_0^2 - \frac{mv^2}{k}
\end{align*}

Với trường hợp $x_0^2 - \frac{mv^2}{k} < 0$ thì phương trình vô nghiệm do $x^2$ không thể âm. Trong trường hợp còn lại, lấy căn bậc hai hai vế để có $$x\in\left\{\sqrt{x_0^2 - \frac{mv^2}{k}}; -\sqrt{x_0^2 - \frac{mv^2}{k}}\right\}.$$ Tại giá trị đặc biệt mà khi $x_0^2 = \frac{mv^2}{k}$ thì tập nghiệm suy biến thành $\left\{0\right\}$.

Vậy, phương trình có nghiệm là 
$$
\boxed{
   \begin{cases}
      \left\{\sqrt{x_0^2 - \frac{mv^2}{k}}; -\sqrt{x_0^2 - \frac{mv^2}{k}}\right\} &\text{ nếu } x_0^2 - \frac{mv^2}{k} \geq 0 \\
      \emptyset &\text{ nếu } x_0^2 - \frac{mv^2}{k} < 0
   \end{cases}.
}
$$

10. Phân tích thừa số với để ý rằng $1$, $\frac{1}{2}$ và $\frac{1}{3}$ là nghiệm:
\begin{align*}
   &x^3 - \frac{11}{6}\cdot x^2 + x - \frac{1}{6} = 0 \\
   \iff &x^3 - x^2 - \frac{5}{6}x^2 + \frac{5}{6}x + \frac{1}{6}x - \frac{1}{6} = 0 \\
   \iff &x^2\left(x - 1\right) - \frac{5}{6}x\left(x - 1\right) + \frac{1}{6}\left(x - 1\right) = 0 \\
   \iff &\left(x - 1\right)\left(x^2 - \frac{5}{6}x + \frac{1}{6}\right) = 0 \\
   \iff &\left(x - 1\right)\left(x^2 - \frac{1}{2}x - \frac{1}{3}x + \frac{1}{6}\right) = 0 \\
   \iff &\left(x - 1\right)\left(x\left(x - \frac{1}{2}\right) - \frac{1}{3}\left(x - \frac{1}{2}\right)\right) = 0 \\
   \iff &\left(x - 1\right)\left(x - \frac{1}{2}\right)\left(x - \frac{1}{3}\right) = 0 \\
   \iff &\left[
      \begin{aligned}
         x - 1 &= 0 \\
         x - \frac{1}{2} &= 0 \\
         x - \frac{1}{3} &= 0
      \end{aligned}
   \right. \\
   \iff &\left[
      \begin{aligned}
         x &= 1 \\
         x &= \frac{1}{2} \\
         x &= \frac{1}{3}
      \end{aligned}
   \right..
\end{align*}
Cuối cùng, như chúng ta đã dự đoán, phương trình có nghiệm là $\boxed{\displaystyle\left\{1; \frac{1}{2}; \frac{1}{3}\right\}}$.

11. Có một cách là chuyển phương trình về một vế rồi nhẩm nghiệm. Dưới đây, tác giả sẽ trình bày một góc nhìn khác để giải bài toán này.
\begin{align}
   2x^3 - 2x^2 + 2x - 2 &= 6 + 6x^2 \nonumber\\
   \iff \left(2x^3 + 2x\right) - \left(2x^2 + 2\right) &= 6x^2 + 6 \nonumber\\
   \iff 2x\left(x^2 + 1\right) - 2\left(x^2 + 1\right) &= 6\left(x^2 + 1\right) \nonumber\\
   \iff \left(2x - 2\right)\left(x^2 + 1\right) &= 6\left(x^2 + 1\right). \label{eq:ham_so_mot_bien:ham_da_thuc:gptdt11}
\end{align}
Để ý rằng, do $x^2 \geq 0$ nên $x^2 + 1 \geq 1 > 0$. Chúng ta đã chỉ ra rằng $x^2 + 1 \neq 0$, và qua đó, chúng ta có thể an toàn chia hai vế của \ref{eq:ham_so_mot_bien:ham_da_thuc:gptdt11} cho $x^2 + 1$ để có:
\begin{align*}
   2x - 2 &= 6 \\
   \iff x &= 4.
\end{align*}
Vậy phương trình có nghiệm là $\boxed{\displaystyle\left\{4\right\}}$. 

\section{Số ảo và số phức}

\ % Lùi đầu dòng

Trước khi đến với sô thực với số phức, chúng ta bắt đầu tiếp cận với định nghĩa đơn vị ảo. Cụ thể, \emph{đơn vị ảo} được kí hiệu là $\mathbf{i}$ \footnote{Phần lớn các tài liệu sẽ kí hiệu số ảo là chữ $i$ thông thường. Tác giả kí hiệu thành chữ $\mathbf{i}$ đứng in đậm để bảo toàn chữ $i$ cho nhiệm vụ khác.} và thỏa mãn $$\mathbf{i}^2 = -1 \text{ hay } \mathbf{i} = \sqrt{-1}.$$

Để có số ảo, nhân một số thực $b \neq 0$ với đơn vị ảo để thành $\mathbf{i}b$. Một số phức bao gồm thành phần thực và $\mathbf{i}$ lần phẩn ảo cộng vào. Viết dưới \emph{dạng chính tắc}, một số phức có dạng là $$z=a+\mathbf{i}b$$ với $a, b$ thực. Từ một số phức, chúng ta cũng có thể lấy ngược lại giá trị phần thực và phần ảo của nó lần lượt qua hai hàm $\Re{(z)}$ và $\Im{(z)}$ (hoặc $\operatorname*{Re}{(z)}$ và $\operatorname*{Im}{(z)}$). Cụ thể, với $z=a+\mathbf{i}b$ thì $\Re{(z)} = a$ và $\Im{(z)}=b$\footnote{Tại sao không gọi cả $\mathbf{i}b$ là phần ảo? Khi nói đến phần ảo, chúng ta đã ngầm định nó sẽ thuộc về số hạng mà có thừa số $\mathbf{i}$. Viết lại đơn vị ảo trở nên thừa thãi. Hơn nữa, sẽ dễ làm việc hơn khi mà cả $\Re{(z)}$ và $\Im{(z)}$ đều thực và không phải chia $\Im{(z)}$ cho $\mathbf{i}$ liên tục.}.

Chúng ta sẽ coi như có thể thực hiện các định luật đại số thông thường trên tập số phức. Coi $\mathbf{i}$ là một biến với $\mathbf{i}^2 = -1$. Để cộng (hay trừ) hai số phức $v = a + \mathbf{i}b$ và $w = c + \mathbf{i}d$, thực hiện cộng (hay trừ) các thành phần tương đương (phần thực với phần thực, phần ảo với phần ảo). Viết dưới dạng toán học:
\begin{equation*}
   \begin{cases}
      v + w = (a + c) + \mathbf{i}(b + d) \\ 
      v - w = (a - c) + \mathbf{i}(b - d) 
   \end{cases}.
\end{equation*}

Nhân hai số phức sẽ yêu cầu sử dụng tính chất phân phối giữa phép nhân với phép cộng, được thực hiện như sau
\begin{align*}
   v\cdot w&=\left(a + \mathbf{i}b\right)\cdot\left(c + \mathbf{i}d\right) \\
      &= ac + \mathbf{i}ad + \mathbf{i}bc + \mathbf{i}^2 bd \\
      &= (ac - bd) + \mathbf{i}(ad + bc).
\end{align*}

Trước khi chia hai số phức, chúng ta cần phải biết đến khái niệm số phức liên hợp và tính chất đặc biệt của nó. Một số phức $z = a+\mathbf{i}b$ sẽ có số phức liên hợp là $$\bar{z} = z^* = a - \mathbf{i}b.$$ Khi này, thực hiện phép nhân số phức $z$ với liên hợp của nó để có $$z\bar{z} = (a+\mathbf{i}b)(a-\mathbf{i}b) = a^2 + b^2.$$ Để ý rằng $a^2 + b^2$ là một số thực do $a, b$ đã là số thực từ định nghĩa, và cũng cần phải nhớ lại rằng khi nhân cả số bị chia và số chia với một số thì thương không đổi. Cho nên, để chia hai số phức, chúng ta nhân cả tử và mẫu với liên hợp của số chia $$\frac{v}{w} = \frac{a + \mathbf{i}b}{c + \mathbf{i}d} = \frac{\left(a + \mathbf{i}b\right)\left(c - \mathbf{i}d\right)}{\left(c + \mathbf{i}d\right)\left(c - \mathbf{i}d\right)} = \frac{\left(ac+bd\right) + \mathbf{i}\left(bc - ad\right)}{c^2+d^2}.$$ Từ đó, chúng ta đưa phép chia hai số phức thành phép chia số phức với số thực và có kết quả là $$\frac{v}{w} = \frac{a + \mathbf{i}b}{c + \mathbf{i}d} = \frac{ac+bd}{c^2+d^2}+\mathbf{i}\cdot\frac{bc - ad}{c^2+d^2}.$$

\begin{wrapfigure}{R}{0.5\textwidth}
    \centering
    \begin{tikzpicture}
      \draw[->] (-2,0) -- (4,0) node[right] {$\operatorname*{Re}(z)$};
      \draw[->] (0,-1) -- (0,3) node[above] {$\operatorname*{Im}(z)$};
      \filldraw (3,2) circle (1pt) node[anchor=west] {$z = 3 + 2\mathbf{i}$};
      \filldraw (0, 0) circle (1pt);
      \node[above] at (3,2) {$Z(3, 2)$};
      \draw[dashed, thick] (0, 0) -- (3, 2);
      \draw (0,0) -- (0.1, -0.15);
      \draw (3, 2) -- (3.1, 1.85);
      \draw[<->] (0.05, -0.075) -- (3.05, 1.925);
      \node[right] at ($(0.1, -0.1)!0.5!(3.1, 1.9)$) {$|z| = \sqrt{3^2 + 2^2} = \sqrt{13}$};
   \end{tikzpicture}

   \caption{Biểu diễn $z = 3 + 2\mathbf{i}$ trên mặt phẳng tọa độ}
   \label{fig:bieu_dien_so_phuc}
\end{wrapfigure}

Ngoài cách biểu diễn đại số, còn có cách biểu diễn hình học trên mặt phẳng tọa độ của số phức qua việc coi trục hoành và trục tung lần lượt biểu diễn phần thực và phần ảo của số phức. Cụ thể, số phức $z = a + \mathbf{i}b$ được biểu diễn bởi một điểm $Z(a,b)$ trên hệ tọa độ vuông góc. Khi này, $Z$ là \emph{ảnh} (hay đơn giản là \emph{điểm biểu diễn}) của $z$ và $(a,b)$ được gọi là \emph{tọa vị} (hay \emph{tọa độ phức}) của $z$.

Hình \ref{fig:bieu_dien_so_phuc} đã biểu diễn số phức $z = 3 + 2\mathbf{i}$ trên mặt phẳng tọa độ. Từ đây, chúng ta có thể phát hiện ra những đặc tính khác của $z$ khác tọa vị. Đầu tiên, chúng ta có thể đo khoảng cách từ ảnh $Z$ đến gốc $(0;0)$, và từ đó, chúng ta sẽ nhận được \emph{mô-đun (module)} của $z$, kí hiệu: $|z|$. Bạn đọc có thể để ý rằng kí hiệu giống như kí hiệu giá trị tuyệt đối của số thực. Cũng có thể hiểu được tại sao lại vậy nếu như bạn đọc nhớ cách biểu diễn khoảng cách hình học của giá trị tuyệt đối trên trục số thực. Khi chúng ta có một điểm biểu diễn một số thực $x$ trên một trục thì giá trị tuyệt đối của $x$ chính là khoảng cách từ $x$ đến điểm $0$.

\begin{figure}[h]
   \centering
   \begin{tikzpicture}
      \draw[<->] (-5, 0) -- (5, 0);

      \foreach \pt/\lbl in {0/0, 4/x, -2.5/y} {
         \filldraw (\pt, 0) circle (1pt);
         \draw (\pt, 0) -- (\pt, -0.2);
         \node[above] at (\pt, 0) {$\lbl$};
      }
      
      \draw[<->] (-2.5, -0.1) -- (0, -0.1);
      \node[below] at (-1.25, -0.1) {$|y|$};
      \draw[<->] (4, -0.1) -- (0, -0.1);
      \node[below] at (2, -0.1) {$|x|$};
   \end{tikzpicture}
   \caption{Giá trị tuyệt đối trên trục thực}
   \label{fig:gia_tri_tuyet_doi_thuc}
\end{figure}

Một cách tương tự, $|z|$ là khoảng cách từ $Z$ đến gốc tọa độ. Công thức Pi-ta-go được sử dụng để tính khoảng cách này: $$|z| = \sqrt{a^2+b^2}.$$ Cũng là vì lí do đó nên trong một số tài liệu, $|z|$ vẫn được gọi là giá trị tuyệt đối để đảm bảo tính nhất quán.

Trên trục số thực, mốt số cụ thể thì giá trị tuyệt đối của nó chỉ có một giá trị, nhưng nếu đầu ra là một giá trị tuyệt đối thì đầu vào có thể là $2$ số khác nhau. Để biết chính xác là số nào thì cần biết thêm dấu của số đó, hay nói một cách khác, hướng của số đó nếu nhìn từ vị trí gốc $0$. Một cách tương tự, một số phức $z$ chỉ ra được một giá trị mô-đun $|z|$ của nó, nhưng với một $|z|$ thì có thể có nhiều $z$ thỏa mãn. Để biết chính xác được giá trị của $z$ thì chúng ta cần phải biết thêm hướng của $z$. Tuy nhiên, việc xác định hướng này không chỉ đơn giản là nằm trái hay phải trên trục một chiều nữa, mà cần phải xác định vị trí trong mặt phẳng hai chiều. Một cách để thực hiện điều này là xác định \emph{góc} (hay \emph{a-gu-men}) của $z$.

Để xác định góc, chúng ta cần phải có $2$ tia. Một tia có thể được nối từ gốc đến điểm biểu diễn. Một tia còn lại có thể bám theo một trục cố định. Về quy ước, phía dương trục hoành, hay trục thực, được sử dụng làm bờ còn lại. Bạn đọc có thể nghĩ rằng là khi này chúng ta đã có đủ điều kiện để xác định góc. Cũng đũng, đã đủ để từ số phức $z$ ra được góc của $z$. Nhưng từ góc của $z$ vẫn chưa đủ để ra được $z$. Hãy nhìn vào hình \ref{fig:hai_truong_hop_goc}:
\begin{figure}[h]
   \centering
   \begin{tikzpicture}
      \draw[->] (-1, 0) -- (5, 0);
      \draw[->] (0, -3) -- (0, 3);

      \filldraw (4,2) circle (1pt) node[anchor=west] {$z = 4 + 2\mathbf{i}$};
      \filldraw (4,-2) circle (1pt) node[anchor=west] {$z = 4 - 2\mathbf{i}$};

      \draw[dashed, ->, thick] (0, 0) -- (4, 2);
      \draw[dashed, ->, thick] (0, 0) -- (4, -2);

      \draw (0.5,0) arc[start angle=0, end angle={atan(2/4)}, radius=0.5];
      \node at (0.75,0.18) {$\theta$};

      \draw (0.75,0) arc[start angle=0, end angle={-atan(2/4)}, radius=0.75];
      \node at (0.9,-0.2) {$\theta$};
   \end{tikzpicture}
   \caption{$4+2\mathbf{i}$ và $4-2\mathbf{i}$ có độ lớn góc bằng nhau.}
   \label{fig:hai_truong_hop_goc}
\end{figure}


Để phân biệt hai góc này, người ta sử dụng khái niệm \emph{góc định hướng}. Một cách đơn giản, quay trục hoành ngược chiều kim đồng hồ cho đến khi chạm vào cạnh còn lại. Góc đã quay là độ lớn của góc định hướng. Khi quay thuận chiều kim đồng hồ thì góc đó quy ước là quay góc âm. Từ đó, chúng ta có thể phân biệt góc nhìn như hình \ref{fig:hai_truong_hop_goc_dinh_huong}:

\begin{figure}[h]
   \centering
   \begin{tikzpicture}
      \draw[->] (-2, 0) -- (5, 0);
      \draw[->] (0, -3) -- (0, 3);

      \filldraw (4,2) circle (1pt) node[anchor=west] {$z = 4 + 2\mathbf{i}$};
      \filldraw (4,-2) circle (1pt) node[anchor=west] {$z = 4 - 2\mathbf{i}$};

      \draw[dashed, ->, thick] (0, 0) -- (4, 2);
      \draw[dashed, ->, thick] (0, 0) -- (4, -2);

      \draw[->] (0.5,0) arc[start angle=0, end angle={atan(2/4)}, radius=0.5];
      \node at (0.75,0.18) {$\theta$};

      \draw[->] (0.75,0) arc[start angle=0, end angle={-atan(2/4)}, radius=0.75];
      \node at (1,-0.22) {$-\theta$};

      \draw[->] (1,0) arc[start angle=0, end angle={360-atan(2/4)}, radius=1];
      \node at (-1.5,0.7) {$2\pi - \theta$};
   \end{tikzpicture}
   \caption{$4+2\mathbf{i}$ và $4-2\mathbf{i}$ có độ lớn góc bằng nhau.}
   \label{fig:hai_truong_hop_goc_dinh_huong}
\end{figure}

Người ta kí hiệu góc của số phức là $\arg{(z)}$ hoặc $\Arg{(z)}$. Cũng giống như góc không định hướng, khi cộng thêm hay bớt đi $2\pi$ ra-đi-an (hay $360$ độ) thì \dblquote{hướng nhìn} cũng không thay đổi. Để cho $\arg{(z)}$ chỉ trả ra một giá trị duy nhất, quy ước là lấy góc trong nửa đoạn $\left(-\pi;\pi\right]$ (hay $\left(-180^\circ;180^\circ\right]$). Như ví dụ trong hình \ref{fig:hai_truong_hop_goc_dinh_huong}, $\arg{(4+2\mathbf{i})} = \theta = \arctan\left(\frac{2}{4}\right) \approx 0,464~\text{rad}$ (hay $26,565^\circ$) và $\arg{(4-2\mathbf{i})} = -\theta = -\arctan\left(\frac{2}{4}\right) \approx -0,464~\text{rad}$ (cũng có thể được viết lại là $-26,565^\circ$).

Như đã viết, có khoảng cách và hướng nhìn thì chúng ta sẽ có được vị trí số phức. Cách biểu diễn này được gọi là \emph{dạng lượng giác} của số phức. Đặt $r = |z|$ và $\varphi = \arg{(z)}$, dạng lượng giác của $z$ được kí hiệu là $z = r \angle \varphi = r \phase{\varphi}$. Quy đổi giữa dạng lượng giác và dạng chính tắc, khi $z = a + \mathbf{i}b = r \phase{\varphi}$ thì
\[
\left\{
\begin{aligned}
   a &= \Re{(z)} = r \cos{(\varphi)} \\ 
   b &= \Im{(z)} = r \sin{(\varphi)}
\end{aligned}
\right.
\]
và từ đó $z = r\left(\cos{(\varphi)} + \mathbf{i}\sin{(\varphi)}\right)$. Liên hợp của $z$ dưới dạng lượng giác là $\bar{z} = r\left(\cos{(\varphi)} - \mathbf{i}\sin{(\varphi)}\right) = r \phase{-\varphi}$.



Thật sự, rất khó cho nhiều người không thường xuyên thường thức về toán ngay lập tức tìm ra và cảm nhận được ý nghĩa thực tiễn của số ảo. Chúng ta không thể tưởng tượng được số ảo một cách trực quan như các số mà chúng ta thường thấy ở ngoài cuộc sống như $5$ cái bút hay $\frac{1}{3}$ giờ. Đi kèm với đó, kể cả trên lí thuyết toán của ghế nhà trường, cũng sẽ không xảy ra trường hợp nào để cho một số nhân với chính nó ra một số âm. 

Nhắc về số âm, theo quan điểm cá nhân, số âm trong đời xuống vốn đã ít khi được sử dụng. Chẳng mấy ai ưa nói \dblquote{lãi $-500000$ đồng} so với \dblquote{lỗ $500000$ đồng}. Một cách tương tự, nhìn về phương diện lịch sử, trong phần lớn quá trình phát triển của toán học, các nhà toán học xưa thường có mặc cảm với những số âm. Các phương trình sẽ luôn được viết lại thành nhiều trường hợp để tránh chúng. Ví dụ, nếu phương trình bậc hai được viết dưới dạng hiện đại là $x^2 + ax+b=0$ với $a,b$ là hai số thực (có thể âm), thì trong quá khữ, phương trình này được chia ra làm ba trường hợp
\begin{align*}
   &x^2+ax = b;\\
   &x^2+b =ax;\\
   &x^2 =ax+b
\end{align*}
với $a,b$ là hai số thực luôn dương. Và cũng từ sự mặc cảm với số âm, họ cho rằng nghiệm của phương trình cũng phải là một số dương. Tương tự với Các-đa-nô \footnote{Gerolamo Cardano (1501-1576).}, khi giải phương trình bậc ba, ông cũng đưa về các trường hợp như trên. Cụ thể, chúng ta xem xét một trường hợp của bài toán: $$x^3 = ax+b.$$ Giải phương trình, chúng ta có được nghiệm $$x=\sqrt[3]{\frac{b}{2} + \sqrt{\frac{b^2}{4}-\frac{a^3}{27}}}+\sqrt[3]{\frac{b}{2} - \sqrt{\frac{b^2}{4}-\frac{a^3}{27}}}.$$ Tuy nhiên, sau khi thay những giá trị cụ thể vào $a$ và $b$, Các-đa-nô đã phát hiện ra một vấn đề. Khi $a=15$ và $b=4$, nghiệm trả ra cho phương trình $x^3 = 15x+4$ theo công thức vừa trên là $$x=\sqrt[3]{2+\sqrt{-121}}+\sqrt[3]{2-\sqrt{-121}}$$ mặc dù phương trình có một nghiệm bình thường là $x=4$ (với kiến thức toán học hiện đại, chúng ta có thể giải ra hai nghiệm cũng thực khác là $-2 \pm \sqrt{3}$). Nhận ra điều đó, Các-đa-nô đã khẳng định rằng công thức này của ông không áp dụng được trong trường hợp xảy ra căn của một số âm. Tuy nhiên, một học trò của ông, Bom-be-li \footnote{Rafael Bombelli (1526-1572).}, lại phủ nhận điều trên. Bom-be-li nhận định rằng tồn tại một kiểu số khác số thực sẽ có giá trị bằng \dblquote{căn âm}. Ông chỉ rõ sự khác biệt giữa kiểu số mới này và kiểu số thực thông thường, và đi kèm theo là phương pháp thực hiện đại số trên kiểu số mới. Áp dụng những nền tảng đó, ông đã tính được căn bậc ba của hai số phức lần lượt là $\sqrt[3]{2+\sqrt{-121}}=2+\sqrt{-1}$ và $\sqrt[3]{2-\sqrt{-121}}=2-\sqrt{-1}$. Cộng hai số vào, hiển nhiên sẽ có được nghiệm $4$ như mong muốn.

Với sự xây dựng ban đầu của Bom-be-li làm gốc, trong những thế kỉ sau, tên gọi và lí thuyết về cách biểu diễn số phức được hình thành.

\section{Bài tập tổng hợp}


\chapter{Cơ bản của xử lí số liệu trong vật lí}
\exercise Khoảng cách trung bình từ trái đất đến mặt trời là $1{,}5 \cdot 10^8$ km. Giả sử quỹ đạo của trái đất quanh mặt trời là tròn và mặt trời được đặt tại gốc của hệ quy chiếu.
\begin{enumerate}
   \item Tính tốc độ di chuyển trung bình của trái đất quanh mặt trời dưới dạng dặm trên giờ ($1 \;\text{dặm}=1{,}6093\;\text{km}$).
   \item Ước lượng góc $\theta$ giữa véc-tơ vị trí của trái đất bây giờ và vị trí sau đó $4$ tháng.
   \item Tính khoảng cách giữa hai vị trí đó.
\end{enumerate}
\solution
\begin{enumerate}
   \item Giả sử trái đất quay quanh mặt trời trong $365{,}25$ ngày. Quãng đường mà trái đất đi được trong thời gian này là chu vi của quỹ đạo tròn $2 \pi \cdot 1{,}5 \cdot 10^8$ km. Từ đó, chúng ta có thể tính được tốc độ trung bình của trái đất quanh mặt trời là $\frac{2 \pi \cdot 1{,}5 \cdot 10^8\ \text{km}}{365{,}25\ \text{ngày}}$. Thực hiện quy đổi để được:
      \[
         \frac{2 \pi \cdot 1{,}5 \cdot 10^8\ \text{km}}{365{,}25\ \text{ngày}}
         \cdot \frac{1\ \text{dặm}}{1{,}6903\ \text{km}}
         \cdot \frac{1\ \text{ngày}}{24\ \text{h}}
         = \boxed{6{,}4\cdot 10^4\ \frac{\text{dặm}}{\text{h}}}.
      \]
   \item Trái đất quay quanh mặt trời trong $12$ tháng, tương đương với một góc quay $360^{\circ}$ so với gốc là mặt trời. Coi như các tháng có độ dài như nhau. Ta có $\theta$ chính là góc quay của trái đất trong $4$ tháng, tương đương với:
   \[
      \theta = \frac{360^{\circ}}{12\ \text{tháng}} \cdot 4\ \text{tháng}= \boxed{120^{\circ}}.
   \]
   \item
\end{enumerate}

\begin{wrapfigure}{R}{0.3\textwidth}
   \centering
   \begin{tikzpicture}
      \draw[thick] (0,0) circle (2cm);
      \filldraw[black] (0,0) circle (\pointSize) node[anchor=north] {O};
      \draw[->, thick, >=latex, line width=0.5mm] (0,0) -- (0:2cm) node[midway, below] {$r$};
      \draw[->, thick, >=latex, line width=0.5mm] (0,0) -- (120:2cm);
      \draw[thick, line width=0.5mm] (0:2cm) -- (120:2cm) node[midway, above] {$d$};
      \filldraw[black] (0:2cm) circle (\pointSize) node[anchor=west] {A};
      \filldraw[black] (120:2cm) circle (\pointSize) node[anchor=south] {B};
      \node at (60:0.3cm) {$\theta$};
      \draw[thick] (0:0.5cm) arc[start angle=0, end angle=120, radius=0.5cm];
   \end{tikzpicture}
   \caption{Quỹ đạo trái đất}
   \label{fig:earth}
\end{wrapfigure}

Gọi $A$ là vị trí của trái đất bây giờ, $B$ là vị trí của trái đất sau $4$ tháng theo như hình \ref{fig:earth}. Coi một đơn vị trên tọa độ bằng độ dài bán kính của quỹ đạo tròn, tức là $r=1{,}5\cdot10^8$ km. Ta có tọa độ điểm $A$ là $(1;0)$. Tọa độ điểm $B$ là $\left(\cos(120^{\circ}); \sin(120^{\circ})\right)=\left(-\frac{1}{2}; \frac{\sqrt{3}}{2}\right)$. Từ đó, chúng ta có khoảng cách giữa hai vị trí đó là: $$d = r\cdot \sqrt{\left(1-\left(-\frac{1}{2}\right)\right)^2 + \left(\frac{\sqrt{3}}{2}\right)^2}=\boxed{2{,}6\cdot10^8\ \text{km}}.$$

\exercise Khối lượng riêng (bằng khối lượng của vật chia cho thể tích của vật đó) của nước là $1{,}00 \;\frac{\text{g}}{\text{cm}^3}$.
\begin{enumerate}
   \item Tính giá trị này theo ki-lô-gam trên mét khối.
   \item $1{,}00$ lít nước nặng bao nhiêu ki-lô-gam, bao nhiêu pao (lb)? Biết $1\ \text{lb} = 0{,}45\ \text{kg}$ (chính xác).
\end{enumerate}

\solution
\begin{enumerate}
   \item Thực hiện quy đổi, chúng ta có:
   \begin{align*}
      1{,}00\;\frac{\text{g}}{\text{cm}^3} &= \left(1{,}00\;\frac{\text{g}}{\text{cm}^3}\right)\cdot\frac{1\ \text{kg}}{1000\ \text{g}}\cdot\left(\frac{100\ \text{cm}}{1\ \text{m}}\right)^3 \\
      &= \boxed{1{,}00\cdot 10^3\ \frac{\text{kg}}{\text{m}^3}}.
   \end{align*}
   \item Khối lượng của $1{,}00$ lít nước là
   \begin{align*}
      1{,}00\ \text{L} \cdot \left(1{,}00\cdot 10^3\ \frac{\text{kg}}{\text{m}^3}\right)&= 1{,}00\ \text{L} \cdot \left(1{,}00\cdot 10^3\ \frac{\text{kg}}{\text{m}^3}\right) \cdot \frac{1\ \text{m}^3}{1000\ \text{L}} \\
      &= \boxed{1{,}00\cdot 10^0\ \text{kg}}.
   \end{align*}
   Theo đơn vị pao (lb), chúng ta có:
   \[
      1{,}00\cdot 10^0\ \text{kg} = 1{,}00\cdot 10^0\ \text{kg} \cdot \frac{1\ \text{lb}}{0{,}45\ \text{kg}} = \boxed{2{,}22\cdot 10^0\ \text{lb}}.
   \]
\end{enumerate}

\exercise Trong hệ thời gian cổ Trung Hoa, từ triều đại Thanh trở về trước (trừ một số năm), một ngày được chia thành $100$ khắc. Sau triều đại này (trừ một số năm), một ngày được chia thành $96$ khắc. Coi một ngày có $24$ giờ và mọi số liệu là chính xác tuyệt đối.
\begin{enumerate}
   \item Tính số giây (hệ đo lường hiện đại) trong một khắc trong cả hai thời kì.
   \item Tính tỉ lệ về độ dài của hai khắc trong hai thời kì.
\end{enumerate}

\solution
\begin{enumerate}
   \item Số giây trong một ngày là $$24\ \text{h} \cdot \frac{60\ \text{phút}}{1\ \text{h}} \cdot \frac{60\ \text{giây}}{1\ \text{phút}} = 86400\ \text{giây}.$$
\end{enumerate}
Từ triều đại Thanh trở về trước, số giây trong một khắc là $$\frac{86400\ \text{giây}}{100\ \text{khắc}_{\text{trước}}} = \boxed{864 \frac{\text{giây}}{\text{khắc}_{\text{trước}}}}.$$
Sau triều đại Thanh, số giây trong một khắc là $$\frac{86400\ \text{giây}}{96\ \text{khắc}_{\text{sau}}} = \boxed{900 \frac{\text{giây}}{\text{khắc}_{\text{sau}}}}.$$
\begin{enumerate}
   \item[2] Tỉ lệ độ dài thời gian một khắc trước và sau là $$\frac{1\ \text{khắc}_{\text{trước}}}{1\ \text{khắc}_{\text{sau}}} = \frac{1\ \text{khắc}_{\text{trước}}}{1\ \text{khắc}_{\text{sau}}}\cdot \frac{864\ \text{giây}}{1\ \text{khắc}_{\text{trước}}}\cdot\frac{1\ \text{khắc}_{\text{sau}}}{900\ \text{giây}}=\boxed{0{,}96}.$$
\end{enumerate}

\exercise Một vòng đĩa tròn như trong hình \ref{fig:vong_dia} có đường kính $4{,}50$ cm rỗng ở giữa một lỗ đường kính $1{,}25$ cm. Đĩa dày $1{,}50$ mm. Biết rằng đĩa được làm từ chất liệu có khối lượng riêng là $8600\;\frac{\text{kg}}{\text{m}^3}$. Tính khôi lượng vòng đĩa theo gram.

\begin{figure}
   \centering
   \begin{tikzpicture}
      % Draw the shape
      \draw[bottom color=gray!20] (-2.25, 0) arc[start angle=180, end angle=360, x radius = 2.25cm, y radius = 2.4cm];
      \draw[bottom color=gray!70, top color=gray!20] (0, 0) circle (2.25cm);
      \draw[fill=white] (0, 0) circle (0.625cm);
      \draw[bottom color = gray!20] (-0.625, 0) arc[start angle=180, end angle=0, radius = 0.625cm];
      \draw[fill=white] (-0.625, 0) arc[start angle=180, end angle=0, x radius = 0.625cm, y radius = 0.5cm];
      % Draw the dimension
      \draw[<->] (-2.25,-2.5) -- (2.25,-2.5);
      \node at (0, -2.7) {$4{,}50$ cm};
      \draw[<->] (-0.625,0) -- (0.625,0);
      \node at (0, -0.2) {$1{,}25$ cm};
      \draw[<->] (0,0.5) -- (0,0.625);
      \node[anchor=west] at (-0.05, 0.7) {$1{,}50$ mm};
   \end{tikzpicture}
   \caption{Vòng đĩa tròn}
   \label{fig:vong_dia}
\end{figure}

\solution

Đặt $D=4{,}50\;\text{cm}=4{,}50\times 10^{-2}\ \text{m}$, $d=1{,}25\ \text{cm}=1{,}25\times 10^{-2}\;\text{m}$, $h=1{,}50\ \text{mm}=1{,}50\times 10^{-3}\ \text{m}$ và $\mathcal{D}=8600\;\frac{\text{kg}}{\text{m}^3}=8{,}6\times 10^3\;\frac{\text{kg}}{\text{m}^3}\cdot \frac{10^3\;\text{g}}{\text{kg}}=8{,}6\times 10^6\;\frac{\text{g}}{\text{m}^3}$.

Nhận thấy rằng đĩa có dạng trụ, diện tích mặt đáy là $$S=\pi\cdot \left(\frac{D}{2}\right)^2-\pi\cdot \left(\frac{d}{2}\right)^2=\frac{\pi \left(D^2-d^2\right)}{4}.$$

Thể tích của đĩa là $V=S\cdot h=\frac{\pi \cdot h\cdot \left(D^2-d^2\right)}{4}.$ Nhân với khối lượng riêng, chúng ta có khối lượng của đĩa là $$m=\mathcal{D}\cdot V=\frac{\pi \cdot h\cdot \mathcal{D}\cdot \left(D^2-d^2\right)}{4}.$$ Thay số trực tiếp với sự để ý đến số chữ số có nghĩa, chúng ta có kết quả $m=\boxed{1{,}89\times 10^1\ \text{kg}}$.

\exercise Khối lượng của một chất lỏng được mô hình hóa bởi phương trình $m=A\cdot t^{0{,}8}-B\cdot t$. Nếu như $t$ được tính bằng giây và $m$ được tính bằng ki-lô-gram, thì đơn vị của $A$ và $B$ là gì?

\solution

Để có thể cộng trừ các phần tử, chúng cần phải có cùng đơn vị. Do vậy, đơn vị của $A\cdot t^{0{,}8}$ và $B\cdot t$ là kg. Từ quy tắc nhân chia các đơn vị, chúng ta có:
\begin{equation*}
   \begin{cases}
     A\cdot \text{s}^{0{,}8} &=\text{kg} \\
     B\cdot\text{s} &=\text{kg}
   \end{cases}
   \iff
   \begin{cases}
      A &=\frac{\text{kg}}{\text{s}^{0{,}8}} \\
      B&=\frac{\text{kg}}{\text{s}}
   \end{cases}.
\end{equation*}

Vậy đơn vị của $A$ là $\boxed{\frac{\text{kg}}{\text{s}^{0{,}8}}}$ và đơn vị của $B$ là $\boxed{\frac{\text{kg}}{\text{s}}}$.

\chapter{Chuyển động}

\exercise Một ô tô đi $40$ km trên một đường thẳng với tốc độ không đổi $40\;\frac{\text{km}}{\text{h}}$. Sau đó, nó đi thêm theo chiều đó $60$ km với tốc độ không đổi $50\;\frac{\text{km}}{\text{h}}$. Các giá trị đo được tính đến hai chữ số có nghĩa.
\begin{enumerate}
   \item Tính vận tốc trung bình trên cả quãng đường.
   \item Tính tốc độ trung bình trên cả quãng đường.
   \item Nếu xe quay đầu trước khi đi $50$ km lúc sau, giữ nguyên các số liệu khác, thì vận tốc trung bình và tốc độ trung bình có thay đổi không. Tại sao?
   \item Vẽ đồ thị vị trí $x$ theo thời gian $t$ và từ đó chỉ ra cách tính vận tốc trung bình.
\end{enumerate}
\solution

Coi chiều chuyển động ban đầu là chiều dương.

\begin{enumerate}
   \item Thời gian đi $40$ km đầu là $$40\ \text{km}\div 40\ \frac{\text{km}}{\text{h}}=1{,}0\ \text{h}.$$
\end{enumerate}

Thời gian đi $50$ km sau là $$60\ \text{km}\div 50\ \frac{\text{km}}{\text{h}}=1{,}2\ \text{h}.$$

Do hai quãng đường là cùng chiều nên chúng ta có độ dịch chuyển của xe tổng cộng là $$\Delta x=40\ \text{km} + 60\ \text{km} = 100\ \text{km}$$ và tổng thời gian đi là $$\Delta t =1{,}0\ \text{h}+1{,}2\ \text{h}=2{,}2\ \text{h}.$$

Từ đó, chúng ta có vận tốc trung bình là $$\bar{v} = \frac{\Delta x}{\Delta t} =\boxed{4{,}5\times10^1\ \frac{\text{km}}{\text{h}}}.$$

\begin{enumerate}
   \item[2.] Dễ thấy tổng quãng đường đi là $d=100\ \text{km}$. Tốc độ trung bình là $\bar{s} = \frac{d}{\Delta t}=\boxed{4{,}5\times10^1\ \frac{\text{km}}{\text{h}}}.$
   \item[3.] Thời gian không thay đổi. Có độ dịch chuyển thay đổi còn $\Delta x = 40\ \text{km} - 60\ \text{km} = -20\ \text{km}$ nhưng tổng quãng đường thì không. Do đó, $\boxed{\text{tốc độ trung bình giữ nguyên}}$ nhưng $\boxed{\text{vận tốc trung bình thay đổi}}$.
   \item[4.] Ta có đồ thị ở hình \ref{fig:do_thi_xe} bằng việc vẽ mối quan hệ $x(t)$ xong nối điểm đầu và điểm cuối. Vận tốc trung bình là độ dốc của đường thẳng nối hai điểm này.
\end{enumerate}

\begin{figure}[h]
   \centering
   \begin{tikzpicture}[scale=1.2]
      \draw[->] (0,0) -- (7,0) node[right] {$t$ (h)};
      \draw[->] (0,0) -- (0,5.5) node[above] {$x$ (km)};
      \node[below] at (3.5,-0.5) {Thời gian};
      \node[rotate=90, above] at (-0.6,2.75) {Vận tốc};
      \draw (0,0) -- (3,2);
      \draw (3,2) -- (6.6,5);
      \filldraw (0,0) circle (1.5pt);
      \filldraw (3,2) circle (1.5pt);
      \filldraw (6.6,5) circle (1.5pt);
      \draw (0,0) -- (-0.08,0) node[left] {$0$};
      \draw (0,2) -- (-0.08,2) node[left] {$40$};
      \draw (0,5) -- (-0.08,5) node[left] {$100$};
      \draw (0,0) -- (0,-0.08) node[below] {$0$};
      \draw (3,0) -- (3,-0.08) node[below] {$1{,}0$};
      \draw (6.6,0) -- (6.6,-0.08) node[below] {$2{,}2$};

      \draw[dashed] (6.6,5) -- (6.6,0);
      \node[left] at (6.6,2.5) {$\Delta x = 100\ \text{km}$};
      \draw[dashed] (0,0) -- (6.6,0);
      \node[above] at (3.3,0) {$\Delta t = 2{,}2\ \text{h}$};
      \draw[ultra thick] (0,0) -- (6.6,5);
   \end{tikzpicture}
   \caption{Đồ thị vị trí xe-thời gian chạy}
   \label{fig:do_thi_xe}
\end{figure}

\exercise Một máy bay phản lực đang bay ngang ở độ cao $h=42$ mét. Đột nhiên nó bay vào vùng đất dốc lên góc $\theta=4{,}2^\circ$ (xem hình \ref{fig:may_bay_doc}). Với tốc độ bay là $v=1300\ \frac{\text{km}}{\text{h}}$, thời gian tính từ lúc bay vào vùng đất dốc mà người phi công có để điều chỉnh máy bay là bao nhiêu? Tất cả các số liệu được đo đến hai chữ số có nghĩa.

\begin{figure}[h]
   \centering
   \begin{tikzpicture}[scale=1.2]
      \draw (0,{7*sin(4.2)}) -- (7, 0);
      \draw (7, 0) -- (11, 0);
      \draw[dashed] (0, 0) -- (7, 0);
      \draw (5,0) arc[start angle=180, end angle=175.8, radius=2];

      \draw[->] (5.2,0.5) -- ({7+2*cos(175.8)},{2*sin(175.8)});
      \node[anchor=west] at (5.2,0.5) {$\theta=4{,}2^\circ$};

      \draw (7,2.5) -- (7.5,2.5) -- (7.5,2.7) -- (7,2.5);
      \filldraw[fill=black] (7.5,2.5) -- (8.1,2.5) -- (7.5,2.7) -- cycle;
      \draw (8.1,2.5) -- (8.3, 2.5) -- (8.3, 2.7) -- cycle;

      \draw[<->, dashed] (7,0) -- (7,2.5);
      \node[anchor=west] at (7,1.25) {$h = 42$ m};
      \draw[->] (7,2.5) -- (4,2.5);
      \node[anchor=south] at (5,2.5) {$v=1300\ \frac{\text{km}}{\text{h}}$};
   \end{tikzpicture}
   \caption{Vị trí máy bay trong vùng dốc lên}
   \label{fig:may_bay_doc}
\end{figure}

\solution

Khoảng cách từ máy bay đến điểm va chạm với mặt đất là $$d=\frac{h}{\tan{(\theta)}}.$$ Từ đó, chúng ta có được thời gian cho phép là $$t=\frac{d}{v}=\frac{h}{v\tan{(\theta)}}.$$

Thay số trực tiếp, với để ý đến sự quy đổi $v=1300\ \frac{\text{km}}{\text{h}}=1300\ \frac{\text{km}}{\text{h}}\frac{1000\ \text{m}}{1\ \text{km}}\frac{1\ \text{h}}{3600\ \text{s}}=361\ \frac{\text{m}}{\text{s}}$, chúng ta có $$t=\boxed{1{,}6\times 10^0\ s}.$$

\exercise Cho biết vị trí của một vật chuyển động thẳng được xác định bằng $x(t) = a\cdot t^2+b\cdot t+c$. Xác định vị trí, vận tốc và gia tốc của vật tại thời điểm $t=t_0$.

\solution

Vị trí của vật tại $t=t_0$ là $$x\left(t_0\right)=\boxed{a\cdot t_0^2+b\cdot t_0+c}.$$

Vận tốc của vật tại $t=t_0$ là $$v\left(t_0\right)=\left.\frac{\mathrm{d}x(t)}{\mathrm{d}t}\right|_{t=t_0}=\boxed{2a\cdot t_0+b}.$$

Gia tốc của vật tại $t=t_0$ là $$a\left(t_0\right)=\left.\frac{\mathrm{d}v(t)}{\mathrm{d}t}\right|_{t=t_0}=\boxed{2a}.$$

\exercise Phác họa đồ thị vị trí - thời gian và gia tốc thời gian của một người chạy bộ nếu đồ thị vận tốc - thời gian của người đó được biểu diễn trên đồ thị
\begin{enumerate}
   \item hình \ref{fig:chay_phan_a};
   \item hình \ref{fig:chay_phan_b}.
\end{enumerate}
Các số liệu được coi như chính xác tuyệt đối. Bạn có thể giả sử người đó bắt đầu chạy từ vị trí $x = 0$.

\begin{figure}[h]
   \centering
   \begin{minipage}[t]{0.48\textwidth}
      \centering
      \begin{tikzpicture}
         \draw[->] (0,0) -- (5.5,0) node[right] {$t$ (s)};
         \draw[->] (0,0) -- (0,4.5) node[above] {$v\left(\frac{\text{m}}{\text{s}}\right)$};
         \node[below] at (2.75,-0.5) {Thời gian};
         \node[rotate=90, above] at (-0.5,2.25) {Vận tốc};
         
         \draw[thick] (0,0) -- (1, 4);
         \draw[thick] (1,4) -- (2, 4);
         \draw[thick] (2,4) -- (4,3);
         \draw[thick] (4,3) -- (5, 0);

         \foreach \x/\y in {1/4, 2/4, 4/3} {
            \draw[dashed] (\x,0) -- (\x,\y);
         }
         \draw[dashed] (0,4) -- (1,4);
         \draw[dashed] (4,3) -- (0,3);
         \foreach \x in {0,1,2,4,5} {
            \draw (\x,0) -- (\x,-0.08) node[below] {$\x$};
         }
         \foreach \y in {0, 3, 4} {
            \draw (0,\y) -- (-0.08,\y) node[left] {$\y$};
         }
      \end{tikzpicture}
      \caption{Phần 1}
      \label{fig:chay_phan_a}
   \end{minipage}
   \hfill
   \begin{minipage}[t]{0.48\textwidth}
      \centering
      \begin{tikzpicture}
         \draw[->] (0,0) -- (5.5,0) node[right] {$t$ (s)};
         \draw[->] (0,0) -- (0,4.5) node[above] {$v\left(\frac{\text{m}}{\text{s}}\right)$};

         \node[below] at (2.75,-0.5) {Thời gian};
         \node[rotate=90, above] at (-0.5,2.25) {Vận tốc};
         
         \draw[domain=0:4, smooth, variable=\x, thick] plot ({\x}, {-\x*(4*\x^3-31*\x^2+77*\x-68)/6});
         \draw[thick] (4,0) -- (5,0);
         \foreach \x in {0,1,2,3,4,5} {
            \draw (\x,0) -- (\x,-0.080) node[below] {$\x$};
         }
         \foreach \y in {0,2,3,4} {
            \draw (0,\y) -- (-0.08,\y) node[left] {$\y$};
         }

         \foreach \x/\y in {1/3, 2/2, 3/4} {
            \draw[dashed] (\x,0) -- (\x,\y);
            \draw[dashed] (0,\y) -- (\x,\y);
         }
         
      \end{tikzpicture}
      \caption{Phần 2}
      \label{fig:chay_phan_b}
   \end{minipage}
\end{figure}

\solution

1. Ta chia quá trình chạy làm $4$ phần.


\begin{itemize}
   \item Phần 1 $\left(0\ \text{s}\leq t \leq 1\ \text{s}\right)$: Vận tốc tăng đều từ $0$ đến $4\ \frac{\text{m}}{\text{s}}$. Chuyển động là nhanh dần với gia tốc không đổi là $\left.a(t)\right|_{t\in\left[0\ \text{s};1\ \text{s}\right]}=\frac{v(1\ \text{s})-v(0\ \text{s})}{1\ \text{s}-0\ \text{s}}=4\ \frac{\text{m}}{\text{s}^2}$.
   
Sau khoảng thời gian $t$, độ dịch chuyển là $\left.x(t)\right|_{t\in\left[0\ \text{s};1\ \text{s}\right]} - x(0\ \text{s}) = \frac{\left.a(t)\right|_{t\in\left[0\ \text{s};1\ \text{s}\right]}\cdot t^2}{2} + \left.v(t)\right|_{t\in\left[0\ \text{s};1\ \text{s}\right]}\cdot t$. Từ đó chúng ta có $x(t) = 2\ \frac{\text{m}}{\text{s}^2}\cdot t^2$ với $0\ \text{s}\leq t \leq 1\ \text{s}$ và $x(1\ \text{s}) = 2\ \text{m}$.
   
   \item Phần 2 $\left(1\ \text{s}\leq t \leq 2\ \text{s}\right)$: Vận tốc không đổi ở $\left.v(t)\right|_{t\in\left[1\ \text{s};2\ \text{s}\right]} = 4\ \frac{\text{m}}{\text{s}}$ (chuyển động thẳng đều). 
   
Qua đó, chúng ta có $\left.x(t)\right|_{t\in\left[1\ \text{s};2\ \text{s}\right]} = x(1\ \text{s}) + \left.v(t)\right|_{t\in\left[1\ \text{s};2\ \text{s}\right]}\cdot\left(t - 1\ \text{s}\right) = 4\ \frac{\text{m}}{\text{s}}\cdot t - 2\ \text{m}$ và $x(2\ \text{s}) = 6\ \text{m}$.
\end{itemize}

Phần 3 $\left(2\ \text{s}\leq t \leq 4\ \text{s}\right)$ và phần 4 $\left(4\ \text{s}\leq t \leq 5\ \text{s}\right)$ làm tương tự như phần 1. Ta được
\begin{equation*}
   \begin{cases}
     \left.a(t)\right|_{t\in\left[2\ \text{s};4\ \text{s}\right]} &= -\frac{1}{2}\ \frac{\text{m}}{\text{s}^2}\\
     \left.a(t)\right|_{t\in\left[4\ \text{s};5\ \text{s}\right]} &= -3\ \frac{\text{m}}{\text{s}^2}\\
   \end{cases}
\end{equation*}
và qua đó
\begin{equation*}
   \begin{cases}
     \left.x(t)\right|_{t\in\left[2\ \text{s};4\ \text{s}\right]} &= -\frac{1}{4}\ \frac{\text{m}}{\text{s}^2}\cdot\left(t - 2\ \text{s}\right)^2 + 4\ \frac{\text{m}}{\text{s}}\cdot \left(t - 2\ \text{s}\right) + 6\ \text{m}\\
     \left.x(t)\right|_{t\in\left[4\ \text{s};5\ \text{s}\right]} &= -\frac{3}{2}\ \frac{\text{m}}{\text{s}^2}\cdot\left(t - 4\ \text{s}\right)^2 + 3\ \frac{\text{m}}{\text{s}}\cdot \left(t - 4\ \text{s}\right) + 13\ \text{m}\\
   \end{cases}
\end{equation*}

\begin{equation*}
         \iff
   \begin{cases}
     \left.x(t)\right|_{t\in\left[2\ \text{s};4\ \text{s}\right]} &= -\frac{1}{4}\ \frac{\text{m}}{\text{s}^2}\cdot t^2 + 5\ \frac{\text{m}}{\text{s}}\cdot t - 3\ \text{m}\\
     \left.x(t)\right|_{t\in\left[4\ \text{s};5\ \text{s}\right]} &= -\frac{3}{2}\ \frac{\text{m}}{\text{s}^2}\cdot t^2 + 15\ \frac{\text{m}}{\text{s}}\cdot t - 23\ \text{m}\\
   \end{cases}.
\end{equation*}

Cuối cùng, chúng ta có thể biểu diễn vị trí của người chạy trên đồ thị như hình \ref{fig:vt_tg1}.

\begin{figure}[h]
   \centering
   \fbox{
      \begin{tikzpicture}
         \draw[->] (0,0) -- (5.5,0) node[right] {$t$ (s)};
         \draw[->] (0,0) -- (0,4.5) node[above] {$x\left(\text{m}\right)$};
         \node[below] at (2.75,-0.5) {Thời gian};
         \node[rotate=90, above] at (-0.5,2.25) {Vị trí};
         
         \draw[domain=0:1, smooth, variable=\t, thick] plot ({\t}, {\t^2 / 2});
         \draw[domain=1:2, smooth, variable=\t, thick] plot ({\t}, {\t-1/2});
         \draw[domain=2:4, smooth, variable=\t, thick] plot ({\t}, {-1/16*(\t-2)^2+(\t-2)+3/2});
         \draw[domain=4:5, smooth, variable=\t, thick] plot ({\t}, {-3/8*\t^2+15/4*\t-23/4});
         \foreach \x/\y in {1/2, 2/6, 4/13, 5/14.5} {
            \draw[dashed] (\x,0) -- (\x,\y/4);
            \draw[dashed] (0,\y/4) -- (\x,\y/4);
         }
         \foreach \x in {0,1,2,4,5} {
            \draw (\x,0) -- (\x,-0.08) node[below] {$\x$};
         }
         \foreach \y in {0, 2, 6, 13, 14.5} {
            \draw (0,\y/4) -- (-0.08,\y/4) node[left] {$\y$};
         }
      \end{tikzpicture}
   }
   \caption{Đồ thị vị trí - thời gian cho phần 1}
   \label{fig:vt_tg1}
\end{figure}

2. Chúng ta có thể phác họa đồ thị vị trí - thời gian bằng việc xấp xỉ đồ thị vận tốc - thời gian dưới dạng đường gấp khúc nối các điểm đã biết thể hiện ở \ref{fig:xx_p2}.

Từ đây, thực hiện tương tự như phần 1 để có phương trình vị trí - thời gian
\begin{equation*}
   x(t) = \begin{cases}
      \frac{3}{2}\ \frac{\text{m}}{\text{s}^2}\cdot t^2 &\quad \text{với } 0 \leq t < 1\ \text{s}\\
      -\frac{1}{2}\ \frac{\text{m}}{\text{s}^2}\cdot t^2 + 4\ \frac{\text{m}}{\text{s}}\cdot t - 2\ \text{m}&\quad \text{với } 1\ \text{s} \leq t < 2\ \text{s}\\
      1\ \frac{\text{m}}{\text{s}^2}\cdot t^2 - 2\ \frac{\text{m}}{\text{s}}\cdot t + 4\ \text{m}&\quad \text{với } 2\ \text{s} \leq t < 3\ \text{s}\\
      -2\ \frac{\text{m}}{\text{s}^2}\cdot t^2+16\ \frac{\text{m}}{\text{s}}\cdot t-23\ \text{m}&\quad \text{với } 3\ \text{s} \leq t < 4\ \text{s}\\
      9\ \text{m}&\quad \text{với } 4\ \text{s} \leq t \leq 5\ \text{s}
   \end{cases}
\end{equation*}
và chúng ta vẽ được đồ thị ở hình \ref{fig:vttgxxp2}.

\begin{figure}[h]
   \centering
   \begin{minipage}[t]{0.48\textwidth}
      \centering
      \begin{tikzpicture}
         \draw[->] (0,0) -- (5.5,0) node[right] {$t$ (s)};
         \draw[->] (0,0) -- (0,5) node[above] {$v\left(\frac{\text{m}}{\text{s}}\right)$};

         \node[below] at (2.75,-0.5) {Thời gian};
         \node[rotate=90, above] at (-0.5,2.25) {Vận tốc};
         
         \draw[thick] (0,0) -- (1,3) -- (2,2) -- (3,4) -- (4,0) -- (5,0);
         \foreach \x in {0,1,2,3,4,5} {
            \draw (\x,0) -- (\x,-0.080) node[below] {$\x$};
         }
         \foreach \y in {0,2,3,4} {
            \draw (0,\y) -- (-0.08,\y) node[left] {$\y$};
         }

         \foreach \x/\y in {1/3, 2/2, 3/4} {
            \draw[dashed] (\x,0) -- (\x,\y);
            \draw[dashed] (0,\y) -- (\x,\y);
         }
      \end{tikzpicture}
      \caption{Vận tốc - thời gian xấp xỉ của phần 2}
      \label{fig:xx_p2}
   \end{minipage}
   \hfill
   \begin{minipage}[t]{0.48\textwidth}
      \centering
      \fbox{
         \begin{tikzpicture}
            \draw[->] (0,0) -- (5.5,0) node[right] {$t$ (s)};
            \draw[->] (0,0) -- (0,5) node[above] {$x\left(\text{m}\right)$};
            \node[below] at (2.75,-0.5) {Thời gian};
            \node[rotate=90, above] at (-0.5,2.25) {Vị trí};
            
            \draw[domain=0:1, smooth, variable=\t, thick] plot ({\t}, {(3*\t^2 / 2) / 2});
            \draw[domain=1:2, smooth, variable=\t, thick] plot ({\t}, {(-\t^2 / 2 + 4*\t - 2)/2});
            \draw[domain=2:3, smooth, variable=\t, thick] plot ({\t}, {(\t^2 -2*\t +4)/2});
            \draw[domain=3:4, smooth, variable=\t, thick] plot ({\t}, {(-2*\t^2+16*\t-23)/2});
            \draw[thick] (4,4.5) -- (5,4.5);

            \foreach \x/\y in {1/1.5, 2/4, 3/7, 4/9} {
               \draw[dashed] (\x,0) -- (\x,\y/2);
               \draw[dashed] (0,\y/2) -- (\x,\y/2);
            }
            \draw[dashed] (5,0) -- (5,4.5);
            \foreach \x in {0,1,2,3,4,5} {
               \draw (\x,0) -- (\x,-0.08) node[below] {$\x$};
            }
            \foreach \y in {0, 1.5, 4, 7, 9} {
               \draw (0,\y/2) -- (-0.08,\y/2) node[left] {$\y$};
            }
         \end{tikzpicture}
      }
      \caption{Vị trí - thời gian (xấp xỉ) cho phần 2}
      \label{fig:vttgxxp2}
   \end{minipage}
\end{figure}

\begin{figure}[h!]
   \centering
   \fbox{
      \begin{tikzpicture}
         \draw[->] (0,0) -- (5.5,0) node[right] {$t$ (s)};
         \draw[->] (0,0) -- (0,6) node[above] {$v\left(\frac{\text{m}}{\text{s}}\right)$};

         \node[below] at (2.75,-0.5) {Thời gian};
         \node[rotate=90, above] at (-0.5,2.25) {Vận tốc};
         
         \draw[domain=0:4, smooth, variable=\x, thick] plot ({\x}, {-\x^2*(48*\x^3-465*\x^2+1540*\x-2040)/720});
         \draw[thick] (4,{496/90}) -- (5,{496/90});
         \foreach \x in {0,1,2,3,4,5} {
            \draw (\x,0) -- (\x,-0.080) node[below] {$\x$};
         }
         \draw (0,{496/90}) -- (-0.08,{496/90}) node[left] {$\approx 11$};

         \foreach \x/\y in {4/{496/45}, 5/{496/45}} {
            \draw[dashed] (\x,0) -- (\x,{\y/2});
         }
         \draw[dashed] (0,{496/90}) -- (4,{496/90});
      \end{tikzpicture}
   }
   \caption{Đồ thị vị trí - thời gian cho phần 2}
   \label{fig:vttgp2}
\end{figure}

Trong thực tiễn, chúng ta hay xấp xỉ những quá trình không tuyến tính qua hữu hạn những điểm đo rồi nội suy tuyến tính (nối các điểm bằng các đoạn thẳng) như đã làm. Còn nhiều phương pháp nội suy nữa còn có thể được tìm thấy trong những tài liệu về phương pháp tính và giải tích số. Thông thường, với càng nhiều điểm thì độ chính xác càng lớn.

Trong trường hợp mà bạn nhận ra phương trình vận tốc - thời gian được cho là
\begin{equation*}
   v(t) =
   \begin{cases}
      \displaystyle \frac{\displaystyle -t\left(4\ \frac{\text{m}}{\text{s}^5}\cdot t^3-31\ \frac{\text{m}}{\text{s}^4}\cdot t^2+77\ \frac{\text{m}}{\text{s}^3}\cdot t-68\ \frac{\text{m}}{\text{s}^2}\right)}{6} &\quad \text{với } 0 \leq t < 4 \\
      0&\quad \text{với } 4 \leq t \leq 5
   \end{cases}
\end{equation*}
thì bạn có thể thực hiện nguyên hàm trên hàm này để tính được vị trí vật là
\begin{equation*}
   \displaystyle 
   x(t) =
   \begin{cases}
      \displaystyle \frac{\displaystyle -t^2\left(48\ \frac{\text{m}}{\text{s}^5}\cdot t^3-465\ \frac{\text{m}}{\text{s}^4}\cdot t^2+1540\ \frac{\text{m}}{\text{s}^3}\cdot t-2040\ \frac{\text{m}}{\text{s}^2}\right)}{360} &\quad \text{với } 0 \leq t < 4 \\
      \displaystyle \frac{496}{45}\ \text{m}&\quad \text{với } 4 \leq t \leq 5
   \end{cases}
\end{equation*}
và chúng ta có đồ thị như hình \ref{fig:vttgp2}.

\exercise Hai xe hơi có tốc độ lần lượt là $v_1 = 50{,}0\ \frac{\text{km}}{\text{h}}$ và $v_2 = 60{,}0\ \frac{\text{km}}{\text{h}}$ đi ngược chiều với nhau trên một con đường hẹp. Hai xe phát hiện lẫn nhau khi khoảng cách giữa hai xe là $d = 400\ \text{m}$. Cả hai xe đồng thời giảm tốc với cùng một gia tốc hãm đều là $a$. Tính giá trị tối thiểu của $a$ nếu biết hai xe không xảy ra va chạm. Số liệu được đo tới $3$ chữ số có nghĩa.

\solution

Gọi quãng đường đi được trong khi hãm phanh của hai xe lần lượt là $d_1$ và $d_2$.

Trong quá trình hãm đến vận tốc bằng $0$, tổng quãng đường đi của cả hai xe phải không vượt quá khoảng cách $d$. Vì vậy, chúng ta có bất đẳng thức $$d_1 + d_2 \leq d.$$

Trong khi đó, quãng đường xe thứ nhất đã di chuyển là $d_1 = \frac{0^2 - v_1^2}{2(-a)} = \frac{v_1^2}{2a}$. Tương tự, chúng ta có quãng đường mà xe thứ hai di chuyển trong khoảng thời gian này là $d_2 = \frac{v_2^2}{2a}$. Từ đó, thay vào phương trình ở trên để được $$
   \frac{v_1^2}{2a} + \frac{v_2^2}{2a} \le d
   \iff a \geq \frac{v_1^2+v_2^2}{2d}.
$$

Thay số trực tiếp, chúng ta có gia tốc hãm tối thiểu phải là $\boxed{7{,}63 \times 10^3 \frac{\text{km}}{\text{h}^2}}$.

\exercise Để dừng xe ban đầu bạn cần một thời gian phản ứng để bắt đầu phanh, rồi xe mới đi chậm dần nhờ có một gia tốc hãm không đổi. Giả sử quãng được đi được trong hai pha này là $186$ ft nếu vận tốc ban đầu là $50\ \frac{\text{dặm}}{\text{h}}$. Còn trong một trường hợp khác, quãng được đi được trong hai pha này là $80$ ft nếu vận tốc ban đầu là $30\ \frac{\text{dặm}}{\text{h}}$. Biết thời gian phản ứng là cố định và $1$ dặm $= 5280$ ft, tính thời gian phản ứng và độ lớn của gia tốc hãm.

\solution

Gọi thời gian phản ứng là $t_p$, vận tốc đầu là $v_0$, gia tốc hãm là $a$.

Trong khoảng thời gian phản ứng, xe đi được $v_0t_p$. Và trong khoảng thời gian hãm, xe đi được $\frac{0^2-v_0^2}{2(-a)}=\frac{v_0^2}{2a}$. Cho nên, tổng quãng đượt đi được trong hai pha là 
\begin{equation}
\Delta x = v_0 t + \frac{v_0^2}{2a}
\label{eq:stopping_distance}
\end{equation}

Trước khi thay số, thực hiện quy đổi $$50\ \frac{\text{dặm}}{\text{h}}=50\ \frac{\text{dặm}}{\text{h}}\cdot\frac{5280\ \text{ft}}{1\ \text{dặm}}\cdot\frac{1\ \text{h}}{3600\ \text{s}}\approx 73\ \frac{\text{ft}}{\text{s}},$$ tương tự, $30\ \frac{\text{dặm}}{\text{h}}=44\ \frac{\text{ft}}{\text{s}}$. Từ đó, thay số vào phương trình \ref{eq:stopping_distance} để có hệ
\begin{equation*}
   \begin{cases}
      186\ \text{ft} = 73\ \frac{\text{ft}}{\text{s}}\cdot t_p + \frac{\left(73\ \frac{\text{ft}}{\text{s}}\right)^2}{2a} \\
      80\ \text{ft} = 44\ \frac{\text{ft}}{\text{s}}\cdot t_p + \frac{\left(44\ \frac{\text{ft}}{\text{s}}\right)^2}{2a} 
   \end{cases}.
\end{equation*}
Giải hệ phương trình, chúng ta có thời gian phản ứng là $t_p=\boxed{0{,}97\ \text{s}}$ và gia tốc hãm là $a = \boxed{26\ \frac{\text{ft}}{\text{s}^2}}$.

\begin{wrapfigure}{R}{0.6\textwidth}
    \centering
    \begin{tikzpicture}
      \draw[->] (0,0) -- (8,0) node[right] {$t$};
      \draw[->] (0,0) -- (0,6) node[above] {$h$};

      \node[below] at (4,-0.25) {Thời gian};
      \node[rotate=90, above] at (-0.25,3) {Độ cao};

      \draw[domain=0.8:7.2, smooth, variable=\x, thick] plot ({\x}, {-4 * (\x - 1) * (\x - 7)/9 + 1});
      \draw[<->] (1,1) -- (7,1);
      \node[anchor=south] at (4,1) {$\Delta T_t$};
      \draw[<->] (3,41/9) -- (5,41/9);
      \node[anchor=north] at (4,41/9) {$\Delta T_c$};
      \draw[<->] (0.5,1) -- (0.5,41/9);
      \node[anchor=west] at (0.5,25/9) {$H$};

      \draw[dashed] (0,1) -- (1,1);
      \draw[dashed] (7,1) -- (8,1);
      \draw[dashed] (0,41/9) -- (3,41/9);
      \draw[dashed] (5,41/9) -- (8,41/9);

   \end{tikzpicture}

   \caption{Đồ thị thời gian - độ cao của quả bóng thủy tinh}
   \label{fig:tgdcqbtt}
\end{wrapfigure}

\exercise Tại Phòng Thí nghiệm Vật lí Quốc gia ở Anh, người ta thực hiện xác định gia tốc trọng trường $g$ theo thí nghiệm sau: Ném một quả bóng thủy tinh lên theo chiều thẳng đứng trong ống chân không và cho nó rơi xuống. Gọi $\Delta T_t$ trên hình \ref{fig:tgdcqbtt} là thời gian khoảng giữa hai lần quả bóng đi qua một điểm thấp nào đó. $\Delta T_c$ là khoảng thời gian giữa hai lần quả bóng đi qua một điểm cao hơn và $H$ là khoảng cách giữa hai điểm. Chứng minh rằng $$g=\frac{8H}{\Delta T_t^2 - \Delta T_c^2}.$$

\solution

Gọi vận tốc khi bóng bắt đầu bay lên từ vị trị thấp là $v_0$. Sau một khoảng thời gian $\Delta T_t$, quả bóng quay lại vị trí cũ, do vậy, chúng ta có phương trình $0 = -\frac{g \Delta T_t^2}{2} + v_0 \Delta T_t$. Thực hiện biến đổi tương đương để có $$v_0=\frac{g \Delta T_t}{2}.$$

Nhận thấy rằng đồ thị có tính đối xứng. Sử dụng điều đó, chúng ta tính được khoảng thời gian quả bóng lên một độ cao $H$ là $t=\frac{\Delta T_t-\Delta T_c}{2}$. Qua đó, có được phương trình thứ hai là $$H = -\frac{g t^2}{2} + v_0 t=-\frac{g \left(\frac{\Delta T_t-\Delta T_c}{2}\right)^2}{2} + v_0 \left(\frac{\Delta T_t-\Delta T_c}{2}\right).$$

Thế giá trị của $v_0$ vào phương trình và tiếp tục thực hiện biến đổi, chúng ta có:
\begin{align*}
   H &= -\frac{g \left(\Delta T_t-\Delta T_c\right)^2}{8} + \frac{g \Delta T_t}{2} \left(\frac{\Delta T_t-\Delta T_c}{2}\right) \\
   &= -g\left(\frac{\Delta T_t^2}{8} - \frac{\Delta T_t\Delta T_c}{4} + \frac{\Delta T_c^2}{8}\right) + g\left(\frac{\Delta T_t^2}{4} - \frac{\Delta T_t\Delta T_c}{4}\right) \\
   &= g\cdot \frac{\Delta T_t^2 - \Delta T_c^2}{8} \\
   \iff g &= \frac{8H}{\Delta T_t^2 - \Delta T_c^2}.
\end{align*}

Ta có điều phải chứng minh.

\exercise Một nghệ sĩ tung hứng các quả bóng lên theo phương thẳng đứng. Quả bóng sẽ lên cao hơn bao nhiêu nếu thời gian bóng trong không khí tăng gấp $n$ lần ($n \in \mathbb{R}^+$)?

\solution

Có thời gian để quả bóng bay từ tay lên trên vị trí cao nhất bằng một nửa thời gian bóng trong không khí. Nếu thời gian bóng trong không khí tăng gấp $n$ lần so với thời gian trong không khí gốc, thì cùng chia cho $2$, chúng ta cũng sẽ có thời gian bóng bay từ tay lên trên vị trí cao nhất cũng tăng gấp $n$ lần so với thời gian gốc để bay lên vị trí cao nhất.

Gọi $t_1$ là thời gian gốc để bóng bay từ tay lên vị trí cao nhất, $t_2 = n t_1$ là thời gian bay khi đã tăng $n$ lần. Gọi $h_1, h_2$ lần lượt là độ cao bóng đi được tương ứng với hai khoảng thời gian $t_1, t_2$. Để ý rằng khi lên vị trí cao nhất thì vận tốc bóng là $0$; chúng ta có hệ phương trình

\begin{equation*}
   \begin{cases}
      h_1 &= \frac{gt_1^2}{2} \\
      h_2 &= \frac{gt_2^2}{2} = \frac{g\left(nt_1\right)^2}{2}
   \end{cases}
   \implies h_2 = n^2 h_1.
\end{equation*}

Từ đó, quả bóng cao lên hơn được $\boxed{n^2 - 1 \text{ lần độ cao gốc}}$.

\begin{wrapfigure}{L}{0.4\textwidth}
   \centering
   \begin{tikzpicture}
      \draw[->] (-0.5, 0) -- (4,0) node[right] {$x$};
      \draw[->] (0, -0.5) -- (0, 3) node[above] {$y$};
      \filldraw (0, 0) circle (1.5pt);
      \draw[-{Latex[width=1.5mm]}] (0, 0) -- (1.3, 0.8) node[above left] {$\vec{v_1}$};
      \draw[-{Latex[width=1.5mm]}] (3, 2) -- (2, 2);
      \node[above] at (2.5,2) {$\vec{u}$};
      \draw (0.5,0) arc[start angle=0, end angle={atan(8/13)}, radius=0.5];
      \node at (0.72,0.20) {$\alpha$};
      
      \node[below left] at (0, 0) {$O$};
      
   \end{tikzpicture}

   \caption{Hình minh họa cho bài \ref{ex:16}}
   \label{fig:mhb16}
\end{wrapfigure}

\exercise[ex:16] Như trong hình \ref{fig:mhb16}, một vật nhỏ có khối lượng $m$ chỉ di chuyển từ gốc $O$ trong mặt phẳng $Oxy$ được cung cấp một vận tốc ban đầu $\overrightarrow{v_1}$ trong vùng không gian có gió thổi với vận tốc $\vec{u} = -u \overrightarrow{e_x}$.



\begin{thebibliography}{1}
\bibitem{Agarwal2011}
Agarwal, R.P., Perera, K., Pinelas, S. (2011). \textit{History of Complex Numbers}. In: An Introduction to Complex Analysis. Springer, Boston, MA. \url{https://doi.org/10.1007/978-1-4614-0195-7_50}
\end{thebibliography}

\end{document}

