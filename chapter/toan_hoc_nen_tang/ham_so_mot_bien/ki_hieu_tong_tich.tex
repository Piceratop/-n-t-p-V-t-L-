\subsection{Kí hiệu tổng và tích của nhiều số}

\ % Lùi đầu dòng

Cho hàm số $f(x)$. Khi cho $x$ là số nguyên chạy từ $a$ đến $b$ (thông thường $a \le b$), chúng ta có tổng của các giá trị $f(x)$ được viết rút gọn là
$$
\defMath{\sum_{x=a}^{b} \left(f(x)\right) = f(a) + f(a + 1) + \cdots + f(b)}.
$$

Ví dụ, nếu cho $f(x) = 3x - 7$, thì
\begin{align*}
   \sum_{x = -1}^3 \left(f(x)\right) &= f(-1) + f(0) + f(1) + f(2) + f(3) \\
   &= \left(3(-1) - 7\right) + \left(3(0) - 7\right) + \left(3(1) - 7\right) + \left(3(2) - 7\right) + \left(3(3) - 7\right) \\
   &= -20.
\end{align*}

Đối với tích của các giá trị $f(x)$, chúng ta có
$$
\defMath{\prod_{x=a}^{b} \left(f(x)\right) = f(a) \times f(a + 1) \times \cdots \times f(b)}.
$$

Ví dụ, cùng với $f(x) = 3x - 7$, chúng ta có
\begin{align*}
   \prod_{x = -1}^3 \left(f(x)\right) &= f(-1) \times f(0) \times f(1) \times f(2) \times f(3) \\
   &= \left(3(-1) - 7\right) \times \left(3(0) - 7\right) \times \left(3(1) - 7\right) \times \left(3(2) - 7\right) \times \left(3(3) - 7\right) \\
   &= 560.
\end{align*}

Mở rộng kí hiệu, nếu chúng ta có cần tính tổng hay tích vào hàm phụ thuộc vào các giá trị $x$ thỏa mãn điều kiện $P$ nào đó, thì chúng ta có thể viết
$$
\defMath{\sum_{P} \left(f(x)\right) \text{ và }\prod_{P} \left(f(x)\right)}.
$$

$P$ có thể được viết theo nhiều kiểu khác nhau, miễn hiểu là được. Ví dụ, thay vì tính tổng và tích của $f(x)$ khi $x$ thay đổi từ $-1$ đến $3$, chúng ta có thể tính tổng và tích của $f(x)$ với $x$ là các số nguyên tố lớn hơn $10$ và nhỏ hơn $20$. Khi đó,

{
   \begin{minipageindent}{0.48\textwidth}
      \begin{align*}
         &\sum\limits_{x \text{ là số nguyên tố lớn hơn } 10 \text{ và nhỏ hơn } 20} \left(f(x)\right) \\
         = &f(11) + f(13) + f(17) + f(19) \\
         = &152\text{, và}
      \end{align*}
   \end{minipageindent}
   \hfill
   \begin{minipageindent}{0.48\textwidth}
      \begin{align*}
         &\prod\limits_{x \text{ là số nguyên tố lớn hơn } 10 \text{ và nhỏ hơn } 20} \left(f(x)\right) \\
         = &f(11) \times f(13) \times f(17) \times f(19) \\
         = &1830400.
      \end{align*}
   \end{minipageindent}
}
