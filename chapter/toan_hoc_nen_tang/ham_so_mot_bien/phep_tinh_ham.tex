\section{Phép tính trên hàm}

\subsection{Phép tính đại số thông thường}

Giống như khi chúng ta làm những phép công, trừ, nhân, chia với số, chúng ta cũng có thể thực hiện các phép tính đại số đó lên hàm số thực. Với kí hiệu tương tự, nếu có hai hàm $f$ và $g$ thì định nghĩa các hàm $f+g$, $f-g$, $f\cdot g$, $\frac{f}{g}$ như sau:
\begin{align*}
    (f + g)(x) &= f(x) + g(x);\\
    (f - g)(x) &= f(x) - g(x);\\
    (f \cdot g)(x) &= f(x) \cdot g(x);\\
    \left(\frac{f}{g}\right)(x) &= \frac{f(x)}{g(x)} \text{ với } g(x) \neq 0.
\end{align*}
Khi này, tập xác định của hàm mới sẽ là tập hợp các giá trị $x$ thuộc tập xác định của cả $f$ và $g$. Trong trường hợp $\frac{f}{g}$, cần loại bỏ các trường hợp khiến cho $g(x) = 0$.

