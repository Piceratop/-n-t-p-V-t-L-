\section{Phép tính trên hàm}

\subsection{Phép tính đại số thông thường}

Giống như khi chúng ta làm những phép công, trừ, nhân, chia với số, chúng ta cũng có thể thực hiện các phép tính đại số đó lên hàm số thực. Với kí hiệu tương tự, nếu có hai hàm $f$ và $g$ thì định nghĩa các hàm $f+g$, $f-g$, $f\cdot g$\footnote{Hạn chế viết \dblquote{$fg$} do có khả năng nhầm lẫn với hàm có tên là \dblquote{fg}. Trong tương lai, hàm có thể có nhiều kí tự trong tên như hàm $\cos$ hay hàm $\tan$.}, $\frac{f}{g}$ như sau:
\begin{align*}
   (f + g)(x) &= f(x) + g(x);\\
   (f - g)(x) &= f(x) - g(x);\\
   (f \cdot g)(x) &= f(x) \cdot g(x);\\
   \left(\frac{f}{g}\right)(x) &= \frac{f(x)}{g(x)} \text{ với } g(x) \neq 0.
\end{align*}
Khi này, tập xác định của hàm mới sẽ là tập hợp các giá trị $x$ thuộc tập xác định của cả $f$ và $g$. Trong trường hợp $\frac{f}{g}$, cần loại bỏ các giá trị $x$ khiến cho $g(x) = 0$.

\def \theF {\textit{世}}

\exercise Xác định tập xác định và tập giá trị của hàm $\theF$ nếu biết
\begin{enumerate}
   \item $f(x) = 4 - 7x$, $g(x) = 2x - 5$ và $\theF(x) = (f + g)(x)$ với mọi $x$ thực;
   \item 
   \begin{tabular}{|c|c|c|c|c|}
      \hline
      $x$ & $-2$ & $0$ & $1$ & $2$ \\
      \hline
      $f(x)$ & $9$ & $1$ & $-5$ & $-4$ \\
      \hline
      $h(x)$ & $7$ & $7$ & $0$ & $3$ \\
      \hline
   \end{tabular} và $\theF(x) = (f - h)(x)$. $f$ và $h$ chỉ nhận các giá trị ở trong bảng.
   \item $f(x) = 4 - 7x$, $g(x) = 2x - 5$ và $\theF(x) = (f \cdot g)(x)$ với mọi $x$ thực;
   \item 
   \begin{tabular}{|c|c|c|c|c|}
      \hline
      $x$ & $-2$ & $0$ & $1$ & $2$ \\
      \hline
      $f(x)$ & $9$ & $1$ & $-5$ & $-4$ \\
      \hline
      $h(x)$ & $7$ & $7$ & $0$ & $3$ \\
      \hline
   \end{tabular} và $\theF(x) = \left(\frac{f}{h}\right)(x)$. $f$ và $h$ chỉ nhận các giá trị ở trong bảng.
\end{enumerate}

\solution

1.
$$
\theF(x) = (f + g)(x) = f(x) + g(x) = (4 - 7x) + (2x - 5) = -5x - 1.
$$

Do $f(x)$ và $g(x)$ đều có tập xác định là $\mathbb{R}$ nên tập xác định của $\theF(x)$ cũng là $\boxed{\mathbb{R}}$. Để ý rằng, mọi $y \in \mathbb{R}$ đều có thể là giá trị của $\theF(x)$ do $$
   \theF\left(\frac{y + 1}{-5}\right) = -5\left(\frac{y + 1}{-5}\right) - 1 = y + 1 - 1 = y.
$$ Qua đó, tập giá trị của $\theF(x)$ là $\boxed{\mathbb{R}}$.

2. Kẻ bảng kết hợp với hàm $\theF$, chúng ta có:

\begin{table}[H]
   \centering
   \begin{tabular}{|c|c|c|c|c|}
      \hline
      $x$ & $-2$ & $0$ & $1$ & $2$ \\
      \hline
      $f(x)$ & $9$ & $1$ & $-5$ & $-4$ \\
      \hline
      $h(x)$ & $7$ & $7$ & $0$ & $3$ \\
      \hline
      $\theF(x) = (f - h)(x) = f(x) - h(x)$ & $2$ & $-6$ & $-5$ & $-7$ \\
      \hline
   \end{tabular}
   \caption{Bảng kết hợp với hàm $\theF= (f - h)(x)$ của phần 2}
   \label{tab:ham_so_mot_bien:phep_tinh_ham:theF_2}
\end{table}

Để ý rằng do $f(x)$ và $g(x)$ đều chỉ nhận đầu vào là $\left\{-2; 0; 1; 2\right\}$ nên tập xác định của $\theF(x)$ cũng là $\boxed{\left\{-2; 0; 1; 2\right\}}$. Ngoài ra, theo bảng \ref{tab:ham_so_mot_bien:phep_tinh_ham:theF_2}, tập giá trị của $\theF(x)$ là $\boxed{\left\{-6; -5; -7; 2\right\}}$.

3.
$$
\theF(x) = (f \cdot g)(x) = f(x) \cdot g(x) = (4 - 7x) \cdot (2x - 5) = -14x^2 + 37x - 20.
$$

Do $f(x)$ và $g(x)$ đều có tập xác định là $\mathbb{R}$ nên tập xác định của $\theF(x)$ cũng là $\boxed{\mathbb{R}}$.

Mặt khác, chúng ta có:
\begin{align*}
   \theF(x) &= -14x^2 + 37x - 20\\
   &= -14\left(x^2 - \frac{37}{14}x\right) - 20\\
   &= -14\left(x^2 - \frac{37}{14}x + \left(\frac{37}{28}\right)^2 - \left(\frac{37}{28}\right)^2\right) - 20\\
   &= -14\left(x - \frac{37}{28}\right)^2 + \frac{249}{56}.
\end{align*}
Suy ra, $\theF(x) \ge \frac{249}{56}$ với mọi $x \in \mathbb{R}$. Ngược lại, với mọi $y \ge \frac{249}{56}$, tồn tại $x \in \mathbb{R}$ sao cho $\theF(x) = y$ với ví dụ là $\displaystyle x=\frac{\sqrt{249-56y}+37}{28}$. Vậy, tập giá trị của $\theF(x)$ là $\boxed{\left[\frac{249}{56}; +\infty\right)}$.

4. Kẻ bảng kết hợp với hàm $\theF$, chúng ta có:

\begin{table}[H]
   \centering
   \begin{tabular}{|c|c|c|c|c|}
      \hline
      $x$ & $-2$ & $0$ & $1$ & $2$ \\
      \hline
      $f(x)$ & $9$ & $1$ & $-5$ & $-4$ \\
      \hline
      $h(x)$ & $7$ & $7$ & $0$ & $3$ \\
      \hline
      $\theF(x) = \left(\frac{f}{h}\right)(x) = \frac{f(x)}{h(x)}$ & $\frac{9}{7}$ & $\frac{1}{7}$ & Không xác định & $-\frac{4}{3}$ \\
      \hline
   \end{tabular}
   \caption{Bảng kết hợp với hàm $\theF = \left(\frac{f}{h}\right)(x)$ của phần 4}
   \label{tab:ham_so_mot_bien:phep_tinh_ham:theF_4}
\end{table}

Khi xác định tập giá trị của $\theF(x)$ cần phải để ý điều kiện $h(x) \neq 0$. Qua bảng \ref{tab:ham_so_mot_bien:phep_tinh_ham:theF_4}, ta thấy $h(x) = 0$ khi $x=1$. Vậy, tập xác định của $\theF(x)$ là $\boxed{\left\{-2; 0; 2\right\}}$. Theo bảng, chúng ta cũng có tập giá trị của $\theF(x)$ được xác định là $\boxed{\left\{\frac{9}{7}; \frac{1}{7}; -\frac{4}{3}\right\}}$.

