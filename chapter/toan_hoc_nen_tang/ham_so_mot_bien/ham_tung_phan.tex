\subsection{Hàm số xác định từng phần}

\ % Lùi đầu dòng

Không phải lúc nào hàm số trong đời sống có thể biểu diễn dưới dạng một biểu thức. Khi này, chúng ta sẽ chia nhỏ đồ thị của hàm số thành các phần nhỏ, và biểu diễn từng phần thông qua biểu thức. Đó cũng là lí do cho tên gọi \defText{hàm số xác định từng phần}.

\exercise Phác thảo đồ thị của những hàm sau:

\begin{multicols}{2}
   \begin{enumerate}
      \item $f(x) = \begin{cases}
         x + 1 \text{ nếu } x \leq 1 \\
         2 \text{ nếu } x > 1
      \end{cases}$;
      \item $f(x) = \begin{cases}
         x^3 + 4 \text{ nếu } x < 0 \\
         -x^2 + 1 \text{ nếu } x \geq 0
      \end{cases}$;
      \item $f(x) = \begin{cases}
         -\frac{4}{x^2} \text{ nếu } -2 > x \geq -3 \\
         \parbox{0.29\textwidth}{$\begin{array}{cl}
            -\frac{5}{x^2 + 1} &\text{nếu } x \geq -2 \text{ thực để} \\
            &\frac{5}{x^2 + 1}\text{ là số nguyên}
         \end{array}$}
      \end{cases}$;
      \item $f(x) = \begin{cases}
         \frac{2x - 1}{x - 1} \text{ nếu } -3 \leq x < 0 \\
         \left(x + 1\right)^2 - 3x \text{ nếu } 0 \leq x < 2 \\
         \frac{2x - 1}{x - 1} \text{ nếu } 2 \leq x \leq 3
      \end{cases}$;
      \item $f(x) = \begin{cases}
         x^3 + 3 \text{ nếu } x \leq 0 \\
         -2x + 2 \text{ nếu } 0 < x < 1 \\
         4 + x - x^2 \text{ nếu } x \geq 1
      \end{cases}$.
   \end{enumerate}
\end{multicols}

\solution

1.

\begin{figure}[H]
	\centering
	\begin{tikzpicture}
		\draw[->] (-4, 0) -- (4, 0) node[right] {$x$};
		\draw[->] (0, -4) -- (0, 4)  node[above] {$f(x)$};
		\draw[graph thickness, samples=80, color=colorEmphasisCyan, domain=-4.000:1] plot (\x, {(((\x)/1) + 1) / 1});
		\draw[graph thickness, samples=80, color=colorEmphasisCyan, domain=1:4] plot (\x, 2);
		\filldraw[color=colorEmphasisCyan] (-3.0, -2.0) circle (\pointSize) node[above left] {$\left(-3;-2\right)$};
		\filldraw[color=colorEmphasisCyan] (-1.0, 0.0) circle (\pointSize) node[above left] {$\left(-1;0\right)$};
		\filldraw[color=colorEmphasisCyan] (1.0, 2.0) circle (\pointSize) node[above] {$\left(1;2\right)$};
      \filldraw[color=colorEmphasisCyan] (3.0, 2.0) circle (\pointSize) node[above] {$\left(3;2\right)$};
	\end{tikzpicture}
	\caption{Đồ thị của $\begin{cases}
         x + 1 \text{ nếu } x \leq 1 \\
         2 \text{ nếu } x > 1
      \end{cases}$}
\end{figure}

2.

\begin{figure}[H]
	\centering
	\begin{tikzpicture}
		\draw[->] (-4, 0) -- (4, 0) node[right] {$x$};
		\draw[->] (0, -4) -- (0, 5)  node[above] {$f(x)$};
      \draw[graph thickness, samples=80, color=colorEmphasisCyan, domain=-2.000:0.000] plot (\x, {(((\x)/1)^3 + 4) / 1});
      \filldraw[color=colorEmphasisCyan] (-2, -4) circle (\pointSize) node[above left] {$\left(-2;-4\right)$};
		\filldraw[color=colorEmphasisCyan] (-1.0, 3.0) circle (\pointSize) node[left] {$\left(-1;3\right)$};
		\draw[color=colorEmphasisCyan, hollow point] (0.0, 4.0) circle (\pointSize) node[right] {$\left(0;4\right)$};
      \draw[graph thickness, samples=80, color=colorEmphasisCyan, domain=0.000:2.236] plot (\x, {(-((\x)/1)^2 + 1) / 1});
		\filldraw[color=colorEmphasisCyan] (0.0, 1.0) circle (\pointSize) node[left] {$\left(0;1\right)$};
		\filldraw[color=colorEmphasisCyan] (1.0, 0.0) circle (\pointSize) node[above right] {$\left(1;0\right)$};
		\filldraw[color=colorEmphasisCyan] (2.0, -3.0) circle (\pointSize) node[left] {$\left(2;-3\right)$};
	\end{tikzpicture}
	\caption{Đồ thị của $\begin{cases}
         x^3 + 4 \text{ nếu } x < 0 \\
         -x^2 + 1 \text{ nếu } x \geq 0
      \end{cases}$}
\end{figure}

Để ý rằng $f(x)$ đứt đoạn tại giá trị $x = 0$. Cụ thể, $f(x)$ không nhận giá trị $x^3 + 4$ khi $x = 0$. Tuy nhiên, không thể vẽ điểm ngay liền trước nó (không có số âm lớn nhất), nên người ta hay dùng đường tròn rỗng để biểu thị điểm đứt đoạn này.

3. Trước hết, cần xác định các giá trị của $x \geq -2$ để $\frac{5}{x^2 + 1}$ là số nguyên. Do với mọi $x \in \mathbb{R}$ thì $$x^2 \geq 0 \iff x^2 + 1 \geq 1 > 0 \iff 5 \geq \frac{5}{x^2 + 1} > 0.$$ Mà cần phải để $\frac{5}{x^2 + 1} \in \mathbb{N}$ cho nên $\frac{5}{x^2 + 1} \in \left\{1; 2; 3; 4; 5\right\}$. Với để ý đến điều kiện $x \geq -2$, kẻ bảng để xác định các giá trị có thể của $x$:

\begin{table}[H]
   \centering
   \begin{tabular}{|c|c|c|c|c|c|}
   \hline
   $\displaystyle \frac{5}{x^2 + 1}$ & $1$ & $2$ & $3$ & $4$ & $5$ \\
   \hline
   $x^2 + 1$ & $5$ & $\displaystyle\frac{5}{2}$ & $\displaystyle\frac{5}{3}$ & $\displaystyle\frac{5}{4}$ & $1$ \\
   \hline
   $x^2$ & $4$ & $\displaystyle\frac{3}{2}$ & $\displaystyle\frac{2}{3}$ & $\displaystyle\frac{1}{4}$ & $1$ \\
   \hline
   $x$ & $\left\{-2; 2\right\}$ & $\left\{-\sqrt{\frac{3}{2}}; \sqrt{\frac{3}{2}}\right\}$ & $\left\{-\sqrt{\frac{2}{3}}; \sqrt{\frac{2}{3}}\right\}$ & $\left\{-\frac{1}{2}; \frac{1}{2}\right\}$ & $0$ \\
   \hline
   \end{tabular}
   \caption{Bảng giá trị của $\frac{5}{x^2 + 1}$ với $x$} 
\end{table}

Từ đây, có được đồ thị của $f(x)$:

\begin{figure}[H]
	\centering
	\begin{tikzpicture}
		\draw[->] (-4, 0) -- (3, 0) node[right] {$x$};
		\draw[->] (0, -6) -- (0, 1)  node[above] {$f(x)$};
		\draw[graph thickness, samples=80, color=colorEmphasisCyan, domain=-3.000:-2.000] plot (\x, {(-4 / (\x)^2)});
      \filldraw[color=colorEmphasisCyan] (-3.0, -0.4444444444444444) circle (\pointSize) node[left] {$\left(-3;- \frac{4}{9}\right)$};
		\filldraw[color=colorEmphasisCyan] (-2.0, -1.0) circle (\pointSize) node[above right] {$\left(-2;-1\right)$};

		\filldraw[color=colorEmphasisCyan] ({ 2.0 }, { -1.0 }) circle (\pointSize) node[above] {$\left({2};{-1}\right)$};
		\filldraw[color=colorEmphasisCyan] ({ 0.0 }, { -5.0 }) circle (\pointSize) node[below] {$\left({0};{-5}\right)$};
		\filldraw[color=colorEmphasisCyan] ({ -0.5*sqrt(6) }, { -2.0 }) circle (\pointSize) node[left] {$\left({- \frac{\sqrt{6}}{2}};{-2}\right)$};
		\filldraw[color=colorEmphasisCyan] ({ 0.5*sqrt(6) }, { -2.0 }) circle (\pointSize) node[right] {$\left({\frac{\sqrt{6}}{2}};{-2}\right)$};
		\filldraw[color=colorEmphasisCyan] ({ -0.333333333333333*sqrt(6) }, { -3.0 }) circle (\pointSize) node[left] {$\left({- \frac{\sqrt{6}}{3}};{-3}\right)$};
		\filldraw[color=colorEmphasisCyan] ({ 0.333333333333333*sqrt(6) }, { -3.0 }) circle (\pointSize) node[right] {$\left({\frac{\sqrt{6}}{3}};{-3}\right)$};
		\filldraw[color=colorEmphasisCyan] ({ -0.500000000000000 }, { -4.0 }) circle (\pointSize) node[left] {$\left({- \frac{1}{2}};{-4}\right)$};
		\filldraw[color=colorEmphasisCyan] ({ 0.500000000000000 }, { -4.0 }) circle (\pointSize) node[right] {$\left({\frac{1}{2}};{-4}\right)$};
	\end{tikzpicture}
	\caption{Đồ thị của $\begin{cases}
      -\frac{4}{x^2} \text{ nếu } -2 > x \geq -3 \\
      \parbox{0.29\textwidth}{$\begin{array}{cl}
         -\frac{5}{x^2 + 1} &\text{nếu } x \geq -2 \text{ thực để} \\
         &\frac{5}{x^2 + 1}\text{ là số nguyên}
      \end{array}$}
   \end{cases}$}
\end{figure}

4. 

\begin{figure}[H]
	\centering
	\begin{tikzpicture}
		\draw[->] (-4, 0) -- (4, 0) node[right] {$x$};
		\draw[->] (0, 0) -- (0, 4)  node[above] {$f(x)$};
		\draw[graph thickness, samples=80, color=colorEmphasisCyan, domain=-3:0] plot (\x, {((2*((\x)/1) - 1) / (((\x)/1) - 1)) / 1});
		\draw[graph thickness, samples=80, color=colorEmphasisCyan, domain=2:3] plot (\x, {((2*((\x)/1) - 1) / (((\x)/1) - 1)) / 1});
		\filldraw[color=colorEmphasisCyan] ({ -3.0 }, { 1.75 }) circle (\pointSize) node[above] {$\left({-3};{\frac{7}{4}}\right)$};
		\filldraw[color=colorEmphasisCyan] ({ 0.0 }, { 1.0 }) circle (\pointSize) node[above right] {$\left({0};{1}\right)$};
		\filldraw[color=colorEmphasisCyan] ({ 2.0 }, { 3.0 }) circle (\pointSize) node[above] {$\left({2};{3}\right)$};
		\filldraw[color=colorEmphasisCyan] ({ 3.0 }, { 2.5 }) circle (\pointSize) node[right] {$\left({3};{\frac{5}{2}}\right)$};
		\draw[graph thickness, samples=80, color=colorEmphasisCyan, domain=0:2] plot (\x, {((((\x)/1) + 1)^2 - 3 * ((\x)/1)) / 1});
	\end{tikzpicture}
	\caption{Đồ thị của $\begin{cases}
         \frac{2x - 1}{x - 1} \text{ nếu } -3 \leq x < 0 \\
         \left(x + 1\right)^2 - 3x \text{ nếu } 0 \leq x < 2 \\
         \frac{2x - 1}{x - 1} \text{ nếu } 2 \leq x \leq 3
      \end{cases}$}
\end{figure}

5.

\begin{figure}[H]
	\centering
	\begin{tikzpicture}
		\draw[->] (-4, 0) -- (4, 0) node[right] {$x$};
		\draw[->] (0, -4) -- (0, 4)  node[above] {$f(x)$};
		\draw[graph thickness, samples=80, color=colorEmphasisCyan, domain=-1.913:0] plot (\x, {(((\x)/1)^3 + 3) / 1});
		\filldraw[color=colorEmphasisCyan] ({ 0.0 }, { 3.0 }) circle (\pointSize) node[above] {$\left({0};{3}\right)$};
		\draw[graph thickness, samples=80, color=colorEmphasisCyan, domain=0.000:1.000] plot (\x, {(-2*(\x) + 2) / 1});
      \draw[color=colorEmphasisCyan, hollow point] (0, 2) circle (\pointSize) node[below left] {$\left({0};{2}\right)$};
	\end{tikzpicture}
	\caption{Đồ thị của $\begin{cases}
      x^3 + 3 \text{ nếu } x \leq 0 \\
      -2x + 2 \text{ nếu } 0 < x < 1 \\
      4 + x - x^2 \text{ nếu } x \geq 1
   \end{cases}$}
\end{figure}
