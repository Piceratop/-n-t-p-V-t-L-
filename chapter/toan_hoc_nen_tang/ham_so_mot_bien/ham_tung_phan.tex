\subsection{Hàm số xác định từng phần}

\ % Lùi đầu dòng

Không phải lúc nào hàm số trong đời sống có thể biểu diễn dưới dạng một biểu thức. Khi này, chúng ta sẽ chia nhỏ đồ thị của hàm số thành các phần nhỏ, và biểu diễn từng phần thông qua biểu thức. Đó cũng là lí do cho tên gọi \defText{hàm số xác định từng phần}.

\exercise Phác thảo đồ thị của những hàm sau:

\begin{multicols}{2}
   \begin{enumerate}
      \item $f(x) = \begin{cases}
         x + 1 \text{ nếu } x \leq 1 \\
         2 \text{ nếu } x > 1
      \end{cases}$;
      \item $f(x) = \begin{cases}
         x^3 + 4 \text{ nếu } x < 0 \\
         -x^2 + 1 \text{ nếu } x \geq 0
      \end{cases}$;
      \item $f(x) = \begin{cases}
         \frac{4}{x} \text{ nếu } -2 > x \geq -3 \\
         \parbox{0.3\textwidth}{$\begin{array}{cl}
            -\frac{4}{x^2} &\text{nếu } x \text{ là số thực để } \frac{4}{x^2} \text{ là}\\
            &\text{số nguyên không quá }$4$
         \end{array}$}
      \end{cases}$;
      \item $f(x) = \begin{cases}
         \frac{2x - 1}{x - 1} \text{ nếu } -3 \leq x < 0 \\
         \left(x + 1\right)^2 - 1 \text{ nếu } 0 \leq x < 1 \\
         \frac{2x - 1}{x - 1} \text{ nếu } 1 \leq x \leq 3
      \end{cases}$;
      \item $f(x) = \begin{cases}
         x^3 + 3 \text{ nếu } x \leq 0 \\
         x + 3 \text{ nếu } 0 < x < 1 \\
         4 + x - x^2 \text{ nếu } x \geq 1
      \end{cases}$.
   \end{enumerate}
\end{multicols}

\solution
