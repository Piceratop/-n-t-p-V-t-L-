\subsection{Hàm số xác định từng phần}

\ % Lùi đầu dòng

Không phải lúc nào hàm số trong đời sống có thể biểu diễn dưới dạng một biểu thức. Khi này, chúng ta sẽ chia nhỏ đồ thị của hàm số thành các phần nhỏ, và biểu diễn từng phần thông qua biểu thức. Đó cũng là lí do cho tên gọi \defText{hàm số xác định từng phần}.

Một ví dụ là hàm giá trị tuyệt đối. Khả năng cao bạn đọc đã biết rằng giá trị tuyệt đối của một số $x$\footnote{Nếu không biết thì bạn đọc đọc qua phần đồ thị vẫn hiểu chứ?} được xác định như sau,

\begin{equation*}
   \defMath{|x|= } \begin{cases}
      \defMath{x \defText{ nếu } x \geq 0 }\\
      \defMath{-x \defText{ nếu } x < 0 }
   \end{cases}.
\end{equation*}

Hai hàm từng phần khác cũng thông dụng nhưng ít khi được đề cập đến là \defText{hàm sàn} (hay \defText{hàm phần nguyên}, \defText{hàm làm tròn xuống}) và \defText{hàm trần} (hay \defText{hàm làm tròn lên}). Hàm sàn tác dụng lên $x$ sẽ làm tròn xuống $x$ đến số nguyên lớn nhất không vượt quá $x$ với kí hiệu là $\defMath{\lfloor x \rfloor}$.
Tương tự, hàm trần của $x$ làm tròn lên $x$ đến số nguyên nhỏ nhất không vượt quá $x$ với kí hiệu là $\defMath{\lceil x \rceil}$.

Ví dụ:
\begin{multicols}{3}
   \begin{itemize}
      \item $\lfloor 2{,}5 \rfloor = 2$;
      \item $\lceil 2{,}5 \rceil = 3$;
      \item $\lfloor -2{,}5 \rfloor = -3$;
      \item $\lceil -2{,}5 \rceil = -2$;
      \item $\lfloor 2 \rfloor = 2$;
      \item $\lceil 2 \rceil = 2$.
   \end{itemize}
\end{multicols}

Và rõ ràng rằng không phải mọi giá trị trong tự nhiên và xã hội đều biểu diễn tốt nhất dưới dạng số thập phân. Chúng ta không mấy khi cắt nửa quả táo để bán cho nhau, hay không có bãi đỗ xe nào nhận đỗ $0{,}2$ cái xe (hoặc ít nhất tác giả chưa thấy bãi nào như vậy).

Để giải những vấn đề với hàm xác định giá trị từng phần, tương tự như cái tên, chúng ta chủ yếu chia bài toán theo các trường hợp phù hợp với từng phần của hàm đầu vào.

\exercise Giải các phương trình và bất phương trình sau trên ẩn $x$ thực:

\begin{multicols}{2}
   \begin{enumerate}
      \item $|x + 4| = 9$;
      \item $|x - 3| = -9$;
      \item $|7 - 2x| < 9$;
      \item $|3 + 6x| \geq 9$;
      \item $2x + 3 + |3x + 4| > 0$;
      \item $|x + 4| = |7x - 12|$;
      \item $|6x + 9| > |6x - 3|$;
      \item $\left|2x + 2\right| + |x + 1| = 9$;
      \item $|3x + 3| + |3x - 4| \leq 7$;
      \item $\left|2(x - 1)^2 - 4\right|$ = 2;
      \item $\left|2x^2 - 2x - 2\right| = \left|3x^2 - 4x - 2\right|$;
      \item $\left|x^3 - 3x^2 + x\right| \leq |x|$.
   \end{enumerate}
\end{multicols}

\solution

\setcounter{subexercise}{1}
\arabic{subexercise}. Xét hai trường hợp:

\textcolor{colorEmphasisCyan}{Trường hợp một --- $x + 4 \geq 0$}. Khi này, phá dấu giá trị tuyệt đối để có $|x + 4| = x + 4$. Cho nên phương trình ban đầu sẽ tương đương với:

\begin{equation*}
   x + 4 = 9 \iff x = 5.
\end{equation*}

\textcolor{colorEmphasis}{Trường hợp hai --- $x + 4 < 0$}. Khi này,

\begin{align*}
   &\begin{cases}
      |x + 4| = 9 \\
      x + 4 < 0
   \end{cases} \\
   \iff &-(x + 4) = 9 \equationexplanation{Phá dấu giá trị tuyệt đối: $|x + 4| = -(x + 4)$.}\\
   \iff &x + 4 = -9 \iff x = -13.
\end{align*}

Kết hợp hai trường hợp, có được $x \in \{5; -13\}$. Thử lại trực tiếp thấy thỏa mãn.

Vậy tập nghiệm của phương trình là $\{5; -13\}$.

2. Đặt $f(x) = |x|$.

Nếu \textcolor{colorEmphasisCyan}{$x \geq 0$} thì $f(x) = |x| = x$ và hiển nhiên $f(x) \geq 0$. Nếu \textcolor{colorEmphasis}{$x < 0$} thì $f(x) = |x| = -x$. Có $x < 0 \iff -x > 0 \iff f(x) > 0$.

Kết hợp lại, chúng ta có $f(x) \geq 0$ với mọi $x \in \mathbb{R}$. Suy ra được rằng $f(x - 3) \geq 0$. Tuy nhiên, phương trình được cho có thể được viết lại là $f(x - 3) = -9$. Do đó, phương trình vô nghiệm.

Sai lầm thường gặp ở dạng bài này là có suy luận như sau:

\begin{equation*}
   |x - 3| = -9 \iff \left[\begin{array}{l}
      x - 3 = -9 \\
      x - 3 = 9
   \end{array}\right..
\end{equation*}

3. Một lần nữa, xét hai trường hợp:

\textcolor{colorEmphasisCyan}{Trường hợp một --- $7 - 2x \geq 0$}. Thực hiện biến đổi:

\begin{align*}
   &\begin{cases}
      |7 - 2x| < 9 \\
      7 - 2x \geq 0
   \end{cases} \\
   \iff &\begin{cases}
      7 - 2x < 9 \\
      7 - 2x \geq 0
   \end{cases} \\
   \iff &0 \leq 7 - 2x < 9 \\
   \iff &-7 \leq -2x < 2 \\
   \iff &\frac{7}{2} \geq x > -1.
\end{align*}

\textcolor{colorEmphasis}{Trường hợp hai --- $7 - 2x < 0$}:

\begin{align*}
   &\begin{cases}
      |7 - 2x| < 9 \\
      7 - 2x < 0
   \end{cases} \\
   \iff &\begin{cases}
      - (7 - 2x) < 9 \\
      7 - 2x < 0
   \end{cases} \\
   \iff & -9 < 7 - 2x < 0 \\
   \iff & -16 < -2x < -7 \\
   \iff & 8 > x > \frac{7}{2}.
\end{align*}

Kết hợp hai trường hợp, có được $x \in \left(-1; 8\right)$. Có biến đổi là tương đương trong tập xác định cho nên phương trình có nghiệm $x \in \left(-1; 8\right)$.

4. \textcolor{colorEmphasisCyan}{Trường hợp một --- $3 + 6x \geq 0$}:

\begin{align*}
   &\begin{cases}
      |3 + 6x| \geq 9 \\
      3 + 6x \geq 0
   \end{cases} \\
   \iff 3 + 6x \geq 9 \\
   \iff x \geq 1.
\end{align*}

\textcolor{colorEmphasis}{Trường hợp hai --- $3 + 6x < 0$}:

\begin{align*}
   &\begin{cases}
      |3 + 6x| \geq 9 \\
      3 + 6x < 0
   \end{cases} \\
   \iff &-(3 + 6x) \geq 9 \\
   \iff &3 + 6x \leq -9 \\
   \iff &x \leq -2.
\end{align*}

Do trong mỗi trường hợp, mọi biến đổi là tương đương, nên chúng ta có tập nghiệm của bất phương trình là $\left(-\infty; -2\right] \cup \left[1; \infty\right)$.

5. \textcolor{colorEmphasisCyan}{Trường hợp một --- $3x + 4 \geq 0$}. Khi này, 

\begin{equation*}
   \begin{cases}
      2x + 3 + |3x + 4| = 2x + 3 + (3x + 4) \\
      3x + 4 \geq 0
   \end{cases} \iff \begin{cases}
      2x + 3 + |3x + 4| = 5x + 7 \\
      x \geq -\frac{4}{3}
   \end{cases}
\end{equation*}

\begin{equation*}
   \implies 2x + 3 + |3x + 4| \geq 5 \left(-\frac{4}{3}\right) + 7 \iff 2x + 3 + |3x + 4| \geq \frac{1}{3} > 0.
\end{equation*}

\textcolor{colorEmphasis}{Trường hợp hai --- $3x + 4 < 0$}:

\begin{equation*}
   \begin{cases}
      2x + 3 + |3x + 4| = 2x + 3 - (3x + 4) \\
      3x + 4 < 0
   \end{cases} \iff \begin{cases}
      2x + 3 + |3x + 4| = -x - 1 \\
      x < -\frac{4}{3}
   \end{cases}
\end{equation*}

Từ đó, có:

\begin{equation*}
   -x > \frac{4}{3} \iff -x - 1 > \frac{1}{3} \implies 2x + 3 + |3x + 4| > 0.
\end{equation*}

Từ hai trường hợp, chúng ta có $2x + 3 + |3x + 4| > 0$ với mọi giá trị thực của $x$. Do đó, tập nghiệm của phương trình là $\mathbb{R}$.

6. Tóm tắt các trường hợp thông qua bảng xét dấu sau:

\begin{table}[H]
   \centering
   \begin{tabular}{|c|ccccccc|}
   \hline
   $x$          & $-\infty$ &     & $-4$ &     & $\frac{12}{7}$ &   & $\infty$ \\
   \hline
   $x+4$        &           & $-$ &  0  &  +  &     & + &           \\
   \hline
   $7x-12$        &           & $-$ &     & $-$ &  0  & + &           \\
   \hline
   \end{tabular}
   \caption{Bảng xét dấu cho $x+4$ và $7x-12$}
   \label{tab:toan_hoc_nen_tang:ham_so_mot_bien:ham_tung_phan:gpt7}
\end{table}

\textcolor{colorEmphasisCyan}{Trường hợp một --- $x < -4$}. Khi này, phương trình ban đầu trở thành

\begin{align*}
   |x + 4| &= |7x - 12| \\
   \iff -(x + 4) &= -(7x - 12) \\
   \iff x = \frac{8}{3}.
\end{align*}

Chúng ta không nhận nghiệm này trong trường hợp này do trái với giả thiết $x < -4$.

\textcolor{colorEmphasis}{Trường hợp hai --- $-4\leq x < \frac{12}{7} $}. Khi này, 

\begin{align*}
   |x + 4| &= |7x - 12| \\
   \iff x + 4 &= -(7x - 12) \\
   \iff x &= 1.
\end{align*}

\textcolor{colorEmphasisGreen}{Trường hợp ba --- $x \geq \frac{12}{7}$}. Từ đây,

\begin{align*}
   |x + 4| &= |7x - 12| \\
   \iff x + 4 &= 7x - 12 \\
   \iff x &= \frac{8}{3}.
\end{align*}

Kết hợp các trường hợp và kiểm tra lại các nghiệm, chúng ta có nghiệm của phương trình là $x \in \left\{1; \frac{8}{3}\right\}$.

7. Kẻ bảng xét dấu

\begin{table}[H]
   \centering
   \begin{tabular}{|c|ccccccc|}
   \hline
   $x$          & $-\infty$ &     & $-\frac{3}{2}$ &     & $\frac{1}{2}$ &   & $\infty$ \\
   \hline
   $6x + 9$        &           & $-$ &  0  &  +  &     & + &           \\
   \hline
   $6x - 3$        &           & $-$ &     & $-$ &  0  & + &           \\
   \hline
   \end{tabular}
   \caption{Bảng xét dấu cho $6x + 9$ và $6x - 3$}
   \label{tab:toan_hoc_nen_tang:ham_so_mot_bien:ham_tung_phan:gpt7}
\end{table}

\textcolor{colorEmphasisCyan}{Trường hợp một --- $x < -\frac{3}{2}$}. Khi này, bất phương trình ban đầu trở thành

\begin{align*}
   |6x + 9| &> |6x - 3| \\
   \iff -(6x + 9) &> -(6x - 3) \\
   \iff -9 &> 3.
\end{align*}

Bất phương trình sai với mọi $x$. Đối với trường hợp này, tập nghiệm là $\emptyset$.

\textcolor{colorEmphasis}{Trường hợp hai --- $-\frac{3}{2} \leq x < \frac{1}{2}$}. Khi này,

\begin{align*}
   |6x + 9| &> |6x - 3| \\
   \iff 6x + 9 &> -(6x - 3) \\
   \iff x &> -\frac{1}{2}.
\end{align*}

\textcolor{colorEmphasisGreen}{Trường hợp ba --- $x \geq \frac{1}{2}$}:

\begin{align*}
   |6x + 9| &> |6x - 3| \\
   \iff 6x + 9 &> 6x - 3 \\
   \iff 9 &> -3
\end{align*}
luôn đúng. Kết hợp với điều kiện, chúng ta có tập nghiệm $\left[\frac{1}{2}; \infty\right)$.

Hợp tập nghiệm của cả ba trường hợp, do mọi biến đổi trong mỗi trường hợp là tương đương cho nên bất phương trình có tập nghiệm là $\left(-\frac{1}{2}; \infty\right)$.

8. Biến đổi cơ bản để có

\begin{align}
   \left|2x + 2\right| + \left|x + 1\right| &= 9 \nonumber\\
   \iff \left|2(x + 1)\right| + \left|x + 1\right| &= 9. \label{eq:toan_hoc_nen_tang:ham_so_mot_bien:ham_tung_phan:pt9}
\end{align}

Với \textcolor{colorEmphasisCyan}{$x \geq -1$} thì $x+ 1 \geq 0 \iff \begin{cases}
   \left|2(x + 1)\right| = 2(x + 1) \\
   \left|x + 1\right| = x + 1
\end{cases}$. Cho nên 
\begin{align*}
   \text{\refeq{eq:toan_hoc_nen_tang:ham_so_mot_bien:ham_tung_phan:pt9}} \iff &2(x + 1) + (x + 1) = 9 \\
   \iff &x = 2.
\end{align*}

Với \textcolor{colorEmphasis}{$x < -1$} thì $x + 1 < 0 \iff \begin{cases}
   \left|2(x + 1)\right| = -\left(2(x + 1)\right) \\
   \left|x + 1\right| = -\left(x + 1\right)
\end{cases}$. Cho nên
\begin{align*}
   \text{\refeq{eq:toan_hoc_nen_tang:ham_so_mot_bien:ham_tung_phan:pt9}} \iff &-\left(2(x + 1)\right) -\left(x + 1\right) = 9 \\
   \iff &x = -4.
\end{align*}

Phương trình có nghiệm là $\left\{2; -4\right\}$.

9. Kẻ bảng xét dấu

\begin{table}[H]
   \centering
   \begin{tabular}{|c|ccccccc|}
   \hline
   $x$          & $-\infty$ &     & $-1$ &     & $\frac{4}{3}$ &   & $\infty$ \\
   \hline
   $3x + 3$        &           & $-$ &  0  &  +  &     & + &           \\
   \hline
   $3x - 4$        &           & $-$ &     & $-$ &  0  & + &           \\
   \hline
   \end{tabular}
   \caption{Bảng xét dấu cho $3x + 3$ và $3x - 4$}
   \label{tab:toan_hoc_nen_tang:ham_so_mot_bien:ham_tung_phan:gpt10}
\end{table}

\textcolor{colorEmphasisCyan}{Trường hợp một --- $x < -1$}. Từ đó,
\begin{align*}
   |3x + 3| + |3x - 4| &\leq 7 \\
   \iff -(3x + 3) - (3x - 4) &\leq 7 \\
   \iff -6x + 1 &\leq 7 \\
   \iff x &\geq -1.
\end{align*}

Điều này trái với điều kiện $x < -1$ của trường hợp này.

\textcolor{colorEmphasis}{Trường hợp hai --- $-1 \leq x < \frac{4}{3}$}. Với điều kiện này,
\begin{align*}
   |3x + 3| + |3x - 4| &\leq 7 \\
   \iff 3x + 3 - (3x - 4) &\leq 7 \\
   \iff 7 &\leq 7.
\end{align*}
Bất phương trình này là luôn đúng.

\textcolor{colorEmphasisGreen}{Trường hợp ba --- $x \geq \frac{4}{3}$}. Khi này,
\begin{align*}
   |3x + 3| + |3x - 4| &\leq 7 \\
   \iff 3x + 3 + 3x - 4 &\leq 7 \\
   \iff x &\leq \frac{4}{3}.
\end{align*}

Kết hợp với điều kiện, có được $x = \frac{4}{3}$.

Qua ba trường hợp, tập nghiệm của bất phương trình là $\left[-1; \frac{4}{3}\right]$.

10.

\begin{align}
   \left|2(x - 1)^2 - 4\right| &= 2 \nonumber\\
   \iff \left|2x^2 - 2x - 2\right| &= 2. \label{eq:toan_hoc_nen_tang:ham_so_mot_bien:ham_tung_phan:pt10}
\end{align}

Xét hai trường hợp. \textcolor{colorEmphasisCyan}{Trường hợp một --- $2x^2 - 2x - 2 \geq 0$}:

\begin{align*}
   \text{\refeq{eq:toan_hoc_nen_tang:ham_so_mot_bien:ham_tung_phan:pt10}} \iff 2x^2 - 2x - 2 &= 2 \\
   2x^2 - 2x - 4 &= 0 \\
   x^2 - x - 2 &= 0.
\end{align*}
Giải phương trình này để có $x\in\left\{-1; 2\right\}$, đều thỏa mãn điều kiện $2x^2 - 2x - 2 \geq 0$.

\textcolor{colorEmphasis}{Trường hợp hai --- $2x^2 - 2x - 2 < 0$}:

\begin{align*}
   \text{\refeq{eq:toan_hoc_nen_tang:ham_so_mot_bien:ham_tung_phan:pt10}} \iff -\left(2x^2 - 2x - 2\right) &= 2 \\
   \iff -2x^2 + 2x + 2 &= 2 \\
   \iff -2x^2 + 2x &= 0.
\end{align*}
Phương trình này có nghiệm $x = 0$ hoặc $x = 1$, đều thỏa mãn điều kiện $2x^2 - 2x - 2 < 0$.

Qua hai trường hợp, tập nghiệm của phương trình là $\left\{-1; 0; 1; 2\right\}$.

11. Vì cả hai đa thức bậc hai $2x^2 -2x - 2$ và $3x^2 - 4x - 2$ đều không có nghiệm đẹp, nên tác giả sẽ không vẽ bảng xét dấu cho bài này mà chia làm bốn trường hợp. Để rút gọn, tác giả sẽ gộp lại như sau:

\textcolor{colorEmphasisCyan}{Trường hợp một --- $
\begin{cases}
   2x^2 - 2x - 2 \geq 0 \\
   3x^2 - 4x - 2 \geq 0
\end{cases}$} và \textcolor{colorEmphasisCyan}{trường hợp hai --- $
\begin{cases}
   2x^2 - 2x - 2 < 0 \\
   3x^2 - 4x - 2 < 0
\end{cases}$}. Cả hai trường hợp này sau khi phá bỏ dấu giá trị tuyệt đối đều cho:
\begin{align*}
   2x^2 - 2x - 2 &= 3x^2 - 4x - 2 \\
   \iff 0 &= x^2 - 2x \\
   \iff x &\in \left\{0; 2\right\}.
\end{align*}

\textcolor{colorEmphasis}{Trường hợp ba --- $
\begin{cases}
   2x^2 - 2x - 2 \geq 0 \\
   3x^2 - 4x - 2 < 0
\end{cases}$} và \textcolor{colorEmphasis}{trường hợp bốn --- $
\begin{cases}
   2x^2 - 2x - 2 < 0 \\
   3x^2 - 4x - 2 \geq 0
\end{cases}$}. Cả hai trường hợp đều suy ra
\begin{align*}
   2x^2 - 2x - 2 &= -\left(3x^2 - 4x - 2\right) \\
   \iff 5x^2 - 6x - 4 &= 0 \\
   \iff x &\in \left\{\frac{3 + \sqrt{29}}{5}; \frac{3 - \sqrt{29}}{5}\right\}.
\end{align*}

Kết hợp các tập nghiệm và kiểm tra trực tiếp, chúng ta có tập nghiệm của phương trình là $$\left\{0; 2; \frac{3 + \sqrt{29}}{5}; \frac{3 - \sqrt{29}}{5}\right\}.$$

12. Xét \textcolor{colorEmphasisCyan}{trường hợp một --- $x < 0$}, có $x^3 - 3x^2 + x = x\left(x^2 - 3x + 1\right)$. Vì $x < 0$ nên $-3x > 0 \implies x^2 - 3x + 1 > 0$ $\implies x^3 - 3x^2 + x < 0$. Do đó,
$$
\begin{cases}
   |x^3 - 3x^2 + x| = -\left(x^3 - 3x^2 + x\right) \\
   |x| = -x
\end{cases}.
$$
Qua đó, bất phương trình ban đầu trở thành:
\begin{align*}
   \left|x^3 - 3x^2 + x\right| &\leq \left|x\right| \\
   \iff -\left(x^3 - 3x^2 + x\right) &\leq -x \\
   \iff -x^3 + 3x^2 &\leq 0 \\
   \iff x^2 (3 - x) &\leq 0 \\
   \iff 3 - x &\leq 0 \equationexplanation{$x^2 \geq 0$ với mọi $x \in \mathbb{R}$} \\
   \iff x &\geq 3.
\end{align*}
Kết quả này mâu thuẫn với điều kiện $x < 0$ nên không có nghiệm trong trường hợp này.

Xét $x \geq 0$, chúng ta chia làm hai trường hợp nhỏ. Cụ thể, \textcolor{colorEmphasis}{trường hợp hai --- $\begin{cases}
   x \geq 0 \\
   x^3 - 3x^2 + x \geq 0
\end{cases}$}. Khi này

\begin{align*}
   |x^3 - 3x^2 + x| &\leq |x| \\
   \iff x^3 - 3x^2 + x &\leq x \\
   \iff x^3 - 3x^2 &\leq 0 \\
   \iff x^2(x - 3) &\leq 0 \\
   \iff x - 3 &\leq 0 \equationexplanation{$x^2 \geq 0$ với mọi $x \in \mathbb{R}$} \\
   \iff x &\leq 3.
\end{align*}

Cần phải kết hợp với điều kiện để xác định nghiệm thỏa mãn. Có

\begin{align*}
   x^3 - 3x^2 + x &\geq 0 \\
   \iff x(x^2 - 3x + 1) &\geq 0 \\
   \iff x^2 - 3x + 1 &\geq 0 \equationexplanation{$x \geq 0$ theo điều kiện}.
\end{align*}

Kẻ bảng xét dấu

\begin{table}[H]
   \centering
   \begin{tabular}{|c|ccccccc|}
   \hline
   $x$           & $-\infty$ &   & $\frac{3 - \sqrt{5}}{2}$ &     & $\frac{3 + \sqrt{5}}{2}$ &   & $\infty$ \\
   \hline
   $x^{2}-3x+1$  &           & + &                 0                 & $-$ &                0                 & + &           \\
   \hline
   \end{tabular}
   \caption{Bảng xét dấu của $x^{2}-3x+1$}
   \label{tab:toan_hoc_nen_tang:ham_so_mot_bien:ham_tung_phan:bxd12_x2_t3x_1}
\end{table}

Qua đó, nghiệm của bất phương trình trong trường hợp này là $\left[0; \frac{3 - \sqrt{5}}{2}\right] \cup \left[\frac{3 + \sqrt{5}}{2}; 3\right]$.

\textcolor{colorEmphasisGreen}{Trường hợp ba --- $
\begin{cases}
   x \geq 0 \\
   x^3 - 3x^2 + x < 0
\end{cases}$} $\iff x^2 - 3x + 1 < 0$. Từ bảng xét dấu \ref{tab:toan_hoc_nen_tang:ham_so_mot_bien:ham_tung_phan:bxd12_x2_t3x_1}, $x$ phải nằm trong đoạn $\left[\frac{3 - \sqrt{5}}{2}; \frac{3 + \sqrt{5}}{2}\right]$. Ngoài ra, từ bất phương trình:
\begin{align*}
   |x^3 - 3x^2 + x| &\leq |x| \\
   \iff -\left(x^3 - 3x^2 + x\right) &\leq x \\
   \iff -x^3 + 3x^2 - 2x &\leq 0 \\
   \iff -x(x-1)(x-2) &\leq 0 \\
   \iff (x - 1)(x - 2) &\geq 0 \equationexplanation{$x < 0$ theo điều kiện}.
\end{align*}

Kẻ bảng xét dấu cho $(x - 1)(x - 2)$:
\begin{table}[H]
   \centering
   \begin{tabular}{|c|ccccccc|}
      \hline
      $x$          & $-\infty$ &     & $1$ &     & $2$ &   & $\infty$ \\
      \hline
      $x-1$        &           & $-$ &  0  &  +  &     & + &           \\
      \hline
      $x-2$        &           & $-$ &     & $-$ &  0  & + &           \\
      \hline
      $(x-1)(x-2)$ &           &  +  &  0  & $-$ &  0  & + &           \\
      \hline
   \end{tabular}
   \caption{Bảng xét dấu của $(x-1)(x-2)$}
\end{table}

Qua đó, nghiệm của bất phương trình trong trường hợp này là $\left[\frac{3 - \sqrt{5}}{2}; 1\right] \cup \left[2; \frac{3 + \sqrt{5}}{2}\right]$.

Qua ba trường hợp, chúng ta có tập nghiệm của bất phương trình: $\left[0; 1\right] \cup \left[2; 3\right]$.

\exercise Phác thảo đồ thị của những hàm sau:

\begin{multicols}{2}
   \begin{enumerate}
      \item $f(x) = \begin{cases}
         x + 1 \text{ nếu } x \leq 1 \\
         2 \text{ nếu } x > 1
      \end{cases}$;
      \item $f(x) = \begin{cases}
         x^3 + 4 \text{ nếu } x < 0 \\
         -x^2 + 1 \text{ nếu } x \geq 0
      \end{cases}$;
      \item $f(x) = \begin{cases}
         -\frac{4}{x^2} \text{ nếu } -2 > x \geq -3 \\
         \parbox{0.29\textwidth}{$\begin{array}{cl}
            -\frac{5}{x^2 + 1} &\text{nếu } x \geq -2 \text{ thực để} \\
            &\frac{5}{x^2 + 1}\text{ là số nguyên}
         \end{array}$}
      \end{cases}$;
      \item $f(x) = \begin{cases}
         \frac{2x - 1}{x - 1} \text{ nếu } -3 \leq x < 0 \\
         \left(x + 1\right)^2 - 3x \text{ nếu } 0 \leq x < 2 \\
         \frac{2x - 1}{x - 1} \text{ nếu } 2 \leq x \leq 3
      \end{cases}$;
      \item $f(x) = \begin{cases}
         x^3 + 3 \text{ nếu } x \leq 0 \\
         -2x + 2 \text{ nếu } 0 < x < 1 \\
         2 + x - x^2 \text{ nếu } x \geq 1
      \end{cases}$;
      \item $f(x) = |x|$;
      \item $f(x) = \left|2x^2 - 4x\right|$;
      \item $f(x) = \left|x^3 - 3x^2 + x\right| - \left|x\right|$;
   \end{enumerate}
\end{multicols}

\solution

\setcounter{subexercise}{1}
\arabic{subexercise}.

\begin{figure}[H]
	\centering
	\begin{tikzpicture}
		\draw[->] (-4, 0) -- (4, 0) node[right] {$x$};
		\draw[->] (0, -4) -- (0, 4)  node[above] {$f(x)$};
		\draw[graph thickness, samples=80, color=colorEmphasisCyan, domain=-4.000:1] plot (\x, {(((\x)/1) + 1) / 1});
		\draw[graph thickness, samples=80, color=colorEmphasisCyan, domain=1:4] plot (\x, 2);
		\filldraw[color=colorEmphasisCyan] (-3.0, -2.0) circle (\pointSize) node[above left] {$\left(-3;-2\right)$};
		\filldraw[color=colorEmphasisCyan] (-1.0, 0.0) circle (\pointSize) node[above left] {$\left(-1;0\right)$};
		\filldraw[color=colorEmphasisCyan] (1.0, 2.0) circle (\pointSize) node[above] {$\left(1;2\right)$};
      \filldraw[color=colorEmphasisCyan] (3.0, 2.0) circle (\pointSize) node[above] {$\left(3;2\right)$};
	\end{tikzpicture}
	\caption{Đồ thị của $\begin{cases}
         x + 1 \text{ nếu } x \leq 1 \\
         2 \text{ nếu } x > 1
      \end{cases}$}
\end{figure}

2.

\begin{figure}[H]
	\centering
	\begin{tikzpicture}
		\draw[->] (-4, 0) -- (4, 0) node[right] {$x$};
		\draw[->] (0, -4) -- (0, 5)  node[above] {$f(x)$};
      \draw[graph thickness, samples=80, color=colorEmphasisCyan, domain=-2.000:0.000] plot (\x, {(((\x)/1)^3 + 4) / 1});
      \filldraw[color=colorEmphasisCyan] (-2, -4) circle (\pointSize) node[above left] {$\left(-2;-4\right)$};
		\filldraw[color=colorEmphasisCyan] (-1.0, 3.0) circle (\pointSize) node[left] {$\left(-1;3\right)$};
		\draw[color=colorEmphasisCyan, hollow point] (0.0, 4.0) circle (\pointSize) node[right] {$\left(0;4\right)$};
      \draw[graph thickness, samples=80, color=colorEmphasisCyan, domain=0.000:2.236] plot (\x, {(-((\x)/1)^2 + 1) / 1});
		\filldraw[color=colorEmphasisCyan] (0.0, 1.0) circle (\pointSize) node[left] {$\left(0;1\right)$};
		\filldraw[color=colorEmphasisCyan] (1.0, 0.0) circle (\pointSize) node[above right] {$\left(1;0\right)$};
		\filldraw[color=colorEmphasisCyan] (2.0, -3.0) circle (\pointSize) node[left] {$\left(2;-3\right)$};
	\end{tikzpicture}
	\caption{Đồ thị của $\begin{cases}
         x^3 + 4 \text{ nếu } x < 0 \\
         -x^2 + 1 \text{ nếu } x \geq 0
      \end{cases}$}
\end{figure}

Để ý rằng $f(x)$ đứt đoạn tại giá trị $x = 0$. Cụ thể, $f(x)$ không nhận giá trị $x^3 + 4$ khi $x = 0$. Tuy nhiên, không thể vẽ điểm ngay liền trước nó (không có số âm lớn nhất), nên người ta hay dùng đường tròn rỗng để biểu thị điểm đứt đoạn này.

3. Trước hết, cần xác định các giá trị của $x \geq -2$ để $\frac{5}{x^2 + 1}$ là số nguyên. Do với mọi $x \in \mathbb{R}$ thì $$x^2 \geq 0 \iff x^2 + 1 \geq 1 > 0 \iff 5 \geq \frac{5}{x^2 + 1} > 0.$$ Mà cần phải để $\frac{5}{x^2 + 1} \in \mathbb{N}$ cho nên $\frac{5}{x^2 + 1} \in \left\{1; 2; 3; 4; 5\right\}$. Với để ý đến điều kiện $x \geq -2$, kẻ bảng để xác định các giá trị có thể của $x$:

\begin{table}[H]
   \centering
   \begin{tabular}{|c|c|c|c|c|c|}
   \hline
   $\displaystyle \frac{5}{x^2 + 1}$ & $1$ & $2$ & $3$ & $4$ & $5$ \\
   \hline
   $x^2 + 1$ & $5$ & $\displaystyle\frac{5}{2}$ & $\displaystyle\frac{5}{3}$ & $\displaystyle\frac{5}{4}$ & $1$ \\
   \hline
   $x^2$ & $4$ & $\displaystyle\frac{3}{2}$ & $\displaystyle\frac{2}{3}$ & $\displaystyle\frac{1}{4}$ & $1$ \\
   \hline
   $x$ & $\left\{-2; 2\right\}$ & $\left\{-\sqrt{\frac{3}{2}}; \sqrt{\frac{3}{2}}\right\}$ & $\left\{-\sqrt{\frac{2}{3}}; \sqrt{\frac{2}{3}}\right\}$ & $\left\{-\frac{1}{2}; \frac{1}{2}\right\}$ & $0$ \\
   \hline
   \end{tabular}
   \caption{Bảng giá trị của $\frac{5}{x^2 + 1}$ với $x$} 
\end{table}

Từ đây, có được đồ thị của $f(x)$:

\begin{figure}[H]
	\centering
	\begin{tikzpicture}
		\draw[->] (-4, 0) -- (3, 0) node[right] {$x$};
		\draw[->] (0, -6) -- (0, 1)  node[above] {$f(x)$};
		\draw[graph thickness, samples=80, color=colorEmphasisCyan, domain=-3.000:-2.000] plot (\x, {(-4 / (\x)^2)});
      \filldraw[color=colorEmphasisCyan] (-3.0, -0.4444444444444444) circle (\pointSize) node[left] {$\left(-3;- \frac{4}{9}\right)$};
		\filldraw[color=colorEmphasisCyan] (-2.0, -1.0) circle (\pointSize) node[above right] {$\left(-2;-1\right)$};

		\filldraw[color=colorEmphasisCyan] ({ 2.0 }, { -1.0 }) circle (\pointSize) node[above] {$\left({2};{-1}\right)$};
		\filldraw[color=colorEmphasisCyan] ({ 0.0 }, { -5.0 }) circle (\pointSize) node[below] {$\left({0};{-5}\right)$};
		\filldraw[color=colorEmphasisCyan] ({ -0.5*sqrt(6) }, { -2.0 }) circle (\pointSize) node[left] {$\left({- \frac{\sqrt{6}}{2}};{-2}\right)$};
		\filldraw[color=colorEmphasisCyan] ({ 0.5*sqrt(6) }, { -2.0 }) circle (\pointSize) node[right] {$\left({\frac{\sqrt{6}}{2}};{-2}\right)$};
		\filldraw[color=colorEmphasisCyan] ({ -0.333333333333333*sqrt(6) }, { -3.0 }) circle (\pointSize) node[left] {$\left({- \frac{\sqrt{6}}{3}};{-3}\right)$};
		\filldraw[color=colorEmphasisCyan] ({ 0.333333333333333*sqrt(6) }, { -3.0 }) circle (\pointSize) node[right] {$\left({\frac{\sqrt{6}}{3}};{-3}\right)$};
		\filldraw[color=colorEmphasisCyan] ({ -0.500000000000000 }, { -4.0 }) circle (\pointSize) node[left] {$\left({- \frac{1}{2}};{-4}\right)$};
		\filldraw[color=colorEmphasisCyan] ({ 0.500000000000000 }, { -4.0 }) circle (\pointSize) node[right] {$\left({\frac{1}{2}};{-4}\right)$};
	\end{tikzpicture}
	\caption{Đồ thị của $\begin{cases}
      -\frac{4}{x^2} \text{ nếu } -2 > x \geq -3 \\
      \parbox{0.29\textwidth}{$\begin{array}{cl}
         -\frac{5}{x^2 + 1} &\text{nếu } x \geq -2 \text{ thực để} \\
         &\frac{5}{x^2 + 1}\text{ là số nguyên}
      \end{array}$}
   \end{cases}$}
\end{figure}

4. 

\begin{figure}[H]
	\centering
	\begin{tikzpicture}
		\draw[->] (-4, 0) -- (4, 0) node[right] {$x$};
		\draw[->] (0, 0) -- (0, 4)  node[above] {$f(x)$};
		\draw[graph thickness, samples=80, color=colorEmphasisCyan, domain=-3:0] plot (\x, {((2*((\x)/1) - 1) / (((\x)/1) - 1)) / 1});
		\draw[graph thickness, samples=80, color=colorEmphasisCyan, domain=2:3] plot (\x, {((2*((\x)/1) - 1) / (((\x)/1) - 1)) / 1});
		\filldraw[color=colorEmphasisCyan] ({ -3.0 }, { 1.75 }) circle (\pointSize) node[above] {$\left({-3};{\frac{7}{4}}\right)$};
		\filldraw[color=colorEmphasisCyan] ({ 0.0 }, { 1.0 }) circle (\pointSize) node[above right] {$\left({0};{1}\right)$};
		\filldraw[color=colorEmphasisCyan] ({ 2.0 }, { 3.0 }) circle (\pointSize) node[above] {$\left({2};{3}\right)$};
		\filldraw[color=colorEmphasisCyan] ({ 3.0 }, { 2.5 }) circle (\pointSize) node[right] {$\left({3};{\frac{5}{2}}\right)$};
		\draw[graph thickness, samples=80, color=colorEmphasisCyan, domain=0:2] plot (\x, {((((\x)/1) + 1)^2 - 3 * ((\x)/1)) / 1});
	\end{tikzpicture}
	\caption{Đồ thị của $\begin{cases}
         \frac{2x - 1}{x - 1} \text{ nếu } -3 \leq x < 0 \\
         \left(x + 1\right)^2 - 3x \text{ nếu } 0 \leq x < 2 \\
         \frac{2x - 1}{x - 1} \text{ nếu } 2 \leq x \leq 3
      \end{cases}$}
\end{figure}

5.

\begin{figure}[H]
	\centering
	\begin{tikzpicture}
		\draw[->] (-4, 0) -- (4, 0) node[right] {$x$};
		\draw[->] (0, -4) -- (0, 4)  node[above] {$f(x)$};
		\draw[graph thickness, samples=80, color=colorEmphasisCyan, domain=-1.913:0] plot (\x, {(((\x)/1)^3 + 3) / 1});
		\filldraw[color=colorEmphasisCyan] ({ 0.0 }, { 3.0 }) circle (\pointSize) node[right] {$\left({0};{3}\right)$};
		\draw[graph thickness, samples=80, color=colorEmphasisCyan, domain=0.000:1.000] plot (\x, {(-2*(\x) + 2) / 1});
      \draw[color=colorEmphasisCyan, hollow point] (0, 2) circle (\pointSize) node[below left] {$\left({0};{2}\right)$};
      \draw[color=colorEmphasisCyan, hollow point] (1, 0) circle (\pointSize) node[below left] {$\left({1};{0}\right)$};
      \draw[graph thickness, samples=80, color=colorEmphasisCyan, domain=1.000:3.000] plot (\x, {(2 + ((\x)/1) - ((\x)/1)^2) / 1});
      \filldraw[color=colorEmphasisCyan] (1, 2) circle (\pointSize) node[above right] {$\left({1};{2}\right)$};
      \filldraw[color=colorEmphasisCyan] (2, 0) circle (\pointSize) node[above right] {$\left({2};{0}\right)$};
	\end{tikzpicture}
	\caption{Đồ thị của $\begin{cases}
      x^3 + 3 \text{ nếu } x \leq 0 \\
      -2x + 2 \text{ nếu } 0 < x < 1 \\
      2 + x - x^2 \text{ nếu } x \geq 1
   \end{cases}$}
\end{figure}

6.

\begin{figure}[H]
	\centering
	\begin{tikzpicture}
		\draw[->] (-4, 0) -- (4, 0) node[right] {$x$};
		\draw[->] (0, -1) -- (0, 4)  node[above] {$f(x)$};
		\draw[graph thickness, samples=80, color=colorEmphasisCyan, domain=0.000:4.000] plot (\x, {(((\x)/1)) / 1});
      \draw[graph thickness, samples=80, color=colorEmphasisCyan, domain=-4.000:0.000] plot (\x, {(-((\x)/1)) / 1});
		\filldraw[color=colorEmphasisCyan] ({ 0.0 }, { 0.0 }) circle (\pointSize) node[below] {$\left({0};{0}\right)$};
		\filldraw[color=colorEmphasisCyan] ({ -2.0 }, { 2.0 }) circle (\pointSize) node[below left] {$\left({-2};{2}\right)$};
		\filldraw[color=colorEmphasisCyan] ({ 2.0 }, { 2.0 }) circle (\pointSize) node[below right] {$\left({2};{2}\right)$};
	\end{tikzpicture}
	\caption{Đồ thị của $|x|$}
\end{figure}

7.

\begin{figure}[H]
	\centering
	\begin{tikzpicture}
		\draw[->] (-3, 0) -- (5, 0) node[right] {$x$};
		\draw[->] (0, -1) -- (0, 4)  node[above] {$f(x)$};
		\draw[graph thickness, samples=80, color=colorEmphasisCyan, domain=-0.732:0] plot (\x, {(2*((\x)/1)^2 - 4*((\x)/1)) / 1});
		\draw[graph thickness, samples=80, color=colorEmphasisCyan, domain=2:2.732] plot (\x, {(2*((\x)/1)^2 - 4*((\x)/1)) / 1});
		\filldraw[color=colorEmphasisCyan] ({ 0.0 }, { 0.0 }) circle (\pointSize) node[below left] {$\left({0};{0}\right)$};
		\filldraw[color=colorEmphasisCyan] ({ 2.0 }, { 0.0 }) circle (\pointSize) node[below right] {$\left({2};{0}\right)$};
      \filldraw[color=colorEmphasisCyan] ({ -0.581 }, { 3.0 }) circle (\pointSize) node[below left] {$\left(-0{,}58;3{,}00\right)$};
      \filldraw[color=colorEmphasisCyan] ({ 2.581 }, { 3.0 }) circle (\pointSize) node[below right] {$\left(2{,}58;3{,}00\right)$};
      \draw[graph thickness, samples=80, color=colorEmphasisCyan, domain=0.000:2.000] plot (\x, {(-2*((\x)/1)^2 + 4*((\x)/1)) / 1});
		\filldraw[color=colorEmphasisCyan] ({ 1.0 }, { 2.0 }) circle (\pointSize) node[above] {$\left({1};{2}\right)$};
	\end{tikzpicture}
	\caption{Đồ thị của $\left|2 x^{2} - 4 x\right|$}
\end{figure}

8.

\begin{figure}[H]
   \centering
   \begin{tikzpicture}
      \draw[->] (-3, 0) -- (6, 0) node[right] {$x$};
		\draw[->] (0, -4) -- (0, 4)  node[above] {$f(x)$};
      \draw[graph thickness, samples=80, color=colorEmphasisCyan, domain=-1.5:5.033] plot (\x, {abs((\x/1.5)^3 - 3*(\x/1.5)^2 + (\x/1.5)) - abs(\x/1.5)});
      \filldraw[color=colorEmphasisCyan] ({ 0.0 }, { 0.0 }) circle (\pointSize) node[below left] {$\left({0};{0}\right)$};
      \filldraw[color=colorEmphasisCyan] ({ 1.5 }, { 0.0 }) circle (\pointSize) node[below right] {$\left({1};{0}\right)$};
      \filldraw[color=colorEmphasisCyan] ({ 3.0 }, { 0.0 }) circle (\pointSize) node[above right] {$\left({2};{0}\right)$};
      \filldraw[color=colorEmphasisCyan] ({ 4.5 }, { 0.0 }) circle (\pointSize) node[below right] {$\left({3};{0}\right)$};
      \filldraw[color=colorEmphasisCyan] ({ 3.927 }, { -2.618 }) circle (\pointSize) node[below right] {$\left({2{,}62};{-2{,}62}\right)$};
      \filldraw[color=colorEmphasisCyan] ({ -1.5 }, { 4.0 }) circle (\pointSize) node[left] {$\left({-1};{4}\right)$};
   \end{tikzpicture}
\end{figure}

\exercise Cho $a$ và $b$ là hai số thực. Chứng minh rằng $|a||b| = |ab|$.

\solution

Để chứng minh $|a||b| = |ab|$ ngắn gọn, chúng ta thực hiện kẻ bảng:

\begin{table}[H]
   \centering
   \begin{tabular}{|c||c|c|c|c|}
      \hline
      Điều kiện & $\begin{cases}a\geq 0\\b\geq0\end{cases}$ & $\begin{cases}a\geq0\\b<0\end{cases}$ & $\begin{cases}a<0\\b\geq 0\end{cases}$ & $\begin{cases}a<0\\b<0\end{cases}$ \\
      \hline
      $|a|$, $|b|$ & $\begin{cases}|a| = a\\|b| = b\end{cases}$ & $\begin{cases}|a| = a\\|b| = -b\end{cases}$ & $\begin{cases}|a| = -a\\|b| = b\end{cases}$ & $\begin{cases}|a| = -a\\|b| = -b\end{cases}$ \\
      \hline
      $|a||b|$ & $ab$ & $(-a)b = -ab$ & $a(-b) = -ab$ & $(-a)(-b) = ab$ \\
      \hline
      Dấu của $ab$ & $\geq 0$ & $< 0$ & $<0$ & $\geq 0$ \\
      \hline
      $|ab|$ & $ab$ & $-ab$ & $-ab$ & $ab$ \\
      \hline
   \end{tabular}
   \caption{Bảng so sánh $|a||b|$ và $|ab|$}
\end{table}

Qua bảng, chúng ta luôn có $|a||b| = |ab|$. Chúng ta có điều phải chứng minh.

\exercise Cho sô thực $a$. Chứng minh rằng
\begin{multicols}{2}
   \begin{enumerate}
      \item $|a| \geq a$;
      \item $|a|^2 = a^2$.
   \end{enumerate}
\end{multicols}

\solution

\setcounter{subexercise}{1}
\arabic{subexercise}. Khi $a \geq 0$ thì $|a| = a$. Trong trường hợp còn lại, nếu $a < 0$, chúng ta có $-a > 0$. Do đó $|a| > a$.

Vậy $|a| \geq a$ với mọi $a$ thực.

2. 
\begin{align*}
   |a|^2 &= |a|\cdot |a| = |a \cdot a| = \left|a^2\right|\\
   &= a^2 \equationexplanation{$a^2$ thì luôn không âm}.
\end{align*}
Chúng ta qua đó có điều phải chứng minh.

\exercise Chứng minh rằng với $a$ và $b$ là hai số thực thì $|a| + |b| \geq |a + b|$.

\solution

Chúng ta có những đẳng thức và bất đẳng thức sau:
\begin{equation*}
   \begin{cases}
      |a|^2 = a^2 \\
      |b|^2 = b^2 \\
      |a||b| = |ab| \geq ab
   \end{cases}.
\end{equation*}
Qua đó, 
\begin{align}
   |a|^2 + 2|a||b| + |b|^2 &\geq a^2 + 2ab + b^2 \nonumber\\
   \iff \left(|a| + |b|\right)^2 &\geq (a + b)^2 \nonumber\\
   \iff \left(|a| + |b|\right)^2 - \left|a + b\right|^2 &\geq 0 \nonumber\\
   \iff \left(|a| + |b| - |a + b|\right)\left(|a| + |b| + |a + b|\right) &\geq 0. \label{eq:toan_hoc_nen_tang:ham_so_mot_bien:ham_tung_phan:bdt23}
\end{align}

Do $|a|$, $|b|$, $|a + b|$ đều không âm nên $|a| + |b| + |a + b| \geq 0$. Cho nên:

\begin{align*}
   \text{\refeq{eq:toan_hoc_nen_tang:ham_so_mot_bien:ham_tung_phan:bdt23}} \iff |a| + |b| - |a + b| &\geq 0 \\
   \iff |a| + |b| &\geq |a + b|.
\end{align*}
Đây là điều phải chứng minh.

\exercise Giải các phương trình sau trên ẩn $x$ thực:

\begin{multicols}{2}
   \begin{enumerate}
      \item $\lfloor x \rfloor = 4$;
      \item $\left\lceil \frac{x}{4} \right\rceil = -2$;
      \item $2\left\lfloor -2x - 3 \right\rfloor - 1 = 1$;
      \item $3\lceil x \rceil^2 - 4\lceil x \rceil - 4 = 0$;
      \item $\lfloor x + 2 \rfloor^3 - \lfloor x \rfloor = 2$;
      \item $\left\lfloor \frac{x}{3} \right\rfloor + \left\lfloor \frac{x}{5} \right\rfloor = 7$. 
   \end{enumerate}
\end{multicols}

\solution

\setcounter{subexercise}{1}
\arabic{subexercise}. Giả sử $x \geq 5$. Khi đó, số nguyên lớn nhất không vượt quá $x$ sẽ phải đạt giá trị tối thiểu là $5$, và suy ra $\lfloor x \rfloor \geq 5$.

Giả sử $x < 4$. Nếu như vậy, số nguyên lớn nhất không vượt quá $x$ cũng không thể từ $4$ trở lên. Do vậy, $\lfloor x \rfloor < 5$. 

Ngoài ra, dễ thấy rằng nếu $4 \leq x < 5$ thì $\lfloor x \rfloor = 4$ theo định nghĩa. Vậy tập nghiệm của phương trình là $\left[4; 5\right)$.

Tổng quát hóa, chúng ta sẽ có $\lfloor x \rfloor = a \iff a \leq x < a + 1$ với $a \in \mathbb{N}$.

\stepcounter{subexercise}
\arabic{subexercise}. Với một lập luận tương tự như phần trước, chúng ta có
\begin{align*}
   &\left\lceil \frac{x}{4} \right\rceil = -2 \\
   \iff &-3 < \frac{x}{4} \leq -2 \\
   \iff &-12 < x \leq -8.
\end{align*}

Tập nghiệm của phương trình là $\left(-12; -8\right]$.

\stepcounter{subexercise}
\arabic{subexercise}. 

\begin{align*}
   2\lfloor -2x - 3 \rfloor - 1 &= 1 \\
   \iff \lfloor -2x - 3 \rfloor &= 1 \\
   \iff 1 \leq -2x - 3 &< 2 \\
   \iff -2 \geq x &> -\frac{5}{2}.
\end{align*}

Tập nghiệm của phương trình là $\left(-\frac{5}{2}; -2\right]$.

\stepcounter{subexercise}
\arabic{subexercise}.

\begin{align*}
   &3\lceil x \rceil^2 - 4\lceil x \rceil - 4 = 0 \\
   \iff &\left(3\lceil x \rceil + 2\right)\left(\lceil x \rceil - 2\right) = 0 \\
   \iff &\left[\begin{array}{l}
      \lceil x \rceil = -\frac{2}{3} \\
      \lceil x \rceil = 2
   \end{array}\right..
\end{align*}

Do kết quả của hàm trần luôn là số nguyên nên chỉ có một trường hợp
\begin{equation*}
   \lceil x \rceil = 2 \iff 1 < x \leq 2.
\end{equation*}
Vậy tập nghiệm của phương trình là $\left(1; 2\right]$.

\stepcounter{subexercise}
\arabic{subexercise}.

Đặt $y = \lfloor x \rfloor$ với $y \in \mathbb{N}$. Theo định nghĩa, $y$ là số nguyên lớn nhất không vượt quá $x$. Theo một cách nói khác, $y$ là số nguyên duy nhất thỏa mãn $y \leq x < y + 1$. Công $2$ vào tất cả các vế, chúng ta có $y + 2 \leq x + 2 < y + 3$. Nhận thấy rằng, $z = y + 2$ là số nguyên duy nhất thỏa mãn $z \leq x + 2 < z + 1$, hay $z$ là số nguyên lớn nhất không quá $x + 2$. Do vậy, 
\begin{equation}
   \lfloor x + 2 \rfloor = y + 2 = \lfloor x \rfloor + 2.
\end{equation}

Sử dụng điều kiện này để biến đổi phương trình:
\begin{align*}
   &\lfloor x + 2 \rfloor^3 - \lfloor x \rfloor = 2\\
   \iff &\left(\lfloor x \rfloor + 2\right)^3 - \left(\lfloor x \rfloor + 2\right) = 0 \\
   \iff &\left(\lfloor x \rfloor + 2\right)\left(\left(\lfloor x \rfloor + 2\right)^2 - 1\right) = 0 \\
   \iff &\left(\lfloor x \rfloor + 2\right)\left(\lfloor x \rfloor + 1\right)\left(\lfloor x \rfloor + 3\right) = 0 \\
   \iff &\left[\begin{array}{l}
      \lfloor x \rfloor = -1 \\
      \lfloor x \rfloor = -2 \\
      \lfloor x \rfloor = -3
   \end{array}\right. \\
   \iff &\left[\begin{array}{l}
      -1 \leq x < 0 \\
      -2 \leq x < -1 \\
      -3 \leq x < -2
   \end{array}\right. \\
   \iff & -3 \leq x < 0.
\end{align*}

Qua đó, chúng ta có nghiệm $x \in \left[-3; 0\right)$.

6. Nếu $x \geq 15$, $$\begin{cases}
   \frac{x}{3} \geq 5 \\
   \frac{x}{5} \geq 3
\end{cases} \implies \begin{cases}
   \left\lfloor \frac{x}{3} \right\rfloor \geq 5 \\
   \left\lfloor \frac{x}{5} \right\rfloor \geq 3
\end{cases} \implies \left\lfloor \frac{x}{3} \right\rfloor + \left\lfloor \frac{x}{5} \right\rfloor \geq 8.$$

Nếu $x < 15$, một cách tương tự, chúng ta cũng có
\begin{equation*}
   \begin{cases}
      \frac{x}{3} < 5 \\
      \frac{x}{5} < 3
   \end{cases} \implies \begin{cases}
      \left\lfloor \frac{x}{3} \right\rfloor < 5 \\
      \left\lfloor \frac{x}{5} \right\rfloor < 3
   \end{cases}.
\end{equation*}
Do $\left\lfloor \frac{x}{3} \right\rfloor$ và $\left\lfloor \frac{x}{5} \right\rfloor$ đều là số nguyên cho nên
$$
\begin{cases}
   \left\lfloor \frac{x}{3} \right\rfloor \leq 4 \\
   \left\lfloor \frac{x}{5} \right\rfloor \leq 2
\end{cases} \implies \left\lfloor \frac{x}{3} \right\rfloor + \left\lfloor \frac{x}{5} \right\rfloor \leq 6.
$$

Qua hai trường hợp, chúng ta thấy không có $x$ để $\left\lfloor \frac{x}{3} \right\rfloor + \left\lfloor \frac{x}{5} \right\rfloor = 7$. Vậy phương trình vô nghiệm.