\subsection{Hàm số xác định từng phần}

\ % Lùi đầu dòng

Không phải lúc nào hàm số trong đời sống có thể biểu diễn dưới dạng một biểu thức. Khi này, chúng ta sẽ chia nhỏ đồ thị của hàm số thành các phần nhỏ, và biểu diễn từng phần thông qua biểu thức. Đó cũng là lí do cho tên gọi \defText{hàm số xác định từng phần}.

Một ví dụ là hàm giá trị tuyệt đối. Khả năng cao bạn đọc đã biết rằng giá trị tuyệt đối của một số $x$\footnote{Nếu không biết thì bạn đọc đọc qua phần đồ thị vẫn hiểu chứ?} được xác định như sau,

\begin{equation*}
   \defMath{|x|= } \begin{cases}
      \defMath{x \defText{ nếu } x \geq 0 }\\
      \defMath{-x \defText{ nếu } x < 0 }
   \end{cases}.
\end{equation*}

Hai hàm từng phần khác cũng thông dụng nhưng ít khi được đề cập đến là \defText{hàm sàn} (hay \defText{hàm phần nguyên}, \defText{hàm làm tròn xuống}) và \defText{hàm trần} (hay \defText{hàm làm tròn lên}). Hàm sàn tác dụng lên $x$ sẽ làm tròn xuống $x$ đến số nguyên lớn nhất không vượt quá $x$ với kí hiệu là $\defMath{\lfloor x \rfloor}$.
Tương tự, hàm trần của $x$ làm tròn lên $x$ đến số nguyên nhỏ nhất không vượt quá $x$ với kí hiệu là $\defMath{\lceil x \rceil}$.

Ví dụ:
\begin{multicols}{3}
   \begin{itemize}
      \item $\lfloor 2{,}5 \rfloor = 2$;
      \item $\lceil 2{,}5 \rceil = 3$;
      \item $\lfloor -2{,}5 \rfloor = -3$;
      \item $\lceil -2{,}5 \rceil = -2$;
      \item $\lfloor 2 \rfloor = 2$;
      \item $\lceil 2 \rceil = 2$.
   \end{itemize}
\end{multicols}

\exercise Giải các phương trình và bất phương trình sau trên ẩn $x$ thực:

\begin{multicols}{2}
   \begin{enumerate}
      \item $|x + 4| = 9$;
      \item $|x - 3| = -9$;
      \item $|7 - 2x| < 9$;
      \item $|3 + 6x| \geq 9$;
      \item $\left|3(x - 5) + 2\right| + 3 = 9$;
      \item $2x + 3 + |3x + 4| > 0$;
      \item $|x + 4| = |7x - 12|$;
      \item $|6x + 9| > |6x - 3|$;
      \item $|2x + 2| + |x + 1| = 9$;
      \item $|3x + 3| + |3x - 4| \leq 7$;
   \end{enumerate}
\end{multicols}

\solution

1. Xét hai trường hợp:

\textcolor{colorEmphasisCyan}{Trường hợp một --- $x + 4 \geq 0$}. Khi này, phá dấu giá trị tuyệt đối để có $|x + 4| = x + 4$. Cho nên phương trình ban đầu sẽ tương đương với:

\begin{equation*}
   x + 4 = 9 \iff x = 5.
\end{equation*}

\textcolor{colorEmphasis}{Trường hợp hai --- $x + 4 < 0$}. Khi này,

\begin{align*}
   &\begin{cases}
      |x + 4| = 9 \\
      x + 4 < 0
   \end{cases} \\
   \iff &-(x + 4) = 9 \equationexplanation{Phá dấu giá trị tuyệt đối: $|x + 4| = -(x + 4)$.}\\
   \iff &x + 4 = -9 \iff x = -13.
\end{align*}

Kết hợp hai trường hợp, có được $x \in \{5; -13\}$. Thử lại trực tiếp thấy thỏa mãn.

Vậy tập nghiệm của phương trình là $\{5; -13\}$.

2. Đặt $f(x) = |x|$.

Nếu \textcolor{colorEmphasisCyan}{$x \geq 0$} thì $f(x) = |x| = x$ và hiển nhiên $f(x) \geq 0$. Nếu \textcolor{colorEmphasis}{$x < 0$} thì $f(x) = |x| = -x$. Có $x < 0 \iff -x > 0 \iff f(x) > 0$.

Kết hợp lại, chúng ta có $f(x) \geq 0$ với mọi $x \in \mathbb{R}$. Suy ra được rằng $f(x - 3) \geq 0$. Tuy nhiên, phương trình được cho có thể được viết lại là $f(x - 3) = -9$. Do đó, phương trình vô nghiệm.

Sai lầm thường gặp ở dạng bài này là có suy luận như sau:

\begin{equation*}
   |x - 3| = -9 \iff \left[\begin{array}{l}
      x - 3 = -9 \\
      x - 3 = 9
   \end{array}\right..
\end{equation*}

3. Một lần nữa, xét hai trường hợp:

\textcolor{colorEmphasisCyan}{Trường hợp một --- $7 - 2x \geq 0$}. Thực hiện biến đổi:

\begin{align*}
   &\begin{cases}
      |7 - 2x| < 9 \\
      7 - 2x \geq 0
   \end{cases} \\
   \iff &\begin{cases}
      7 - 2x < 9 \\
      7 - 2x \geq 0
   \end{cases} \\
   \iff &0 \leq 7 - 2x < 9 \\
   \iff &-7 \leq -2x < 2 \\
   \iff &\frac{7}{2} \geq x > -1.
\end{align*}

\textcolor{colorEmphasis}{Trường hợp hai --- $7 - 2x < 0$}:

\begin{align*}
   &\begin{cases}
      |7 - 2x| < 9 \\
      7 - 2x < 0
   \end{cases} \\
   \iff &\begin{cases}
      - (7 - 2x) < 9 \\
      7 - 2x < 0
   \end{cases} \\
   \iff & -9 < 7 - 2x < 0 \\
   \iff & -16 < -2x < -7 \\
   \iff & 8 > x > \frac{7}{2}.
\end{align*}

Kết hợp hai trường hợp, có được $x \in \left(-1; 8\right)$. Có biến đổi là tương đương trong tập xác định cho nên phương trình có nghiệm $x \in \left(-1; 8\right)$.

4. \textcolor{colorEmphasisCyan}{Trường hợp một --- $3 + 6x \geq 0$}:

\begin{align*}
   &\begin{cases}
      |3 + 6x| \geq 9 \\
      3 + 6x \geq 0
   \end{cases} \\
   \iff 3 + 6x \geq 9 \\
   \iff x \geq 1.
\end{align*}

\textcolor{colorEmphasis}{Trường hợp hai --- $3 + 6x < 0$}:

\begin{align*}
   &\begin{cases}
      |3 + 6x| \geq 9 \\
      3 + 6x < 0
   \end{cases} \\
   \iff &-(3 + 6x) \geq 9 \\
   \iff &3 + 6x \leq -9 \\
   \iff &x \leq -2.
\end{align*}

Do trong mỗi trường hợp, mọi biến đổi là tương đương, nên chúng ta có tập nghiệm của bất phương trình là $\left(-\infty; -2\right] \cup \left[1; \infty\right)$.

5.

\begin{align}
   \left|3(x-5) + 2\right| + 3 &= 9 \nonumber\\
   \iff \left|3x - 13\right| &= 6. \label{eq:toan_hoc_nen_tang:ham_so_mot_bien:ham_tung_phan:pt5}
\end{align}

Đến đây, chúng ta xét dấu của $3x - 13$ để chia làm hai trường hợp. Nhằm mục đích rút gọn, tác giả sẽ gộp cả hai trường hợp vào để có phương trình \refeq{eq:toan_hoc_nen_tang:ham_so_mot_bien:ham_tung_phan:pt5} tương đương với:

\begin{equation*}
   \text{\refeq{eq:toan_hoc_nen_tang:ham_so_mot_bien:ham_tung_phan:pt5}} \iff \left[\begin{array}{l}
      3x - 13 = 6 \\
      -(3x - 13) = 6
   \end{array}\right. \iff \left[\begin{array}{l}
      x = \frac{19}{3} \\
      x = \frac{7}{3}
   \end{array}\right.
\end{equation*}

Và qua đó, tập nghiệm của phương trình là $\left\{\frac{19}{3}; \frac{7}{3}\right\}$.

6. \textcolor{colorEmphasisCyan}{Trường hợp một --- $3x + 4 > 0$}. Khi này, 

\begin{equation*}
   \begin{cases}
      2x + 3 + |3x + 4| = 2x + 3 + (3x + 4) \\
      3x + 4 > 0
   \end{cases} \iff \begin{cases}
      2x + 3 + |3x + 4| = 5x + 7 \\
      x > -\frac{4}{3}
   \end{cases}
\end{equation*}

\begin{equation*}
   2x + 3 + |3x + 4| > 5 \left(-\frac{4}{3}\right) + 7 \iff 2x + 3 + |3x + 4| > \frac{1}{3} > 0.
\end{equation*}

\textcolor{colorEmphasis}{Trường hợp hai --- $3x + 4 < 0$}:



\exercise Phác thảo đồ thị của những hàm sau:

\begin{multicols}{2}
   \begin{enumerate}
      \item $f(x) = \begin{cases}
         x + 1 \text{ nếu } x \leq 1 \\
         2 \text{ nếu } x > 1
      \end{cases}$;
      \item $f(x) = \begin{cases}
         x^3 + 4 \text{ nếu } x < 0 \\
         -x^2 + 1 \text{ nếu } x \geq 0
      \end{cases}$;
      \item $f(x) = \begin{cases}
         -\frac{4}{x^2} \text{ nếu } -2 > x \geq -3 \\
         \parbox{0.29\textwidth}{$\begin{array}{cl}
            -\frac{5}{x^2 + 1} &\text{nếu } x \geq -2 \text{ thực để} \\
            &\frac{5}{x^2 + 1}\text{ là số nguyên}
         \end{array}$}
      \end{cases}$;
      \item $f(x) = \begin{cases}
         \frac{2x - 1}{x - 1} \text{ nếu } -3 \leq x < 0 \\
         \left(x + 1\right)^2 - 3x \text{ nếu } 0 \leq x < 2 \\
         \frac{2x - 1}{x - 1} \text{ nếu } 2 \leq x \leq 3
      \end{cases}$;
      \item $f(x) = \begin{cases}
         x^3 + 3 \text{ nếu } x \leq 0 \\
         -2x + 2 \text{ nếu } 0 < x < 1 \\
         2 + x - x^2 \text{ nếu } x \geq 1
      \end{cases}$.
   \end{enumerate}
\end{multicols}

\solution

1.

\begin{figure}[H]
	\centering
	\begin{tikzpicture}
		\draw[->] (-4, 0) -- (4, 0) node[right] {$x$};
		\draw[->] (0, -4) -- (0, 4)  node[above] {$f(x)$};
		\draw[graph thickness, samples=80, color=colorEmphasisCyan, domain=-4.000:1] plot (\x, {(((\x)/1) + 1) / 1});
		\draw[graph thickness, samples=80, color=colorEmphasisCyan, domain=1:4] plot (\x, 2);
		\filldraw[color=colorEmphasisCyan] (-3.0, -2.0) circle (\pointSize) node[above left] {$\left(-3;-2\right)$};
		\filldraw[color=colorEmphasisCyan] (-1.0, 0.0) circle (\pointSize) node[above left] {$\left(-1;0\right)$};
		\filldraw[color=colorEmphasisCyan] (1.0, 2.0) circle (\pointSize) node[above] {$\left(1;2\right)$};
      \filldraw[color=colorEmphasisCyan] (3.0, 2.0) circle (\pointSize) node[above] {$\left(3;2\right)$};
	\end{tikzpicture}
	\caption{Đồ thị của $\begin{cases}
         x + 1 \text{ nếu } x \leq 1 \\
         2 \text{ nếu } x > 1
      \end{cases}$}
\end{figure}

2.

\begin{figure}[H]
	\centering
	\begin{tikzpicture}
		\draw[->] (-4, 0) -- (4, 0) node[right] {$x$};
		\draw[->] (0, -4) -- (0, 5)  node[above] {$f(x)$};
      \draw[graph thickness, samples=80, color=colorEmphasisCyan, domain=-2.000:0.000] plot (\x, {(((\x)/1)^3 + 4) / 1});
      \filldraw[color=colorEmphasisCyan] (-2, -4) circle (\pointSize) node[above left] {$\left(-2;-4\right)$};
		\filldraw[color=colorEmphasisCyan] (-1.0, 3.0) circle (\pointSize) node[left] {$\left(-1;3\right)$};
		\draw[color=colorEmphasisCyan, hollow point] (0.0, 4.0) circle (\pointSize) node[right] {$\left(0;4\right)$};
      \draw[graph thickness, samples=80, color=colorEmphasisCyan, domain=0.000:2.236] plot (\x, {(-((\x)/1)^2 + 1) / 1});
		\filldraw[color=colorEmphasisCyan] (0.0, 1.0) circle (\pointSize) node[left] {$\left(0;1\right)$};
		\filldraw[color=colorEmphasisCyan] (1.0, 0.0) circle (\pointSize) node[above right] {$\left(1;0\right)$};
		\filldraw[color=colorEmphasisCyan] (2.0, -3.0) circle (\pointSize) node[left] {$\left(2;-3\right)$};
	\end{tikzpicture}
	\caption{Đồ thị của $\begin{cases}
         x^3 + 4 \text{ nếu } x < 0 \\
         -x^2 + 1 \text{ nếu } x \geq 0
      \end{cases}$}
\end{figure}

Để ý rằng $f(x)$ đứt đoạn tại giá trị $x = 0$. Cụ thể, $f(x)$ không nhận giá trị $x^3 + 4$ khi $x = 0$. Tuy nhiên, không thể vẽ điểm ngay liền trước nó (không có số âm lớn nhất), nên người ta hay dùng đường tròn rỗng để biểu thị điểm đứt đoạn này.

3. Trước hết, cần xác định các giá trị của $x \geq -2$ để $\frac{5}{x^2 + 1}$ là số nguyên. Do với mọi $x \in \mathbb{R}$ thì $$x^2 \geq 0 \iff x^2 + 1 \geq 1 > 0 \iff 5 \geq \frac{5}{x^2 + 1} > 0.$$ Mà cần phải để $\frac{5}{x^2 + 1} \in \mathbb{N}$ cho nên $\frac{5}{x^2 + 1} \in \left\{1; 2; 3; 4; 5\right\}$. Với để ý đến điều kiện $x \geq -2$, kẻ bảng để xác định các giá trị có thể của $x$:

\begin{table}[H]
   \centering
   \begin{tabular}{|c|c|c|c|c|c|}
   \hline
   $\displaystyle \frac{5}{x^2 + 1}$ & $1$ & $2$ & $3$ & $4$ & $5$ \\
   \hline
   $x^2 + 1$ & $5$ & $\displaystyle\frac{5}{2}$ & $\displaystyle\frac{5}{3}$ & $\displaystyle\frac{5}{4}$ & $1$ \\
   \hline
   $x^2$ & $4$ & $\displaystyle\frac{3}{2}$ & $\displaystyle\frac{2}{3}$ & $\displaystyle\frac{1}{4}$ & $1$ \\
   \hline
   $x$ & $\left\{-2; 2\right\}$ & $\left\{-\sqrt{\frac{3}{2}}; \sqrt{\frac{3}{2}}\right\}$ & $\left\{-\sqrt{\frac{2}{3}}; \sqrt{\frac{2}{3}}\right\}$ & $\left\{-\frac{1}{2}; \frac{1}{2}\right\}$ & $0$ \\
   \hline
   \end{tabular}
   \caption{Bảng giá trị của $\frac{5}{x^2 + 1}$ với $x$} 
\end{table}

Từ đây, có được đồ thị của $f(x)$:

\begin{figure}[H]
	\centering
	\begin{tikzpicture}
		\draw[->] (-4, 0) -- (3, 0) node[right] {$x$};
		\draw[->] (0, -6) -- (0, 1)  node[above] {$f(x)$};
		\draw[graph thickness, samples=80, color=colorEmphasisCyan, domain=-3.000:-2.000] plot (\x, {(-4 / (\x)^2)});
      \filldraw[color=colorEmphasisCyan] (-3.0, -0.4444444444444444) circle (\pointSize) node[left] {$\left(-3;- \frac{4}{9}\right)$};
		\filldraw[color=colorEmphasisCyan] (-2.0, -1.0) circle (\pointSize) node[above right] {$\left(-2;-1\right)$};

		\filldraw[color=colorEmphasisCyan] ({ 2.0 }, { -1.0 }) circle (\pointSize) node[above] {$\left({2};{-1}\right)$};
		\filldraw[color=colorEmphasisCyan] ({ 0.0 }, { -5.0 }) circle (\pointSize) node[below] {$\left({0};{-5}\right)$};
		\filldraw[color=colorEmphasisCyan] ({ -0.5*sqrt(6) }, { -2.0 }) circle (\pointSize) node[left] {$\left({- \frac{\sqrt{6}}{2}};{-2}\right)$};
		\filldraw[color=colorEmphasisCyan] ({ 0.5*sqrt(6) }, { -2.0 }) circle (\pointSize) node[right] {$\left({\frac{\sqrt{6}}{2}};{-2}\right)$};
		\filldraw[color=colorEmphasisCyan] ({ -0.333333333333333*sqrt(6) }, { -3.0 }) circle (\pointSize) node[left] {$\left({- \frac{\sqrt{6}}{3}};{-3}\right)$};
		\filldraw[color=colorEmphasisCyan] ({ 0.333333333333333*sqrt(6) }, { -3.0 }) circle (\pointSize) node[right] {$\left({\frac{\sqrt{6}}{3}};{-3}\right)$};
		\filldraw[color=colorEmphasisCyan] ({ -0.500000000000000 }, { -4.0 }) circle (\pointSize) node[left] {$\left({- \frac{1}{2}};{-4}\right)$};
		\filldraw[color=colorEmphasisCyan] ({ 0.500000000000000 }, { -4.0 }) circle (\pointSize) node[right] {$\left({\frac{1}{2}};{-4}\right)$};
	\end{tikzpicture}
	\caption{Đồ thị của $\begin{cases}
      -\frac{4}{x^2} \text{ nếu } -2 > x \geq -3 \\
      \parbox{0.29\textwidth}{$\begin{array}{cl}
         -\frac{5}{x^2 + 1} &\text{nếu } x \geq -2 \text{ thực để} \\
         &\frac{5}{x^2 + 1}\text{ là số nguyên}
      \end{array}$}
   \end{cases}$}
\end{figure}

4. 

\begin{figure}[H]
	\centering
	\begin{tikzpicture}
		\draw[->] (-4, 0) -- (4, 0) node[right] {$x$};
		\draw[->] (0, 0) -- (0, 4)  node[above] {$f(x)$};
		\draw[graph thickness, samples=80, color=colorEmphasisCyan, domain=-3:0] plot (\x, {((2*((\x)/1) - 1) / (((\x)/1) - 1)) / 1});
		\draw[graph thickness, samples=80, color=colorEmphasisCyan, domain=2:3] plot (\x, {((2*((\x)/1) - 1) / (((\x)/1) - 1)) / 1});
		\filldraw[color=colorEmphasisCyan] ({ -3.0 }, { 1.75 }) circle (\pointSize) node[above] {$\left({-3};{\frac{7}{4}}\right)$};
		\filldraw[color=colorEmphasisCyan] ({ 0.0 }, { 1.0 }) circle (\pointSize) node[above right] {$\left({0};{1}\right)$};
		\filldraw[color=colorEmphasisCyan] ({ 2.0 }, { 3.0 }) circle (\pointSize) node[above] {$\left({2};{3}\right)$};
		\filldraw[color=colorEmphasisCyan] ({ 3.0 }, { 2.5 }) circle (\pointSize) node[right] {$\left({3};{\frac{5}{2}}\right)$};
		\draw[graph thickness, samples=80, color=colorEmphasisCyan, domain=0:2] plot (\x, {((((\x)/1) + 1)^2 - 3 * ((\x)/1)) / 1});
	\end{tikzpicture}
	\caption{Đồ thị của $\begin{cases}
         \frac{2x - 1}{x - 1} \text{ nếu } -3 \leq x < 0 \\
         \left(x + 1\right)^2 - 3x \text{ nếu } 0 \leq x < 2 \\
         \frac{2x - 1}{x - 1} \text{ nếu } 2 \leq x \leq 3
      \end{cases}$}
\end{figure}

5.

\begin{figure}[H]
	\centering
	\begin{tikzpicture}
		\draw[->] (-4, 0) -- (4, 0) node[right] {$x$};
		\draw[->] (0, -4) -- (0, 4)  node[above] {$f(x)$};
		\draw[graph thickness, samples=80, color=colorEmphasisCyan, domain=-1.913:0] plot (\x, {(((\x)/1)^3 + 3) / 1});
		\filldraw[color=colorEmphasisCyan] ({ 0.0 }, { 3.0 }) circle (\pointSize) node[right] {$\left({0};{3}\right)$};
		\draw[graph thickness, samples=80, color=colorEmphasisCyan, domain=0.000:1.000] plot (\x, {(-2*(\x) + 2) / 1});
      \draw[color=colorEmphasisCyan, hollow point] (0, 2) circle (\pointSize) node[below left] {$\left({0};{2}\right)$};
      \draw[color=colorEmphasisCyan, hollow point] (1, 0) circle (\pointSize) node[below left] {$\left({1};{0}\right)$};
      \draw[graph thickness, samples=80, color=colorEmphasisCyan, domain=1.000:3.000] plot (\x, {(2 + ((\x)/1) - ((\x)/1)^2) / 1});
      \filldraw[color=colorEmphasisCyan] (1, 2) circle (\pointSize) node[above right] {$\left({1};{2}\right)$};
      \filldraw[color=colorEmphasisCyan] (2, 0) circle (\pointSize) node[above right] {$\left({2};{0}\right)$};
	\end{tikzpicture}
	\caption{Đồ thị của $\begin{cases}
      x^3 + 3 \text{ nếu } x \leq 0 \\
      -2x + 2 \text{ nếu } 0 < x < 1 \\
      2 + x - x^2 \text{ nếu } x \geq 1
   \end{cases}$}
\end{figure}

\exercise Giải các phương trình và bất phương trình sau trên ẩn $x$ thực:

\begin{multicols}{2}
   \begin{enumerate}
      \item 
   \end{enumerate}
\end{multicols}

\solution

