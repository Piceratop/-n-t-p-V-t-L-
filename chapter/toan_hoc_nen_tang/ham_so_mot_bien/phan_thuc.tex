\subsection{Hàm phân thức}

\ % Lùi đầu dòng

Hàm cộng, hàm trừ và hàm nhân của hai hàm đa thức là những hàm đa thức. Tuy nhiên, hàm thương lại không như vậy. Do khi chia hai đa thức có những tính chất đặc biệt, nên chúng ta xây dựng một khái niệm mới là hàm \defText{phân thức}. Một hàm $f$ được gọi là phân thức nếu $\defMath{f = 0}$, hoặc: $$\defMath{f = \left(\frac{p}{q}\right)}$$ với $p$ và $q$ là hai đa thức. Trong trường hợp $f \neq 0$, tập xác định của $f$ là tập hợp các giá trị $x$ sao cho $q(x) \neq 0$. 

Khái niệm về phân thức dẫn chúng ta một cách tự nhiên đến khái niệm về một dạng phân thức đặc biệt mang tên \defText{số mũ âm}. Khi mũ một số bằng số âm, chúng ta có thể viết lại là $$\defMath{x^{-n} = \frac{1}{x^n}}.$$ Và đương nhiên, để có thể chia được thì $x \neq 0$.

\exercise Cho biết tập xác định, tập giá trị và phác thảo đồ thị của những hàm sau:
\begin{multicols}{3}
   \begin{enumerate}
      \item $\displaystyle f(x) = \frac{2}{x}$;
      \item $\displaystyle f(x) = \frac{1}{x^2 + 4x + 4}$;
      \item $\displaystyle f(x) = \frac{2x - 5}{x - 3}$;
      \item $\displaystyle f(x) = \frac{x + 1}{2x^2 + 5x - 3}$;
      \item $\displaystyle f(x) = \frac{x^2 - 3x - 2}{x^2 + 2x + 1}$;
      \item $\displaystyle f(x) = \frac{2x^2 + 2}{x - 2}$;
      \item $\displaystyle f(x) = \frac{x^2 + 4x - 5}{x - 1}$;
      \item $\displaystyle f(x) = \frac{x - 1}{x^2 + 4x - 5}$;
      \item $\displaystyle f(x) = \frac{1}{x^2 + x + 1}$.
   \end{enumerate}
\end{multicols}

\solution

{
   \begin{minipageindent}{0.44\textwidth}
      1. Theo định nghĩa hàm phân thức, tập xác định của hàm $f(x) = \frac{2}{x}$ là $\mathbb{R} \setminus \left\{0\right\}$.
      
      Kết quả của $f(x)$ phải khác $0$ do nếu như vậy thì $f(x) = \frac{2}{x} = 0 \implies 2 = 0\times x = 0$, vô lí.
      
      Tuy nhiên, mọi số $y$ khác $0$ đều có thể là giá trị của $f(x)$ do $$f\left(\frac{2}{y}\right) = \frac{2}{\frac{2}{y}} = y.$$
      
      Vậy tập giá trị của $f(x)$ là $\mathbb{R} \setminus \left\{0\right\}$.
   \end{minipageindent}
   \hfill
   \begin{minipageindent}{0.55\textwidth}
      \begin{figure}[H]
         \centering
         \begin{tikzpicture}
            \draw[->] (-4, 0) -- (4, 0) node[right] {$x$};
            \draw[->] (0, -4) -- (0, 4) node[above] {$f(x)$};
            \draw[color=colorEmphasisCyan, graph thickness, smooth, samples=100] plot[domain=-4:-0.5] (\x, {2/\x});
            \draw[color=colorEmphasisCyan, graph thickness, smooth, samples=100] plot[domain=0.5:4] (\x, {2/\x});
            \foreach \x/\y/\pos in {1/2/right, -1/-2/left, -2/-1/above, 2/1/below} {
               \filldraw[color=colorEmphasisCyan] (\x, \y) circle (\pointSize) node[\pos] {$\left(\x; \y\right)$};
            }
         \end{tikzpicture}
         \caption{Đồ thị của hàm $f(x) = \frac{2}{x}$}
         \label{fig:ham_so_mot_bien:phan_thuc:2_x}
      \end{figure}
   \end{minipageindent}
}

{
   \begin{minipageindent}{0.44\textwidth}
      2. Để phân thức có nghĩa thì mẫu số của phân thức phải khác $0$. Viết và bất phương trình này:

      \begin{align*}
         x^2 + 4x + 4 &\neq 0\\
         \iff \left(x + 2\right)^2 &\neq 0\\
         \iff x + 2 &\neq 0 \\
         \iff x &\neq -2
      \end{align*}
      Vậy tập xác định của $f(x)$ là $\mathbb{R} \setminus \left\{-2\right\}$.

      Có mẫu số $x^2 + 4x + 4 = (x + 2)^2 \geq 0$, mà mẫu số phải khác $0$ nên có $x^2 + 4x + 4 > 0$. Chia hai số dương luôn được số dương, cho nên $f(x)$ chỉ nhận giá trị dương.
   \end{minipageindent}
   \hfill
   \begin{minipageindent}{0.55\textwidth}
      \begin{figure}[H]
         \centering
         \begin{tikzpicture}
            \draw[->] (-6, 0) -- (2, 0) node[right] {$x$};
            \draw[->] (0, 0) -- (0, 5)  node[above] {$y$};
            \draw[graph thickness, samples=80, color=colorEmphasisCyan, domain=-6.000:-2.447] plot (\x, {1/((\x)^2 + 4*(\x) + 4)});
            \draw[graph thickness, samples=80, color=colorEmphasisCyan, domain=-1.553:2.000] plot (\x, {1/((\x)^2 + 4*(\x) + 4)});
            \filldraw[color=colorEmphasisCyan] (1, {1/9}) circle (\pointSize) node[below] {$\left(1; \frac{1}{9}\right)$};
            \filldraw[color=colorEmphasisCyan] (0, {1/4}) circle (\pointSize) node[below] {$\left(0; \frac{1}{4}\right)$};
            \filldraw[color=colorEmphasisCyan] (-1, 1) circle (\pointSize) node[right] {$\left(-1; 1\right)$};
            \filldraw[color=colorEmphasisCyan] (-3, 1) circle (\pointSize) node[left] {$\left(-3; 1\right)$};
            \filldraw[color=colorEmphasisCyan] (-{5 / 2}, 4) circle (\pointSize) node[left] {$\left(-\frac{5}{2}; 4\right)$};
         \end{tikzpicture}
         \caption{Đồ thị của hàm $f(x) = \frac{1}{x^2 + 4x + 4}$}
         \label{fig:ham_so_mot_bien:phan_thuc:1_x2_4x_4}
      \end{figure}
   \end{minipageindent}
}

Ngược lại, mọi giá trị dương $y$ đều có thể biểu diễn thông qua $f(x)$ do \begin{align*}
   &f\left(-2 + \frac{1}{\sqrt{y}}\right) = \frac{1}{\left(-2 + \frac{1}{\sqrt{y}}\right)^2 + 4\left(-2 + \frac{1}{\sqrt{y}}\right) + 4}\\
   =& \frac{1}{\left(\left(-2 + \frac{1}{\sqrt{y}}\right) + 2\right)^2} \\
   =&\frac{1}{\left(\frac{1}{\sqrt{y}}\right)^2} =\frac{1}{\frac{1}{y}} =y.
\end{align*}
Vậy tập giá trị của $f(x)$ là $\mathbb{R}^+$.

{
   \begin{minipageindent}{0.48\textwidth}
      3. Để $x$ thuộc tập xác định của hàm $f(x) = \frac{2x - 5}{x - 3}$ thì $x - 3 \neq 0 \implies x \neq 3$. Vậy tập xác định của $f(x)$ là $\mathbb{R} \setminus \left\{3\right\}$.

      Giả sử có $y$ sao cho $y = f(x)$. Khi này, chúng ta có \begin{align*}
         y &= \frac{2x - 5}{x - 3}\\
         \implies y(x - 3) &= 2x - 5\\
         \iff yx - 3y &= 2x - 5\\
         \iff yx - 2x &= 3y - 5\\
         \iff x(y - 2) &= 3y - 5.
      \end{align*}

      Nếu $y = 2$ thì chúng ta sẽ có $x(y - 2) = 3y - 5 \implies x(2 - 2) = 3\times 2 - 5 \implies 0 = 1$, vô lí.

      Nếu $y \neq 2$ thì $x = \frac{3y - 5}{y - 2}$. Thay ngược lại giá trị $x$ này: 
   \end{minipageindent}
   \hfill
   \begin{minipageindent}{0.5\textwidth}
      \begin{figure}[H]
         \centering
         \begin{tikzpicture}
            \draw[->] (-0.5, 0) -- (6.5, 0) node[right] {$x$};
            \draw[->] (0, -1) -- (0, 6)  node[above] {$y$};
            \draw[graph thickness, samples=80, color=colorEmphasisCyan, domain=-0.500:2.667] plot (\x, {(2*(\x) - 5)/((\x) - 3)});
            \draw[graph thickness, samples=80, color=colorEmphasisCyan, domain=3.250:6.500] plot (\x, {(2*(\x) - 5)/((\x) - 3)});

            \filldraw[color=colorEmphasisCyan] (0, {5/3}) circle (\pointSize) node[above right] {$\left(0; \frac{5}{3}\right)$};
            \filldraw[color=colorEmphasisCyan] (1, {3/2}) circle (\pointSize) node[above right] {$\left(1; \frac{3}{2}\right)$};
            \filldraw[color=colorEmphasisCyan] ({5/2}, 0) circle (\pointSize) node[above right] {$\left(\frac{5}{2}; 0\right)$};

            \filldraw[color=colorEmphasisCyan] (4, 3) circle (\pointSize) node[below left] {$\left(4; 3\right)$};
            \filldraw[color=colorEmphasisCyan] (5, {5/2}) circle (\pointSize) node[below left] {$\left(5; \frac{5}{2}\right)$};

         \end{tikzpicture}
         \caption{Đồ thị của $f(x) = \frac{2 x - 5}{x - 3}$}
      \end{figure}
   \end{minipageindent}
}

\begin{equation*}
   f\left(\frac{3y - 5}{y - 2}\right) = \frac{2\left(\frac{3y - 5}{y - 2}\right) - 5}{\left(\frac{3y - 5}{y - 2}\right) - 3} = \frac{\frac{6y - 10 - 5y + 10}{y - 2}}{\frac{3y - 5 - 3y + 6}{y - 2}} = \frac{\frac{y}{y - 2}}{\frac{1}{y - 2}} = y.
\end{equation*}

Qua lập luận vừa rồi, chúng ta có kết luận rằng tập giá trị của $f(x)$ là $\mathbb{R} \setminus \left\{2\right\}$.

{
   \begin{minipageindent}{0.48\textwidth}
      4. Để $f(x)$ có nghĩa thì mẫu số phải khác $0$. Có:
      \begin{align*}
         2x^2 + 5x - 3 &\neq 0 \\
         \iff (2x - 1)(x + 3) &\neq 0 \qquad \parbox[c]{0.36\textwidth}{\textcolor{colorEmphasis}{(Phân tích đa thức thành nhân tử.)}}\\
         \iff x &\notin \left\{\frac{1}{2}; -3\right\}.
      \end{align*}

      Qua đó, tập xác định của $f(x)$ là $\mathbb{R} \setminus \left\{\frac{1}{2}; -3\right\}$.

      Bây giờ, chúng ta cần tìm những giá trị $y$ sao cho tồn tại $x$ để $y = f(x)$. Với $y = 0$ thì có $f(-1) = 0$ từ đồ thị \ref{fig:ham_so_mot_bien:phan_thuc:1_x2_5x_3}.
      Với $y \neq 0$, đặt $$x = \frac{\sqrt{49y^2-2y+1}-5y+1}{4y}.$$ $x$ luôn nhận giá trị thực do mẫu số khác $0$ ($4y\neq 0$) và phần tử bên trong dấu khai căn $49y^2 - 2y + 1 = 48y^2 + y^2 - 2y + 1 = 48y^2 + (y - 1)^2$ luôn không âm. Thay giá trị $x$ này vào tử số của $f(x)$:
      \begin{align*}
         x + 1 &= \frac{\sqrt{49y^2-2y+1}-5y+1}{4y} + 1\\
         &= \frac{\sqrt{49y^2-2y+1}-y+1}{4y}.
      \end{align*}
   \end{minipageindent}
   \hfill
   \begin{minipageindent}{0.5\textwidth}
      \begin{figure}[H]
         \centering
         \begin{tikzpicture}
            \draw[->] (-5, 0) -- (2, 0) node[right] {$x$};
            \draw[->] (0, -4) -- (0, 4)  node[above] {$y$};
            \draw[graph thickness, samples=80, color=colorEmphasisCyan, domain=-5.000:-3.073] plot (\x, {((\x) + 1)/(2*(\x)^2 + 5*(\x) - 3)});
            \draw[graph thickness, samples=80, color=colorEmphasisCyan, domain=-2.930:0.448] plot (\x, {((\x) + 1)/(2*(\x)^2 + 5*(\x) - 3)});
            \draw[graph thickness, samples=80, color=colorEmphasisCyan, domain=0.555:2.000] plot (\x, {((\x) + 1)/(2*(\x)^2 + 5*(\x) - 3)});

            \filldraw[color= colorEmphasisCyan] (-1, 0.0) circle (\pointSize) node[below] {$\left(-1;0\right)$};
            \filldraw[color= colorEmphasisCyan] (0, -0.3333333333333333) circle (\pointSize) node[right] {$\left(0;- \frac{1}{3}\right)$};
            \filldraw[color= colorEmphasisCyan] (-2, 0.2) circle (\pointSize) node[above right] {$\left(-2;\frac{1}{5}\right)$};
            \filldraw[color= colorEmphasisCyan] (1.5, 0.2777777777777778) circle (\pointSize) node[above right] {$\left(\frac{3}{2};\frac{5}{18}\right)$};
            \filldraw[color= colorEmphasisCyan] (-5, -0.18181818181818182) circle (\pointSize) node[below] {$\left(-5;- \frac{2}{11}\right)$};
            \filldraw[color= colorEmphasisCyan] (-2.75, 1.0769230769230769) circle (\pointSize) node[above right] {$\left(- \frac{11}{4};\frac{14}{13}\right)$};
            \filldraw[color= colorEmphasisCyan] (-4, -0.3333333333333333) circle (\pointSize) node[right] {$\left(-4;- \frac{1}{3}\right)$};
         \end{tikzpicture}
         \caption{Đồ thị của $f(x) = \frac{x + 1}{2 x^{2} + 5 x - 3}$}
         \label{fig:ham_so_mot_bien:phan_thuc:1_x2_5x_3}
      \end{figure}
   \end{minipageindent}
}

Thay giá trị của $y$ vào mẫu:

\begin{align*}
   2x^2 + 5x - 3 &= (x + 3)(2x - 1)\\
   &= \left(\frac{\sqrt{49y^2-2y+1}-5y+1}{4y}+3\right)\left(2\cdot\frac{\sqrt{49y^2-2y+1}-5y+1}{4y}-1\right) \\
   &= \frac{\sqrt{49y^2-2y+1}+7y+1}{4y}\cdot\frac{\sqrt{49y^2-2y+1}-7y+1}{2y} \\
   \displaybreak[1]
   &= \frac{\left(\sqrt{49y^2-2y+1} + 1\right)^2 - (7y)^2}{8y^2}\\
   &= \frac{49y^2-2y+1 + 2\sqrt{49y^2-2y+1} + 1 - 49y^2}{8y^2}\\
   &= \frac{2\sqrt{49y^2-2y+1} - 2y + 2}{8y^2}\\
   &= \frac{\sqrt{49y^2-2y+1} - y + 1}{4y^2}.
\end{align*}

Mẫu số này khác $0$ do nếu bằng $0$ thì chúng ta sẽ có
\begin{align*}
   \sqrt{49y^2-2y+1} - y + 1 &= 0\\
   \iff \sqrt{49y^2-2y+1} &= y - 1\\
   \implies 49y^2 - 2y + 1 &= (y - 1)^2\\
   \iff 49y^2 - 2y + 1 &= y^2 - 2y + 1\\
   \iff 48y^2 &= 0\\
   \iff y &= 0
\end{align*} mâu thuẫn với giả thiết $y\neq 0$. Lấy tử số chia cho mẫu số và khử bỏ thừa số chúng để có
$$f\left(\frac{\sqrt{49y^2-2y+1}-5y+1}{4y}\right) = \frac{\frac{\sqrt{49y^2-2y+1}-y+1}{4y}}{\frac{\sqrt{49y^2-2y+1} - y + 1}{4y^2}} = y.$$

Chúng ta đã thể hiện rằng mọi số $y$ đều có thể biểu diễn thông qua $f(x)$. Vậy tập giá trị của $f(x)$ là $\mathbb{R}$.

5. Giải tập xác định:

\begin{align*}
   x^2 + 2x + 1 &\neq 0 \\
   \iff (x + 1)^2 &\neq 0 \\
   \iff x &\neq -1.
\end{align*}

Qua đó, chúng ta có tập xác định của $f(x)$ là $\mathbb{R} \setminus \left\{-1\right\}$.

Giải tập giá trị sẽ khó hơn. Gọi $y\in\mathbb{R}$ và giả sử $y = f(x)$. Khi này,

\begin{align}
   y &= \frac{x^2 - 3x - 2}{x^2 + 2x + 1} \nonumber\\
   \implies y(x^2 + 2x + 1) &= x^2 - 3x - 2 \nonumber\\
   \iff yx^2 + 2yx + y &= x^2 - 3x - 2 \nonumber\\
   \iff (y - 1)x^2 + (2y + 3)x + (y + 2) &= 0. \label{eq:ham_so_mot_bien:phan_thuc:p5}
\end{align}

Nếu $y = 1$ thì từ \ref{eq:ham_so_mot_bien:phan_thuc:p5}, $5x + 3 = 0 \iff x = -\frac{3}{5}$. Vậy $1$ có thể là kết quả của $f(x)$.

Trong trường hợp còn lại, coi \ref{eq:ham_so_mot_bien:phan_thuc:p5} là phương trình bậc hai với $x$ là nghiệm. Để tồn tại nghiệm thì $\Delta \geq 0$, với $\Delta$ là 
\begin{align*}
   &= (2y + 3)^2 - 4(y - 1)(y + 2) \\
   &= 4y^2 + 12y + 9 - 4(y^2 + y - 2) \\
   &= 4y^2 + 12y + 9 - 4y^2 - 4y + 8 \\
   &= 8y + 17.
\end{align*}
Từ đó, để $\Delta \geq 0$ thì $8y + 17 \geq 0 \iff y \geq -\frac{17}{8}$.

Kiểm tra ngược tập giá trị, chúng ta đã biết $1$ thuộc tập giá trị này. Với mọi giá trị $y \geq -\frac{17}{8}$ khác $1$, đặt $x = \frac{2y + 3 - \sqrt{8y + 17}}{2(1 - y)}$, khi này

\begin{align*}
   f(x) &= \frac{x^2 - 3x - 2}{x^2 + 2x + 1} = \frac{x^2 - 3x - 2}{x^2 + 2x + 1} - y + y = \frac{x^2 -3x - 2 - yx^2 - 2yx - y}{\left(x + 1\right)^2} + y\\
   &= \frac{(1-y)x^2 - (2y+3)x - (2+y)}{\left(x + 1\right)^2} + y = \frac{x^2 - \left(\frac{2y+3}{1-y}\right)x - \frac{y+2}{1-y}}{\left(x + 1\right)^2} + y\\
   \displaybreak[1]
   &= \frac{x^2 - 2\cdot x\cdot \left(\frac{2y+3}{2(1-y)}\right) + \left(\frac{2y+3}{2(1 - y)}\right)^2 - \left(\frac{2y+3}{2(1-y)}\right)^2 - \frac{y+2}{1 - y} }{(x + 1)^2} + y \\
   &= \frac{\left(x - \frac{2y + 3}{2(1-y)}\right)^2-\frac{8y + 17}{4(1 - y)^2}}{(x + 1)^2} + y \\
   &= \frac{\left(\frac{2y + 3 - \sqrt{8y + 17}}{2(1 - y)} - \frac{2y + 3}{2(1-y)}\right)^2-\frac{8y + 17}{4(1 - y)^2}}{(x + 1)^2} + y\\
   &= \frac{\left(\frac{-\sqrt{8y+17}}{2(1-y)}\right)^2 - \frac{8y + 17}{4(1 - y)^2}}{(x + 1)^2} + y = \frac{\frac{8y + 17}{4(1 - y)^2} - \frac{8y + 17}{4(1 - y)^2}}{(x + 1)^2} + y = y.
\end{align*}

Vậy tập giá trị của $f(x)$ là $\left[-\frac{17}{8}; +\infty\right)$. Đồ thị của $f(x) = \frac{x^2 - 3x - 2}{x^2 + 2x + 1}$ được thể hiện trong \ref{fig:ham_so_mot_bien:phan_thuc:1t3t2_121}.

\begin{figure}[H]
	\centering
	\begin{tikzpicture}
		\draw[->] (-6, 0) -- (6, 0) node[right] {$x$};
		\draw[->] (0, -2.5) -- (0, 5.5)  node[above] {$y$};
		\draw[graph thickness, samples=80, color=colorEmphasisCyan, domain=-6.000:-2.423] plot (\x, {((\x)^2 - 3*(\x) - 2)/((\x)^2 + 2*(\x) + 1)});
		\draw[graph thickness, samples=80, color=colorEmphasisCyan, domain=-0.688:6.000] plot (\x, {((\x)^2 - 3*(\x) - 2)/((\x)^2 + 2*(\x) + 1)});
		\filldraw[color= colorEmphasisCyan] (-3.0, 4.0) circle (\pointSize) node[below right] {$\left(-3{,}0;4{,}0\right)$};
		\filldraw[color= colorEmphasisCyan] (-4.5, 2.5918367346938775) circle (\pointSize) node[below right] {$\left(-4{,}5;2{,}592\right)$};
		\filldraw[color= colorEmphasisCyan] (0.0, -2.0) circle (\pointSize) node[right] {$\left(0{,}0;-2{,}0\right)$};
		\filldraw[color= colorEmphasisCyan] (-0.2, -2.125) circle (\pointSize) node[below] {$\left(-\frac{1}{5};-\frac{17}{8}\right)$};
		\filldraw[color= colorEmphasisCyan] (1.0, -1.0) circle (\pointSize) node[below right] {$\left(1{,}0;-1{,}0\right)$};
		\filldraw[color= colorEmphasisCyan] (-0.5, -1.0) circle (\pointSize) node[left] {$\left(-0{,}5;-1{,}0\right)$};
		\filldraw[color= colorEmphasisCyan] (3.0, -0.125) circle (\pointSize) node[above left] {$\left(3{,}0;-0{,}125\right)$};
      \filldraw[color= colorEmphasisCyan] (-0.6, 1) circle (\pointSize) node[left] {$\left(-\frac{3}{5};1\right)$};
	\end{tikzpicture}
	\caption{Đồ thị của $f(x) = \frac{x^{2} - 3 x - 2}{x^{2} + 2 x + 1}$}
   \label{fig:ham_so_mot_bien:phan_thuc:1t3t2_121}
\end{figure}


\exercise Phác thảo đồ thị của những hàm sau:

\begin{multicols}{2}
   \begin{enumerate}
      \item $\displaystyle f(x) = \frac{2x}{x^2 + 1} + 1$;
      \item $\displaystyle f(x) = \frac{x^4 + 1}{3x^2} - x$;
      \item $\displaystyle f(x) = \frac{15x^3 + x^2 - 22x - 8}{3x^2 + 3x + 8}$;
      \item $\displaystyle f(x) = \frac{x}{x + 2} + \frac{1}{x - 2}$;
      \item $\displaystyle f(x) = \frac{x + 2}{x} \cdot \frac{x + 3}{x + 1}$;
      \item $\displaystyle f(x) = \frac{\frac{x^3 + 3x^2 + 3x + 1}{x^4 + 4}}{\frac{2x^2 + 2}{3x^2 + 6x + 6}}$.
   \end{enumerate}
\end{multicols}

\solution