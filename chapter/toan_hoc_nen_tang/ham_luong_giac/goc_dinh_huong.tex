\subsection{Góc định hướng}

\ % Lùi đầu dòng

\defText{Góc không định hướng} (gọi tắt là \defText{góc}) là hình gồm hai tia có chung gốc. Gốc chung của hai tia là \defText{đỉnh} của góc. Hai tia chắn góc được gọi là \defText{cạnh} của góc.

\defText{Số đo góc} về mặt trực quan là độ mở của góc. Để đo số đo góc, chúng ta sử dụng đơn vị \defText{độ} hoặc \defText{ra-đi-an}. Gọi \defText{đường tròn đơn vị} là đường tròn có bán kính bằng $1$ đơn vị độ dài. Vẽ đường tròn đơn vị với tâm nằm ở đỉnh của góc, khi đó, số đo góc được đo bằng độ dài của cung chắn góc. Định nghĩa hai đơn vị đo góc như sau:
\begin{itemize}
   \item Độ: Chia đường tròn làm $360$ phần bằng nhau. 
\end{itemize}
