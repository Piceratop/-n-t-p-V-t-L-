\subsection{Định nghĩa hàm lượng giác}

\ % Lùi đầu dòng

Để định nghĩa được hàm lượng giác của một góc, cần phải vẽ góc đó trong hệ trục tọa độ vuông góc như sau: Vẽ cạnh đầu của góc trùng với trục hoành, gốc tọa độ là đỉnh của góc, cạnh cuối của góc nằm trong mặt phẳng tọa độ. Khi đó, giao điểm của cạnh cuối với đường tròn đơn vị là điểm có tọa độ $(\cos \left(\theta\right), \sin \left(\theta\right))$, trong đó $\theta$ là số đo của góc.