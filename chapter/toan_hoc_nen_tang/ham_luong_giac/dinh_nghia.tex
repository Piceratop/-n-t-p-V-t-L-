\subsection{Định nghĩa hàm lượng giác}

\ % Lùi đầu dòng

{
   \begin{minipageindent}{0.48\textwidth}
      Để định nghĩa được hàm lượng giác của một góc, cần phải vẽ góc đó trong hệ trục tọa độ vuông góc như sau: Vẽ cạnh đầu của góc trùng với trục hoành, gốc tọa độ là đỉnh của góc, cạnh cuối của góc nằm trong mặt phẳng tọa độ. Khi đó, giao điểm của cạnh cuối với đường tròn đơn vị là điểm có tọa độ $\defMath{\left(\cos \left(\theta\right); \sin \left(\theta\right)\right)}$, trong đó $\theta$ là số đo của góc.

      Cần phải cẩn thận rằng, ví trí của giao điểm nằm trong các góc phần tư khác góc thứ nhất thì $\cos \left(\theta\right)$ và $\sin \left(\theta\right)$ có thể nhận giá trị âm.

      Ngoài ra, cón những hàm lượng giác thông dụng khác với định nghĩa như sau:
      \begin{equation*}
         \begin{array}{ccc}
            \defMath{\tan \left(\theta\right) = \frac{\sin \left(\theta\right)}{\cos \left(\theta\right)}}; &\qquad& \defMath{\cot \left(\theta\right) = \frac{\cos \left(\theta\right)}{\sin \left(\theta\right)}}; \\
            \defMath{\sec \left(\theta\right) = \frac{1}{\cos \left(\theta\right)}}; &\qquad& \defMath{\csc \left(\theta\right) = \frac{1}{\sin \left(\theta\right)}}.
         \end{array}
      \end{equation*}

      Những giá trị $\theta$ làm cho $\cos \left(\theta\right) = 0$ sẽ khiến cho $\tan \left(\theta\right)$ và $\sec \left(\theta\right)$ không xác định. Tương tự, nếu $\sin \left(\theta\right) = 0$ thì $\cot \left(\theta\right)$ và $\csc \left(\theta\right)$ không xác định. Từ định nghĩa đó, dễ dàng nhận thấy rằng $$\tan \left(\theta\right) = \frac{1}{\cot \left(\theta\right)}.$$
   \end{minipageindent}
   \hfill
   \begin{minipageindent}{0.48\textwidth}
      \begin{figure}[H]
         \centering
         \begin{tikzpicture}
            \draw[->] (-3, 0) -- (4, 0) node[right] {$x$};
            \draw[->] (0, -3) -- (0, 4) node[above] {$y$};
            
            \draw[very thick] (2, 4) -- (0, 0) -- (4, 0);
            \filldraw (0, 0) circle (\pointSize) node[below left] {$O$};
            \draw (0, 0) circle (2);
            \pgfmathsetmacro{\denominator}{sqrt(5)}
            \pgfmathsetmacro{\xcoord}{2/\denominator}
            \pgfmathsetmacro{\ycoord}{4/\denominator}

            \draw[color=colorEmphasisCyan, dashed] ({\xcoord}, {\ycoord}) -- (0, {\ycoord});
            \draw[color=colorEmphasisCyan, measuring arrow] ({\xcoord}, {\ycoord + \guideLineLength / 2}) -- (0, {\ycoord + \guideLineLength / 2});
            \draw[\guideLineThickness] ({\xcoord}, {\ycoord}) -- ({\xcoord}, {\ycoord + \guideLineLength});
            \node[above, color=colorEmphasisCyan] at ({\xcoord / 2}, {\ycoord + \guideLineLength / 2}) {$\cos (\theta)$};

            \draw[color=colorEmphasis, dashed] ({\xcoord}, {\ycoord}) -- ({\xcoord}, 0);
            \draw[color=colorEmphasis, measuring arrow] ({\xcoord + \guideLineLength / 2}, {\ycoord}) -- ({\xcoord + \guideLineLength / 2}, 0);
            \draw[\guideLineThickness] ({\xcoord + \guideLineLength}, {\ycoord}) -- ({\xcoord}, {\ycoord});
            \node[right, color=colorEmphasis] at ({\xcoord + \guideLineLength / 2}, {\ycoord / 2}) {$\sin (\theta)$};

            \draw[single measuring arrow] (0.75, 0) arc[start angle=0, end angle={atan(2)}, radius=0.75];
            \node[above] at (0.4, 0) {$\theta$};

            \filldraw ({\xcoord}, {\ycoord}) circle (\pointSize) node[above right] {$P\left(\textcolor{colorEmphasisCyan}{\cos (\theta)}; \textcolor{colorEmphasis}{\sin (\theta)}\right)$};
            \filldraw ({\xcoord}, 0) circle (\pointSize) node[below right] {$S$};
            \filldraw (0, {\ycoord}) circle (\pointSize) node[below left] {$C$};
         \end{tikzpicture}
         \caption{Biểu diễn $\cos (\theta)$ và $\sin (\theta)$}
         \label{fig:toan_hoc_nen_tang:dinh_nghia:dn_sin_cos}
      \end{figure}
   \end{minipageindent}
}


